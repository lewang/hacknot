% Created 2015-11-24 K 09:55
\documentclass{article}
\usepackage[utf8]{inputenc}
\usepackage[T1]{fontenc}
\usepackage{graphicx}
\usepackage{longtable}
\usepackage{hyperref}
\usepackage{natbib}
\usepackage{amssymb}
\usepackage{amsmath}
\usepackage{geometry}
\geometry{a4paper,left=2.5cm,top=2cm,right=2.5cm,bottom=2cm,marginparsep=7pt, marginparwidth=.6in}
\usepackage[utf8]{inputenc}
\usepackage[T1]{fontenc}
\usepackage{fixltx2e}
\usepackage{graphicx}
\usepackage{grffile}
\usepackage{longtable}
\usepackage{wrapfig}
\usepackage{rotating}
\usepackage[normalem]{ulem}
\usepackage{amsmath}
\usepackage{textcomp}
\usepackage{amssymb}
\usepackage{capt-of}
\usepackage{hyperref}
\author{Ed Johnson}
\date{\today}
\title{Hacknot\\\medskip
\large Essays on Software Development}
\hypersetup{
 pdfauthor={Ed Johnson},
 pdftitle={Hacknot},
 pdfkeywords={},
 pdfsubject={},
 pdfcreator={<a href="http://www.gnu.org/software/emacs/">Emacs</a> 24.4.1 (<a href="http://orgmode.org">Org</a> mode 8.3.2)}, 
 pdflang={English}}
\begin{document}

\maketitle
\tableofcontents


\section{Hacknot}
\label{sec:orgheadline423}

ASIN: B002EVD14U This work is licensed under the Creative Commons
Attribution License. To view a copy of this license:

\begin{itemize}
\item Visit \url{http://creativecommons.org/licenses/by/2.5/} or
\item Send a letter to: Creative Commons, 543 Howard Street, 5th Floor, San
Francisco, California, 94105, USA
\end{itemize}

\subsection{Foreword}
\label{sec:orgheadline3}

\subsubsection{The Hacknot Web Site}
\label{sec:orgheadline1}

Hacknot began life in 2001 as an internal mailing list at the
multinational telecommunications company I was then working for. As part
of the activities of the local Software Engineering Process Group, I was
looking for a way to promote discussion amongst staff about software
engineering related issues, and hopefully encourage people to learn
about the methods and techniques that could be used to improve the
quality of their work. A creative colleague came up with the name
“Hacknot” for the mailing list \ldots{} a pun on the geek web site
“slashdot.”

A few years later, when I left the company, I restarted Hacknot as an
externally hosted mailing list, with many of the same members as in its
last incarnation. In 2003, I was looking for a coding exercise in J2EE,
the main technologies of which had passed me by while I was busy working
in other areas. Growing tired of building play-applications like bug
trackers and online store simulations, I decided to create a web version
of the Hacknot mailing list. I figured it would give me a “real world”
context in which to learn about J2EE, and also a project that I could
pursue without the interference of the usually inept management that so
plagued the development efforts of my working life.

So in 2003 the Hacknot web site was born. In Australia, domain name
registration rules restrict ownership of “.com” domains to commercial
enterprises, so I chose the next best top-level domain, which was
“.info”.

Initially, I imagined that the web site would host works by a variety of
authors, myself included. But when it came time to put pen to paper,
almost all of those who had previously expressed interest in
participating suddenly backed off, leaving me to write all the content
myself.

Many of the essays on Hacknot take a stab at some sacred cow of the
software development field, such as Extreme Programming, Open Source and
Function Point Analysis. These subjects tend to attract fanatical
adherents who don't take kindly to someone criticizing what for them has
become an object of veneration. The vitriol of some of the e-mail I
receive is testament to the fact that some people need to get out more
and get a sense of perspective. It is partially because of the
controversial nature of these topics that I have always written behind a
pseudonym; either “Ed” or “Mr. Ed”. I also favor anonymity because it
makes a nice change from the relentless self-promotion engaged in by so
many members of the IT community.

\subsubsection{The Hacknot Book}
\label{sec:orgheadline2}

This book contains 46 essays originally published on the Hacknot web
site between 2003 and 2006. The version of each essay appearing in the
book is substantially the same as the online version, with some minor
revisions and editing.

You can freely download PDFs of this book with page sizes of 6” x 9” or
A4, by visiting \url{http://www.hacknot.info}. There you will also find
instructions on how you can obtain a hard-cover copy, for the price of
the binding and postage. Please send any comments or corrections to
editor@hacknot.info.

\subsection{Peopleware}
\label{sec:orgheadline135}

\subsubsection{The A to Z of Programmer Predilections  \footnote{First published 24 Jan 2006 at
\url{http://www.hacknot.info/hacknot/action/showEntry?eid=81}}}
\label{sec:orgheadline30}

There is a realization that comes with the accrual of software
development experience across a reasonable number of organizations, and
it is this:

\begin{quote}
\emph{Though the names change, the problems remain the same.}
\end{quote}

Traveling from project to project, from one organization to another,
across disparate geographies, domains and technologies, I am repeatedly
struck more by the similarities between the projects I work on than
their differences. Scenes from one job seem to replay in the next one,
only with a different set of actors.

You might finish a gig in which you've seen a project flop due to
inadequate consultation with end users, only to find your next project
heading down the same path for exactly the same reason. And it generally
doesn't matter how much you jump up and down and try and warn your new
project team that you've seen the disastrous results of similar actions
in the past. They will ignore you, insisting that their situation is
somehow different. You will stand back and watch in horror as the whole
scenario plays out as you knew it would, all the while unable to do
anything more to prevent it. The IT contractor's career can be like some
cruel matinee of "Groundhog Day" -- without the moral resolution at the
end.

But this technological déjà vu is not limited to technical scenarios -
it extends to people. I find myself working with the same programmers
over and over again. Their names and faces change, but their
personalities and predilections are immediately recognizable. I find
myself playing mental games of "Snap" with my fellow developers. "Bob
over there is just like Ian from Acme. James is this workplace's
equivalent of Charles from that financial services gig I had last year"
-- and so on.

Sometimes I fancy that I have met them all. There will be no new
developers for me to work with in future -- only the reanimated ghosts
of projects past. The same quirks and foibles that I've endured in the
past will haunt me the rest of my days.

I've listed below the cast of characters that have been following me
around for some years now. Coincidentally, there are exactly twenty six
of them, one for each letter of the alphabet. Perhaps you've encountered
some of them yourself. Perhaps you're one of them. If so -- please go
away and find someone else to bug.

\begin{enumerate}
\item Arrogant Arthur
\label{sec:orgheadline4}

The three hardest words in any techie's vocabulary are "I don't know".
Arthur never has to struggle with them, for he knows everything. Any
technology you might name - he's an expert. Any problem you might have
-- he's solved it before. No matter what challenge he's assigned -- he's
sure it will be easy. Whenever Arthur appears to have made a mistake,
closer investigation will reveal that the fault in fact lies with
someone or something else. Arthur is a pretty handy conversationalist.
Whenever you're having a technical discussion with someone and he is
within earshot, Arthur will generally join in and quickly dominate the
discussion with his displays of erudition. Uncertainty and self-doubt
are states of mind that Arthur is entirely unfamiliar with. Arthur has a
tendency to make big generalizations and sweeping statements, as if to
imply that he has the certainty that only comes from vast experience.

\item Belligerent Brian
\label{sec:orgheadline5}

Nobody in the office is particularly fond of Brian. Sure, he's a smart
guy and seems to be technically well informed, but he has such a
strident and aggressive manner that it's difficult to talk with him for
any length of time without feeling that you are under attack. Brian
likes it that way and his hostile manner is entirely intentional. You
see, Brian is a go-getter. Highly ambitious and energetic, he is
determined to advance up the corporate ladder, no matter who he has to
step on in the process. Whenever any action is undertaken or decision
made, there is always a part of him thinking "How will this make me look
to my manager?" It's not surprising then that not all of Brian's
decisions are good ones. He has been known to select cutting edge
technologies simply for their buzzword compliance, betting that cool
acronyms and shiny new methodologies will make him appear progressive
and forward-looking. Although he regularly makes mistakes, Brian never
admits to any of them, and generally blames third parties, vendors and
colleagues for errors that are actually his own.

\item \texttt{C++} Colin
\label{sec:orgheadline6}

Colin is the local language bigot, whose language of preference is C++.
He began programming in C, moved on to C++ when commercial forces threw
the OO paradigm at him, and has been working in C++ ever since. Colin
has watched the ascent of Java with a mixture of disdain and veiled
jealousy. Initially, it was easy to defend C++ against criticisms from
the Java camp, by pointing to C++'s superior performance. But with the
growing speed of JVMs, this advantage has been lost. Now, most of the
advantages that Colin claims for C++ are the same language features that
Java enthusiasts see as disadvantages. Java developers (or, "Java
weenies" as Colin is fond of calling them) point to automatic memory
reclamation as an eliminator of a whole category of bugs that C++
developers must still contend with. Colin sees garbage collection as
disempowering the programmer, referring to the random intrusion of
garbage collection cycles as payback for those too lazy to free memory
themselves. Java weenies consider the absence of multiple inheritance in
Java an advantage because it avoids any confusion over the rules used to
resolve inheritance of conflicting features; Colin sees it as an
unforgivable limitation to effective and accurate domain modeling. Java
weenies consider C++'s operator overloading to be an archaic syntax
shortcut, rife with potential for error; Colin sees it as a concise and
natural way to capture operations upon objects. Colin displays a certain
bitterness, resulting from the dwindling variety of work available to
him within the language domain he is comfortable with.

\item Distracted Daniel
\label{sec:orgheadline7}

Daniel's mind is only ever half on the job, or to put it another way, he
doesn't have his head in the game. Daniel lives a very full life --
indeed, so full that his private life overflows copiously into his
professional one. He has several hobbies that he is passionate about,
and he is always ready to regale a colleague with tales of his weekend
exploits in one of them. It looks as if his job is just a way of funding
his many (often expensive) hobbies. His work is strictly a nine to five
endeavor, and it would be very rare to find him reading around a
particular work-related topic in his own time, or putting in an
extraordinary effort to meet a deadline or project milestone. He is
constantly taking off at lunch times to take care of one task or
another, and does not seem to be particularly productive even when he is
in the office. Daniel refers to this as "leading a balanced life". He
may be right.

\item Essential Eric
\label{sec:orgheadline8}

Eric knows that knowledge is power. Partly by happenstance but mostly by
design, Eric has become irreplaceable to his employer. There just seems
to be a vast amount of technical and procedural arcana that only Eric
knows. If he should ever leave, the company would be in a mess, as he
would take so much critical information with him. This gives him a good
deal of bargaining power with management, and good job security. A few
of the company's managers have recognized the unhealthy dependence that
exists upon him, and have attempted to document some of the valuable
knowledge about certain pieces of software central to the business, but
Eric always finds a way to get out of it. There always seems to be
something more pressing for him to do, and if he is forced to put pen to
paper, what results tends to be incoherent nonsense. It seems that he
just can't write things down - or rather, that he chooses to be so poor
at it that no one even bothers to ask him to document things any more.
Eric is not keen to help others in those domains that he is master of,
as he doesn't want to dilute the power of his monopoly.

\item Feature Creep Frank
\label{sec:orgheadline9}

Most of the trouble that Frank has got himself into over the years has
been heralded by the phrase "Wouldn't if be cool if \ldots{} ". No matter how
feature-laden his current project may be, Frank can always think of one
more bell or whistle to tack onto it that will make it so much cooler.
Having decided that a particular feature is critical to user acceptance
of the application, it is a very difficult task to stop him adding it
in. He has been known to work nights and weekends just to get his
favorite feature incorporated into the code base -- whether he has got
permission to do so or not. Part of Frank's cavalier attitude to these
"enhancements" comes from his unwillingness to consider the long term
consequences of each addition. He tends to think of the work being over
once the feature has been coded, but he fails to consider that this
feature must now be tested, debugged and otherwise maintained in all
future versions of the product. Once the users have seen it, they may
grow accustomed to it, and so removing it from future versions may well
be impossible. They may even like the feature so much that they begin
requesting extensions and modifications to it, creating further burden
on the development team. Frank justifies his actions to others in terms
of providing value to users, and often professes a greater knowledge of
the user demographic than what he actually possesses, so that he can
claim how much the users will need a particular feature. But Frank's
real motivations are not really about user satisfaction, but are about
satisfying his own ego. Each new feature is an opportunity for him to
demonstrate how clever he is, and how in touch with the user community.

\item Generic George
\label{sec:orgheadline10}

George delights in the design process. Pathologically incapable of
solving just the immediate problem at hand, George always creates the
most generic, flexible and adaptable solution possible, paying for the
capabilities he thinks he will need in the future with extra complexity
now. Sadly, George always seems to anticipate incorrectly. The castles
in the air that he continually builds rarely end up with more than a
single room occupied. Meanwhile, everyone must cope with the inordinate
degree of time and effort that is needlessly invested in managing the
complexity of an implementation whose flexibility is never required. It
is a usual characteristic of George's work that it takes at least a
dozen classes working together to accomplish even trivial functionality.
He is generally the first to declare "Let's build a framework" whenever
the opportunity presents itself, and the last to want to use the
framework thus created.

\item Hacker Henry
\label{sec:orgheadline11}

Henry considers himself to be a true hacker -- a code poet and geek
guru. Still in the early stages of his career, he spends most of his
life in front of a keyboard. Even when not at work, he is working on his
own projects, participating in online discussion forums and learning
about the latest languages and utilities. Software is his principal
passion in life. This single-minded pursuit of technical knowledge has
made him quite proficient in many areas, and has engendered a certain
arrogance that generally manifests as a disdain directed towards those
of his colleagues whom he regards as not being "true hackers". For his
managers, Henry is a bit of a problem. They know that they can rely on
him to overcome pretty much any technical challenge that might be
presented to him, provided that the solution can be reached by doing
nothing but coding. For unless it's coding, Henry's not interested. He
won't document anything; certainly not his code, because he feels that
good code is self-documenting. He is early enough into his career to
have not yet been presented with the task of adopting a large code base
from someone who subscribes to that same belief, and to have thereby
seen the problems with it. Also, Henry can generally only be given
"mind-size" tasks to do. His tasks have to be small and well defined
enough for him to fit all their details in his head at once, as he
simply refuses to write anything down. The architecture of
enterprise-scale systems will likely forever be a mystery to him as he
does not possess, and has no interest in developing, the facility with
abstractions and modeling that is necessary to manage the design of
large systems.

\item Incompetent Ian
\label{sec:orgheadline12}

Ian is a nice enough guy but is genuinely incapable of performing most
of the job functions his position requires. It's not clear whether this
is a result of inadequate education, limited experience or simply a lack
of native ability. Either way, it is clear to anyone who works with Ian
for any length of time that he is not really on the ball, and takes a
very long time to complete even basic tasks. Worst of all, Ian seems to
be blissfully unaware of his own incompetence. This can make for some
embarrassing situations for everyone, as Ian's attempts to weigh in on
technical discussions leave him looking naive and ignorant -- which he
also fails to notice. Ian tends to get work based upon his personable
manner and the large number of friends he has working in the industry.
Most of his employers have come to view him as a "retrospective hiring
error".

\item Jailbird John
\label{sec:orgheadline13}

John has been working for his current employer a long time. A very long
time. Longer than most of the senior management in fact. John has been
working here so long that it is highly unlikely he will ever be able to
work anywhere else. Over the years, his skill set has deteriorated so
greatly and become so stale that he has become an entirely unmarketable
commodity. He knows all there is to know about the company's legacy
applications -- after all, he wrote most of them. He has been keeping
himself employed for the last decade just patching them up and making
one piecemeal addition after another in order to try and keep them
abreast of the business's changing function. Tired of chasing the latest
and greatest technologies, he has not bothered learning new ones,
sticking to the comfortable territory defined by the small stable of
dodgy applications he has been shepherding for some years. John gets
along with everyone, particularly those more senior to him. He can't
afford the possibility of getting into conflict with anyone who might
influence his employment status, as he knows that this will likely be
the last good job he ever has. So he tries to stay under the radar,
hoping that the progressive re-engineering of his pet applications with
more modern technologies takes long enough for him to make it over the
finish line.

\item Kludgy Kevin
\label{sec:orgheadline14}

Kevin is remarkably quick to fix bugs. It seems that he's no sooner
started on a bug fix than he's checking in the solution. And then, as if
by magic, the very same bug reappears. "I thought I fixed that",
declares Kevin -- and indeed he did -- but not properly. In his rush to
move on to something else, Kevin invariably forgets to check that his
"fix" works correctly under some boundary condition or special case, and
ends up having to go back and fix it again. Sometimes a third or even
fourth attempt will be necessary. This is Kevin's version of "iterative
development."

\item Loudmouth Lincoln
\label{sec:orgheadline15}

Terror of the cubicle farm, Lincoln incurs the ire of all those who sit
anywhere near him, but remains blissfully unaware that he is so
unpopular. His voice is louder than anyone else's by a least a factor of
two, and he seems unable to converse at any volume other than full
volume. When Lincoln is talking, everyone else is listening, whether
they want to or not. People in his part of the office know a great deal
more about Lincoln's personal life than they would like, as they have
heard one end of the half dozen or so telephone calls that he seems to
receive from his wife every day. Lincoln's favorite instrument of
torture is the speakerphone. He always listens to his voicemail on
speakerphone each morning, so that he can unpack his briefcase while
doing so. He also likes to place calls on speakerphone so that his hands
are free to type at his keyboard while conversing with someone else. He
either doesn't realize or doesn't care that he is disturbing those
nearby. Nobody seems to be game enough to tell him how inconsiderate he
is being.

\item Martyr Morris
\label{sec:orgheadline16}

Morris is very conscious of the impression others form of him. Probably
a little too concerned. He has observed that many of his colleagues
associate long hours with hard work and dedication. The longer the
hours, the harder you're working -- and having a reputation as a hard
worker can only be a good thing when it comes performance review time.
So Morris makes sure he is at the office when his boss arrives of a
morning, and that he is still working away when his boss leaves of an
afternoon. Everyone agrees that Morris certainly puts in the hard yards,
but are a little perplexed as to why his code is so often buggy and
poorly structured. In fact, it seems like Morris has to put in extended
hours in order to compensate for the poor quality of his work. The net
result is that he gets almost as much achieved as his team mates who
work more sensible hours. Morris hasn't yet twigged to the fact that his
defect injection rate rises dramatically as he fatigues, meaning that
the extra hours he works often have a negative effect on his
productivity. Worse yet, his know-nothing manager rewards him for his
dedication, thereby reinforcing the faulty behavior.

\item Not-Invented-Here Nick
\label{sec:orgheadline17}

Nick has an overwhelming drive to write everything himself. Due to
hubris and ambition, he is rarely satisfied with buying a third party
utility or library to help in his development efforts. It seems to him
that the rest of the industry must be incompetent, for every time he
looks to buy rather than build, he finds so many shortcomings in the
products on offer that he invariably concludes that there's nothing for
it but to write the whole thing himself. It also seems that his
particular requirements are always so unique that no generally available
tool has just the functionality that he needs. Not wanting to work
inefficiently, he insists on only using tools that do exactly what he
wants -- nothing more, nothing less. Little wonder then that he finds
himself having to write such fundamental utilities as text editors, file
transfer programs, string and math utility libraries. The real problem
is not that Nick's requirements are so unique, but that he deliberately
fabricates requirements so specific that he can find commercial
offerings lacking, and thereby justify reinvention of those offerings
himself. In short, he is looking for excuses to write what he considers
to be the "fun stuff" (the development tools) rather than the "boring
stuff" (the application code). He generally has little difficulty in
finding such justifications. Most people who work with Nick note with
interest that the tools that he writes himself are rarely of the quality
of the equivalent commercial offerings.

\item Open Source Oliver
\label{sec:orgheadline18}

Oliver is very enthusiastic about open source software development. He
contributes to several open source projects himself, and tries to
incorporate open source products into his projects wherever possible --
and it's always possible; mainly because Oliver begins a project for the
principal purpose of providing himself with an opportunity to try out
the latest and greatest CVS build from Apache, Jakarta or wherever.
Oliver rarely has to justify his technology selections to his
colleagues, as he is always sure to surround himself with other open
source believers. On occasions when he needs to explain the failure or
buggy nature of some open source package, he relies upon the old saw "we
can always fix it ourselves". However there never seems to be enough
time in the schedule for this to actually occur; so every release of his
project bristles with the underlying warts of its open source
components. If all else fails, it can at least be said that the price is
right.

\item Process Peter
\label{sec:orgheadline19}

If you want to see Peter get worked up, just start a discussion with him
about the poor state of software development today. He will hold forth
at length, and with passion, on where it has all go wrong. And Peter has
decided that all of software's woes have a common genesis -- a lack of
disciplined process. Peter's career history reads like a marketing
brochure of process trends. BPR, Clean Room, Six Sigma, ISO -- he's been
a whole-hearted enthusiast of them all at one time or another. His
dedication to strict process adherence as a panacea to a project's
quality ills is absolute, and he will do almost anything to ensure that
ticks appear in the relevant boxes. Unfortunately, this uncompromising
approach is often self-defeating, as it denies him the flexibility to
adapt quality levels on a case-by-case basis. It has also made him more
than a few enemies over the years. He is prone to considering the people
component of software development as a largely secondary consideration,
and views programmers a little like assembly line production workers --
interchangeable parts whose individual talents and proclivities are not
so important as the procedures they follow to do their work. Those
subject to such views tend to find it more than a little dehumanizing
and impersonal.

\item Quiet Quincy
\label{sec:orgheadline20}

Quincy is one of those guys who has no need to brag about his technical
skills or the depth of his technical knowledge. He's not much interested
in being "alpha geek" at the office, he just wants to do a good job and
then go home to his wife and children. Quietly spoken and unassuming, he
looks on with amusement at Zealous Zack's ever-changing enthusiasms and
shakes his head, knowing that in a few more years Zack will have gained
enough experience to know that the computing industry is full of "next
big things" that generally aren't. Given a task, he just sits down and
does it. He doesn't succumb to heroic bug-fixing and late night coding
efforts -- his code is good enough to begin with that won't get many
pats on the back from management, whose attention will largely be
captured by the technical prima donnas that swan around the project
space, dropping buzzwords and acronyms like they were the names of
celebrities they knew personally. But without Quincy and those of his
ilk, the project would fail -- because someone has to get the work done.

\item Rank Rodger
\label{sec:orgheadline21}

Rodger is very good at what he does. He's a techie through and through,
and delights in problem solving. The problem is that Rodger lives in his
head. At times he feels like a brain on legs, so focused is he upon
intellectual pursuits. His body is a much neglected container for
cortical function that he generally pays little attention to, except to
meet its basic functional requirements for food and clothing. As a
result, there is a certain funk surrounding Rodger which nearby
colleagues are all too aware of, but of which Rodger is olfactorily
ignorant. Halitosis is his constant companion and dandruff a regular
visitor. In general, he has unkempt appearance -- his shirt often
buttoned incorrectly, hair not combed and tie (which he wears only under
the greatest duress) knotted irregularly. Rodger doesn't really care
what others think of him and is largely unaware of the message his poor
grooming and hygiene is sending to others. Rodger is likely to remain
unaware for a long time, as nobody can think of a way of broaching the
topic with him that wouldn't cause offense.

\item Skill Set Sam
\label{sec:orgheadline22}

Sam is just passing through. If he is a contractor, everyone will
already be aware of this. If he is permanent staff, his colleagues might
be a little surprised to know just how certain he is that he won't be
working here in a year's time. Sam is committed to accumulating as much
experience with as many technologies as he possibly can, in order to
make himself more attractive to future employers. His career objective
is simply that he remain continually employed, earning progressively
higher salaries until he is ready to retire.

\item Toolsmith Trevor
\label{sec:orgheadline23}

Trevor loves to build development tools. He can whip you up a build
script in a few minutes and automate just about any development task you
might mention. In fact, Trevor can't be stopped from doing these things.
He is actively looking for things to automate -- whether they need it or
not. For some reason, Trevor doesn't see the writing of development
tools as a means to an end, but an end in itself. The living embodiment
of the "Do It Yourself" ethic, Trev insists on writing common
development tools himself, even if an off-the-shelf solution is readily
available. Rather than chose one of the million commercially available
bug tracking applications, you can rely on Trevor to come up with an
argument as to why none of them are adequate for your purposes, and
there is no solution but for him to write one. At the very least, he
will have to take an open source tool and customize it extensively. So
too with version management, document templates and editor configuration
files. Trevor is right into metawork, with the emphasis on the meta.

\item Unintelligible Uri
\label{sec:orgheadline24}

English is not Uri's native tongue. This is blatantly obvious to anyone
who attempts to communicate with him. He speaks with a thick accent and
at such a rapid pace that listeners can go several minutes in
conversation with him without having a clear idea of what he has said.
Trying to work with Uri can be an excruciating experience. He cannot
contribute to technical discussions effectively, regardless of how well
informed he might be, because he is always shouted down by those with
more rhetorical flair, regardless how uninformed they might be.
Delegating work to him is a dangerous undertaking because you can never
be certain that he has really understood the description of his
assignment; he tends to respond with affirmative clichés that can be
easily said, but don't necessarily reflect that information has been
successfully communicated. Very often, people choose simply not to
bother communicating with Uri, because they find it both exhausting and
frustrating. Whoever hired Uri has failed to appreciate that fluency in
a natural language is worth ten times as much as fluency in a
programming language.

\item Vb Victor
\label{sec:orgheadline25}

Sometime in the nineties Victor underwent what is colloquially referred
to as a "Visual Basic Lobotomy". He found himself a programmer on a
misconceived and overly ambitious VB project, and fought to write a
serious enterprise application for some years in a language that was
never conceived for more than small scale usage. Visual Basic Land is a
warm and soothing place, and Victor let his skill set atrophy while he
slaved away at VB, until eventually VB was all he was good for. Now,
dispirited and deskilled, he is a testament to the hazards of building
your career upon a narrow technological basis. Victor will likely
survive a few more years, pottering from one VB project to the next,
until he loses the enthusiasm even for that.

\item Word Salad Warren
\label{sec:orgheadline26}

Unlike Uri, Warren's native tongue is English; but it does him little
good. Listening to Warren explain something technical is like listening
to Dr Seuss -- all the words make sense when taken individually, but
assembled together they seem to be mostly gibberish with no coherent
message. Such is Warren's talent for obfuscation, he can take simple
concepts and make them sound complex; take complex topics and make them
sound entirely incomprehensible. This is big problem for everyone
attempting to collaborate with Warren, for they generally find it
impossible to understand the approach Warren is taking in solving his
part of the problem, which virtually guarantees it won't work properly
in conjunction with other's work. On those rare occasions when he tries
to document his code, the comments aren't useful, as they make no more
sense than Warren would if he were explaining the code verbally.
Management has made the mistake of assuming that Warren's diatribes are
inscrutable because he is so technically advanced and is describing
something that is inherently complex. That's why he is in a senior
technical position. But his pathetic communication skills are a major
impediment to the duties he must perform as a senior developer, which
routinely involve directing and coordinating the technical work of
others by giving instructions and feedback. Warren is a source of great
frustration to his colleagues, who would give anything for precise and
concise communication.

\item X-Files Xavier
\label{sec:orgheadline27}

Xavier takes a little getting used to. Although his programming skills
are decidedly mature, his personality seems to be lagging behind. He has
an unhealthy fascination with Star Trek, Dr Who and Babylon 5. Graphic
novels and Dungeons and Dragons rule books are littered about his
cubicle, and he can often be found reading them during his lunch break,
which he always spends in front of his computer, surfing various science
fiction fan sites and overseas toy stores. Project meetings involving
Xavier are generally \ldots{} interesting, but somewhat tiring. He regularly
interjects quotations from Star Wars movies and episodes of Red Dwarf,
laughing in an irritating way at his own humor, oblivious to the fact
that others without his rich fantasy life are not amused by his obscure
pop culture references. Xavier seems to spend most of his time by
himself. No one has ever heard him mention a girl-friend. Those who have
worked with him for any length of time know that he is best kept away
from customers and other "normal people" who would not understand his
eccentricities.

\item Young Yasmin
\label{sec:orgheadline28}

Yasmin has only been out of University for a few years. She is
constantly surprised by the discrepancy between what she was taught in
lectures and what actually appears to happen in industry. In fact, there
seems to be a good deal that happens in practice that was not
anticipated at all by her tertiary education. She concludes that the
numerous shortcuts, reactive management and run-away bug count of her
projects are just localized eccentricities, rather than a widespread
phenomenon. Yasmin fits well into the startup company environment, with
its prevailing attitude of "total dedication." Indeed, she is the target
employee demographic of such firms. She is at that stage of life where
she has the stamina to work 60 and 70 hour weeks on a regular basis. She
is not distracted by family commitments, and is ambitious and eager
enough to still be willing to do what is necessary to impress others.
Lacking industry experience and the perspective that comes with
maturity, she is not assertive enough to stand up to management when
they make excessive demands of her.

\item Zealous Zack
\label{sec:orgheadline29}

Zack is a very enthusiastic guy. In fact, there seems to be very little
going on in the world of computing that Zack is not enthusiastic about.
Like a kid staring in the candy store window, Zack gazes longingly at
every new buzzword, acronym and advertising campaign that crosses his
path, immediately becoming a disciple of every new movement and
technology craze that comes along. Sometimes these enthusiasms bring
with them certain ideological conflicts, but Zack is too busy
downloading the Beta version of the next big thing to be worried about
such matters. He runs Linux on his home PC, has a Mac Mini in his living
room, and worships at the church of Agile. Having Zack on your project
can be challenging, particularly if he exercises any control over
technology selection. He will invariably try and load down your project
with whatever "cool" technologies he is presently over-enthused about,
and delight in the interoperability problems that result as an
opportunity to introduce even more technologies to save the day. Zack
never quite learnt to distinguish work from play.
\end{enumerate}

\subsubsection{The Hazards of Being Quality Guy  \footnote{First published 3 Sep 2003 at
\url{http://www.hacknot.info/hacknot/action/showEntry?eid=20}}}
\label{sec:orgheadline31}

Perhaps you've seen the Dilbert comic about Process Girl. At a meeting,
the Pointy Haired Boss introduces Process Girl as "the one who has the
answer to everything", at which point Process Girl chimes in parrot-like
with "Process!" She then denounces the meeting as inefficient because
the participants have no process to describe how to conduct a meeting.
By a unanimous vote she is expelled from the meeting. As he escorts her
out of the room, Dilbert offers by way of consolation "at least you
lasted longer than Quality Guy."

And now I must reveal a shocking truth \ldots{} ladies and gentlemen (rips
open shirt to reveal spandex body suit with "Q" emblazoned on the front)
\ldots{} I am Quality Guy. I am that much maligned coworker that you love to
hate. I am your local ISO champion, the leader of the Software
Engineering Process Group and the mongrel who overflows your inbox with
links to articles about process improvement. I'm the trouble maker that
asks embarrassing questions in meetings like "why aren't we doing code
reviews?" and "where's the design documentation?" I am the one that
dilutes your passionate discussions on J2EE and SOAP with hideously
unfashionable prattle about CMM and the SEI.

And like my namesake in the Dilbert comics, I am ostracized by my peers
and colleagues. I am renounced as being a "quality bigot" and dismissed
as impractical and too focused upon meta-issues to actually achieve
anything worthwhile. I am perceived as an impediment to real work and
characterized as a self-righteous, holier-than-thou elitist. My
suggestions of ways to improve my team's work habits are interpreted as
personally directed criticisms and thereby evidence that I am "not a
team player".

From my point of view at the periphery of the team, the earnest activity
of you and your geek friends seems somewhat farcical. You seem to be
perpetually distracted by the shiny new technology toys that the vendors
are constantly grunting out. You are hopelessly addicted to novelty and
consumed by the frenetic pursuit of the latest bandwagon. You seem to be
entirely unconcerned that "beta" is synonymous with "buggy" and "new"
with "unproven". The projects of my successive employers march by me
like a series of straight-to-video movies, each baring the same
formulaic plot wherein only the names of the participating technologies
have been changed to protect the innocent. I feel compelled to yell out
"stop!", "think!" and "why?", but it is hard to be heard when you're in
geostationary orbit around Planet Cool and in space, no one can hear you
scream.

Friends, this is what it is to be Quality Guy, and it ain't no party.

If you think you or a loved one might be in danger of becoming a Quality
Guy sidekick, let me offer you this one piece of advice -- never reveal
your true identity to your coworkers. It is a sure recipe for alienation
and isolation. Keep your shirt closed to the top button, so that your
superhero garb will go unnoticed. Eschew all quality-related terminology
from your public vocabulary and substitute terms from the jargon file
 \footnote{\url{http://www.jargonfile.com/}}. Hide any books you might have that do not relate directly to a
technology.

When it comes to development practice, with a little ingenuity you can
institute a number of quality-related practices within the sandbox of
your own development machine, without needing to reveal to others that
your sphere of concern extends beyond the acronym's:

\begin{itemize}
\item If you find yourself in an environment without version control,
install a free version control system such as CVS or CS-RCS on your
own machine. You can at least maintain control over those files that
you are immediately involved with.
\item If there is no prevailing coding standard, employ one for your own
code without revealing to others that there is any guiding hand of
consistency in your code (that would be uncool).
\item If there is no unit testing, write your own in a parallel source tree
visible only to yourself using the free \texttt{xUnit} package appropriate
to your platform.
\item If there is no design documentation, reverse engineer the existing
code into some hand-drawn UML diagrams and then stash them away where
others won't find them, keeping them just for your own reference.
\item No requirements? Start your own mini-requirements document as a local
text file, and question the developers and senior team members around
you to try and flesh it out. You can at least try and restrict
uncertainty with regard to your own development objectives.
\end{itemize}

Remember, the secret to surviving as a Quality Guy is to keep your true
identity a closely guarded secret. That way you can still be one of the
gang and remain non-threatening whilst still being able to take some
satisfaction from the limited degree of quality enforcement you can
achieve through isolated effort.

\subsubsection{A Dozen Ways to Sustain Irrational Technology Decisions  \footnote{First published 5 Oct 2005 at
\url{http://www.hacknot.info/hacknot/action/showEntry?eid=79}}}
\label{sec:orgheadline51}

External observers often think of programmers as being somewhat cold and
emotionless. Because our day-to-day activities are largely analytical in
nature, it has become a part of the developer stereotype that we are
dispassionate and rational in our manner and decision making. Those who
have watched programmers up close for any length of time will know that
this is far from the case. I believe that emotion plays a far larger
part in IT decision making than many would be willing to admit.
Frequently developers try and disguise the emotive nature of their
thinking by retrospectively rationalizing their decisions, but not being
well-skilled in interpersonal communication, are often unconvincing. If
you've ever witnessed or taken in part in a technological "holy war",
then you'll already have witnessed the unhealthy way that stances held
by emotional conviction can be misrepresented as being the result of
rational analysis.

\begin{enumerate}
\item The Causes
\label{sec:orgheadline36}

\begin{enumerate}
\item Novelty
\label{sec:orgheadline32}

The majority of irrational technical selections I've seen have their
origin in a senior techie's fascination with a new technology. For an
uncommon number of developers, the lure of an untried API or the novelty
of a new development model is simply irresistible. Such folks seem to be
focused on the journey rather than the destination -- which is
philosophically delightful but practically frustrating. The urge to play
with a new toy seems to overwhelm the ability to rationally evaluate a
technology on its merits, as if it's "newness" excused any faults and
weaknesses it might have. There seems to be a strong "grass is greener"
effect at work here. The weaknesses of existing technologies are known
because they have been teased out by the development community's
experience with it. But a new technology has an unblemished record. The
absence of community experience means that no one has encountered its
inevitable flaws, or pushed the boundaries of its capabilities.
Psychologically, it is easy to be drawn to the new technology based on
the implied promise of perfection, as compared to the manifest
imperfections of current technologies.

\item Ego
\label{sec:orgheadline33}

Programmers are not a group lacking in self-confidence; at least when it
comes to technical matters. In fact, the intellectual arrogance of some
can be quite stunning. For those with decision-making authority, the
burden of ego can be a substantial liability. A technology selection
based solely upon technical merit is easily defended by dispassionate
reference to facts, but once the outcome is identified with the
individual who made it, ego comes into play. Any challenge to the
decision tends to be interpreted as a challenge to the authority of the
decision maker. Any criticism of the selected technology tends to be
emotionally defended, because the party who selected it feels that fault
is being found with them personally. They are likely also sensitive to
the potential for injury to their image and reputation that might come
from being responsible for a poor technology decision. It is difficult
to retain status as the alpha geek when you are known to have made poor
technical decisions. Managers, in particular, are acutely aware of the
way their behavior and ability is perceived by others. Having been drawn
in by the false promises of glossy product brochures, the misinformed
technical manager is poorly positioned to subsequently defend technology
decisions. Such managers are frequently those to be found most
passionately and aggressively defending their decisions.

\item Fashion
\label{sec:orgheadline34}

An alarming number of developers seem to be slaves to technical fashion.
Plagued by a "gotta get me some of that" mentality, the arrival of
almost any new product or development tool is accompanied by an almost
salivatory response. They rush to evaluate the new offering and to share
their experiences with like-minded others who also like to be at the
leading edge. These programmers fit well and truly into the "early
adopter" category, or as I like to call them "crash test dummies." Like
their mannequin counterparts, they are forever running head long into
collisions -- in this case, with technologies. By observing the results,
the rest of us can learn from their often hard-won experiences, without
having to suffer the frequent injuries that tend to result.

\item Ideology
\label{sec:orgheadline35}

As frequent as it is unrecognized, ideological conviction seems to be a
major driver behind many technology decisions. Many developers remain
convinced that open source software will save the world, enable black
and white peoples to live in racial harmony, cure cancer and eliminate
hunger and poverty. They may be right, but none of these are rational
reasons to select a particular offering over a proprietary alternative
for a particular commercial application. But for many, it is automatic
and unquestioned that open source software is the way to go, as a matter
of moral imperative, regardless of the merits or otherwise of that
software.
\end{enumerate}

\item The Techniques
\label{sec:orgheadline49}

Once the commitment to a particular technology has been publicly made,
its proponents must then be prepared to defend their decision in the
light of any negative development experience. If the technology was
selected for irrational reasons, then those identified with its
selection must now become apologists for the technology, seeking to
minimize and quash any information that might reflect poorly on the
technology and transitively, upon themselves. Here are twelve techniques
I have seen used to sustain a bad technology decision in the face of
experience that puts that technology's selection in doubt.

\begin{enumerate}
\item 1. Deny That Negative Experiences Exist
\label{sec:orgheadline37}

This is a common technique amongst the "kick ass" school of management.
When faced with evidence that casts your technology selection in an
unfavorable light, simple deny that the evidence exists. Even if someone
can demonstrate to you first hand the problems that have been
encountered, you can employ a "shoot the messenger" approach to distract
attention away from the evidence being presented, and put the messenger
on the defensive. You will need to be in a position of sufficient
authority, and surrounded by suitably spineless colleagues, to make
"black is white" declarations hold fast and create a localized reality
distortion zone. It may sound fantastic, but in practice it is quite
common for authority to usurp reality.

It is not a technique unique to the IT profession. In his memoirs
"Inside the Third Reich", Albert Speer relates a situation in which
Hermann Göering employed exactly this technique. When Göering was
advised that American fighters had began to encroach upon German skies,
he refused to accept the report, despite being presented with
irrefutable evidence by one of his generals. He simply issued an
official order stating that nobody had seen any fighters.

\item 2. Claim "We'll Fix It Ourselves"
\label{sec:orgheadline38}

When an open source product is selected but ultimately found wanting,
the "we can fix it ourselves" apology is often the first one that is
trotted out. The availability of the source code means that you can
ostensibly patch the product yourself, submit that patch to the open
source project, and then carry on. Whenever a colleague finds a bug in
the technology, just dismiss their complaints with the directive to
"just fix it yourself", and the problem will go away \ldots{} for you,
anyway.

\item 3. Claim That Bugs Are Intellectual Property
\label{sec:orgheadline39}

This is a sneaky but effective one. Make it known to your colleagues
that they cannot report any problems they find with the new technology
to the vendor (or the community, in the case of open source software) as
that would equate to divulgence of information that has been gathered at
company expense. In the strictest sense, the knowledge of the bug's
existence is the company's intellectual property. Exactly what kind of
intellectual property it is, is open to question. It could be
"confidential", but it seems doubtful that it is of enough significance
to possess the necessary "quality of confidence". In any case, it
doesn't really matter. You can rely upon others being sufficiently
intimidated by the implied threat of prosecution for IP infringement to
remain silent.

\item 4. Claim "It Will Be Fixed In The Next Release"
\label{sec:orgheadline40}

This piece of misdirection can be used to postpone problems almost
indefinitely. It is particularly handy for products that are on a short
release cycle, as the promise of a fix is always just around the corner
(and with it, the potential for the introduction of new bugs -- but
ignore that). If the bug is not actually fixed in the next release, then
it's hardly your fault. Blame the vendor, blame the development
community, lament the state of software development in general \ldots{} do
anything to divert attention away from the original source of the
technology's selection.

\item 5. Make The Bug Reporting Process Unwieldy And Onerous
\label{sec:orgheadline41}

A worthwhile bug report takes a bit of effort to produce. Sample code,
screenshots and instructions to reproduce the buggy behavior are all
part of a conscientiously compiled bug report. But if that is all that
is required, there will be some developers willing to take the time to
write them. You can make the lodging of a bug report more daunting by
requiring developers to lodge an entire specification of the desired
(non-buggy) behavior, including requirements, a mock-up or prototype,
design specification and test specification. This can take days. They'll
quickly learn that it's simply not worth the effort to report bugs via
such a lengthy process, and to move directly from discovery of a bug to
the search for workarounds or alternative approaches.

\item 6. Claim "It Works For Me"
\label{sec:orgheadline42}

An indirect form of denial exists in claiming that you have been unable
to reproduce the bug yourself, so the complainant must be doing
something wrong. Due to the almost unlimited potential for interactions
between software components, libraries and operating system functions,
it is easy to simply point somewhere in the direction of this
programmatic thicket and declare "the problem's probably in there."

\item 7. Appeal To Non-Quantifiable Benefits Yet To Be Realized
\label{sec:orgheadline43}

If enough difficulties are encountered with your chosen technology, it's
only a matter of time until someone starts suggesting alternatives. When
your opponents open fire with the feature list of their favorite
competing technology or product, you need a reply. It is best to appeal
to non-quantifiable and non-functional benefits as it is impossible to
prove that they have not been realized. "Flexibility" and
"maintainability" are a few non-functional favorites that you can claim
are being realized by your technology selection, regardless of what the
reality may be.

\item 8. Employ The Power Of Standards
\label{sec:orgheadline44}

A technology that has been embodied in a standard already has a
significant head start on non-standardized competitors. If the standard
is one that has been accepted by major vendors as a basis for their own
product offerings, then all the better. The psychological principal
being appealed to here is that of "social proof" - the belief that
popularity is indicative of worth. Indeed, widespread acceptance of a
standard (or a technology implementing a standard) is unlikely to occur
if the notion is completely without value, but there is no guarantee of
you achieving the same success in your own context as others have
achieved in theirs. However, many will ignore the need to consider
application-specific issues in deciding the merit of a technology. If
IBM, Microsoft or some other big name says it's good, then it must be
good - for everyone, all the time, regardless of what the constraints of
their particular problem may be. To appreciate how seductive this faulty
reasoning can be, consider how many times you've seen a J2EE application
that was written simply for the sake of using J2EE, even though there
was no real need for a solution with a distributed architecture.

\item 9. Maximize Investment
\label{sec:orgheadline45}

One of the best ways to get a technology on a solid foothold in your
organization is to maximize your investment in it as quickly as
possible. This can be achieved by forward-scheduling tasks that use the
technology the most, so that the number of hours invested in using it
accrue quickly. You might justify this by presenting the host project to
management as a "pilot" of some sort, where the technology is being
evaluated on its merits. But so long as you can silence any negative
findings that might emerge from that ersatz "evaluation", you are also
strengthening the project's commitment to the continued use of that
technology. What project wants to incur the schedule burden of having to
swap technologies and re-implement those parts of the project based upon
the now defunct technology? If you can just suppress criticism for long
enough, the project will soon reach a point of no return, after which it
becomes infeasible to make technology changes without incurring an
unacceptable schedule penalty.

The bigger a company's financial investment in a technology, the more
reticent it will be to discard it. So you will find it easier to keep
expensive technologies in use. You can increase expenditure by
purchasing entire product suites, or choosing products so complex that
you can justify hiring highly paid consultants to tailor them to your
project environment or teach your staff how to use them. Once all that
time and money has been invested, it will become extremely difficult for
anyone to abandon the technology due the financial inertia it has
acquired.

\item 10. Exclude The Technically Informed From The Decision Making
\label{sec:orgheadline46}

As a self-appointed evangelist for your chosen technology, your worst
enemy is the voice of reason. The technology's inability to fulfill the
promises its vendor makes should be no obstacle to its adoption in your
organization -- and indeed, it won't be, so long as you can keep those
who make the decisions away from those who know about the technology's
failings. Let their be no delusion amongst your staff and colleagues
that it is management's purview to make these decisions, and the
techie's job to implement their decision. Some will try and argue that
those who know the technology most intimately (technical staff) are in
the best position to judge its value. Assure them that this is not so
and that only those with an organizational perspective (management) are
in a position to assess the technology's "fit" with the corporate
strategy. Allude to unspoken factors that influence the decision to use
this technology, but are too sensitive for you to discuss openly
(conveniently making that decision unassailable).

\item 11. Sell The Positives To Upper Management, Hide The Negatives
\label{sec:orgheadline47}

\emph{Question: How does a fish rot?}

\emph{Answer: From the head down.}

If you can get those in senior management to develop some identification
with the technology then you will have made some powerful allies.
Assuming they are technically uninformed, make your management a sales
pitch for the technology in which you emphasize all the positives and
completely neglect the negatives. Give them glossy brochures advocating
the technology, and appeal to their competitiveness by providing
testimonials from big-name managers, as if to suggest "this technology
is what the best managers are getting behind"; the implication being
that your own management are not amongst "the best" unless they follow
suit. The ego-driven push from above is almost impossible to counter
with a factual push from below. Authority trumps reason in many
organizations.

\item 12. Put A Head On A Pike
\label{sec:orgheadline48}

It is part of the barbarian tradition to place a head on a pike at the
entrance to your domain, to warn those approaching of the fate that
awaits them if they don't follow the rules. It's crude, but undeniably
effective. Actual decapitation is frowned upon in most office
environments, but you can still put a figurative "head on a pike" to
make it clear to others that dispute over your chosen technology will
not be tolerated. If you have the authority, firing someone who
expresses a dissenting opinion should be adequate to ensure the
remaining staff fall into line. Otherwise, some form of public
humiliation -- a verbal dressing down in a common area of the office,
for instance -- will have to do. In either case, it is important that
you adopt some pretense for your actions that is not directly related to
the issue of technology selection. Unfair dismissal laws being what they
are, you need to be a bit careful here. Witnesses will know, however,
from the greater context that the real reason for this retribution is
the target's opposition to the technology decision you made, and will
make a note to themselves not to express their own concerns about the
technology, lest they also be made an example of.
\end{enumerate}

\item Conclusion
\label{sec:orgheadline50}

IT managers, developers and other technical staff are no less
susceptible to self-deception and political ambition, simply because
they work in a field in which analytical thought is traditionally
valued. When it comes to the selection of a technology from a field of
competitors, the complexity and number of factors to consider often
leads to a tendency to abandon detailed, rational analysis and make
decisions on an arbitrary, emotive basis. If the technology selected
fails to live up to its promise, those who selected it then face the
difficult task of rationalizing its continued usage, lest their decision
be overturned and they lose face as a result. By employing one or more
of the techniques identified above, a skilful manager or senior
technician can avoid this embarrassment and force the continued usage of
an unsuitable technology, while they work by other means to distance
themselves from the original decision.
\end{enumerate}

\subsubsection{My Kingdom for a Door  \footnote{First published 11 Sep 2005 at
\url{http://www.hacknot.info/hacknot/action/showEntry?eid=78}}}
\label{sec:orgheadline57}

\begin{quote}
\emph{“All men's miseries derive from not being able to sit in a quiet room
alone.” -- Blaise Pascal}
\end{quote}

In some interviews there comes a point where you realize that you don't
want the job. It might be the moment you discover that the employer has
conveniently omitted from the published job description the requirement
for the incumbent to spend 50\% of their time maintaining a one million
line legacy application, written in Visual Basic. It may be shortly
after you state your salary expectation, only to be greeted with a look
of blank astonishment. For me, it is often the point at which the
interviewer reaches into their bag of interview clichés and asks a
question so trite that it betrays the total absence of advance
preparation and original thought. Once the role has been safely
relegated to the "no thanks" pile, it is difficult to resist adopting a
certain playfulness while waiting out the duration of the interview, as
courtesy demands.

For example, when asked "Where do you see yourself in five years time?"
I like to borrow a witticism from comedian Steven Wright, and respond "I
don't know -- I don't have any special powers like that." If asked "Why
are manhole covers round?" I might reply "Because God made them that
way", simply to see if they will dare broach a topic traditionally
considered taboo in interviews. And if they should enquire "What are
your career goals?" I will almost certainly reply "I have only one -- I
want a door."

But in this last I'm only partially being facetious, for one of the most
consistently difficult aspects of every software development effort I've
been a part of has been the physical environment in which it is
conducted. Having abandoned the lofty career goals of my youth (such as
producing quality software) I have deliberately set my sights a little
lower. These days, my sole ambition is to have an office with a door. My
professional nirvana would then be to close that door, so I can get on
with my work undisturbed.

As challenging as technical issues can be, they are at least considered
approachable by most organizations. But environmental problems,
particularly noise levels, seem to universally receive short shrift, and
are often dismissed as an unfortunate but unavoidable part of office
life and beyond anyone's ability to deal with.

Of course, the problem of office noise is far from intractable. Numerous
approaches can be taken to relieve or at least ameliorate the problem,
the most obvious of which involves the reintroduction of an antiquated
and long neglected piece of spatial division technology -- the door. The
real reasons that environmental issues go unattended are somewhat
different.

\begin{enumerate}
\item Brain Time Versus Body Time
\label{sec:orgheadline52}

Software developers are knowledge workers. Our job is to produce
intellectual property. You would think it self-evident that work of this
nature requires sustained concentration, and that it is easier to
concentrate when things are quiet.

Back in my school days, these facts seemed to be widely known and
accepted. When you went to the library, the school librarian (who, in my
school, was a particularly ferocious woman the students referred to as
"Conan The Librarian") would do her best to see that the library was
quiet. Why? \emph{Because people were trying to study, to think, to
concentrate}. When there was an exam to be done, the exam would be
conducted in complete silence. Why? Because it's easier to concentrate
on your exam when it's quiet. When the teacher gave the class time to
work on an assignment, the class was expected to be silent. Why? Because
it's easier to think about your assignment when it's quiet.

In university too, there was little dispute about the necessity for a
quiet environment when doing intellectual work. The libraries and exam
halls are silent, the lecture theaters and tutorial rooms are quiet so
that the speaker may be heard and their message understood.

Prior to entering the workforce, I thought nothing of it. It all seemed
to be just common sense. Imagine my surprise then to discover that the
corporate world had decided that none of it was true. That, in fact, you
don't need quiet in order to concentrate effectively -- you can work
just as well when immersed in an environment that is a noisy as your
local shopping center. Or so I infer is the reasoning, because the
standards in both office accommodation and behavior seem to have been
determined with such an assumption in mind.

Sitting at my desk at work, I am surrounded by distraction and
diversion, which everyone just seems to accept will not impair my
ability to work at all. But my own impression is very much to the
contrary. I find myself constantly frustrated and annoyed by the
ceaseless chatter around me and the incessant whir of printers and
photocopiers. I have never known a workplace to be any different.

How is it that the corporate and academic worlds seem to have completely
different ideas about what characterizes an environment conducive to
intellectual activity? Why is it that the academic community seems to
have got it right, and the corporate community ubiquitously has it
wrong? Surely employers are not knowingly paying their staff to be only
semi-productive, are they? Unless the corporate world is consistently
behaving in a self-defeating and irrational way, I must simply be
mistaken about the effect this office noise is having on me.

Perhaps I am actually quite unaffected by the conversations that my
cubicle neighbors are having, on matters unrelated to my work \ldots{} all
day. Perhaps the four foot high partition which separates me from them
is actually enough to reduce their inane chatter and laughter to a
distant whisper -- I guess the sound dampening cloth on it must have
some effect. Although the partition only covers two of the four sides of
my "cubicle", perhaps adopting a "glass half full" attitude would make
the lack of privacy less disturbing. Perhaps the sound of the printers
and copiers in the facilities area, just three feet away from my desk,
really isn't that loud. Perhaps the guy in the next cubicle who insists
on checking his voice mail through the speakerphone isn't the sociopath
he appears to be, and I'm just not sufficiently tolerant of others.
Perhaps it's not really all that visually distracting to have people
walking through the corridor beside my cubicle every few minutes. Maybe
some blinkers, like those given to cart horses, would lessen the effect
of constant movement at the periphery of my vision. And perhaps the ten
mobile phone calls that my surrounding cubies seem to get every day,
each one heralded by a distinctive and piercing ring-tone sampled from
some Top 10 dance hit, really isn't as wearing as what I think it is.
And maybe having a pair programming partner leaning over your shoulder,
barking in your ear and correcting your every typographic error isn't an
obnoxious novelty that removes what little remaining chance there is of
thoughtful consideration occurring in the modern workplace, but a
mechanism for solving complex problems by having a chat over a nice cup
of tea.

Or perhaps, just perhaps, the cubicle farm is a fundamentally unsuitable
work environment for software developers. But how could that be, when
the "open plan" office is the corporate norm? Could organizations really
be so blind as to routinely give their staff an environment which is not
conducive to the conduct of their work?

How could such a patently irrational trend develop and persist?

\item It's About Money
\label{sec:orgheadline53}

The modern cubicle had its genesis in 1968, when University of Colorado
fine-arts professor Robert Propst came up with the "Action Office" --
later commercialized by Herman Miller \footnote{\emph{Death to the Cubicle!}, Linda Tischler, FastCompany, June 2005}. At the time, offices usually
contained rows of desks, without any separation between them. At least
cubicles were an improvement. But once the facilities management people
cottoned onto the idea of putting people in boxes, their focus became
achieving maximum packing density and consideration of noise and
interruption went out the window (if you could find one). That mentality
persists today, largely because the costs associated with office
accommodation and office space rental are concrete expenditures that
appear on a balance sheet somewhere. Somebody is accountable for those
costs, and therefore seeks to minimize them. But the costs of lost
productivity due to an unsuitable work environment aren't readily
quantified, they just disappear "into the air", and so are easily
forgotten or disregarded. There are also tax breaks in some localities,
where legislation exists making it quicker to write off the depreciation
of cubicles more quickly than traditional offices. \footnote{\emph{The Man Behind the Cubicle}, Yvonne Abraham, Metropolis, November
1998}

\item It's About Rationalization
\label{sec:orgheadline54}

The ostensible benefits of an open-plan office are its moderate cost,
flexibility, facilitation of teamwork and efficient use of space. These
are the attributes by which cubicle systems are marketed \footnote{\emph{“Resolve” product literature}, Herman Miller}. Note that
the ability to create an environment suitable for knowledge workers is
not amongst those features.

Flexibility, although a possibility, is seldom realized in IT-centric
environments where the need to re-route power and network cabling makes
people reticent to re-arrange cubicles to any significant extent. Even
individual variation and customization is discouraged in many
workplaces, where such non-conformity is viewed as a threat to the
establishment.

It is also commonly held that cubicles "promote communication" amongst
staff. Unfortunately, one man's "communication" is another man's
"distraction", the difference being whether the desire to participate is
mutual. Alistair Cockburn, never one stuck for a metaphor, describes the
wafting of conversation from one cube to the next as "convection
currents of information" \footnote{\emph{Agile Software Development}, Alistair Cockburn, Addison Wesley,
2002} and promotes the benefits that might arise
from incidental communication. But when one is trying to concentrate,
these currents of information become rip-tides of noise pollution that
one cannot escape. The result is frustration and aggravation for the
party on the listening end.

Unsurprisingly, companies that produce modular office furniture claim
that cubicles are fabulous, and choose to selectively ignore their
manifest disadvantages. In the advertising literature 7 for their
"Resolve" furniture system, Herman Miller lauds the necessity of
teamwork:

\begin{quote}
/All the accepted research in this field says you have to have more
visual and acoustic openness to get the benefits of a team-based
organization./
\end{quote}

\ldots{} and downplays the need for individual work:

\begin{quote}
/Although there will always be types of work that require intense
concentration and protection from distraction, our research suggests
that these needs can be effectively met outside assigned, enclosed
workstations -- through remote work locations or on-site, shared,
"quiet rooms" for instance./
\end{quote}

In other words, the workplace should be optimized for collaborative
work, and those who want to concentrate can go elsewhere. Indeed, it
seems to be a growing misconception amongst designers and managers that
a high level of interaction and collaboration is a universal good, the
more the better, and that the downsides don't matter.

For knowledge workers, who spend the vast majority of their time in
isolated contemplation, this is decidedly bad news. Those who fit out
offices seem to be either gullible enough to believe glib rhetoric such
as the above, or more likely, choose to remain willfully ignorant of the
fundamental requirements of their staff. Herman Miller would have you
believe that the cubicle environment is good for your software
development effort as well:

\begin{quote}
\emph{But the benefits of physical openness are gaining recognition even
among the "gold-collar" engineers and programmers of Silicon Valley.}

/"The programming code we write has to work together seamlessly, so we
should work together seamlessly as well", says a Netscape
Communication programmer and open-plan advocate quoted recently in the
New York Times./
\end{quote}

Clearly, it is inane to suggest that software can be invested with
desirable runtime behavior by adopting parallel behavior in the team
that develops it. Does the code execute more quickly if we write it more
quickly? Will it be more user friendly if the developers are more
friendly toward each other? No -- it is just nonsensical wordplay. But
the use of such faulty "proof by metaphor" techniques is illustrative of
how desperate the furniture industry is to ignore the workplace
realities they are producing, and the superficial level of thought that
they employ in promoting their ostensible success.

Consider the following statement, again from Herman Miller:

\begin{quote}
/Recent studies also indicate that people become habituated to
background office noise after prolonged exposure. Over time, people
get used to the sounds of a given environment, and noises that
initially have a negative impact on performance eventually lose their
disruptive effect./
\end{quote}

Or perhaps, workers simply give up on the issue of office noise after
their prolonged attempts to deal with it are continually met with
stonewalling and denial. No references are given, so it is impossible to
gauge the validity or relevance of these studies. But it sounds so
inconsistent with known research in this area that one cannot help but
be suspicious.

Many studies have examined the effect of background speech on human
performance. \footnote{\emph{Human Performance Lecture}, Dr Nick Neave, Northumbria University} One phenomena that consistently recurs is the
"Irrelevant Speech Effect" (ISE). In ISE experiments, participants are
given tasks to do while being subject to speech that is unrelated to the
task at hand. Susceptibility to ISE varies between individuals, but in
general ISE is found to be "detrimental to reading comprehension,
short-term memory, proofreading and mathematical computations." \footnote{\emph{Collaborative Knowledge Work Environments}, J. Heerwagen,
K.Kampschroer, K. Powell and V. Loftness} In
general, work that requires focus and ongoing access to short-term
memory will suffer in the presence of ISE and other distractions and
interruptions.

\item It's About Status
\label{sec:orgheadline55}

Real estate has always been an indicator of status. Whether you're a
feudal lord or a middle manager, the area in your command is usually
proportional to your perceived status and importance. Those who suggest
that the cubicle is an unavoidable part of the office landscape are
often those whose status precludes them from ever having to occupy one,
and who have a vested interest in the distribution of office space
remaining exactly as it is -- in their favor. The unstated purpose of
the cubicle is to serve as a container for the "have-nots", to more
obviously distinguish them from the "haves." The preoccupation with
offices (and the number of windows therein) and car parking spaces is
often quite baffling to techies, who think first in terms of utility
rather than perception. But for those more "image oriented," the true
worth of corporate real estate has nothing to do with functionality and
everything to do with positioning.

\item Float Your Mind Upstream
\label{sec:orgheadline56}

I would like to be able to say that companies are gradually realizing
that knowledge workers such as software developers need support for both
team interaction and distraction-free individual work, and are making
changes to workplace accommodation accordingly. But I would be lying.

In truth, the workplace's suitability as a place to work is likely to
sink below even its currently deplorable standard. The trend is towards
ever smaller cubicles with fewer and lower dividing partitions. A 1990
study by Reder and Schwab found that the average duration of
uninterrupted work for developers in a particular software development
firm was 10 minutes. That's revealing, because it generally takes about
15 minutes to descend into that deep state of contemplative involvement
in work called "flow". During the period in which one is transitioning
to a state of flow, one is particularly sensitive to noise and
interruption \footnote{\emph{Peopleware}, T. DeMarco and T. Lister, Dorset House, 1987}. If you're interrupted every 10 minutes or so, chances
are you spend your day struggling to focus on what you're doing, being
constantly prevented from thoughtful contemplation of the problem before
you by visual and auditory distractions around you \ldots{} and that's the
typical working day of many software developers. As DeMarco and Lister
comment "In most of the office space we encounter today, there is enough
noise and interruption to make any serious thinking virtually
impossible." With the addition of some doors into the environment,
developers could at least control their noise exposure.

Look around you now, and what do you see? Chances are there will be at
least one and probably many of your colleagues wearing headphones. It's
common practice for software developers to retreat into an isolated
sonic world as the only way they have of overcoming the incessant
distraction around them. Some companies pipe white noise into individual
cubicles to try and mask the surrounding noise. I've found it helpful to
run a few USB-powered fans from my computer -- their quiet hum serves
much the same purpose, as well as compensating for the often inadequate
air conditioning.

Why don't developers revolt? Why is it so rare to hear them vocalize
their complaints? Talk to them in private and they'll likely concede
that their work environment is too noisy to enable them to work
effectively. But they're unlikely to make those concerns public, for
fear of retribution or simply because they know that the noise level
will be dismissed as being an inherently intractable problem.

So we will continue to grind our teeth and shake our heads in disbelief
while listening to the dull roar of the combined efforts of the
printers, fax machines, photocopiers, telephones, speakerphones,
inconsiderate coworkers, slamming doors, hallway conversations
immediately beside our desks and wonder how we can be expected to work
effectively amidst such a furor. And as long as developers continue to
tolerate unsatisfactory noise levels, and work longer hours to
compensate for their negative effect on their productivity,
organizations will continue to ignore their dissatisfaction.
\end{enumerate}

\subsubsection{Interview with the Sociopath  \footnote{First published 24 Nov 2004 at
\url{http://www.hacknot.info/hacknot/action/showEntry?eid=70}}}
\label{sec:orgheadline72}

Recently I have had the misfortune to be playing the interview circuit
again; parading from one interrogation to the next like some prisoner of
technical war. The experience has been both frustrating and humiliating
-- and unpleasant reminder of how appallingly most technical interviews
are conducted.

So ignorant is the conduct of many interviewers, one could be forgiven
for thinking they have undertaken the interview process with the
deliberate intent of minimizing the chances of finding the right person
for the job, and maximizing the opportunity for their own ego
gratification. Such behavior is a common feature of the sociopathic
personality.

Based on my recent interview experiences, I've assembled below a list of
the techniques commonly practiced by the sociopathic interviewer.

\begin{enumerate}
\item Put No Effort Into The Position Description
\label{sec:orgheadline58}

The best way to ensure you don't accidentally get the right person for
the job is to have no idea who you're looking for and what role they
will be fulfilling in your organization. A meager and perfunctory PD
(position description) helps to convey that "don't care" attitude right
from the start of the hiring process. If you're working through a
recruiting service, simply tell the recruiter that you don't have time
to write out a decent PD. Rattle off a few buzzwords and acronyms and
leave them to patch something together themselves.

If you are somehow compelled to write a PD, fill it out with the usual
platitudes about "excellent communication skills", "ability to work well
in a team", "delivering high quality code" \ldots{} and other such nonsense
that 90\% of programmer PDs include and which nobody can effectively
appraise in an interview situation.

\item Conduct Phone Interviews With A Poor Quality Speakerphone
\label{sec:orgheadline59}

Phone interviews provide an excellent opportunity to explore the aural
aspects of discourtesy. Always use a low quality speakerphone; even if
you are the sole interviewer. Make the call from the largest,
echo-filled room that you have access to, and sit a long way from the
speakerphone. If there is more than one interviewer, make sure you
constantly interrupt and talk over each other, making it impossible for
the candidate to distinguish who they are currently talking to. The
frustration of the constant struggle to understand and be understood
will eventually wear down even the most ardent of candidates, often with
comic effect.

\item Be Poorly Organized
\label{sec:orgheadline60}

Some candidates have the audacity to view the organization of an
interview as being representative of the organizational capabilities of
your company as a whole. They reason that finding someone to fill a role
is effectively a mini-project in itself, and if you can't schedule and
coordinate even a minor project like that, how could you manage a larger
and more complex undertaking like a software project? These people are
clearly thinking too hard and too critically. They are exactly the ones
that you want to turn off. Therefore you should make every effort to
have the interviewing process reflect the abysmal state of project
management in your company as closely as possible.

Demonstrate your inability to estimate and track tasks by scheduling
candidates' interviews too close together, booting one candidate out the
door just as the next is about to give up hope that their own interview
will ever commence. Having started the interview late, make it clear
from the outset that you don't have much time to devote to each
individual so you will have to rush. This will demonstrate your tendency
to meet deadlines by making heroic efforts rather than rational
adjustments of scope.

Then reveal that you have no questions prepared for the candidate. Just
“um” and “ah” your way through a random series of queries that reveal no
overall structure or intent, thereby conveying your inability to
structure a work effort appropriately.

\item Focus On Technical Arcana
\label{sec:orgheadline61}

Technical interviews are a sociopath's utopia, for they provide you with
infinite opportunity to humiliate a candidate while engendering feelings
of supreme inadequacy. Even if a candidate has been using a particular
technology for many years, chances are that they have only dealt with
the most commonly used 80\% or so of that technology's features.
Therefore your questions relating to that technology should target the
seldom encountered 20\% at the periphery. Identify those aspects of the
technology so infrequently used that most developers have either never
been called upon to use them, or if they have, have not done so
sufficiently to internalize the finer points of its operation. Drill the
candidate mercilessly on these obscure and largely irrelevant details.
When they fail to provide the correct answers, assume a facial
expression that betrays your amazement that they have managed to survive
in the industry without having immediate recall on every aspect of the
technology they deal with.

\item Hire A List Of Products And Acronyms, Not A Person
\label{sec:orgheadline62}

The topic of "business value" should be avoided at all costs. Do not ask
about the candidates' contributions to the businesses they have worked
in, as this implies that all that boring business stuff is actually of
concern to you. The sort of person you want is one who is solely focused
upon decorating their CV with the latest buzzwords, and playing around
with whatever "cool" technologies that vendors have most recently
grunted out. You'll get such a person by ignoring the business aspect of
software development, and assessing candidates solely on the amount of
technical trivia they know. Clearly, those who take a "technology first"
approach are motivated more by self-interest than professional
responsibility, and are more likely to be suitable company for the
sociopathic interviewer.

\item Pose Unsolvable Problems
\label{sec:orgheadline63}

A favorite ploy of sociopathic interviewers everywhere is to ask
questions that have no concrete answer. The standard defense of this
technique is the claim that it verifies the candidates' ability to take
a logical approach to problem solving. Of course, there is no empirical
evidence correlating the ability to solve logic puzzles with the ability
to develop software - but no matter.

The real reason for asking questions that permit no solution is to watch
the candidate squirm "on the hook", and to experience that feeling of
smug self-satisfaction that you get when you finally acknowledge that
there is no solution to the problem -- it's just an exercise.

Such questions include:

\begin{itemize}
\item "How would you count the number of gas stations in the US?"
\item "How would you measure the number of liters of water in Sydney
Harbor?"
\item "How would you move Mount Fuji?"
\end{itemize}

\ldots{} which are all variants on the classic quandary "How long is a piece
of string?" and equally deserving of serious consideration.

\item Ask About MVC
\label{sec:orgheadline64}

For some reason, it has become accepted in technical circles that all
programming interviews must contain a question about the
Model-View-Controller pattern. Every candidate expects it, every
interviewer asks it -- and there's no good reason for you to challenge
the tradition. At least it chews up some interview time and spares you
having to think of your own questions.

\item Ask General Questions But Expect A Specific Answer
\label{sec:orgheadline65}

This technique is the staple of anti-social interviewers everywhere.
It's particularly handy if you want to devote no cognitive energy
whatsoever to the proceedings. Ask a question that is general enough to
permit multiple answers, but badger the candidate until they provide the
specific answer that you have in mind. Thus a technical query turns into
a guessing game, which is great fun for everyone -- providing you're not
the one doing the guessing.

\item Take Every Opportunity To Demonstrate How Clever You Are
\label{sec:orgheadline66}

For the sociopath, the interview is mainly about them and only
peripherally about the candidate. They view an interview as an
opportunity to demonstrate their natural intellectual and technical
superiority. That they control the questions and have had time to
research the answers doesn't hurt either.

You should make frequent, derogatory references to the quality of the
candidates you have previously interviewed, the implication being that
the current candidate can expect to be discussed in similarly negative
terms once they are absent.

Don't hesitate to mock the candidate if they answer a question
incorrectly. If it looks like they are about to provide a correct
answer, interrupt them and change or augment the original question with
additional complexities, creating a moving target that they will
eventually abandon hope of ever hitting.

A technique that will certainly annoy the candidate (and people react in
so much more interesting ways once they're angry, don't they?) is to
deliberately misinterpret the candidates answer, exaggerate or distort
it, then throw it back to them as a challenge i.e. create a straw man
from their answer. Here is an example from one of my recent interviews:

\begin{description}
\item[{Interviewer}] Have you participated in code reviews before?
\item[{Ed}] Yes. I've reviewed other team member's code on many occasions.
\item[{Interviewer}] So you don't trust your colleagues, then?
\end{description}

An attitude of willful antagonism will enable you to goad even the most
dispassionate of candidates into an angry (and entertaining) response.

\item Set COMP101 Programming Problems
\label{sec:orgheadline67}

Companies intent upon creating the impression that they really care
about the quality of their people will give potential candidates a hokey
COMP101-level programming problem to solve prior to granting them an
audience. The solution provided is then dissected carefully and assessed
according to criteria that the candidate was not made aware of at the
time the assignment was set. Ridiculous extrapolations and inferences
about the author's general programming ability are then made based upon
the given code sample.

The beauty of this technique is that because the problem has been
offered context-free, the candidate has no idea what design forces
should influence their solution. They don't know what importance to
assign to non-functional criteria such as performance, extensibility,
genericity and memory consumption. The weight of these factors might
significantly influence the form of the solution. By withholding them,
and because these factors are often in conflict with each other, it is
impossible for the candidate to submit a solution that is correct.
Simply change the criteria for evaluation to the opposite of whatever
qualities their solution actually contains.

For example, if their solution is readily extensible, claim that it is
too complex. If they have favored clarity over efficiency, criticize
their solution for its verbosity and memory footprint. If they have
provided you only with code, select documentation-level and
handover-readiness as the criteria-du-jour -- question the absence of
release notes.

\item Treat Senior Candidates The Same As Junior Candidates
\label{sec:orgheadline68}

Those who have been in the industry for a few decades will probably
arrive at the interview expecting you to draw upon their extensive
experience as a source of examples of problems you have solved,
applications you have implemented and difficulties you have overcome. A
sociopathic interviewer should demonstrate their contempt for the
candidates' life's work by completing ignoring their work history. Make
it clear that you don't care about the past by treating even the most
senior of candidates like a fresh-faced rookie, demonstrating an
appropriately condescending and patronizing attitude. After all, even
the most worldly-wise candidate appears naive when put alongside your
own towering genius.

The most effective means of convey your disdain for the candidate that I
have witnessed is to ask them to take an IQ test, thereby implying that
it is not their professional qualifications which are in doubt, but
their native intelligence.

\item Make The Interview Process Long And Arduous
\label{sec:orgheadline69}

There is a lot of folk wisdom surrounding the hiring process. One common
misperception is that the more arduous the interview process (i.e. the
more rounds it contains, the greater the size of the interview panel
etc.) then the more worth the position actually has. In other words, the
harder the journey the better the destination must be. Clearly, the
logic is flawed -- it is quite possible for a long and demanding journey
to conclude in a cesspit.

In an organizational context, a protracted interview process may simply
indicate that the company is disorganized, indecisive and have failed to
gather the information they needed in an efficient manner. But the myth
persists, so you can exploit it to maximum effect, creating ever greater
hoops for the candidate to jump through, on the pretext that you are
being thorough or somehow testing their commitment. Be careful not to
let on that you are really only demonstrating your own ineptitude and
disrespect for the candidate's time.

\item Don't Hire Too Smart
\label{sec:orgheadline70}

One of the biggest hiring mistakes you can make is to hire someone who
is better than you, and whose subsequent performance makes you look bad
by comparison. As soon as you've formed an impression of the candidate's
ability, adjust your interview technique accordingly. If the candidate
is too good, step up the difficulty and obscurity of the questions you
ask until you reach the point where they are struggling, and thereby
creating a bad impression with any other interviewers present. If you
sense the candidate is just good enough to do the job but not so good
that they could do your job, then ease up on the questions and let them
shine.

Remember that there may also be some career advantage in simply not
filling the position at all; concluding that you simply couldn't find a
suitable candidate. You may be able to emphasize how lucky your company
was to have hired the last decent software developer out there -- you.

\item Conclusion
\label{sec:orgheadline71}

The senior ranks of the software development community seems to attract
more than it's fair share of sociopaths. Such people undertake the
interview process with the same intent as they approach all activities
-- to create advantage for themselves. Whether you are amongst the
self-adoring community of psychopaths, or just anti-social with
psychopathic ambitions, the technical interview is a professional
construct designed with your particular needs in mind. Using the
techniques described above, interviews can be both a means of
self-gratification and a fulcrum for leveraging your own career
advantage.
\end{enumerate}

\subsubsection{The Art of Flame War  \footnote{First published 13 Mar 2005 at
\url{http://www.hacknot.info/hacknot/action/showEntry?eid=72}}}
\label{sec:orgheadline89}

The word "argument" has negative connotations for many people. It is
associated with heated exchanges and passionate disagreement. But your
experience of argument need not be so negative. Consider that the word
'argument' also means 'a line of reasoning'. By approaching a verbal or
electronic discussion, even a hostile one, with this definition in mind,
you can learn to separate the logical content of the exchange from its
emotional content and thereby deal with each more effectively. You may
even find the process of so doing an agreeable one.

The following are a few tips and techniques that I've learnt in the
course of a great many arguments, flame wars and other "vigorous
discussions" that may help you argue more purposefully, and thereby come
to view argument as a stimulating activity to be relished, rather than
an ordeal to be avoided.

\begin{enumerate}
\item You Can Be Right, But You Can't Win
\label{sec:orgheadline73}

At the end of a formal debate, one or more adjudicators decides which
team are the victors. If only it were that clean cut in real life. A
good portion of the time, arguments arise spontaneously, continue in a
haphazard manner and then fizzle out without any clear resolution or
outcome. When you cannot force your opponent to concede their losses or
acknowledge your victories, it becomes impossible to keep score.
Therefore you should not enter any dispute, particularly an online one,
with visions of your ultimate rhetorical triumph, in which you lord your
argumentative superiority over your opponent, who shirks away, cap in
hand and ego in tatters. It's not going to happen.

So why engage in argument at all, if you can never win? Here are a few
possible motivations:

\begin{itemize}
\item To hone your rhetorical and logical skills i.e. your attitude will be
more playful than combative
\item To get something off your chest
\item To gratify your ego
\item To restore the balance of opinion
\item To humiliate your opponent
\item To defend your own beliefs against a real or perceived attack
\item To learn about your opponent
\item To learn about yourself
\item To explore the subject matter
\item To protect your reputation against a real or perceived slight
\end{itemize}

\item Remain As Dispassionate As Possible
\label{sec:orgheadline74}

This is at once the most difficult and the most valuable aspect of
arguing effectively. Strong emotion can cloud your thinking and inhibit
your ability to reason objectively and thoroughly. Anger is what turns a
discussion into an argument and then into a flame war. Responses you
give while angry are likely to be poorly considered, so it is invaluable
to have techniques at your disposal to moderate that anger so that you
can argue at your best and even begin to enjoy the dispute. Here are a
few techniques that might be useful:

\begin{itemize}
\item When you're not arguing in real-time (e.g. via email or discussion
forums), print out the email or message that you've found
inflammatory. Read it somewhere away from the computer and plan how
you will respond. Delay making your actual response as long as
possible.
\item When arguing in person, make a deliberate effort to slow down the
pace of the discussion and lower its volume. If you're uncomfortable
with the silence created, adopt a thoughtful expression and pretend
to be considering your reply carefully. Use the time created to take
a few deep breaths and calm down.
\item Adopt a different mental posture towards the email or message.
Pretend that the message is for someone else. This helps to
de-personalize the argument and put it at a distance.
\end{itemize}

Realizing that your opponent is as susceptible to emotion as you are,
you may choose to use this to your advantage. Here we venture out of the
realm of the logical and into the rhetorical. If you can identify your
opponent's "hot buttons," then you may be able to goad them into making
an unconsidered response. Once made, the response cannot be retracted
and you may be able to play that advantage for the remainder of the
argument. When being inflammatory or provocative, be careful not to
overdo it. Lest you appear vitriolic or juvenile, make your barbs short
and well targeted. Ensure that they are offered as parenthetical asides
rather than as a basis for argument.

Perhaps the most effective means of disarming your opponent's insults is
with wit, as demonstrated by the following exchange between Winston
Churchill and Lady Asbury:

\begin{description}
\item[{Lady Asbury}] Mr. Churchill, if you were my husband, I would put
poison in your wine.
\item[{Winston Churchill}] Madam, if you were my wife, I would drink it.
\end{description}

\item Be Familiar With The Basic Logical Fallacies
\label{sec:orgheadline81}

Those not skilled in argument are often prone to employing logical
fallacies and being unaware that they are doing so. It is vital that you
be able to recognize at least the basic logical fallacies so that you
don't end up trying to attack an insensible argument, or formulating one
yourself. Common logical fallacies include:

\begin{enumerate}
\item Straw Man Arguments
\label{sec:orgheadline75}

Your opponent restates your argument inaccurately and in a weaker form,
then refutes the weaker argument as if it were your own.

\item Argumentum Ad Hominem
\label{sec:orgheadline76}

Ad hominem means 'to the man'. Your opponent attacks you rather than
your argument. If you choose to insult your opponent in order to provoke
an emotional reaction, be sure that your insults are not used as part of
your argument, otherwise you will be guilty of argumentum ad hominem
yourself.

\item Appeal To Popularity
\label{sec:orgheadline77}

The suggestion that because something is popular it must be good, or
because something is widely believed it must be true.

\item Hasty Generalization
\label{sec:orgheadline78}

Making an unjustified generalization from too little evidence or only a
few examples.

\item Appeal To Ignorance
\label{sec:orgheadline79}

Claiming that something is true because there is no evidence that it is
false.

\item Appeal To Authority
\label{sec:orgheadline80}

Claiming that something is true because someone important says that it
is.
\end{enumerate}

\item Seek Precision
\label{sec:orgheadline82}

It's easy to end up arguing at cross-purposes with someone simply
because you each have different definitions in mind for component terms
of the subject being debated. So a good starting point when engaging in
debate is to first ensure that you and your opponent have precisely the
same understanding of the topic being argued. Remarkably often, the act
of precisely defining the topic will serve to circumvent any subsequent
argument, as it becomes clear that the warring parties do not have
conflicting positions on a given subject, but instead are talking about
different subjects entirely.

\item Ask Pointed Questions
\label{sec:orgheadline83}

There are several reasons why you might choose to ask your opponent
questions:

\begin{itemize}
\item To seek clarification on a point that they have made
\item In the hope that some of the information volunteered will be faulty,
thereby providing you with fuel for rebuttal.
\item To save effort on your part. It often takes less effort to ask a
question than answer it. In a protracted exchange, this economy of
effort can be important. It also gives you time to think about your
next move.
\item Because you know the answer. A powerful rhetorical technique is to
ask a series of questions that leads your opponent, by degrees, to
the realization that their answer is in contradiction with statements
they have previously made.
\end{itemize}

For example, suppose you are arguing the merits of free software with
one of Richard Stallman's disciples. You might use questioning to tease
out the inconsistencies in their philosophy:

\begin{description}
\item[{Free Software Advocate}] All software should be "free", as in
"freedom".
\item[{You}] How do define "free", exactly?
\item[{FSA}] "Free" means that you can do with it whatever you want.
\item[{You}] With no restrictions at all?
\item[{FSA}] Yes - you have absolute freedom to do with it whatever you
please. Anything else is an attempt to take away your freedom.
\item[{You}] Then I would be free to make it non-free if I wanted to?
\item[{FSA}] Ummm \ldots{} I guess so.
\item[{You}] But wouldn't that contradict your original statement that "all
software should be free"?
\end{description}

If the last response from the FSA had been different, the argument might
have headed in a different direction:

\begin{description}
\item[{You}] Then I would be free to make it non-free if I wanted to?
\item[{FSA}] No - that's the exception. You can't inhibit the freedom of
others.
\item[{You}] But doesn't that mean that I'm not really free? Specifically,
I'm not free to inhibit the freedom of others?
\item[{FSA}] Sure, but you have to draw the line when it comes to
fundamental liberties.
\item[{You}] And what basis do you have for claiming that free use of
software is a fundamental liberty?
\end{description}

\ldots{} and so the FSA is led to an awareness of the circular reasoning they
are employing.

\item Don't Claim More Than You Have To
\label{sec:orgheadline84}

A common error is to extend the claims you're making to a broader scope
than is really necessary to make your point. In doing so, you extend the
logical territory that you have to defend and permit counterargument on
a broader front. This is one of the primary benefits of maintaining a
skeptical attitude. Skeptics assume as little as possible, and therefore
have less to defend than True Believers who are prone to making broad
assumptions and sweeping generalizations.

Suppose you're arguing about the quality of open source software versus
proprietary software. An open source zealot may make a broad claim such
as "Open source software is always of higher quality than proprietary
software". A universal qualifier such as "always" makes their claim easy
to disprove -- all that is required is a single counter-example. A more
cautious open source enthusiast might claim "Open source software is
usually of higher quality than proprietary software", which is a
narrower claim than the one made by the zealot, but one still requiring
evidential support. A skeptic might ask "How do you define quality?"

Claims can be accidentally over-extended by provision of a flawed
example of the general point you're making. Your opponent counters the
particular example you've provided and then assumes victory over the
general claim it was supposed to be illustrating. Before choosing to
illustrate your general claim with a specific example, be very sure the
example is a true instance of your general case. It may be more prudent
to leave out your example all together.

\item Seek Evidence
\label{sec:orgheadline85}

It's easy to make bold claims and impressive assertions; it's not so
easy to back them up with proof. A common problem in argument is the
failure to identify which party carries the burden of proof, and to what
extent that burden exists. The general rule is this: He who makes the
claim carries the burden of proving it. If you claim "Linux is more
reliable than Windows", then it is your responsibility to not only
specify your definition of "more reliable" but to provide evidence that
supports your claim. Your claim is not "provisionally true" until
someone can prove you wrong; and neither is it false. It's truth or
otherwise is simply unknown.

This is an area of common misunderstanding amongst those with
pseudo-scientific beliefs. For instance, UFO believers will look at a
history of UFO sightings for some region and note that although 99\% have
been attributed to aircraft, weather balloons and such, 1\% of them are
still unexplained. They delight in this 1\% figure as if it were
vindication of their beliefs. But 1\% being "unknown" does not equate to
"1\% being alien beings in spaceships". It might also mean that the 1\% of
reports were simply too vague or incomplete to permit any kind of
conclusion being reached. Those claiming by implication that the 1\%
represent alien beings carry the burden of proving that with evidence.

But always remain aware of the context in which claims are made.
Different contexts bring with them different levels of formality, and
consequently different evidentiary standards. If your friend remarks
"Boy it's hot outside", it's obviously not appropriate to insist upon
meteorological data to back up their claim. But if an environmental
activist claims "average daytime temperature world-wide has risen an
average of 0.5 degrees in the last century" then the first thing you'll
be wanting to know is where the data came from that supports that claim.

\item When Your Opponent Is Irrational
\label{sec:orgheadline86}

Finally, there is a delicate ethical issue to consider when arguing.
Every so often you find yourself locking horns with someone who appears
to have a fairly shaky grip on reality. I'm not referring to simple
eccentricity or religious fervor, but psychiatric illness. For examples,
you can refer to some of the emails received by the James Randi
Educational Foundation  \footnote{\url{http://www.randi.org/}} (JREF) in response to their million dollar
challenge. James Randi is a well known skeptic and magician. Since 1994,
the JREF has offered a prize of one million dollars to anyone able to
demonstrate paranormal or supernatural abilities or phenomena under
controlled observational conditions. To date, no one has successfully
claimed that prize. But some of the applications \footnote{\url{http://forums.randi.org/forumdisplay.php?f=43}} they receive
suggest that the respondent is unwell, perhaps delusional. If you should
find yourself in online discussion with someone whom you suspect is
unencumbered by the restrictions of rational thought, then perhaps the
best you can do is exit the discussion immediately. To continue is to
risk antagonizing someone who may be genuinely dangerous. This is one of
the prime reasons for conducting online arguments anonymously, where
possible.

\item Knowing When To Quit
\label{sec:orgheadline87}

There comes a point when you want to exit an argument. Perhaps you've
grown bored with it; perhaps it has become clear that your opponent's
views are so heavily entrenched that progress is impossible; perhaps
your opponent is offering only insults without any logical content. Here
are a few ways of bringing the argument to a definite conclusion, rather
than just letting it peter out:

\begin{itemize}
\item Simply walk away. For online arguments, refuse to respond.
\item Insist that any topics covered thus far be resolved before the
argument continues. This prevents your opponent switching subjects
and responding to your rebuttals by simply making a new batch of
assertions.
\item Ask your opponent what they hope to gain by continuing the argument.
To what end are they arguing.
\end{itemize}

\item Reconstruct Your Opponent's Argument
\label{sec:orgheadline88}

Argument reconstruction is the process of analysis the verbal or written
form of an argument and identifying the premises (both explicit and
implied) and the conclusion/s it contains. To effectively rebut your
opponent's arguments you need to know exactly what they are claiming,
and upon what basis they are claiming it. For each premise you identify,
consider whether the premise is true or false. If you think one or more
of them is false, call attention to each of them and ask your opponent
to justify them with evidence. If the conclusions don't follow logically
from the premises, call attention to the logical error. If the
conclusion cannot be true without one or more unstated premises also
being true, then call your opponent's attention to their reliance upon
implicit premises and, where those premises are in doubt, insist that
evidence be provided in support of them.
\end{enumerate}

\subsubsection{Testers: Are They Vegetable or Mineral?*  \footnote{First published 13 Oct 2004 at
\url{http://www.hacknot.info/hacknot/action/showEntry?eid=68}}}
\label{sec:orgheadline102}

There are real advantages to having a group of people, separate from
developers, whose job is solely to find fault with your work. They have
an emotional and cognitive distance from the product that a developer
can never fully imitate. Testing is a task requiring patience, attention
to detail and a fairly devious mindset. Sometimes managers make the
mistake of regarding testing as a second class activity, suitable to be
performed by less skilled or more junior staff members. Such
misimpressions are a disservice to the project and the testing
community.

But a common byproduct of having a distinct testing team is the
development of an adversarial dynamic between testers and developers. I
can understand completely how easily this situation occurs. I recently
had the misfortune to work with a testing team whose methods left myself
and other developers ready to kill them.

Below, I have listed the main work habits this team engaged in, that
made them so difficult to work with. I hope that these items may serve
as a brief catalog of bug reporting "anti-patterns" that testers can use
as a checklist to make sure they are not accidentally annoying the
developers they work with, and that developers can use to identify
sources of friction between themselves and their testing team.

\begin{enumerate}
\item Abbreviating Instructions For Reproducing The Bug
\label{sec:orgheadline90}

\textbf{Problem}: Some testers believe that they can save themselves some time
by describing the circumstances under which the bug appears in the
briefest terms possible. Often the bug report degrades into a contracted
narrative that only specifies the milestones in the series of actions
necessary to reproduce the bug. Being unfamiliar with the application's
internal structure, a tester can not know which of the series of actions
they have followed is most significant when diagnosing the underlying
fault. By neglecting actions they consider unimportant, there is a
significant risk they are omitting important information.

\textbf{Solution}: The best way to avoid this is to simply enumerate all the
actions that are necessary to reproduce the buggy behavior, starting
with the launch of the application. Put the first step in a bug
reporting template to remind testers to do this e.g. "1) Launch the
application. 2) \emph{your text here}"

\item Not Identifying The Erroneous Behavior
\label{sec:orgheadline91}

\textbf{Problem}: The description in the bug report ends in a simple statement
of application state without identifying what aspect of that state is
actually in error. For example, the bug report concludes "The Properties
dialog appears", but the tester fails to add "\ldots{} and the property
controls are enabled, even though the selection is read-only".

\textbf{Solution}: Put the heading "Erroneous behavior:" or "Actual behavior:"
in your bug report template, to remind the tester to include that
information.

\item Not Identifying The Expected Behavior
\label{sec:orgheadline92}

\textbf{Problem}: Even when the bug report contains a description of the
erroneous behavior, testers sometimes forget to explain what the
expected (correct) behavior is. For example, the bug report concludes
"The file saves silently", but the tester fails to add "\ldots{} but there is
no visual indication that the application is busy performing the save.
The cursor should change to an hour glass and a modal progress dialog
should appear.

\textbf{Solution}: Put the heading "Expected behavior: " in your bug report
template, to remind the tester to include that information.

\item Not Justifying The Expected Behavior
\label{sec:orgheadline93}
\textbf{Problem}: It is not always clear why a tester has decided that a
particular behavior is buggy. The bug report may simply claim "X should
happen" without making it clear why X is the correct behavior. A
reference to a requirement specification is an appropriate
justification. If that requirement is for adherence to an externally
specified standard, then a reference to the relevant portion of that
standard is appropriate.

\textbf{Solution}: Put the heading "Requirement reference:" in your bug report
template, to remind the tester to include that information.

\item Re-Opening Old Bug Reports For New Bugs With Similar Symptoms
\label{sec:orgheadline95}

\textbf{Problem}: A bug report is marked as FIXED and everyone thinks it is
done with. But in the course of subsequent testing, a tester sees faulty
behavior occurring that is very similar to that produced by the bug that
was thought FIXED. Reasoning that the behavior is so similar that it
must have the same underlying cause, the tester concludes that the bug
previously marked FIXED has resurfaced. They REOPEN the FIXED bug
report. This is problematic for the developer, because the re-opening of
the bug implies that the original symptoms are re-occurring, not the
similar symptoms that the tester is now observing. The tester has
communicated to the developer their incorrect diagnosis of the fault,
rather than simply reporting the faulty behavior they have observed.

\textbf{Solution}: Insist that testers refrain from reusing old bug reports
unless the erroneous behavior they see is exactly the same as that
described in the old bug report. Even then, there is some chance of
confusing two separate bugs that just happen to produce identical
observed behavior. If there is any doubt, create an entirely new bug
report. The develop can always mark it as a duplicate of the old bug
report and re-open the old bug report themselves, if investigation
demonstrates that the new and old bugs have the same underlying cause.
See also \hyperref[sec:orgheadline94]{Diagnosing Instead Of Reporting}

\item Testing An Old Version Of The Software
\label{sec:orgheadline96}

\textbf{Problem}:

\begin{description}
\item[{Developer}] It's fixed!
\item[{Tester}] It's NOT fixed!
\item[{Developer}] It's fixed! Here's a screen shot showing it fixed!
\item[{Tester}] I don't care about your screen shot. It's NOT fixed for me!
\end{description}

This developer / tester exchange quickly escalates into justifiable
homicide and arises far more often than it should. In a testing process
which permits the version of the software being tested to change
underfoot, the conflict often arises from a developer fixing a bug in a
version yet to be released to the tester. Both developer and tester are
correct in their assessment of the bug's status, with respect to the
version of the software that is front of them.

\textbf{Solution}: Institute a process to enable version coordination between
developers and testers. Label each new version with a unique number and
make the version numbers currently being tested and developed readily
available to all. Ensure someone has the responsibility to update this
version number whenever a new version is released to the testers. When a
bug report is declared FIXED, ensure developers include the version
number in which the fix will appear.

\item Inventing Requirements Based Upon Personal Preference
\label{sec:orgheadline97}

\textbf{Problem}: Generally a set of requirements is not so complete as to
explicitly specify program behavior in every possible circumstance.
Quite aside from inevitable oversights by those assembling the
requirements, some requirements are left to "common sense". A
requirement such as "shall conform to Microsoft Windows User Interface
Guidelines" is broad and may be difficult to interpret in any particular
instance. Rather than interrogate the standard thoroughly, some testers
will try and substitute their own version of "common sense" for the
requirement, bringing with it their mistakes and misinterpretations. For
instance, I received a UI bug report indicating that "a sub-menu should
not appear if all menu items within it are disabled." The tester
regarded this as "common sense". However, the UI standards explicitly
dictated that such sub-menus should always appear, even when all of
their menu items are disabled, so that the user could at least see the
contents of the sub-menu and would know where to find a particular
option when it did become available. Yet the bug report stated quite
emphatically that the behavior "should" be different. The tester had
fabricated the requirement, and decided to lend it authority by using
the word "should", so as to imply the presence of such a requirement.

\textbf{Solution}: See "\hyperref[sec:orgheadline93]{Not Justifying the Expected Behaviour}"

\item Omitting Screen Shots
\label{sec:orgheadline98}

\textbf{Problem}: Many bug tracking systems provide the facility to attach a
file to a bug report, the way one might attach a file to an email. But
testers frequently forget (or can't be bothered) making use of this
facility. Particularly for GUI-related bugs, a screen shot showing the
bug occurring, or illustrating a step in its reproduction, is an
efficient way of capturing information.

\textbf{Solution}: Make sure testers are aware of the "attach" functionality in
your bug tracking system and are encouraged to use it. Image attachments
can also be a convenient way of proving to a disbelieving developer that
a bug occurs, or to a tester that a bug has been fixed.

\item Using Vague Or Ambiguous Wording
\label{sec:orgheadline99}

\textbf{Problem}: In the text of the bug report, the tester employs terminology
that is imprecise or ambiguous. For example: the tester refers to "this
dialog" in the bug report, intending the word "dialog" to mean "an
exchange between parties"; but the developer interprets "dialog" as
referring to a secondary window in the interface. Another example: The
tester describes a text field as being "enabled when it should be
disabled", but really intended that the text field is "editable when it
should be uneditable".

\textbf{Solution}: None -- however a large, blunt object applied with extreme
prejudice can at least have a cautionary effect.

\item Diagnosing Instead Of Reporting
\label{sec:orgheadline94}
\textbf{Problem}: Either through arrogance or a misguided attempt to be
helpful, the tester describes what they believe is the underlying fault
exposed by the bug, rather than simply reporting the observed behavior.
For example, the tester examines a log file and deduces from the name of
an exception appearing in a stack trace that the application is running
out of memory. Having provided this insight, they omit the rest of the
bug report, thinking that they have already provided the crucial
information. \textbf{Solution}: See "Solution" above.

\item Exaggerating The Priority Of A Bug
\label{sec:orgheadline100}

\textbf{Problem}: Some testers exhibit a tendency to elevate the priority of
the bug reports they lodge later in the testing process. As testing
proceeds and the identification of new bugs becomes harder and harder,
it seems that the extra effort involved in their location is justified
by raising their priority - by way of psychological compensation, I
suppose. Developers find that bugs which would have been regarded minor
in early testing are suddenly becoming major issues. This effect may
also be attributable to increasing stress or approaching deadlines.
\textbf{Solution}: For each priority level your bug reporting system allows,
provide a clear definition that can be referred to in order to resolve
disputes over bug priority.

\item Justifying Partial Coverage With Appeals To Bad Assumptions
\label{sec:orgheadline101}

\textbf{Problem}: Rather than exhaustively test all possible combinations of
inputs or circumstances, testers choose a limited subset of these for
testing, reasoning that the chosen subset will be sufficient to exercise
the underlying code. In effect, they are making assumptions about the
code coverage that results from manipulating the application's interface
in various ways.

\textbf{Solution}: Sometimes assumptions of this nature can legitimately be
made. If there is insufficient time to perform exhaustive testing, then
it is the developers who should be choosing the representative subset of
operations to test, not the testers. See \hyperref[sec:orgheadline94]{Diagnosing Instead Of Reporting}
\end{enumerate}

\subsubsection{Corporate Pimps: Dealing With Technical Recruiters  \footnote{First published 12 Jul 2003 at
\url{http://www.hacknot.info/hacknot/action/showEntry?eid=1}}}
\label{sec:orgheadline124}

Anyone who has had any substantial dealings with technical recruiters
invariably has a poor opinion of them. This is because the standard of
practice in the recruiting industry is so low. To be a recruiter you
don't need any formal qualification, or any particular experience.

Recruiting, as it is generally practiced, is little more than
telemarketing. As with telemarketing, people are drawn to it because of
the opportunity to make money without having to satisfy any particular
educational requirements. A recruiter's commission is generally 15-20\%
of the candidate's first year's salary, which explains why recruiters
are not generally altruistically motivated. They share the ethical and
moral shortcomings of workers in other commission-based occupations such
as used car salesmen, real estate agents and pimps.

In your interaction with recruiters, it pays to keep the following
firmly in mind:

\begin{itemize}
\item The recruiter is first and foremost a salesman, so their prime
objective is to make money. They do this by finding someone who
satisfies their client's requirements for long enough to earn them a
commission.
\item You don't need the recruiter's good favor, you just need to convince
them to pass your resume onto their client. Because recruiters are
universally maligned, their clients have no more respect for their
opinions than you do.
\item The recruiter has no technical knowledge. The skills you've spent
years acquiring are just empty keywords and acronyms to them.
\item Never allow yourself to be talked into doing something you don't want
to. Recruiters are good talkers, and know how to railroad the
introverted techie into a particular course of action. They will
speak quickly, loudly and with unwarranted familiarity in order to
influence you into doing what they want.
\item Above all, remember that it's your career you're dealing with. You
are the only one who exercises any control over that, not the
recruiter.
\end{itemize}

When I began speaking with recruiters again recently, I went in search
of a guide to help me deal with them more effectively. Finding no such
guide available, I decide to write one. The following presents some tips
on dealing with that most useless of creatures, the IT recruiter.

\begin{enumerate}
\item Phone Calls
\label{sec:orgheadline116}

\begin{enumerate}
\item Tip: Don't Bother Leaving Voicemails
\label{sec:orgheadline103}

You will find that recruiters rarely return your voicemail messages. The
perceived justification for this discourtesy is "I'm too busy,” although
the real reason is "Contacting you doesn't hold the immediate promise of
financial reward". Therefore, don't bother to leave messages -- keep
calling until you can speak to them in person.

\item Tip: Be Cautious When Answering Certain Questions
\label{sec:orgheadline109}

Recruiters will try and gather more information than is necessary, in
the hope of learning something that can be used to their advantage. Only
discuss what is strictly relevant to the job in question. In particular,
look out for the following questions:

\begin{enumerate}
\item Do You Have Any Other Opportunities In Hand?
\label{sec:orgheadline104}

Recruiters will often make a "friendly enquiry" about how your job
hunting prospects are at the moment. This is not idle small talk. The
recruiter is trying to gauge:

\begin{itemize}
\item How desperate you are i.e. how much leverage they have
\item The number of opportunities out there for people with your skill set.
At best, this enquiry could be called "market research."
\item The names of companies that are currently hiring -- so they can
approach them.
\end{itemize}

It is of no advantage to you to provide any of this information to the
recruiter, and it could weaken your bargaining position in future. A
suitable response might be “I'd prefer not to discuss the status of my
job search.” Above all, never appear desperate -- it will be a signal to
the recruiter that they can get away with dramatically cutting your
rate, thereby increasing their profit margin.

\item What Recruiter Did You Apply Through?
\label{sec:orgheadline105}

If you tell them you have already made application for the position
through another recruiter, they may try and find out who that recruiter
is, and what agency they work for. It's none of their business -- tell
them so. The same response as above will suffice.

\item Do You Know Anyone Else Who Might Be Interested In This Job?
\label{sec:orgheadline106}

Here, the recruiter is trying to get you to refer them to another
candidate. Never do this, if you want to keep your friends. Once that
information gets into the recruiter's hands, there is no telling what
will happen to it. The only appropriate answer to the above question is
“no.” If you do know someone who is interested, still tell the recruiter
“no”, and then contact that person yourself so they can approach the
recruiter at their leisure, if they so choose.

\item Who Did You Work For While You Were At Company X?
\label{sec:orgheadline107}

A common technique recruiters use to broaden their client base is to use
candidates to get contacts within companies the candidate has worked
for. For example:

\begin{description}
\item[{Recruiter}] Did you work for \emph{fictional-name} while you were at
J-Corp?
\item[{You}] No -- I've never heard of \emph{fictional-name}. I reported to John
Smith.
\end{description}

Now the recruiter has a contact name within JCorp that they can use to
get past the company switchboard (companies often have switchboard
blocks on recruiters). They can ring J-Corp's switchboard, ask to speak
to John Smith -- without revealing that they are a recruiter -- and be
in a position to market their services directly to someone who is
reasonably senior.

\item What Was Your Rate/Salary In Your Last Contract/Job?
\label{sec:orgheadline108}

The danger in quoting a contract rate is that the rate at which you
actually work (assuming you're awarded the contract) is yet to be
negotiated. If the recruiter can subsequently negotiate a higher rate
with his client, he can keep that information to himself and absorb the
surplus into his margin.
\end{enumerate}

\item Tip: Learn A Few Rote Answers
\label{sec:orgheadline112}

All recruiters tend to ask the same questions. It may surprise you to
know that recruiters often follow scripts -- the same way that
telemarketers follow scripts when cold calling potential customers. They
may have worked with the script so long that they've now internalized
it, or perhaps they've developed the script themselves, refining it over
the course of hundreds of phone calls. The point is, the recruiter is
far more rehearsed in asking questions than you are in providing
answers. To level the playing field, you can prepare your own scripts by
rehearsing answers to some commonly asked questions:

\begin{enumerate}
\item Why Did You Leave Your Last Job?
\label{sec:orgheadline110}

Some recruiters will ask this, as if they had the right to know and
could put the info to any sensible use. Prepare a brief and suitably
vague answer that suggests you bear no animosity towards your last
employer, and that your performance wasn't questioned in any way. A
tried and true comeback is “It was just time for a change” -- which is
impossible to refute or question further.

\item What Is Your Ideal Job?
\label{sec:orgheadline111}

Occasionally a recruiter asks this, just on the off chance that your
ideal job is currently on their books. Not surprisingly, it never is.
They're not really interested in your response, so much as that you have
one and asking it makes it sound like they're displaying due diligence.
Learn a brief and dismissive answer.
\end{enumerate}

\item Tip: Determine The Purpose Of The Call Early In The Conversation
\label{sec:orgheadline113}

It's not uncommon to have recruiters contact you even though they don't
actually have a suitable position to discuss with you -- the operative
word being “suitable.” You may find that they have a position that is
clearly unsuitable for you, but will try and use that position to
establish contact with you, ask you to come and see them for a chat, and
generally begin the recruiting process. These recruiters are desperate
and are trying to match the few positions they have to whatever
candidature they can dig up, no matter how inappropriate the match.
Don't let them waste your time. If they're not prepared to put a job
specification down on the table, walk away.

\item Tip: Protect Your Referees From Unnecessary Interruption
\label{sec:orgheadline114}

There's no need to put “references available upon request” on your
resume -- that is understood. Out of consideration for your referees,
you should aim to minimize the number of occasions they are contacted.
Therefore, never give away your references until there is a job offer on
the table, for the following reasons:

\begin{itemize}
\item Some recruiters will use your referees as contact points for
marketing their services.
\item If the recruiter contacts your referees, there is no guarantee that
their client will not also want to contact them. Then your referees
end up getting hounded with phone calls.
\item If the recruiter contacts your referees prior to a job offer being
made, and the client does not decide to hire you, then your referees
have been pestered for nothing.
\end{itemize}

Some recruiters will try to tell you that they can't even submit your
resume to their client without references. This is nonsense, and
certainly an attempt to collect your referees as contacts.

\item Tip: Be Suspicious Of Phone Calls From Agents You've Never Heard
\label{sec:orgheadline115}
Of

Once you have been circulating your resume for a while, and it has been
entered in the résumé databases of enough agencies, you'll find that you
start getting cold calls from agents that you've never heard of. What's
happened in these cases is that the agent has done a keyword search on
their agency's résumé database for a particular skill set, got back
several dozen matches, and then placed a phone call to every person
whose resume was a match. Your resume happens to be in the agencies
database as a result of your previous contact with some other agent
working at that agency.

If an unknown recruiter leaves you a message, if you do call them back,
you can expect the following:

\begin{itemize}
\item The recruiter doesn't remember who you are.
\item The recruiter doesn't remember what job description they rang you in
relation to.
\item Once they've worked out those two things, they search their database
for your résumé.
\item Then they read out their job's skill requirements and you have to
respond “yes” or “no” to each \ldots{} even though that info is on the
screen in front of them.
\end{itemize}

For this reason I generally don't return calls from recruiters I've
never heard of. I have better things to do than read out my résumé over
the phone.
\end{enumerate}

\item Tricks Of The Trade
\label{sec:orgheadline123}

\begin{enumerate}
\item Trick: Bait And Switch
\label{sec:orgheadline117}

This is an old salesman's scam that still finds application in the
recruiting industry. The practice consists of luring in a candidate with
an inviting (but inaccurate or incomplete) job description, and once the
candidate is “hooked”, revealing the true nature of the position. The
hope is that the sense of positive expectation already created will make
the candidate more receptive to the true job description.

\item Trick: Salary/Contract Rate Negotiation
\label{sec:orgheadline118}

Never forget that the recruiter is paid by the client company to find
employees, and he who pays the bill gets the service. Perhaps this is
the way recruiters self-justify their poor treatment of candidates. It
is also significant when the recruiter is negotiating a salary/rate on
your behalf -- they are negotiating with the same party that pays their
commission, so it is as well to have a good idea of what money you're
worth and to set definite boundaries for the recruiter so that you don't
get sold out. Recruiters will try and get you to lower your rate by
claiming that their client has one or more alternatives of similar
experience/ability as yourself, and they are willing to work at a lower
rate. You can never tell whether your competitors are real or are
phantoms created by the recruiter. Any enquiries you might make to
determine the authenticity of these competitors will be foiled by the
recruiter's claims of privileged information.

\item Trick: Vague Job Descriptions
\label{sec:orgheadline119}

At times, recruiters will publish deliberately vague job descriptions in
the hope of garnering as wide a response as possible. Their motivation
is in part to refresh their internal resume database, and in part to
assess the amount of interest associated with particular skills sets
(market research). There may be an actual job behind it all, or there
may not.

\item Trick: Agent Interviews
\label{sec:orgheadline120}

The “agent interview” is one of the biggest conceits in the recruiting
industry. A small percentage of recruiters will want to speak with you
in person before putting your résumé forward to their client. Some will
even claim that they are required by company policy to do so. The
ostensible purpose of these chats is for the recruiter to get a better
idea of who you are, thereby enabling them to present your strengths
more effectively to their client. If you were wondering exactly what a
recruiter will learn about you in a 20 minute chat that they can't
gather over the phone, then you wouldn't be the first. The real purpose
of agent interviews are:

\begin{itemize}
\item For the recruiter to see how attractive you are.
\end{itemize}

Statistically, good-looking candidates are more likely to interview
successfully. If the recruiter has a choice of candidates to put
forward, they are better off choosing the more attractive ones. Of
course, discrimination based on appearance is illegal, so you'll never
hear any public admission that this sort of assessment occurs.

\begin{itemize}
\item To increase your degree of investment in the agent and the job. Once
you've gone to the effort of meeting with a recruiter, you will have
a natural tendency in future to act in a way that retrospectively
justifies having made that investment. In future you are more likely
to favor that agent, and to be more kindly disposed towards positions
put forward by that agent. If this sort of psychological manipulation
strikes you as being beyond the average recruiter's capability,
remember that most recruiters have at least an intuitive grasp of
sales techniques. Exploiting your need to appear consistent with
previous actions is a common technique employed by salesmen. The
door-to-door salesman who offers a free demonstration of his product
knows that the hidden expense is the cost of your time, which is only
justified if you later make a purchase. The car salesman who lets you
take a vehicle for a test drive is relying upon the same principle.
\item To establish a power dynamic. It is significant that you go to the
recruiter, and not the other way around. This suggests that the
recruiter is in control, as they would like to believe, and as they
would like you to believe.
\end{itemize}

\item Trick: X-Rayers And Phone Lists
\label{sec:orgheadline121}

Recruiters will go to extraordinary lengths to get leads to clients and
candidates. There are a number of software packages available, called
web site “x-rayers” or “flippers”, designed to automatically probe
corporate websites for names and phone numbers. Lurking on Usenet groups
is another way of getting relevant email addresses. Looking to fill a
Java job? A few weeks lurking on \texttt{comp.lang.java} enables the recruiter
to identify the technically savvy and geographically appropriate
posters. I suspect the vast majority of recruiters are not technically
savvy enough to use these sorts of techniques. However, that such
possibilities exist does illustrate why it's worthwhile being very
careful with how much information you give away.

\item Trick: Wooden Ducks
\label{sec:orgheadline122}

Particularly unscrupulous recruiters will submit candidates to their
client to act as placeholders -- for the purposes of making another
candidate appear good by comparison. It's going to be difficult to
determine when you are being used as a wooden duck because you have no
knowledge of the other candidates your recruiter is putting forward.
Tell tale signs may be:

\begin{itemize}
\item The recruiter is pushing hard for you to attend an interview, even
though they have previously expressed doubts about your chances
against other candidates.
\item The recruiter makes no effort to coach you about the interview, what
to expect or how to prepare.
\item The recruiter has hinted that you may be competing against internal
candidates i.e. candidates already employed by the client.\\
\item The recruiter has made statements such as “not getting your hopes up”
or similar, indicating they are anticipating failure.
\end{itemize}
\end{enumerate}
\end{enumerate}

\subsubsection{Developers are from Mars, Programmers are from Venus  \footnote{First published 9 Oct 2006 at
\url{http://www.hacknot.info/hacknot/action/showEntry?eid=90}}}
\label{sec:orgheadline134}

Many of us use the terms "programmer" and "developer" interchangeably.
When someone asks me what I do for a living I tend to describe my
vocation as "computer programmer" rather than "software developer",
because the former seems to be understood more readily by those
unfamiliar with IT. Even when writing pieces for this site, I tend to
swap back and forth between the two terms, to try and avoid sounding
repetitive. But in truth, there is a world of difference between a
computer programmer and a software developer.

The term "programmer" has historically referred to a menial, manual
input task conducted by an unskilled worker. Predecessors of the
computer, such as the Hollerith machine, would be fed encoded
instructions by operators called "programmers". Early
electro-mechanical, valve and relay-based computers were huge and
expensive machines, operated within an institutional environment whose
hierarchical division of labor involved, at the lowest level, a "button
pusher" whose task was to laboriously program the device according to
instructions developed by those higher up the technical ladder. So the
programmer role is traditionally concerned only with the input of data
in machine-compatible form, and not with the relevance or adequacy of
those instructions when executed.

A modern programmer loves cutting code -- and \emph{only} cutting code. They
delight in code the way a writer delights in text. Programmers see their
sole function in an organization as being the production of code, and
view any task that doesn't involve having their hands on the keyboard as
an unwanted distraction.

Developers like to code as well, but they see it as being only a \emph{part}
of their job function. They focus more on delivering value than
delivering program text, and know that they can't create value without
having an awareness of the business context into which they will deploy
their application, and the organizational factors that impact upon its
success once delivered. More specifically \ldots{}

\begin{enumerate}
\item Developers Have Some Knowledge Of The Domain And The Business
\label{sec:orgheadline125}

Programmers like to stay as ignorant as possible of the business within
which they work. They consider the problem domain to be the realm of the
non-technical, and neither their problem or concern. You'll hear
programmers express their indifference to the business within which they
operate - they don't care if it's finance, health or telecommunications.
For them, the domain is just an excuse to exercise a set of programming
technologies.

Developers view the business domain as their "second job." They work to
develop a solid understanding of those aspects of it that impact upon
their software, then use that knowledge to determine what the real
business problems are that the application is meant to be solving. They
make an effort to get inside the heads of their user base -- to see the
software as the users will see it. This perspective enables them to
anticipate requirements that may not have occurred to the users, and to
discover opportunities to add business value that the users may have
been unaware was technically possible.

\item Developers Care About Maintenance Burden
\label{sec:orgheadline126}

Programmers crave new technologies the way children crave sweets. It's a
hunger that can never be satiated. They are forever flitting from one
programming language, framework, library or IDE to the next; forever
gushing enthusiastically about the latest silver bullet to have been
grunted out by some vendor or open source enthusiast, and garnished with
naive praise and marketing hype. They won't hesitate to incorporate the
newest technology into critical parts of their current project, for no
reason other than that it is "cool", and all the other kids are doing
it. They will be so intent on getting this new technology working, and
overcoming the inevitable troubles that immature technologies bring,
that there will be no time to spare for documentation of their effort.
Which is exactly how they like it -- because documentation is, they
believe, of no use to them. Sure, it might be useful to future
generations of programmers, but who cares about them?

Developers have a much more cautious approach to new technology. They
know that a new technology is inevitably hyped through the roof by those
with a vested interest in its success, but that the reality of the
technology's performance in the field often falls short of the
spectacular claims made by proponents. They know that a technology that
is new is also unproven, and that its weaknesses and shortcomings are
neither well known or publicized. They know that part of the reason it
takes time for the negative experiences with technologies to become
apparent is that many developers will be hesitant to say something
critical amongst that first flush of community enthusiasm, for fear that
they will be shouted down by the newly-converted zealots, or dismissed
as laggards who have fallen behind the curve. So developers know to
stand back and wait for the hype to die down, and for cooler heads to
prevail. Developers also know the organizational chaos that can result
from too many changes in technical direction. A company can quickly
accumulate a series of legacy applications, each written in a host of
once-popular technologies, that few (if any) currently on staff possess
the skills to maintain and extend. Those that first championed those
technologies and forced them into production may have long since moved
onto other enthusiasms, perhaps other organizations, leaving behind the
byproduct of their fleeting infatuation as a maintenance burden for the
organization and future staff to bare.

\item Developers Know That Work Methods Are More Important Than Technical
\label{sec:orgheadline127}
Chops

Programmers often focus so intently upon the technologies they use that
they come to believe that technology is the dominant factor influencing
the ultimate success or otherwise of their projects. The mind set
becomes one of constantly looking over the horizon for the next thing
that might solve their software development woes. The expectation
becomes "Everything will be better once we switch to technology X."

Developers know that this "grass is greener" effect is a falsehood --
one often promulgated by vendors, marketers and technology evangelists
in their quest to sell a product. The dominant factors influencing the
quality of your application, and ultimately its success or otherwise,
are the quality of the people doing the development and the work methods
that they follow. In most cases, technology choice is almost incidental
(the one possible exception being where there is a generational,
revolutionary change in technology, such as the transition from low
level to high level programming languages). Therefore developers
frequently posses an interest in QA and software engineering techniques
that their programmer counterparts do not.

\item Programmers Try To Solve Every Problem By Coding
\label{sec:orgheadline128}

It is characteristic of the programmer mentality that every problem they
encounter is perceived as an opportunity to write more code. A typical
manifestation is the presence of a "tools guy" on a development team.
This is the guy who is continually writing new scripts and utilities to
facilitate the development process, even if the process he is automating
is only performed once in a blue moon, meaning that there is more effort
expended in writing the tool than the resulting automation will ever
save.

Developers know that coding effort is best reserved for the application
itself. After all, this is what you are being paid to produce. They know
that tool development is only useful to a point, after which it becomes
just a self-indulgent distraction from the task at hand. Typically, a
retreat sought by those with a love of "plumbing" and
infrastructure-level development. Developers know that there are many
development tasks that it is simply not worth automating and, where
possible, will buy their development tools rather than roll their own,
as this is the most time- and cost-efficient way of meeting their needs.

\item Developers Seek Repeatability, Programmers Like One-Off Heroics
\label{sec:orgheadline129}

If development were an Aesop's fable, then programmers would be the
hares, and developers the tortoises. Programmers, prone to an
over-confidence resulting from excessive faith in technology's ability
to save the day, will find themselves facing impending deadlines with
work still to go that was meant to be made "easy" by that technology,
but was unexpectedly time-consuming. Not surprisingly, the technology
doesn't ameliorate the impact of too little forethought and planning.
These last-minute saves, and the concentrated effort they require, are
later interpreted as evidence of commitment and conviction, rewarded as
such, and thereby perpetuated.

Developers are very aware that there are no silver bullets, be they
methodological or technological. Rather than pinning their hopes on new
methods or tools, they settle down to a period of detailed analysis and
planning, during which they do their best to anticipate the road ahead
and the sorts of obstacles they will encounter. They only proceed when
they feel that they can do so without entertaining too much risk of
making faulty assumptions, and having to later throw work away.

\item Programmers Like Complexity, Developers Favor Simplicity
\label{sec:orgheadline130}

It's not uncommon for programmers to deliberately over-engineer the
solutions they produce, simply because they enjoy having a more complex
problem to solve. They may introduce requirements that are actually
quite unnecessary, but which give them the opportunity to employ some
technology that they have been itching to play with. Their users will
have to bear this extra complexity in their every interaction with the
system; maintenance programmers will have to wade through it in every
fix and patch; the company will have to finance the extensions to the
project schedule necessary to support the additional implementation
effort; but the programmers care about none of this -- as long as they
get to play with a shiny new tech toy.

Developers continually seek the simplest possible resolution to all the
design forces impinging on their project, regardless of how cool or
trendy the technology path it takes them down. If the project's best
interests are served by implementing in Visual Basic, then VB is what
you use, even though VB isn't cool and may not be something you really
want to see on your CV. If the problem doesn't demand a distributed
solution, with all the scalability that such an architecture provides,
then you don't foist a distributed architecture upon the project just so
you can get some experience with the technologies involved, or just
because it is possible to fabricate some specious "what if" scenario to
justify its usage, even though this scenario is never likely to occur in
a real business context.

\item Developers Care About Users
\label{sec:orgheadline131}

Programmers often view their user base with disdain or even outright
contempt, as if they are the ignorant hordes to whose low technical
literacy they must pander. They refer to them as "lusers", and laugh at
their relative inexperience with computing technology. Their attitude is
one of "What a shame we have to waste our elite programming skills
solving your petty problems" and "You'll take whatever I give you and be
thankful for it." Programmers delight in throwing technical jargon at
the user base, knowing that it won't be understood, because it enables
them to feel superior. They are quick to brush off the user's requests
for help or additional functionality, justifying their laziness by
appealing to "technical reasons" that are too involved to go into.

Developers don't consider users beneath them, but recognize and respect
that they just serve the organization in a different capacity. Their
contribution is no less important for that. When speaking with users,
they try to eliminate unnecessary technical jargon from their speech,
and instead adopt terminology more familiar to the user. They presume
that requests for functionality or guidance are well intended, and
endeavor to objectively appraise the worth of user's requests in terms
of business value rather than personal appeal.

\item Developers Like To Satisfy A Need, Programmers Like To Finish
\label{sec:orgheadline132}

Programmers tend to rush headlong into tasks, spending little time
considering boundary conditions, low-level details, integration issues
and so on. They are keen to get typing as soon as possible, and convince
themselves that the details can be sorted out later on. The worst that
could happen is that they'll have to abandon what they've done and
rewrite it -- which would simply be an opportunity to do more coding and
perhaps switch technologies as well. They enjoy this trial and error
approach, because it keeps activity focused around the coding.

Developers know that the exacting nature of programming means that "more
haste" often leads to "less speed." They are also mindful of the
temptation to leap into coding a solution before having fully understood
the problem. Therefore they will take the time to ensure that they
understand the intricacies of the problem, and the business need behind
it. Their intent is to solve a business problem, not just to close an
issue in a bug tracking system.

\item Developers Work, Programmers Play
\label{sec:orgheadline133}

Many software developers enter the work force as programmers, having
developed an interest in software from programmer-like, hobbyist
activities. Once they learn something of the role that software plays in
an organizational context, their sphere of concern broadens to encompass
all those other activities that constitute the difference between
programmer and developer, as described above.

However, some never make the attitudinal transition from the amateur to
the professional, and continue to "play" with computers in the same way
they always have, but do so at an employer's expense. Many will never
even appreciate that there could be much more to their work, if only
they were willing to step up to the challenge and responsibility.

Software engineering, not yet a true profession, places no minimum
standards and requirements upon practitioners. Until that changes,
hobbyist programmers will remain free to masquerade as software
development professionals.

It is the developers that you want working in your organization.
Programmers are a dime a dozen, but developers can bring real value to a
business. Wise employers know how to tell the difference.
\end{enumerate}

\subsection{Management}
\label{sec:orgheadline189}

\subsubsection{To The Management  \footnote{First published 12 Jul 2003 at
\url{http://www.hacknot.info/hacknot/action/showEntry?eid=7}}}
\label{sec:orgheadline142}

I am frequently frustrated and disappointed in the standard of
management I am subject to. Discussions with my peers in the software
industry lead me to believe that I am not alone in my malaise. So on
behalf of the silent multitude of software professionals who are
disappointed with their management, I would like to remind you - the
project manager, team leader or technical manager - of those basic
rights to which your staff are entitled.

\begin{enumerate}
\item The Right To Your Courtesy And Respect
\label{sec:orgheadline136}

Of the complaints I hear directed towards management, the most frequent
concern dishonest, abusive or otherwise inappropriate behavior. Remember
that no matter how angry or frustrated you may be feeling, it is never
okay to direct that anger towards your associates. Intemperate outbursts
only engender disrespect and generate ill feeling. As a leader, you are
obliged to behave in an exemplary manner at all times.

Respecting your staff implies valuing their opinions, and being prepared
to accept their determinations in areas where their expertise is greater
than your own. It means acknowledging and accommodating the technical
obstacles they encounter, rather than trying to usurp reality with
authority.

\item The Right To Adequate Resources
\label{sec:orgheadline137}

Skimping on software and hardware resources is an obviously false
economy, as a deficit of either impedes your most expensive resource --
your people. More commonly overlooked are such environmental resources
as lighting, storage space, desk space, ergonomic aids and whiteboards.
Workers quickly become dissatisfied if the basic elements of a
productive environment are absent.

The resource generally in shortest supply in a software development
environment is time. It's not surprising then that unrealistic
scheduling is one of the greatest sources of conflict between technical
staff and their management. Please keep this in mind

\begin{itemize}
\item successful scheduling is a process of negotiation, not dictation.
Nobody knows more about how long a particular task will take to
complete than the person who is to complete it. Your team has the
right to be consulted on the scheduling of those tasks they are
responsible for, and to be able to meet their commitments without
undue stress or hardship.
\end{itemize}

\item The Right To Emotional Safety
\label{sec:orgheadline138}

In many corporate cultures there is a stigma associated with being the
bearer of bad news. To ensure that individuals feel safe in expressing
unpopular truths, you must not only accept, but also welcome bad news as
an opportunity to avert a more serious problem later. Ultimately, you
are reliant upon others to keep you apprised of the project's technical
progress, so it is obviously beneficial to obtain their insights in
uncensored form. For their part, technical staff need to feel that they
can openly seek help with their problems, without risk of punitive
repercussions.

In a human-based endeavor like software development, mistakes and
failures are inevitable. Staff rightfully expects a compassionate
attitude from you when dealing with their own failures. When they
underestimate a task's completion time, or inject a defect into the code
base, they need help in correcting the underlying problem, not
castigation for the symptom.

\item The Right To Your Support
\label{sec:orgheadline139}

Your staff has the right to expect your assistance in dealing with the
issues they encounter. Responsiveness is paramount - issues need to be
dealt with in a timely manner, before they can fester into full-blown
crises. You must be willing to put aside self-regard and do whatever is
necessary to resolve the issue as quickly as possible. This may mean
making an unpopular decision, or entering into conflict with other
managers. Without the courage and integrity to support your team in this
manner, you compromise the well being of the project and the people on
it.

Failure to proactively support your team's efforts will necessarily
disadvantage them. They have a right to presume you will use your
experience and your high level view of the project to forecast the risks
they may encounter, and prepare mitigation strategies accordingly.

\item The Right To Know
\label{sec:orgheadline140}

Your team expects decisions affecting project staffing, scheduling and
scope to be communicated to them quickly and honestly. Unnecessary
delays can limit their ability to respond effectively to changing
conditions, with consequent stress and time pressure.

Some managers feel they have to shield their subordinates from the
political machinations of their organization. This attitude betrays
little respect for their team member's maturity, and a basic ignorance
of the technical personality, which values painful truths over
comforting lies. Your staff has the right to know about anything that
impacts on their work, so that they can maximize the chances of
achieving their goals.

\item The Right To Self-Determination
\label{sec:orgheadline141}

There is nothing so disempowering as to be set goals, but have no
control over the means by which one is to achieve them. This is the
predicament in which you place your staff if you deny them the
flexibility to tailor their work practices to the problem at hand,
insisting instead on rigid adherence to methodological or corporate
dogma. You may find political safety in playing it by the book, but your
people want to work in a way that makes best use of their time and
energy, and expect your support in achieving that goal.

It is my recurring observation that management practices that infringe
upon the abovementioned rights are common. Equally common is the
software professional's lamentation that their management "doesn't have
a clue." The two may well be causally related.

So I urge you to put aside your spreadsheets and Gantt charts for a
moment and consider the rights of your subordinates. Focus on the basic
principles of a humane management style - integrity, respect, courtesy
and compassion. Their application cannot guarantee your success as a
manager, but their absence will guarantee your failure.
\end{enumerate}

\subsubsection{Great Mistakes in Technical Leadership  \footnote{First published 11 Jun 2006 at
\url{http://www.hacknot.info/hacknot/action/showEntry?eid=87}}}
\label{sec:orgheadline162}

\begin{quote}
\emph{“If you are a good leader who talks little, they will say when your
work is done and your aim fulfilled, ‘We did it ourselves.'”} --
Lao-Tse  \footnote{cited in \emph{Becoming A Technical Leader}, G. M. Weinberg, Dorset
Hourse, 1986}
\end{quote}

Perhaps the most difficult job to do on any software development project
is that of Technical Lead. The Technical Lead has overall responsibility
for all technical aspects of the project -- design, code, technology
selection, work assignment, scheduling and architecture are all within
his purview. Positioned right at the border of the technical and
managerial, they are the proverbial "meat in the sandwich." This means
that they have to be able to speak two languages -- the high-level
language of the project manager to whom they report, and the low-level
technical language of their team. In effect, they're the translator
between the two dialects.

Observation suggests that there are not that many senior techies who
have the skills and personal characteristics necessary to perform the
Technical Lead role well. Of those I have seen attempt it, perhaps ten
percent did a good job of it, twenty percent just got by, and the
remaining seventy percent screwed it up. Therefore most of what I have
learnt about being a good Technical Lead has been learnt by
counter-example. Each time I see a Technical Lead doing something
stupid, I make a mental note to avoid that same behavior or action when
I am next in the Technical Lead role.

What follows is the abridged version of the list of mistakes I have
assembled in this manner over the last thirteen years of watching
Technical Leads get it wrong. It is my contention that if you can just
avoid making these mistakes, you are well on your way to doing a good
job as a Technical Lead. You might consider it a long-form equivalent of
the Hippocratic Oath "First do no harm," although given the self-evident
nature of many of these exhortations, it is more like "First do nothing
stupid."

\begin{enumerate}
\item Mistake \#0: Assuming The Team Serves You
\label{sec:orgheadline143}

Perhaps the most damaging mistake a Technical Lead can make is to assume
that their seniority somehow gives them an elevated status in their
organization. Once their ego gets involved, the door is open to a host
of concomitant miseries such as emotional decision making, defensiveness
and intra-team conflict.

I can't emphasize enough how important it is to realize that although
the Technical Lead role brings with it many additional responsibilities,
it does not put you "above" the other team members in any meaningful
sense. Rather, you are on an exactly equal footing with them. It's just
that your duties are slightly different from theirs.

If anything, it is \emph{you} that is in service of \emph{them}, given that it is
part of your role to facilitate their work. To put it another way, you
are there to make them look good, not the other way around.

\item Mistake \#1: Isolating Yourself From The Team
\label{sec:orgheadline144}

In some organizations, having the title of Technical Lead gives you
entitlements that the rank and file of your team do not enjoy.
Sometimes, the title is considered sufficiently senior to entitle you to
an office of your own, or at least a larger workspace if you must still
dwell in cubicle land.

It is a mistake to take or accept such perquisites, as they serve to
distance you (both physically and organizationally) from the people that
you work most closely with. As military leaders know, it creates an
artificial and ultimately unhealthy class distinction between soldiers
and officers if the latter are afforded special privileges. To truly
understand your team's problems and be considered just "one of the guys"
(which you are), you need to be in the same circumstances as they are.

\item Mistake \#2: Employing Hokey Motivation Techniques
\label{sec:orgheadline145}

Different sorts of people are motivated by different sorts of rewards.
Programmers and managers certainly have very different natures, yet it
is surprising the number of managers and aspiring managers who ignore
those differences and try to reward technical staff in the same way they
would like to be rewarded themselves.

For example, managers value perception and status, so being presented
with an award in front of everyone, or receiving a plaque to display on
their wall where everyone can see it, may well be motivating to them.
However programmers tend to be focused on the practical and functional,
and value things that they can use to some advantage. Programmers regard
the sorts of rewards that managers typically receive as superficial and
trite. They have a similar view of "team building" activities,
motivational speeches and posters and the like.

So if you want to motivate a developer, don't start cheering "Yay team"
or force him to wear the team t-shirt you just had printed. Instead,
give him something of use. A second monitor for his computer will be
well received, as will some extra RAM, a faster CPU, cooler peripherals,
or a more comfortable office chair. It's also hard to go wrong with cash
or time off.

Developers are also constantly mindful of keeping their skill sets up to
date, and so will value any contribution you can make to their technical
education. Give them some time during work hours to pursue their own
projects or explore new technologies, a substantial voucher from your
local technical book store, or leave to attend a training course that
interests them -- it doesn't have to be something that bears direct
relationship to company work, just as long as it has career value to
them.

\item Mistake \#3: Not Providing Technical Direction And Context
\label{sec:orgheadline146}

A common mode of failure amongst Technical Leads is to focus on their
love of the "technical" and forget about their obligation to "lead."
Leading means thinking ahead enough that you can make informed and
well-considered decisions before the need for that decision becomes an
impediment to team progress.

The most obvious form of such leadership is the specification of the
software's overall architecture. Before implementation begins, you
should have already considered the architectural alternatives available,
and have chosen one of them for objective and rationally defensible
reasons. You should also have communicated this architecture to the
team, so that they can always place the units of work they do in a
broader architectural context. This gives their work a direction and
promotes confidence that the team's collective efforts will bind
together into a successful whole.

A Technical Lead lacking in self-confidence can be a major frustration
to their team. They may find themselves waiting on the Lead to make
decisions that significantly effect their work, but find that there is
some reticence or unwillingness to make a firm decision. Particularly
when new in the role, some Technical Leads find it difficult to make
decisions in a timely manner, for they are paralyzed by the fear of
making that decision incorrectly. Troubled that a bad decision will make
them look foolish, they vacillate endlessly between the alternatives,
while their team-mates are standing by wondering when they are going to
be able to move forward. In such cases, one does well to remember that a
good enough decision now is often better than a perfect decision later.
Sometimes there is no choice amongst technical alternatives that jumps
out at you as being clearly better than any other -- there are merely
different possibilities, each with pros and cons. Don't belabor such
decisions indefinitely. In particular, don't hand over such decisions to
the team and hope to arrive at some consensus. Such consensus is often
impossible to obtain. What is most important is that you make a timely
decision that you feel moderately confident in, and then commit to it.
If all else fails, look to those industry figures whose opinions you
trust, and follow the advice they have to give.

Finally, always be prepared to admit that a decision you've made was
incorrect, if information to that effect should come to light. Some of
the nastiest technical disasters I've witnessed have originated with a
senior techie with an ego investment in a particular decision, who lacks
the integrity necessary to admit error, even when their mistake is
obvious to all.

\item Mistake \#4: Fulfilling Your Own Needs Via The Team
\label{sec:orgheadline147}

You will occasionally hear people opine that one should not let the
personal interfere with the professional. In other words, difficulties
at home should not interfere with the execution of duties in the
workplace. In some environments, the obvious expression of emotion is
simply taboo. But such ideas don't mesh with reality too well. People
are holistic creatures and our life experience is not so conveniently
compartmentalized, no matter how desirable some Taylorist ideal may be.

Just the same, there are practical and social limitations upon workplace
behavior which some may be tempted to flaunt, to the discomfort and
embarrassment of their colleagues. The broader one's influence, the
greater the opportunity to co-opt activities that should be focused on
work, and turn them to personal effect.

For example, meetings (complete with buffet) make a fine social occasion
for those not concerned with making best use of company time.
Teambuilding exercises provide an easily excused opportunity to get away
from the office and out into the sun, as do off-site training courses
and conferences.

Pair programming seems to be most appealing to those who like to chat
about their work \ldots{} continually. An excessive focus on group
consensus-based decision-making for all technical aspects of the
project, even the trivial ones, may be a sign that a Technical Lead is
more concerned with the sociology of the project and their place amongst
it, than with leadership and making efficient use of people's time and
effort.

\item Mistake \#5: Focusing On Your Individual Contribution
\label{sec:orgheadline148}

Changing roles from developer to Technical Lead requires a certain
adjustment in mindset. As a developer you tend to be focused upon
individual achievement. You spend your time laboring on units of work,
mainly by yourself, and can later point to these discrete pieces of the
application and say, with some satisfaction, "I did that."

But as a Technical Lead your focus shifts from individual achievement to
group achievement. Your work is now to facilitate the work of others.
This means that when others come to you for help, you should be in the
habit of dropping everything and servicing their requests immediately. A
fatal mistake some Technical Leads make is to try and retain their
former role as an individual contributor, which tends to result in the
Technical Lead duties suffering, as they become engrossed in their own
problems and push the concerns of others aside.

The constant alternation between helping individuals with low-level
technical problems and thinking about high-level project-wide issues is
very cognitively demanding. I've come to call the problem "zoom fatigue"
\begin{itemize}
\item the mental fatigue which results from rapidly changing between the
\end{itemize}
precise and the abstract on a regular basis. It's like the physical
fatigue that the eye experiences when constantly switching focus from
long distance to short distance. The muscular effort required within the
eye to change focal length eventually leads to fatigue, making the eye
less responsive to subsequent demands. Similarly, you get cognitive
fatigue when in one moment you are helping someone with an intricate
coding issue, and in the next you're examining the interaction between
subsystems at the architectural level. The latter requires a more
abstract mental state than the former, and alternating between the two
is quite taxing.

As a result, people may come to you seeking help with something that has
been the sole focus of their attention for several hours or days, and
you will find it difficult to "task switch" from what you were just
doing into a mindset where you can discuss the problem with them on
equal terms. I find it helpful to just ask the person to give me ten
minutes to get my head into the problem space, during which I might
retreat to my own machine and study the problem body of code in detail,
before attempting to help them with it.

\item Mistake \#6: Trying To Be Technically Omniscient
\label{sec:orgheadline149}

Just because you have the last word in technical decisions, don't think
that it is somehow assumed that you are the programming equivalent of
Yoda. With the variety and complexity of development technologies always
growing, it is increasingly difficult to maintain a mastery of any given
subset of that domain. As in most growing fields, those who call
themselves "expert" will progressively know more and more about less and
less.

It is therefore entirely possible that you will be learning new
technologies at the same time as you are first applying them. The
mistakes you make and the gaps in your knowledge will be abundantly
obvious to your team members, so it is best to abandon at the outset any
pretext of having it all figured out.

Be open and honest about what you do and don't know. Don't try and
overstate or otherwise misrepresent the extent and nature of your
familiarity with a technology, for once you are found out, the trust
lost will be very difficult to regain.

There is an opportunity here to widen the knowledge and experience of
all team members. You might like to appoint certain people as
specialists in particular technologies, giving them the time and task
assignments necessary to develop a superior knowledge of their assigned
area. To avoid boredom and unnecessary risk, be sure to give these
resident experts plenty of opportunity to spread their knowledge around
the team, and to exchange specialties with others.

Adopting this "collection of specialists" approach makes it clear that
you are not presuming to be all things to all people; and that you have
faith in the abilities of your colleagues. But it will require you to
park your ego at the door and be prepared to say "I don't know" quite
frequently.

But be careful not to lean on others too heavily. It is still vitally
important for you to have a good overarching knowledge of the
technologies you are employing, particularly those elements of them that
are critical to their successful interoperation in service of your
system's architecture.

\item Mistake \#7: Failing To Delegate Effectively
\label{sec:orgheadline150}

To successfully lead a group, there must be an attitude of implicit
trust and assumed good intent between the leader and those being led.
Therefore a Technical Lead must be willing to trust his team to be
diligent in the pursuit of their goals, without feeling the need to
watch over their shoulder and constantly monitor their progress. This
sort of micromanagement is particularly loathed by programmers, who
recognize it as a tacit questioning of their abilities and commitment.

But ineffective delegation can also arise for selfish reasons. Several
times now I've seen Technical Leads who like to save all the "fun" work
for themselves, leaving others the tedious grunt work. For example, the
Technical Lead will assign themselves the task of evaluating new
technologies, constructing exploratory and "proof of concept"
prototypes, but once play time is over and the need for disciplined work
arrives, hand over the detailed tasks to others.

Not only is effective delegation desirable with respect to team morale
and project risk, on large projects it is simply a necessity, as there
will be too much information to be managed and maintained at once for
one person to be able to cope.

\item Mistake \#8: Being Ignorant Of Your Own Shortcomings
\label{sec:orgheadline151}

Some people simply don't have the natural proclivities necessary to be
good Technical Leads. It's not enough to have good technical knowledge.
You must be able to communicate that knowledge to others, as well as
translate it into a simpler form that your management can understand.

You also need good organizational skills. Coordinating the efforts of
multiple people to produce a functionally consistent outcome is not
easy, and demands a methodical and detail-oriented approach to planning
and scheduling. If you can't plan ahead successfully, you will find
yourself constantly in reactive mode, which is both stressful and
inefficient.

If you don't have these qualities naturally, you may be able to develop
them to some extent, through training and deliberate effort. But it may
ultimately be necessary for you to lean on others in your team to
support you, should they have strengths in areas in which you have
weaknesses.

\item Mistake \#9: Failing To Represent The Best Interests Of Your Team
\label{sec:orgheadline152}

Perhaps the most nauseating mistake a Technical Lead can make is to
become a puppet of the management above them. As the interface between
management and technicians, it is the Technical Lead's role to go into
bat with their management to represent the best interests of their team.
This means standing up to the imposition of unreasonable deadlines,
fighting for decent tools and resources, and preventing the
prevarications of management from disturbing the rhythm of the project.
A weak-willed or easily manipulated Technical Lead will incur the
disrespect of his team.

Unfortunately, such spineless behavior is quite common amongst the ranks
of the ambitious, and you don't have to look far to find obsequious
Technical Leads who will gladly promise the impossible and impose
hardship on their team, in the interests of creating a "can do" image
for themselves.

\item Mistake \#10: Failing To Anticipate
\label{sec:orgheadline153}

An essential part of the Technical Lead's role is keeping an eye on the
"big picture" -- those system-wide concerns that are easily forgotten by
programmers whose attention is consumed by the coding problem they
currently face. These "big picture" issues include those non-functional
requirements sometimes called "-ilities" - maintainability, reliability,
usability, testability and so on. If you don't make a conscious effort
to track your progress against these requirements, there is a high
probability of them slipping through the cracks and being forgotten
about until they later emerge as crises.

If you don't have a dedicated project manager, it may also fall to you
to handle the scheduling, tracking and assignment of tasks. It isn't
uncommon for Technical Leads to find themselves playing dual roles in
this manner. You may not be very fond of such "administrative" duties,
but their efficient performance is critical to the smooth running of the
project, and for the developers to know where they are and where they're
going. Don't make the mistake of ignoring or devaluing these tasks
simply because they are non-technical in nature.

\item Mistake \#11: Repeat Mistakes Others Have Already Made
\label{sec:orgheadline154}

It is common for developers to dismiss the experience reports of others
as having no relevance to their own situation. Indeed, it is wise to
approach all anecdotal evidence with skepticism. But it is unwise to
\emph{completely} disregard the advice of others, particularly when it is
accompanied by sound reasoning, or can be independently verified.
Ignoring good advice can be very expensive; as Benjamin Franklin said,
"Experience keeps a dear school but fools will learn in no other."

The unwillingness of developers to learn from the mistakes of others,
and the ease with which you can encounter software project horror
stories in the literature and recognize your own projects in them, is
evidence suggesting that the software industry as a whole is not getting
any wiser. \footnote{\emph{Facts and Fallacies of Software Engineering}, Robert L. Glass,
Addison-Wesley, 2003} You need not contribute to that collective stupidity.

\item Mistake \#12: Using The Project To Pursue Your Own Technical
\label{sec:orgheadline155}
Interests

Remarkably, developers can reach quite senior levels in their
organization without having learnt to appreciate the difference between
work and play. Many are attracted to programming to begin with because,
as hobbyists, they enjoyed fooling around with the latest and greatest
technologies. Somehow they carry this tendency to "play" with
technologies into their working lives, and it becomes the aspect of
their jobs that they value most. From their perspective, the purpose of
a development effort is not to create something of value to the
business, but to create an opportunity to experiment with new
technologies and pad their CV with some new acronyms.

Their technology selection is based upon whatever looks "cool". But a
rational approach to technology selection may yield quite a different
result to one guided by technical enthusiasm or a fascination with
novelty. New technologies are often riskier choices, as the development
community has not had much time to apply the technology in varying
circumstances and thereby discover its weaknesses and shortcomings.
Putting an immature technology on a project's critical path is
especially risky. So an older, tried and true technology may be a more
rational choice than a new, unproven one.

\item Mistake \#13: Not Maintaining Technical Involvement
\label{sec:orgheadline156}

In order to fully appreciate the current status of the project as well
as the difficulties your team is facing, it is vital that you maintain a
coding-level involvement in the project. If you're not cutting code, it
is too easy to become divorced from the effects of your own decision
making, and to be seen by other developers as being out of touch with
the technical realities of the project.

\item Mistake \#14: Playing The Game Rather Than Focusing On The Target
\label{sec:orgheadline157}

In some organizations, being a Technical Lead is a politically sensitive
position. Technology choices, work assignments and project outcomes are
all just tools to be used in the pursuit of personal agendas. To some,
this "game" of political influence is both fascinating and addictive.
They play it in the hope of gaining some advantage for themselves, and
do so to the detriment of the project and the individuals upon to
Machiavellian maneuverings than to the technical difficulties of the
project, then the project inevitably suffers.

\item Mistake \#15: Avoiding Conflict
\label{sec:orgheadline158}

Many people find interpersonal conflict distasteful. Some dislike it so
much that they will do practically anything to avoid it, including
giving up in technical disputes. Such people are prone to being walked
over by those more aggressive and forthright.

This is bad enough for the individual, but worse if that person is meant
to be representing the best interests of a team. A meek Technical Lead
can be a real liability to a development team, who will find themselves
buffeted about by external forces that they should have been shielded
from, and burdened by demands and goals that are not informed by the
project's reality.

With such a disposition, a Technical Lead may be unable to even deal
effectively with unruly behavior or inadequate performance from members
of their own team.

\item Mistake \#16: Putting The Project Before The People
\label{sec:orgheadline159}

It's one thing to be focused on the project's goals, but quite another
to adopt a "succeed at all costs" attitude. Ambitious Technical Leads,
concerned with the image they project to their management, sometimes
accept impossible goals or unreasonable demands, because they lack the
courage or integrity to say "no." These goals then become the
development team's burden to shoulder, leading to increased stress,
higher defect injection rates, longer working hours and lower morale.
There is a tendency to be so focused on the end goal that the effects of
the project on the developers gets overlooked. It is not uncommon for
successful delivery on a high pressure project to be followed by the
resignations of several disgruntled team members, making the project's
triumph a pyrrhic victory indeed.

Given the costs of hiring and training staff, treating developers as
expendable resources makes no financial sense, quite aside from the
ethical implications of such treatment. A wise Technical Lead will know
that putting the well-being of the developers first also produces the
best results for the project and the business. Project success should
leave the participants satisfied with their achievement, not burnt out
and demoralized.

\item Mistake \#17: Expecting Everyone To Think And Act Like You
\label{sec:orgheadline160}

Being a Technical Lead may be the first time you are exposed so
frequently and directly to the problem solving styles and low-level work
habits of others. Programming is traditionally an individual activity.
Programmers are often able to face the technical difficulties of their
work in isolation, emerging sometime later with the completed solution.
But as a Technical Lead you will frequently be called on to help those
who are stuck part way through the problem-solving process, unable to
proceed. Seeing a solution that is "under construction" might be a bit
of a shock to you at first, as you may find your colleagues approach to
problem solving dramatically different to your own. Some people work
"outside in", others "inside out", others jump all over the place, some
work quickly with lots of trial and error, others slowly and
methodically. It is tempting to stand in judgment of approaches and
methods that don't gel for you, pronouncing them somehow inferior. Avoid
the temptation. Learn to accept the varieties of cognitive styles on
your team, and recognize that this cognitive diversity may actually be
an asset, for the variety of perspective it brings.

\item Mistake \#18: Failing To Demonstrate Compassion
\label{sec:orgheadline161}

Although I've put this last, it is in some ways the most important of
all the mistakes listed here. Always remember that your team members are
people first and programmers second. You can expect them to be
temperamental, inconsistent, proud, undisciplined and cynical -- perhaps
all in the same day. Which is to say they are flawed and imperfect, just
like you and everyone else. So cut them some slack. Everyone has good
and bad days, strengths and weaknesses; so tolerance is the order of the
day.

If someone breaks the build, it's no big deal. If a regression is
introduced, learn something by finding out how it got there, but don't
get upset over it or attempt to assign blame. If a deadline is missed,
stand back from the immediate situation and appreciate that in the grand
scheme of things, it really doesn't matter. Mistakes happen and you
should expect your colleagues to make many, as you will surely make many
yourself.
\end{enumerate}

\subsubsection{The Architecture Group  \footnote{First published 29 Mar 2005 at
\url{http://www.hacknot.info/hacknot/action/showEntry?eid=73}}}
\label{sec:orgheadline164}

An organizational antipattern that I have seen a few times now is the
formation of an Architecture Group. Architecture Groups generally have
the following purposes:

\begin{itemize}
\item To design the enterprise architecture shared by a group of
applications within an organization
\item To review the design of projects to ensure they are consistent with
the enterprise architecture
\item To prescribe the standard technologies to be used across projects in
the organization
\end{itemize}

In summary, the Architecture Group is an internal "governing body" and
"standards group" rolled into one. Membership of the group tends to be
restricted by seniority -- the architects and senior technical staff.

In general, the Architecture Groups I've witnessed in action have been
disastrous. That's not to say that it need necessarily be so -- I have
no legitimate basis for generalizing beyond my direct experience -- but
based on the reasons that I've seen these groups fail, I conject that
failure is a likely outcome of any such group.

The negative impact of an Architecture Group often originates from the
tendency to create an "us and them" mentality amongst staff. Because the
group makes technology and design decisions which are then imposed upon
other projects, those working on individual projects come to resent the
architecture group for the constraints they have placed upon the
project. Working at the overview level, as an architecture group does,
it is difficult or impossible to keep track of the low level details of
a variety of projects. And yet the details of those projects are key
determinants of the suitability of the technologies and designs that the
architecture group deals with. Project staff come to view the
architecture group as dwelling in an ivory tower, from where they can
afford to overlook the troublesome aspects of the projects in their
influence.

Members of the architecture group can begin to share this view. They
consider their decision making more objective and sensible precisely
because it is not influenced by the low level concerns of individual
projects. Once high level consideration has occurred, any difficulties
encountered while implementing those decisions are dismissed as
"implementation details" that are beneath the group's level of concern.

The major source of trouble with architecture groups seems to be the
social dynamic that builds up around them. They have a tendency to
become a clique that is in overestimation of its own collective
abilities, because it is deprived of any negative feedback concerning
the consequences of the decisions it makes. The absence of feedback
results in part from the unwillingness of project staff to criticize
those senior to them, and in part of the self-imposed isolation of the
architecture group, which makes its decisions from behind closed doors.

The issue of seniority is a real stumbling block, because senior staff
may have great difficulty in admitting that they have made a poor
decision, even when it is perfectly obvious to project staff that this
is the case. Any adjustment to the decrees of the architecture group,
once made, results in a perceived loss of face which the members of the
architecture group can ill afford. Being senior, they are perhaps more
cognizant of the political forces at work in the organization. Perhaps
they are more ambitious, and therefore reticent to concede wrong doing
for fear of the impact it might have on their reputation. Perhaps they
view the objections of project staff as a challenge to their authority.
In any case, members of the architecture group develop an ego
identification with the decisions they make, which leads them to ignore
or devalue negative feedback from project staff -- leading to the
reinforcement of the architecture group's external image as being
isolated from the project community.

Consider also that people working in architectural roles tend to be
abstractionist by nature. They are comfortable working at a high level
and just trusting that the low level details will work themselves out.
When project staff object that a decision made in the abstract has
resulted in concrete difficulties at the implementation level, the
abstractionist is prone to characterizing the situation as one of a well
conceived plan that has been fumbled in the execution. In other words,
they shoot the messenger, preferring to blame the implementation of
their decision rather than the decision itself, which is perfect -- as
long as it is only considered in the abstract.

\begin{enumerate}
\item Conclusion
\label{sec:orgheadline163}

Those who institute an architecture group in their organization may be
courting disaster. There is a strong tendency for the group to become
cliquish, divorced from the consequences of its decision making, and the
object of wide-spread resentment within the organization. Coordination
of projects and adherence to enterprise architectures should occur in a
way that does not impinge upon individual project's chances of success,
nor rob them of the ability to solve the particular problems of their
project in an effective way.
\end{enumerate}

\subsubsection{The Mismeasure of Man  \footnote{First published 6 Aug 2006 at
\url{http://www.hacknot.info/hacknot/action/showEntry?eid=88}}}
\label{sec:orgheadline178}

Software developers are drawn to metrics for a variety of reasons.
Generally, their motivations are good. They want to find out something
meaningful about the way their project is progressing or the way they
are doing their job. Managers are also drawn to metrication for a
variety of reasons, but their motives are not necessarily honorable.
Some managers view metrics as an instrument for getting more work out of
their team and detecting if they are slacking off.

Performance metrics -- metrics intended to quantify individual or group
performance -- can be useful if they are employed sensibly and in full
awareness of their limitations. Unfortunately, it is very common for
performance metrics to be gathered and interpreted in ways that are
ultimately harmful to a project and its developers. Many is the metrics
program that, through inept implementation and application, has
engendered anger and resentment amongst those it was intended to
benefit.

Below, we consider various performance metrics commonly encountered in
development environments, the ways they are abused, and illustrate their
misuse with some examples taken from my own experience and the
experience of others as they have related it to me.

\begin{enumerate}
\item The Number Of The Counting
\label{sec:orgheadline176}

\begin{enumerate}
\item Face Time
\label{sec:orgheadline165}

This is perhaps the most commonly abused "metric" in the software
development world. For reasons of both tradition and convenience, many
managers and developers alike persist in considering the number of hours
spent in front of the screen as being some indication of how devoted a
programmer is to their work. Those that work long hours are considered
"hard workers," those that keep regular hours are considered "clock
watchers."

The fault behind such thinking is the assumption that software
development is a manufacturing-like process, rather than a
problem-solving process. If a worker on a production line works an extra
hour then the result is an extra hours' worth of stuff. If they work an
extra three hours then the result is an extra three hours worth of
stuff; which will be exactly three times the quantity of extra stuff
they would've produced had they only worked a single extra hour. If
their role on the production line is menial assembly work, then the
quality of the stuff they produce in their third hour of overtime will
be the same as the quality of the work from their first hour of
overtime. In such an environment, it is reasonable to see productivity
as a direct function of time on the job.

But software development is nothing like this mechanistic process. It is
a complex, intellectual effort conducted by knowledge workers, not a
menial assembly task performed by laborers. So more hours spent in front
of the screen does not necessarily equate to more progress. For example,
long work hours might be a result of problems such as:

\begin{itemize}
\item Relying on trial and error rather than thinking ahead
\item Goofing off surfing the web or socializing
\item Solving the wrong problem, and having to start again
\item Gold-plating (extending scope beyond what is required, simply for the
satisfaction of it)
\item Using a lengthy, inefficient algorithm rather than a smaller, elegant
one
\item Writing functionality that should have been purchased in a third
party library
\item Making the solution more generic than is necessary
\item Poor understanding of the technologies employed, resulting in a lot
of thrashing
\item Losing a lot of time to debugging, because of the higher defect
injection rates that occur when working while fatigued
\item Overly ambitious scheduling resulting from poor self-insight and lack
of experience
\end{itemize}

So by expecting or encouraging long working hours, we may simply be
rewarding poor performance and inefficient work practices.

I first encountered the obsession with working hours at a small "dot
com" company I once had the misfortune to work for. Full of bright and
enthusiastic young people, the CTO of this company considered his stable
of go-getters a resource to be exploited to the fullest. Not being the
most technically aware of CTOs he was unable to assess the performance
of the technical staff that reported to him in any meaningful way, so he
was forced to rely on what he considered to be secondary indicators of
performance -- the number of weekly hours each employee logged in their
electronic time-sheet.

Those with more experience of his somewhat indirect approach to
assessment were quite generous when it came to such time-keeping tasks,
logging some spectacular hours -- some of which they actually worked.
Those unfamiliar with the man's chronological obsession, such as myself,
made the mistake of working efficiently and recording their work hours
accurately. This did not go down so well.

In my letter of resignation I cited unscrupulous and irrational
management practice as one of the principal reasons I was leaving. On my
last day at said company I received what is, to date, the only written
response to a resignation that I have ever encountered. The response
contained a month-by-month tabulation of average daily working hours --
both the company average and my personal figures. Of course, my
"performance metric" was disgustingly normal, whereas the company
averages seemed to indicate that many staff were dedicating all their
waking hours to work. The conclusion was obvious -- I was not putting in
the sort of effort that was expected of me. How right they were.

\item Lines Of Code
\label{sec:orgheadline166}

It should be common knowledge that lines of code (LOC) and non-comment
lines of code (NLOC) are not measures of size, productivity, complexity
or anything else particularly meaningful. It is none-the-less very
common to find them being used in the field to quantify exactly these
characteristics. This is probably because these metrics are so easily
gathered and there is an intuitive appeal to equating the amount of code
written with the amount of progress being made.

But it is a big mistake to consider large quantities of code necessarily
a good thing, for large volumes of code may also be symptomatic of
problematic development practices such as:

\begin{itemize}
\item Unnecessarily complex or generic design
\item Cut-and-paste reuse
\item Duplication of functionality
\end{itemize}

Large quantities of code can also bring such problems as:

\begin{itemize}
\item A greater opportunity for bugs
\item A greater maintenance burden
\item A greater testing effort
\item Poor performance
\end{itemize}

So by rewarding those who produce larger quantities of code, we may
simply be encouraging the production of a burdensome code base.

The story is told of a team of developers whose well-meaning but
uninformed manager decided that he would start measuring their
individual contributions to the code base by counting the number of
lines of code each of them wrote per week. Fancying himself as more
technically informed than most other middle managers, he wrote a simple
script to count the number of lines of code in a file.

The project was written in C. Figuring that most statements in C ended
in a semicolon, he presumed that his script could just count the number
of semicolons in the file and that would give him the number of C
statements. He congratulated himself on thinking of this clever counting
method, which would not be susceptible to differences in coding style
between developers, nor any of the techniques developers sometimes
employed to try and manipulate metrics in their favor by changing the
layout of their code.

However a few of the developers got wind of the technique their manager
was using, and started writing function comments containing long rows of
semicolons to delineate the beginning and end of the comment block.

Their measured rate of code production skyrocketed \ldots{} so much so that
their manager became suspicious and, looking at the code to manually
verify that his script was working correctly, discovered what was going
on. But the developers simply claimed that their recent change in
comment style was just an innocent search for greater code readability
The manager could not prove otherwise.

\item Function Points
\label{sec:orgheadline168}

In some circles, Function Points (FPs) have currency as a way of
measuring the size of a piece of software. There are complex counting
procedures that enable functionality to be expressed as a number of FPs
in an ostensibly language-independent way. The formation of the IFPUG
(International Function Point Users Group) and the amount of
semi-academic study they have received has invested FPs with a certain
amount of faux credibility. However, this credibility is undeserved, as
FPs are a fundamentally flawed metric. They are not a valid unit of
measurement, nor can they validly be manipulated mathematically. Any
metric involving them is approximately meaningless. FPs have been
discussed at length in a previous article \footnote{See \hyperref[sec:orgheadline167]{Function Points: Numerology for Software
Developers}}.

\item Screens
\label{sec:orgheadline169}

Having worked principally in the area of rich-client and desktop
applications, I've witnessed numerous mismeasures of progress from this
domain. The most foolish of them was to use a "screen" (dialog / window)
as a unit of measurement. Thus, if programmer A implemented two dialogs
in the time programmer B implemented one, A was considered to be twice
as productive as B.

The faults with such an approach are alarmingly obvious, but often
ignored by an unthinking management that is too impressed by the fact
that they can attach numbers to something, which creates a false
impression that they are measuring something. Such are the perils of
metrication in the hands of the ignorant.

To labor the obvious, here are a few reasons one programmer might
produce more "screens" than another, that have nothing to do with
productivity:

\begin{itemize}
\item Their screens were simpler in appearance and/or behavior.
\item Their screens were sufficiently similar in appearance and/or
behavior, so there could be code re-use between them.\\
\item Their screens could be constructed with standard GUI components,
without the need for custom components being developed.\\
\item Their screens were not the end result of a usability-based design
process, but were whatever was most programmatically expedient.
\end{itemize}

By counting "screens" as a measure of progress, we encourage programmers
to race through their tasks, giving short shrift to issues of usability
and reuse.

I once worked for a small firm in the finance industry. Their flagship
product was a client/server application for managing investment
portfolios. I was brought in, together with another GUI guy, to extend
the functionality of the system and clean up a few of the existing
screens and dialogs. Under the hood, this product was a disaster. Poorly
coded, undocumented and architecturally inconsistent, it was the end
result of the half-hearted, piece-meal hacking of many previous
generations of contractors.

The gentleman who had shepherded all these contractors through the
company doors, and who considered himself both Technical Lead and
Project Manager, was not heavily into software. Indeed, he never
actually bothered to look at the application's code. He had only one way
to gauge individual or collective progress and that was on the basis of
appearance. If a piece of work involved lots happening on the screen,
then he figured that it represented a lot of work. If it wasn't visually
significant, then he figured there probably wasn't much to it. Let's
call him Senior Idiot.

He and I did not get on so well, right from the start. I'm told I don't
suffer fools lightly and as fools go, this guy was an exceptional
specimen. My fellow GUI guy was no better. Examining the code that he
wrote and the work he delivered, it was clear he was working at a level
consistent with the noxious quality of the existing code base. Let's
call him Junior Idiot.

A few months after I started, Big Idiot took me aside and asked why my
progress was "so slow." I thought this was an interesting comment, given
that by my own analysis I was generating good quality code at a rate
several times the industry average. Both the code and the resulting
interfaces were some of the best they had in the entire, sorry product.
When I enquired how he had determined my progress was "slow" given that
he never actually looked at code, he explained that he was comparing the
"number of screens" Little Idiot had managed to grunt out, to what I had
developed in the same time. Little Idiot was some way in front.

He was correct. Little Idiot had produced several rather large screens
(large in the sense that they occupied many pixels, not in the sense
that they represented a lot of functionality). They were usability
disasters, every one of them, and the product of some pretty deft
cut-and-paste but, scatological in quality as they were, they were there
to be seen.

After some chuckling, I tried to carefully explain to him the
"discrepancy" that he saw was because Little Idiot was spitting out
rubbish as quickly as possible, and I was taking some time to do a
decent job. Additionally, Little Idiot was producing non-reusable code ,
whereas I was writing general purpose code, reuse of which would mean
that future work, both my own and others, would progress much more
quickly than Little Idiot could ever do. He was not convinced and my
time at this little company came to an end shortly thereafter, much to
our mutual relief.

\item Iterations
\label{sec:orgheadline170}

Unbelievable as it is, I can honestly say that I've seen entire projects
compared on the basis of what iteration they are up to in their
respective schedules. Suppose projects A and B both employ an iterative
methodology. A is in the third of five planned iterations, B is in the
fourth of seven planned iterations. Some observers may then conclude
that project A is behind project B because "three" is less than "four."
Others might conclude that project A is ahead of project B because it
has completed 60\% of its iterations and B only 57\%.

I recall the organization in which I first encountered this. A rather
hubristic, research oriented environment in which some very clever
people worked. Sadly, the quality of the management was not on a par
with the quality of the technical staff. As they say, "A fish rots from
the head down," so it was no surprise that the manager at the top was
not as clued up in many areas as one might like.

At this time, "data warehousing", "knowledge management", "project
cross-fertilization" and "knowledge repositories" were the buzzwords
that substituted for critical thought. Mashing all these concepts
together in his head, the top guy decided to establish a "project wall"
in the office, upon which the project managers were required to post the
Gantt charts for their respective projects, and keep them up to date.
This strategy was meant to promote some sort of comparison and knowledge
sharing between projects, although exactly how this was to be done
meaningfully was never quite made clear. The device became widely known
as "The Wall Of Shame", as that was its obvious but unstated purpose --
to publicly shame those managers whose projects were running behind
schedule. Presumably, the potential for embarrassment was meant to
encourage individual project's to maintain schedule.

It came as a surprise to no-one but the man who instituted the scheme,
that it had precisely no effect on anything, except to become the focus
of widespread derision.

\item Tasks / Bugs
\label{sec:orgheadline171}

Many software development teams allocate work to individuals on a
per-task basis. Typically, these tasks are tracked in some electronic
form -- perhaps as bugs in a bug tracking system or tickets in a trouble
ticket system. XP projects like to track tasks on pieces of card because
the arts-and-crafts association creates the illusion of simplicity (an
illusion which disappears when reports of any kind are required, or when
the first strong breeze comes along).

Regardless of the mechanism used, "the task" is so useful as a unit of
work allocation that it is very tempting and convenient to think of it
as a unit of measurement. Of course, it is not a unit of measurement, as
no two tasks are the same. A tiny, one-line bug fix might be captured as
one task, as might the implementation of an entire subsystem. The
granularity is ever-varying, making any mathematical comparison of task
counts meaningless.

But convenience outweighs reason and so one frequently finds,
particularly amongst the ranks of management, the tendency to equate
high rates of task completion with high productivity and effort, and
lower rates with lower productivity and effort. The mistake is so common
that developers become quite practiced at gaming the system to make
themselves look good. Common image enhancement techniques include:

\begin{itemize}
\item Breaking work down into unusually small tasks, thereby enabling a
greater number of tasks to be completed at a faster rate.\\
\item Registering tasks as completed before they have been properly tested.
This enables bugs to be readily found in the work, each of which will
be considered a separate task. These tasks can be completed
relatively quickly because the programmer is familiar with the code
at fault, having just written it.\\
\item Registering tasks multiple times, describing it in slightly different
ways each time. Once completed, all the tasks are closed, with all
but one marked as duplicates. If the management forgets to exclude
duplicate tasks from their reporting, the programmer's rate of task
completion is artificially inflated. He might also "forget" to mark
some of the duplicate tasks as being duplicates, to further enhance
the effect.
\item When a task is found to be more involved than originally thought,
rather than revise the scope of the existing task, new tasks are
spawned to capture the unanticipated work. Their eventual completion
will mean that the number of "completed" tasks registered against the
programmer's name is greater.\\
\item When selecting work to do, programmers gravitate towards the short
tasks which can be easily dispensed with, enabling them to quickly
get runs on the board.
\end{itemize}

When invalid metrics are gathered, the result is often to contort the
team member's work practice so as to create the best perceived
performance, regardless of what their actual performance might be.

A colleague once related to me the story of two teams of developers in a
multinational company who reported to the same manager. One team
contained three developers working mainly on maintenance tasks,
documentation and bug fixing. The other, containing six developers,
worked on per-client product customizations. Both happened to use a
common issue tracking system.

A developer from the smaller team complained to the manager about the
discrepancy in work loads between the two teams. He felt that his own
team was dreadfully overburdened while the larger one just seemed to be
taking it easy. Although uncertain that the developer's complaint was
valid, the manager felt compelled to "handle" the situation in a
managerial kind of way. Turning to the issue tracking system he did a
few simple queries and discovered that the small team was closing issues
at nearly twice the rate of the larger team. This struck him as
confirmation of the developer's complaint. After all, a team twice as
large should be getting through issues much faster than a team half its
size.

So the manager sent an e-mail to all members of both teams, and CC'd the
general manager. In this e-mail he highlighted the discrepancy in issue
closure rate for the two teams, chastised the larger team for slacking
off and praised the smaller team for their hard work.

The original complainant was suitably appeased, but the other members of
his team, along with the entirety of the larger team, were not quite so
happy. The following day, the leader of the larger team came to the
managers office and explained to him, in a tone of barely suppressed
hostility, that the two teams worked on completely different sized
issues, and so comparing issue closure rates across the two was quite
meaningless. The smaller team addressed issues that could generally be
resolved in a single day, two days at the most, and so naturally they
got through them at a fairly rapid pace. His team, the larger one,
addressed implementation issues that might legitimately involve weeks of
effort, including design, requirements gathering and testing. He was
more than a little offended that his hard working team was being
reprimanded on such an irrational basis.

The manager admitted his error -- but of course, never apologized to
those he had offended.

\item Version Control Operations
\label{sec:orgheadline172}

Astonishing as it may seem, some developers like to commit changes to
their version control system frequently to create the impression that
they are hard at work. This only works if you are managed by the
technically incompetent. In other words, it works more frequently than
you would like.

\item Requirements Completed
\label{sec:orgheadline173}

Regardless of whether you capture your requirements in tabular, use case
or story card format, individual requirements make spectacularly bad
units of measurement.

Consider the enormous variation in scope that can exist between one
requirement and another. "The user shall not be able to enter an age
greater than 120 or less than 0" counts as "one requirement"; so does
"The system shall reserve the section of track for the given vehicle in
accordance with safe-working procedure SP-105A." But the latter is
probably a far greater undertaking than the former, and we would expect
it to take significantly more time and effort to complete. Pity the
developer who is assigned the task of satisfying this requirement, only
to have his labors viewed as an achievement "equal" to that of his
colleague who was assigned the simpler age-related requirement.

\item Noise Generated
\label{sec:orgheadline174}

Some programmers just get the job done. Others seem to find it necessary
to let others know that they are getting the job done. You've probably
met the type before. Every little obstacle and difficulty they encounter
seems to be a major drama to them -- almost a theatric opportunity.
These are the same programmers who will work overtime to fix problems of
their own creation, then seek credit for the extra hours they've put in.
Although there is no number associated with their vociferations, they
effectively multiply the amount of perceived work they are doing, and
inflate the perceived effort they are making by drawing attention to
their actions.

I once worked with such a programmer. He was a hacker of the first
order; and I use the word "hacker" in the pejorative sense. Each day
over the lunch room table he would regale us with stories of his mighty
development efforts, the technical heights to which he had scaled, and
the complex obstacles he had overcome -- all of these adventures
apparently having happened since the previous day's story-telling
episode. But when you actually looked in the source code for evidence of
these mighty exploits, you would find only an amateurish and confused
mess, and be left wondering how so much difficulty could have been
encountered in the achievement of such modest results.

\item Pages Of Documentation
\label{sec:orgheadline175}

Used intelligently, documentation makes a useful component of the
development process. But when seen as an end in itself, documentation
becomes a time-consuming ritual for comforting self-serving
administration. Strange then that we should so frequently see, most
often in heavily bureaucratic environments, people striving to generate
technical specifications that are as voluminous as possible, apparently
fearing that brevity will be interpreted as evidence of laziness. A page
fails to measure either effort or progress for all the same reasons that
"Lines of Code" fails. Stylistic variations mean there is little
relationship between volume of text and effective communication as there
is between volume of code and functionality.
\end{enumerate}

\item Conclusion
\label{sec:orgheadline177}

In the above you will have noticed the same problems occurring again and
again. All these scenarios reflect a poor understanding of the basics of
measurement theory, together with a willingness to rationalize a
metric's invalidity because of the ease with which it can be collected.

Essentially, a valid unit of measurement is a way of consistently
dividing some real world quantity into a linear scale. In other words, X
is a valid unit of measurement if X is half as much of something real as
2X is, one third as much of something real as 3X, and so on. For this to
be true, all instances of X must be the same. For example, the "meter"
is a valid unit of measurement because 2 meters is twice the linear
distance of 1 meter, and all instances of the "meter" are the same. The
"1 meter" that exists between the 0 and "1 meter" marks on your tape
measure is the same quantity of something real as the "1 meter" between
the "4 meters" and "5 meters" marks. Compare this to an invalid metric
like a "task." A task doesn't divide any real world quantity into equal
portions. In particular, it doesn't divide effort or work into equal
portions, because different tasks might require different amounts of
work to complete. So "2 tasks" is not twice "1 task" in any meaningful
sense. Put more simply, when comparing tasks, you're not comparing like
with like.

The attraction to metrics, even false ones, perhaps stems from the false
sense of control they offer. Once we pin a number on something, we feel
that we know something about it, that we can manipulate it
mathematically, and that we can make comparisons with it. But these
statements are only true for valid metrics. For false metrics like bugs,
tasks, function points, pages, lines of code, iterations etc., we create
only the illusion of knowledge. The illusion may be comforting,
particularly to those of an analytical bent, but it is also an
invitation to misinterpretation and false conclusions.

We might try and rationalize these invalid metrics, figuring that they
may not be perfect, but they are "close enough" to still have some
significance. But really this is just wishful thinking. You might think,
"our tasks may not be exactly the same, but they're close enough in
scope that 'tasks completed' still means something." Really? What
evidence do you have that these tasks are of approximately equal scope?
If you're honest with yourself, you'll find you've got nothing more than
gut feel to justify that statement. Yet the very reason we use metrics
is to obtain greater surety than that provided by gut feel. So we see we
are really just trying to convince ourselves that our own guesswork can
be somehow made better by hiding it behind a number -- borrowing the
credibility often associated with quantification.

Metrics are a tool easily abused. A common cause of mismeasurement is
their punitive application with the intent of motivating higher
productivity. In their zeal to find some way to meet a deadline,
managers sometimes sacrifice reason for expediency, hoping that some
hastily contrived metric can be used to convince someone that they need
to be working harder. Of course, such tactics frequently backfire,
resulting only in developers feeling resentful of such numeric bullying.
\end{enumerate}

\subsubsection{Meeting Driven Development  \footnote{First published 30 Mar 2006 at
\url{http://www.hacknot.info/hacknot/action/showEntry?eid=84}}}
\label{sec:orgheadline188}

The software development arena is the land of the perpetual "me too."
Populated by an eager community of "joiners," every band wagon that
comes along is soon laden down by a collection of hype merchants who,
recognizing the next big thing when they see it, are keen to milk it for
all it is worth. Extreme Programming -- that marketing campaign in
search of a product -- was a particularly fruitful source of commercial
spin-offs. When Extreme Testing, Extreme Database Design, Extreme
Debugging and Extreme Project Management had run their course; when XP's
agile prequel had fostered a small industry based on old saws spruced up
with a few neologisms; those looking to make a name for themselves
turned to another member of the XP franchise -- Test Driven Development
-- for entrepreneurial inspiration.

\begin{enumerate}
\item TDD: The Progenitor Of MDD
\label{sec:orgheadline179}

If you have not read Kent Beck's insufferable tome "Test Driven
Development,” \footnote{\emph{Test Driven Development}, Kent Beck, Addison Wesley, 2003} let me spare you the time and insult by presenting
the expurgated version here:

\begin{quote}
/Hello boys and girls. Once upon a time there was a thing called Test
Driven Development -- it looked for all the world like an impoverished
rendering of Design by Contract \footnote{\emph{Object Oriented Software Construction, 2nd Ed., Ch 11}, Bertrand
Meyer, Prentice Hall, 1997} only much cooler./
\end{quote}

The ditto brigade latched onto TDD and got to work. We soon had,
sprouting like weeds from between the pavement stones, "\emph{Blah} Driven
Development", for all conceivable values of Blah. It became \emph{de rigueur}
to have something driven by something else. Not since Djikstra's "Goto
Statement Considered Harmful" had there been such a rash of imitation.

The appeal of such development models is in the simplistic and
unrealistic view that a complex activity can be reduced to consideration
of, or focus upon, a single factor. But software development is an
inherently multivariate process requiring intelligent compromise between
competing forces. Unfortunately, such a view is hard to sell.

The fantasy is more appealing \ldots{} focus on \emph{blah}, make it the basis of
your development effort, and the rest will fall into place as a natural
consequence. If you can convince yourself that \emph{blah} is analogous to a
set of requirements or an abstract model then you can also dispense with
the unpleasantness of requirements elicitation and design. With
sufficiently zealous adherence to \emph{BlahDD}, combined with a healthy dose
of metaphor and supposition, the formerly complex and uncertain
undertaking of developing a piece of software turns into the routine
application of a silver bullet. Or so some would have you believe.

Such "one stop" philosophies are a recipe for disappointment, but will
no doubt continue to sell well, for the same reasons that "get rich
quick" and "lose weight fast" schemes do -- the promise of an easy fix.

To show how it's done and perhaps make an obtuse point or two, let's
look at the latest blah to exhibit in the software development road show
-- Meeting Driven Development.

\item An Introduction To MDD
\label{sec:orgheadline180}

MDD is more than an approach to software development, it is a cultural
force. If you're lucky, you are already working in an environment
conducive to the meeting mindset. In some corporate cultures meetings
are so endemic that they have become an integral part of the corporate
identity. For example, an IBM insider tells me that most staff consider
IBM to stand for "I've Been to Meetings".

If your corporate culture is not so amenable to MDD, do not despair. You
can surreptitiously introduce it into your project without much effort
and when others see how successful you have been, it will quickly spread
through the rest of your organization like a virus.

I suggest you begin by creating a localized "meeting zone" in your
project area. Put a table and some chairs right in the middle of your
project's work area, so that project staff need only turn their chairs
around and wheel them a short distance in order to assume the meeting
position. You will enjoy the disgruntled mutterings of nearby
programmers as they struggle to concentrate amidst the noise such
meetings create.

The only practical skill MDD entails is the ability to recognize and
achieve \emph{meeting mode}. Meeting mode is the colloquial name for what is
more properly known as \emph{corporate catatonia} -- the mental state
achieved by those meeting attendees who cannot or will not participate,
instead turning their attention inward. MDD veterans describe the state
as being peaceful, meditative and excruciatingly dull. Some claim to
have undergone "Out of Body Corporate" experiences while in deep states
of meeting mode, during which they separate from their physical bodies,
leave the meeting room and go on annual leave.

External indications that an MDD practitioner is in meeting mode
include:

\begin{itemize}
\item Vacant staring into the middle distance.
\item Methodical doodling upon note paper.
\item Slowing or cessation of respiration.
\item Extended periods of silence.
\end{itemize}

\item Types Of Meetings
\label{sec:orgheadline186}

In MDD, we encourage the use of meetings at every opportunity and for
every purpose. Our motto is "Every Meeting Is a Good Meeting". While you
can hold a meeting for almost any purpose that comes to mind, there are
certain types of meetings that tend to feature commonly in software
development environments. It is important that you develop some facility
with each of them.

\begin{enumerate}
\item Type \#1: The Morning Stand-Up Meeting
\label{sec:orgheadline181}

You should begin the day with a team meeting, and in this respect MDD is
in agreement with XP's practice of holding daily "stand-up" meetings.
Like many meetings that are driven by the calendar rather than by a
need, your morning meeting will probably devolve into a pointless ritual
that serves only to give the organizer a sense of control and influence.
For those desperately trying to fulfill a management or leadership role,
but lacking the basic proclivities that such roles demand, these
ritualistic meetings can also help sustain their delusions of
competence, as holding and attending meetings seems like a very
managerial thing to do.

\item Type \#2: The Requirements Meeting
\label{sec:orgheadline182}

A typical requirements meeting involves some technical staff and
stakeholders sitting down to discuss the functional requirements for a
unit of work. If there are any questions concerning requirements
previously elicited, they are tabled here. It is a chance for potential
users to lobby technical staff and their managers for the inclusion of
their favorite features. However, developers and domain specialists
speak different languages, have different priorities and widely
disparate agendas. The developers want to cut scope down to the minimum
that will be functionally adequate so they will have some chance of
meeting the schedules imposed upon them; potential users want an
application that will make their working lives as easy as possible.

The tension between these two forces inevitably brings an adversarial
dynamic to requirements meetings that can be very entertaining. Domain
experts can take the opportunity to express their resentment at the
developer's intrusion into their domain and to laugh at the folly of the
developer's attempts to capture the expertise and judgment acquired in a
lifetime's professional endeavor in a few minutes of discussion. In
turn, developers can mock the stakeholders for their lack of technical
knowledge, their inability to express their know-how in a succinct and
consistent manner, and to proclaim requests for even simple
functionality as being impossible to implement for technical reasons
that would take too long to go into.

\item Type \#3: The Technical Meeting
\label{sec:orgheadline183}

MDD prescribes that all technical problems be solved "by committee". The
basic method is:

\begin{enumerate}
\item Select a group of techies having maximum variation in technical
opinion and preferences.
\item Put said techies together in a meeting room.
\item Direct them to reach consensus on the "best" solution to the
technical problem.
\item Observe resultant fireworks and carnage.
\end{enumerate}

MDD practitioners are not afraid to thrash out all technical issues
amongst themselves, comparing the merits of varying approaches in an
unstructured session of verbal sparring. As with many meeting-based
outcomes, the determining factor is the relative rhetorical skill or
obstinacy of the protagonists. Victory goes to whoever can best "ad lib"
an argument to support their proposition, rather than whoever actually
proposes the best solution.

Of course, there may not even be a "best" solution to the problem. It's
likely there will only be a set of alternatives having different
strengths and weakness. You'll find that if you let the fighting go on
for long enough, eventually a compromise emerges that nobody is happy
with, but which they will settle for simply for the sake of having the
issue done with and getting out of the tense meeting room. This is how
MDD forces issues to resolution -- by escalating tension until it
becomes unbearable.

From a technical lead's perspective, the MDD approach to design is also
an excellent way to disguise your own incompetence. If you're in over
your head in some technical arena, delegating all decisions to a meeting
enables you to hide your lack of understanding and appear egalitarian at
the same time. When the resulting design is implemented and found to be
inadequate, the blame is spread amongst all the meeting participants
rather than being focused upon yourself. It's a win-win situation for
you.

The real magic of meetings is that they are like mini-corporations. Just
as shareholders enjoy limited liability for the failure and misdeeds of
the corporation, meeting participants enjoy a limited liability for the
mistaken outcomes of the meeting. The meeting becomes an artificial
entity unto itself; an additional, synthetic developer who is always
willing to take the blame when something goes wrong.

\item Type\#4: The Progress Meeting
\label{sec:orgheadline184}

Progress meetings are at once the most uneventful and easiest to
institute type of meeting. Their ostensible purpose is for team members
to gather together and somehow collectively "update" their mutual
awareness of the state of the project. Their real purposes are both
symbolic and exculpatory. They provide an opportunity for the meeting
organizer to give themselves the impression of active involvement with a
project (even though they may see little of the team or its work at any
other time), and also provide a way for the "hands off" manager to find
out what is going on with their own project.

The most ineffective types of progress meetings are structured like
this:

\begin{enumerate}
\item A chairman, usually the person who convened the meeting, reads
through the action items from the previous progress meeting.\\
\item The assignee of each action item offers some excuse as to why they
haven't attended to it, and then makes some vague resolution to do it
before the next progress meeting.\\
\item The chairman reads out any new agenda items.
\item Each new agenda item is turned into a new action item and assigned to
one of the meeting attendants, who promptly forgets about it.\\
\item The meeting is dismissed and the chairman writes up the minutes of
the meeting and distributes them to the participants, who ignore
them.
\end{enumerate}

For most of the meeting then, there is only oneway communication from a
speaker to a group of disinterested listeners. The same effect could be
achieved through judicious use of a text-to-speech engine and Valium.

But there is great power hidden behind this apparently meaningless
ritual. The chairman, in later distributing the minutes of the meeting,
is in a position to engage in some historical revisionism. The minutes
are supposed to detail the activities of the meeting and the decisions
reached. But the one writing the minutes can generally write anything
that they want, safe in the knowledge that hardly anyone will actually
bother to read them. So if a decision doesn't go your way in the
meeting, just change the way it is recorded in the minutes. You can even
introduce items that were never discussed in the meeting, together with
your preferred outcomes, safe in the knowledge that any participant who
reads such an item but can't remember it from the meeting will probably
conclude that they must have fallen asleep or been otherwise distracted
during that part of the proceedings. Their unwillingness to admit their
inattention means that your fabricated version of events will go
unchallenged. The minutes are also invaluable for assigning blame when
trouble occurs, as they can be used to substantiate claims that a
particular resolution was arrived at with the agreement of all parties
present (remembering that many will choose not to say anything at these
meetings, lest they end up with work assigned to them, But their silence
will forever condemn them to having offered implicit support for any
decision you chose to put into the minutes).

Should the more rational members of the gathering ever object that these
progress meetings seem pointless, you can always justify them by
pointing out that they are an opportunity for communication to occur,
and that communication is good. The complainant will be hard pressed to
argue that communication is bad, and your point is won.

\item Type \#5: Review Meetings
\label{sec:orgheadline185}

Technical artifacts should always be reviewed by a group, for the
practice offers numerous advantages \ldots{} to the reviewers, not the author
of the work being reviewed. Reviews are a good opportunity to gang up on
your enemies and humiliate them in front of an audience. Developers have
a notoriously strong ego investment in their work, so tearing apart the
finely tuned code they have been poring over for weeks is sure to
provoke an interesting reaction. This is the principle goal of group
code reviews. The reviewers function like a self-appointed council of
inquisitors looking for evidence of witchcraft in the accused. And like
a witchcraft trial, incriminating evidence can always be found, as few
developers can write code or produce a design that cannot be criticized
in some way for something. Review meetings also allow individuals to
find fault with impunity, as any degree of pettiness or vindictiveness
they might exhibit can be excused as a diligent attempt to make
constructive criticism.

Once you can conduct all of the above types of meetings, and enter
meeting mode at will, you may consider yourself a competent MDD
practitioner.
\end{enumerate}

\item Conclusion
\label{sec:orgheadline187}

So that's a brief overview of the magic that is Meeting Driven
Development. This approach to software development has been around since
the beginning of corporate activity in the programming arena. In many
corporations, the developmental norm is indistinguishable from MDD.
Meetings are so much a part of the corporate culture it would not occur
to anyone to take any other approach.

You will find that many programmers are afraid of meetings, having come
to view them as pointless, "busy work" activities. This is simply
because they have not yet learnt to appreciate that futility is actually
a strength of meetings, not a weakness. The ability to convincingly
create the illusion of coordinated effort and activity is invaluable in
many situations.

Meetings are not a knee-jerk reaction to problem solving as some
suggest, but a vehicle for creating a synthetic corporate entity -- a
virtual member of the development team -- that can adopt the
responsibility for the participant's poor decision making and manifest
inabilities. Only when they have abandoned their reflexive animosity
towards meetings and recognized them for the ritual scapegoat that they
are, can developers really appreciate the benefits of MDD.
\end{enumerate}

\subsection{Extreme Programming and Agile Methods}
\label{sec:orgheadline223}
\subsubsection{Extreme Deprogramming  \footnote{First published 29 Jul 2003 at
\url{http://www.hacknot.info/hacknot/action/showEntry?eid=11}}}
\label{sec:orgheadline197}

In recent weeks I've read two books by cult survivors. The first,
"Inside Out” by Alexandra Stein \footnote{\emph{Inside Out}, Alexandria Stein, North Star Press, 2002}, describes her ten year embroilment
in a Minneapolis political cult called “The O.” The second, "Seductive
Poison" by Deborah Layton \footnote{\emph{Seductive Poison}, Deborah Layton, Anchor Books, 1999}, details the author's involvement with
the “Peoples Temple,” the religious cult lead by Jim Jones, who
engineered the mass suicide of 900 of his followers in 1978.

Reading each I became aware of the similarities in the methods for
control, manipulation and persuasion that both cults employed. It also
occurred to me that those techniques were not just features of groups
that would conform to the traditional definition of a cult, but also
extended to what might be called benign cults. Think of the fierce
loyalty of members of pyramid organizations such as Amway and Mary Kay;
think of brands with a loyal consumer base like Apple and Harley
Davidson \footnote{\emph{The Power of Cult Branding}, M. Ragas and B. Bueno, Prima
Publishing, 2002}; and finally, think of the ardent supporters of Extreme
Programming.

By examining some of the characteristic features of cults (benign and
otherwise) and calling out their presence in the recently popular XP
movement, I hope to throw some light on why this technical cult incites
such fervor and emotion in certain members of the development community.

Drawing on the work of thought reform specialist Robert Lifton and
others, consider the following characteristics of a cult, all of which
are displayed by XP:

\begin{itemize}
\item Sense of higher purpose
\item Loaded language
\item Creation of an exclusive community
\item Persuasive leadership
\item Revisionism
\item Aura of sacred science
\end{itemize}

\begin{enumerate}
\item Sense Of Higher Purpose
\label{sec:orgheadline190}

\begin{quote}
/Cult members believe that they are privy to special truths and
insights not known to the general community, and that it is their
mission to spread this knowledge to others./
\end{quote}

I could only laugh when I read Scott Ambler's response \footnote{\emph{Software Development}, July 2003} to a letter
taking issue with an article on outsourcing that he wrote for Software
Development magazine. In the July 2003 issue he wrote "While it's nice
that so many Indian companies have high CMM ratings, it doesn't reflect
modern thinking about software development. CMM and Six Sigma have a
tendency to lead to prescriptive, documentation-heavy processes." These
are the words of a zealot, who is so convinced of the righteousness of
his beliefs that he is willing to elevate them to the status of being
representative of "modern thinking about software development." In
unguarded moments, it is occasionally conceded that XP is not the answer
to all software development problems, but that is certainly the attitude
portrayed by many of its devotees. Spend any time reading
\texttt{comp.software.extreme-programming} and you will not be able to help but
notice the thinly veiled arrogance and elitist attitude behind the
postings of many of XP's most zealous followers. This is definitely a
group of people who think they have \emph{got it}, and that anyone else not
similarly enthused is a laggard.

\item Loaded Language
\label{sec:orgheadline191}

\begin{quote}
/Cults create a custom vocabulary for their members. New words are
invented, existing words are redefined, and a jargon of trite and pat
clichés is developed./
\end{quote}

Perhaps XP's most egregious effect on the broader software development
community has been to infect communication with cutesy slogans and
acronyms. No one could overlook the overuse the word "extreme" has been
put to in the marketing of a host of unrelated products and concepts.
The only common meaning amongst Extreme Programming, Extreme Project
Management, Extreme Design and Extreme Testing is the implication of
identifying a product that is sufficiently different from previous
offerings to warrant purchase.

"Refactoring" has been abducted from its proper home in the algebraic
texts and elevated to the status of an essential work method, which one
must apply "ruthlessly." If we consider that "rework" or "restructuring"
are essentially synonyms for "refactoring", we see that this piece of
custom terminology is only dignifying the act of investing effort to
correct ill-considered implementation decisions for no functional gain.
In general usage, I have noticed the term being used as an even broader
euphemism to disguise and minimize bug fixing and functional extension.

Particularly offensive is the frequent characterization of XP as
"disciplined". XP may satisfy the weakest definitions of the word
"disciplined" in so far as there is some regularity and control in its
methods. But these minor concessions to true rigor are in fact just the
leftovers remaining after the elimination of particular activities from
a truly disciplined development process -- one that includes formal
documentation and design. The abandonment of these activities is
precisely where XP's principal appeal to many lies -- that there are
fragments of a rigorous development process remaining after the
unpleasant stuff has been cast aside is hardly sufficient basis upon
which to claim that the overall work pattern exhibits discipline --
unless one considers the determined pursuit of the path of least
resistance to evidence discipline.

The XP jargon serves the same purpose as it does in any cult, to elevate
the mundane to the significant through relabelling, and to misdirect
attention away from failings and inconsistencies in the cult's value
system. It is a shame that the XP community did not apply its own YAGNI
(You Ain't Gonna Need It) principle to the invention of such novel
terminology.

\item Creation Of An Exclusive Community
\label{sec:orgheadline192}

\begin{quote}
\emph{A cult provides a surrogate family for its members, who feel somehow
separated and at odds with mainstream society.}
\end{quote}

Cults are a refuge for the uncertain. For those feeling lost or without
direction, the faux certainty of a cult provides welcome relief.
Software development is a field full of uncertainty. The increasing
societal reliance upon software and the attendant but conflicting
requirements for speedy and reliable development, has outpaced our
ability to learn better ways to do our work. Faced with this
unsatisfactory situation and desperate for a solution, the development
community is vulnerable to the claims and promises made by XP. The fact
that there is a community of enthusiastic proponents behind XP serves
only to enhance its credibility via the principle of \emph{social
proof} \footnote{\emph{Influence: The Psychology of Persuasion}, Robert Cialdini, Quill,
1993}. In truth, the presence of such a community only evidences
the widespread confusion about software development methods, coupled
with the hope that there is some answer that doesn't entail unpleasant
activities such as documentation.

\item Persuasive Leadership
\label{sec:orgheadline193}

\begin{quote}
/Central to almost all cults is the founding member, a figure who
through the strength of their own conviction is able to attract others
to their cause./
\end{quote}

The leaders of the XP movement are three members of the C3 project where
XP was piloted -- Kent Beck, Ron Jeffries and Ward Cunningham -- and to
a lesser extent the industry figures who have adopted it as their
personal cause -- Scott Ambler and Martin Fowler being amongst these.
These people have generated an impressive amount of literature which
forms the basis for the ever growing XP canon. They also serve as the XP
community's ultimate arbiters of policy and direction. Reading the
\texttt{comp.software.extreme-programming} newsgroup I notice people
continually directing questions about their own interpretations of the
XP doctrine to these central figures, seeking their approval and the
authority of their advice. That there is a need for personal
consultation in addition to the information provided by the large amount
of literature on XP speaks of the imprecise and variable definition of
the subtleties of XP practice. That knowledge of what is and isn't OK is
seen to be held by a central authority and is not in the hands of the
practitioners themselves, echoes the authoritarian distribution of
sacred knowledge that is present in most cults.

\item Revisionism
\label{sec:orgheadline194}

\begin{quote}
\emph{Cults often craft alternative interpretations of world events, both
present and historical, that serve to reinforce their belief system.}
\end{quote}

There are a number of examples of revisionism in XP. The most blatant
concern the C3 project -- the original breeding ground for XP.
Proponents of XP repeatedly use this project as their poster child, the
tacit claim being that its success is evidence of the validity of XP.
However the reality is that the C3 project was a failure -- ultimately
being abandoned by the project sponsor and replaced with an
off-the-shelf solution \footnote{\emph{Extreme Programming Refactored}, M. Stephens and D. Rosenberg,
Apress, 2003}. XP advocates have chosen to cast this
failure as a success, by carefully defining the criteria for success
that they claim is relevant. It is typical cult behavior to interpret
real world events in a light that confirms existing beliefs, and to deny
contrary evidence as being inauthentic.

One of the advantages of having a central authority is the ability to
reconceive fundamental beliefs when necessary. The change in the
attitude of the XP "inner circle" with regard to the production of
documentation is an example of this. In its initial conception,
documentation was regarded as unnecessary. In the light of real world
experiences with XP, this stance softened to include the production of
documentation "if you are required to." More recently, the philosophy
has been stated as "if it's valuable to you, do it." Some would dismiss
this as a result of XP's infancy, claiming that it is still being
developed and refined; but I believe these shifts in position are the
thought reformer's attempts to incorporate unflattering real world
experience into their original ideation. Whatever real practitioner's
experiences are, we can be sure that the primacy of XP doctrine will
remain.

\item Aura Of Sacred Science
\label{sec:orgheadline195}

\begin{quote}
\emph{Which implies that the laws and tenets of the cult are beyond
question.}
\end{quote}

Central to XP is the notion of the 12 core practices. These technical
equivalents of the Ten Commandments are considered interdependent and so
the removal of any one of them is likely to cause the collapse of the
whole. This all-or-nothing thinking is typical of cults. Members must
display total dedication to the cult and its objectives, or they are
labeled impure and expelled from the community. This discourages members
from questioning the cult's fundamental beliefs.

In the case of XP, the organizational circumstances required to perform
all the core practices are so particular that it is doubtful if more
than a handful of companies could ever host an authentic XP project.
Therefore practitioners are forced to perform partial implementations of
XP. If they are unsuccessful, then failure is attributed to the impurity
of their implementation rather than any failing or infeasibility of XP
itself. The quest for individual purity is a feature common to many
cults, as is the contrivance of circumstances that render it ultimately
unachievable.

Much is made of the "humanity" of the methodology, the transition from
"journeyman" to "master", and the focus upon individual qualities and
contributions. Consideration of these softer, cultural aspects of XP has
devolved into the sort of pseudoscience we often find in new age cults
centered on the notion of "personal power" and "personal growth". To
quote one zealot "XP is a culture, not a method." \footnote{\emph{Enculturating Extreme Programmers}, David M. West} The elevation of
a new and unproven methodology to the philosophical status of a Zen-like
belief system demonstrates the skewed perspective that typifies cult
mentality.

\item Conclusion
\label{sec:orgheadline196}

Whether you choose to label XP a cult is not as important as whether you
recognize that it displays cult-like attributes. I believe that the
psychological and social phenomenon underlying these six characteristics
account in no small part for the current popularity that XP enjoys. I
also believe that they point to its future.

Cults tend to have a very limited life. The hype and fervor can only
sustain the devotion of the members for so long, and eventually they
will look to other sources for inspiration -- those leaving a cult are
frequently drawn into another within a short time.

I believe that XP will eventually lose its luster and fall into
disrepute like so many other religious, commercial and technical cults
of the past. Many of the current adherents will cast about for a new
cause to follow, and no doubt the marketing departments of the technical
book publishers and software vendors will be only too happy to provide
them with a new subject upon which to focus their devotion. Meanwhile,
software projects will continue to fail or succeed with the same
frequency as always, as our industry continues its search for a panacea
to the ills of software development.
\end{enumerate}

\subsubsection{New Methodologies or New Age Methodologies?  \footnote{First published 10 Nov 2003 at
\url{http://www.hacknot.info/hacknot/action/showEntry?eid=34}}}
\label{sec:orgheadline204}

I first encountered the coincidence of the aesthetic and the technical
in a secondary school mathematics class. After leading the class through
an algebraic proof, my teacher said "You have to admit there's a certain
beauty to that." As I recall, he was met by a room of blank stares, one
of which was my own. I remember thinking "You sad, sad man." I really
couldn't see how a mathematical proof could be called "beautiful".
Beauty was an attribute reserved for the arts -- a song could be
beautiful, a painting could be beautiful, but a mathematical proof might
at best be called "ingenious."

It wasn't until some years later at University, while studying data
structures and algorithms that I would come to some appreciation of what
my mathematics teacher had meant. An appreciation of certain algorithms
would leave me with a smile on my face, and an ineffable feeling of
satisfaction. I believe that to appreciate the beauty of something
technical first requires the observer to care a lot about the subject at
hand, and that what we experience has something to do with a sense of
admiration for the mind that produced the thing, rather than the thing
itself.

That it is possible to appreciate the technical in an aesthetic way is a
realization that I suspect comes to many people after spending long
enough in a particular technical field. But that aesthetic is a quality
of an existing artifact, not a basis for its production. The sense of
"rightness" that we associate with an elegant solution to a problem is
the end result of a rather less romantic, technical struggle. It is not
the starting point for that struggle, but rather a flag that indicates
that we have arrived at a good resolution.

\begin{enumerate}
\item The New Age Methodologies
\label{sec:orgheadline198}

One of the more disturbing characteristics of the New Methodologies of
software development is the tendency to impose a new aesthetic upon
existing knowledge, and then interpret that aesthetic as evidence that
something new has been discovered. Hence, we find the literature of the
New Methodologies littered with references to Zen philosophy,
craftsmanship, martial arts and personal empowerment. This is the stuff
of pseudo-science and mysticism. By indulging in this sort of "discovery
by metaphor," we risk descending into a stasis of vague,
self-referential navel gazing that characterizes the delusional New Age
movement.

In the following sections I look at a number of the software development
metaphors that recent authors have proposed as a means of gaining
insight into the software development process.

\item Personal Empowerment
\label{sec:orgheadline199}

The New Methodologies purport to be more focused on people than on
process. This is often construed as empowering the programmers against a
harsh and dictatorial management. The New Methodologies have values and
principles at their foundation, on an equal footing with actual
techniques and practices. Commonly touted values are communication,
simplicity, feedback, courage and humility. No doubt these are
worthwhile values, not only in software development but in practically
every other field of human endeavor. So why would we chose to focus on
these values particularly, and their relationship to software
development? Perhaps the biggest effect of highlighting this arbitrary
selection of values is to add a certain faux credibility to a
methodology by associating it with noble concepts.

The irony of the "empowerment" message is that the vagueness of this
values-based approach actually has the opposite effect -- it disempowers
the programmer. The power is placed instead in the hands of the
methodologists, who must be consulted as to what the appropriate
interpretation of these values is, in the situations the programmers
actually encounter in the field. These spokesmen have become moral
arbiters. A more precise and objective methodological foundation would
empower individuals to unambiguously interpret the methodology's
recommendations in their local environment, without the need to
continuously seek clarification from the methodologists.

For more rational discussion of the predilections and working habits of
software developers see:

\begin{itemize}
\item "\emph{The Psychology of Computer Programming}" by Gerald Weinberg
\item "\emph{Peopleware}" by Tom DeMarco and Timothy Lister
\item "\emph{Constantine on Peopleware}" by Larry Constantine
\item "\emph{Understanding the Professional Programmer}" by Gerald Weinberg
\end{itemize}

\item Eastern Mysticism
\label{sec:orgheadline200}

Nowhere do the New Methodologies and the New Age movement intersect to
more egregious effect than in the area of Zen philosophy. In an attempt
to elevate the ordinary to the profound, or to disguise
self-contradiction as sagacity, the New Methodologists will often invoke
the inexplicable wisdom of Zen.

In the new edition of "Agile Software Development", Alistair Cockburn
offers us this:

\begin{quote}
"/It is paradoxical, because it is not the case, and at the same time
it is very much the case, that software development is mathematical
\ldots{} engineering \ldots{} craft \ldots{} a mystical act of creation/".
\end{quote}

Worse yet, this obfuscating nonsense is later followed by:

\begin{quote}
"\emph{The trouble with using engineering as a reference is that we, as a
community, don't know what that means.}"
\end{quote}

So the "engineering" metaphor is unacceptably difficult to understand,
but koan-like homilies are OK?

Cockburn then introduces his Shu-Ha-Ri model of software development
practice. Shu, Ha and Ri are the three levels of practice in Aikido, and
roughly translate into \emph{learn}, \emph{detach} and \emph{transcend}. In drawing
this obtuse metaphor, Cockburn manages to simultaneously insult the
intelligence of his readers and the martial arts tradition whose
authenticity he is trying to co-opt. Much is made of the fact that
software developers can be considered to pass through successive stages
of facility that correspond to Shu, Ha and Ri. Nothing is made of the
fact that the same analogy can be drawn with every other occupation
whose practitioners grow in expertise over time.

One keeps waiting for the admission that all this armchair
philosophizing is just self-deprecating jest, but it seems it is not
going to be forthcoming. If you need a laugh, I'd encourage you read
Kent Beck's message to a young extremist \footnote{\url{http://c2.com/cgi/wiki?ToAyoungExtremist}} and the comments that
follow it. A greater pile of pseudo-intellectual backslapping you will
not find anywhere outside of the self-congratulatory annals of the New
Age movement.

\item Craftsmanship
\label{sec:orgheadline201}

The portrayal of "software development as craft" reached its most
irksome zenith in Pete McBreen's loathsome book "Software
Craftsmanship" \footnote{\emph{Software Craftsmanship}, Pete McBreen, Addison Wesley, 2002}. The book presents a false dichotomy between
engineering and craft. Engineering is mischaracterized as a soul-less
and impersonal undertaking that ignores the contribution of, and
variations between, individuals. However craftsmanship values the
individual and nurtures their development through apprenticeship-like
relationships with other practitioners.

McBreen makes the profound observation: "\ldots{} large methodologies and
formal structures don't write software; people do." Who'd have thought?
I rather thought these structures were there to support the people in
their efforts, not to supplant them. But apparently the Big M
Methodologists are conspiring to eliminate the human contribution
altogether and our only chance to save our jobs and our identities is to
embrace our "craft" and our role in its development.

I'm sure many developers like to think of themselves as craftsmen -- it
strokes their egos and elevates their self-perceived status. However the
notion of a craft is usually reserved for activities where artifacts are
produced through manual skill and dexterity e.g. carpentry, painting,
sculpture. In common usage you will also find it applied to certain
intellectual artifacts (as in "well crafted prose") but not those
artifacts of a more technical origin, of which software is surely one
(we don't speak of "well crafted formulae")

To liken software developers to craftsmen may be superficially
appealing, but it represents a retreat into the vague and inscrutable
domain of the New Age theorist.

\item This Is Engineering
\label{sec:orgheadline202}

Engineering is the use of scientific knowledge to solve practical
problems. It is characterized by activities such as planning and
construction. Engineers maintain such values as precision, realism and
integrity. Taking an engineering-based approach to software development
in no way denies the significant influence that individual abilities and
social dynamics exert over the outcomes we produce.

I believe engineering remains a suitable basis upon which we can make
concrete advances in software development practices. The kind of New Age
humanism we are seeing incorporated into the New Methodologies only
encourages endless philosophizing, metaphysical thinking and wasted
effort spent in the exploration of non-falsifiable premises.

\item Follow The Money
\label{sec:orgheadline203}

If the New Methodologies continue to follow the examples of their New
Age counterparts, it can only be a matter of time before they begin to
employ some of the same merchandising tactics. Only half in jest, I
contend that before too long we will see the following items available
for your convenient online purchase:

\begin{itemize}
\item Tapes and CDs of lectures given by notable New Methodologists, that
you can listen to in your car on the way to work. Titles may include
"The Path To Agility" and "Power Programming".\\
\item Office decorations in the mould of the Successories products.
Inspirational plaques with panoramic landscapes and themes like
courage, simplicity, humility etc. Matching mouse pads, mugs and
badges.\\
\item The "Agile Thought of the Day" email services
\item Hokey accessories like diaries and calendars featuring slogans like
"You Ain't Gonna Need It" and "Do The Simplest Thing That Could
Possibly Work". The XP Programmer's cube \footnote{\url{http://xp123.com/xplor/xp0006/index.shtml}} may be an early
prototype.
\end{itemize}

Finally, let me leave you with a Zen parable. Make of it what you will:

\begin{quote}
/Bazen and an Engineer were out walking together. Bazen turned to the
Engineer and said, "Tell me Engineer, what is the sound of one hand
clapping?" The Engineer, swatting at the air near one ear, replied
"It's sort of a 'wooshing' noise, isn't it?" At this, Bazen was
enlightened./
\end{quote}
\end{enumerate}

\subsubsection{Rhetorical AntiPatterns in XP  \footnote{First published 19 Apr 2004 at
\url{http://www.hacknot.info/hacknot/action/showEntry?eid=51}}}
\label{sec:orgheadline210}

Over the past few years, I've spent more time in consideration of XP and
its followers than is in the best interests of one's mental health. My
preoccupation with it springs from my broader interest in skepticism.
It's fascinating to watch the same forces that drive cults,
pseudo-science and other popular delusions at work in one's own
profession. It's like driving past a road accident. It's tragic and
disturbing, but so entrancing that you just can't look away.

One of the aspects of XP that is particularly intriguing is the way that
certain rhetorical devices are used repeatedly to prop up the XP belief
system in the face an uncooperative reality.

This post describes the four main rhetorical devices that XPers use to
influence their audience and each other. Once you see how it's done,
you'll find yourself able to "talk XP" like a native.

The four techniques are:

\begin{itemize}
\item Adopt A Tone Of Authority And Eschew Equivocation
\item Make Bold Assertions And Broad Generalizations
\item Use Evidence Whose Veracity Can Not Be Challenged
\item Create Slogans And Neologisms
\end{itemize}

\begin{enumerate}
\item Adopt A Tone Of Authority And Eschew Equivocation
\label{sec:orgheadline205}

No matter what questions you might have, there is someone out there that
is willing to sell you the answers. And although the vendors come in
many different forms they have one characteristic in common -- they all
appear absolutely sincere and absolutely sure of themselves. So must you
be if you are to talk like a true XPer.

Fortunately, the impression of authority is easily created with some
linguistic sleight of hand:

\begin{itemize}
\item Never qualify your statements or concede error. If you say "I don't
think that is true" nobody will notice. But if you say "That is
absolutely false" you can capture people's interest and attention.\\
\item Intimate that you are speaking on behalf of others. For example, the
statement "Software developers don't work that way" is more
compelling than the statement "I don't work that way." Stating that
"Everybody knows X" is more impressive than stating "I know X."
\end{itemize}

Exercise some restraint with these techniques. It's easy to go too far
and sound like a born-again prophet. You will find it useful to temper
your pontifications with the occasional self-deprecatory statement, just
to make it clear to your audience that although you know you are very
wise, you don't think you're the Messiah.

Another way of elevating your own perceived authority is to denigrate
others. For example, those not enamored of pair programming may be
accused of being socially inept or sociopathic. More recently, we have
seen attempts to attribute a distaste for pair programming to genetic
disorders such as autism and Asperger's syndrome. Statements so personal
are delightfully controversial, and can also be used to goad detractors
into overly emotive responses, which can be interpreted as further
evidence of mental instability. Applied frequently enough, such
pathologizing will discourage your detractors from making public
criticisms, knowing that they will be virtually waving their "freak
flag" for all the world to see.

Finally, boost your own credibility by borrowing it from elsewhere. Make
occasional references to:

\begin{itemize}
\item Eastern philosophies and spiritual traditions
\item Movies, literature and personalities from pop culture
\item Advanced mathematics and physics, particularly chaos theory and
quantum mechanics
\item Political ideologies
\end{itemize}

\item Make Bold Assertions And Broad Generalizations
\label{sec:orgheadline206}

XP rhetoric is characterized by broad and sweeping generalizations about
software development practice, projects and developers. A classic
example is the following, from Kent Beck:

\begin{quote}
\emph{Unacknowledged fear is the source of all software project
failures}. \footnote{\emph{Planning Extreme Programming}, Kent Beck and Martin Fowler, p8}
\end{quote}

It takes a special kind of person to make such claims -- specifically,
one that is breathtakingly arrogant. If this arrogance doesn't come
naturally to you, then you will have to affect it. The more spectacular
and entertaining your statements, the better the chance that they will
be turned into a sound bite or quoted by a journalist. The media loves
attention grabbing one-liners and there is little you can say that is so
ridiculous that the determined reader will not find some way to
interpret it as both meaningful and insightful.

Do not let an absence of supporting evidence constrain your imagination.
If detractors point out exceptions to your generalizations, simply
dismiss those exceptions as being so atypical or statistically
insignificant as to not warrant revision of an otherwise useful rule of
thumb.

In argument, coupling these generalizations with baseless assertions is
an effective "one-two" punch to your opponent's frontal lobes. If they
should be rendered speechless at the audacity of your statements, seize
the opportunity to change the subject or offer some non-sequitur, so
that they will not have the opportunity to challenge you.

Most importantly, remember that the credibility of your propositions
rests almost exclusively on your ability to deliver them with absolute
conviction. The software development community are a gullible lot, and
provided that you sound like you know what you're talking about, a great
number of them will simply assume that you've got the facts to back it
up. For those unencumbered by integrity, this is the ideal flock to lead
out of the programmatic wilderness, if only you can make the cattle-call
compelling enough.

To get you started, here are some bold assertions and baseless
generalizations that are anti-XP in nature. Feel free to use them in
your next exchange with an XPer.

\begin{itemize}
\item It is inevitable that XP will fade into technical obscurity, just
like every other fad the software industry has witnessed in the last
thirty years.\\
\item The fervor with which XPers cling to their codecentric methodology
betrays the underlying fear which drives them: the fear that if they
should ever stop typing someone might realize that coding is their
only skill. In a modern business context the ability to code is
useless if not accompanied, in equal or greater measure, by the
ability to perform a whole host of non-coding activities that XP does
not even address.\\
\item Extreme programming is not about programming. It is about the
attempts of a small group of attention-seeking individuals to make
their mark on the computing landscape. • The irony of Extreme
Programming is that to make it work in the real world, you have to
moderate the "extremeness" to such an extent that you're left with
just "programming."
\end{itemize}

\item Use Evidence Whose Veracity Can Not Be Challenged
\label{sec:orgheadline207}

The software development community has a very low evidentiary standard
-- somewhere approaching zero. In other words, personal observations and
testimonials are the only corroboration that most will require for any
statement you might make. Empirical software engineering is not a
popular field and the task of gathering empirical data sounds altogether
like too much hard work for most to be bothered with it. All the numbers
and statistics that it generates make really boring reading.
Additionally, it takes time to conduct experiments, and who has that
sort of time when you're busy "riding the wave" of the latest technology
fad?

These factors are a gift to you, the burgeoning XP orator. With suitably
contrived "anecdotal evidence" you can justify any claim you might make,
no matter how preposterous. Whether such evidence has any basis in fact
is almost entirely irrelevant. Anecdotal evidence is qualitative in
nature, which lends itself readily to exaggeration and confabulation.
You can create anecdotal statistics, safe in the knowledge that nobody
has any better information with which to challenge you. Here's an
example from Robert Martin:

\begin{quote}
/We find that only one in twenty programmers dislike pairing so much
that they refuse to continue after trying it. About one in ten
programmers start out being strongly resistant to the idea, but after
trying for a couple of weeks about half of them find that pairing
helps them./ \footnote{Artima web logs forum, posted November 15, 2003, R. Martin}
\end{quote}

If anyone does try to challenge your statistics, just ask them why they
are so hung up on numbers, and suggest that an emphasis upon
quantification in software development is unreasonable and impractical.

If the purported evidence originates from your own experiences, prefix
it with "in my experience" and claim "I've seen it with my own eyes."
Who could doubt that? If you want evidence to have come from someone
else, to create the impression of independence, remember that you can
always get the answers you want by asking the right questions of the
right people.

\item Create Slogans And Neologisms
\label{sec:orgheadline208}

If you've ever wondered why the XP lexicon contains so many trite catch
phrases like "embrace change" and cutesy terms like "planning game" and
"YAGNI", then you've hit upon two of the most important features of the
vernacular -- slogans and neologisms.

Slogans are a frequently used marketing device. They're like the "hook"
in a pop song -- they are music to the ears of the masses. As an added
bonus, they lend themselves to being parroted off dogmatically -- which
will discourage people from thinking (critically or otherwise) about the
validity of the propositions they embody. XP slogans are the rhetorical
equivalent of the pre-prepared meals that TV cooking show hosts
introduce with the phrase "here's one I made earlier."

To get you started, here are a few anti-XP slogans you might like to put
on a t-shirt or poster:

\begin{itemize}
\item Pair programming -- for those with only half a brain
\item eXtreme Propaganda not welcome here
\item Embrace Change (You're Gonna Need It after you get fired)
\item IfXPIsSoGreatWhyCan'tTheyFindTheSpaceBar?
\end{itemize}

Neologisms are a trademark of many methodologies. By creating new terms
you also create the impression of invention; of having discovered or
created something so novel that no existing term adequately describes
it. Conveniently then, neologisms allow you to take old knowledge, give
it a new name, and then portray it as being something new. What's more,
if you created the term, then you have a monopoly over its definition,
which you are free to change from time to time as suits your purpose.
You can even furnish common terms like "success" and "simple" with
methodology-specific definitions, if this is what it takes to preserve
the truth of some rather brash statements you made earlier. Do not be
hampered by the bug-bear of consistency. Feel free to develop
conflicting definitions of terms, giving you the freedom to later invoke
whatever definition is most convenient for the situation you're in. If
anyone should highlight your self-contradiction, simply excuse it as
evidence of a deeper wisdom that defies even your complete
understanding.

\item A Catechism
\label{sec:orgheadline209}

To illustrate how these techniques can be used in combination, I offer
you the following dialog that I may or may not have had recently (hey,
it's anecdotal evidence -- how are you going to challenge me?) with a
hard-core XPer. I chose to abandon my usual skeptical mode of argument
and get "down and dirty" with some XP lingo. I encourage you to try it
sometime. It's quite liberating to be free of the constraints of logic,
and the burden of proof.

\begin{description}
\item[{XPer}] Hey Ed, want to do some pair programming with me?
\item[{Ed}] No thanks - pair programming isn't for me.
\item[{XPer}] Have you tried it?
\item[{Ed}] Briefly, but I disliked it - which wasn't surprising. It's
quite at odds with my personality.
\item[{XPer}] How long did you try it for?
\item[{Ed}] Oh - about four days or so.
\item[{XPer}] (laughing) That's not nearly long enough. And you've got to
make sure you're doing it right, otherwise it won't work.
\item[{Ed}] No \ldots{} really. No amount of persistence is going to change the
situation. I know enough about my own nature to say that with some
confidence.
\item[{XPer}] But why not try it again? What are you afraid of?
\item[{Ed}] [switching to XP lingo] I'm afraid of ending up in a state of
total cognitive surrender, like yourself and other similarly
disillusioned XP zealots. Anyway -- why do you need to program with
someone else? Aren't you good enough to work by yourself?
\item[{XPer}] [taken aback] It's not about "good enough", it's about
"better". I'm more productive when I work with someone else.
\item[{Ed}] So you claim. If I claimed to be more productive with a whiskey
and soda by my side, would that warrant charging up a bottle of Jack
Daniels to the project? Playing around with novel work methods at the
customer's expense is professionally irresponsible.
\item[{XPer}] But pair programming works! I've experienced it for myself!
\item[{Ed}] No, what you've experienced is having a nice time with a buddy.
Then you justified it to yourself by claiming a productivity
improvement. People see what they want to see.
\item[{XPer}] I don't think you can comment -- you haven't really tried
pair programming.
\item[{Ed}] Or to put it another way -- I'm not the slave to technical
fashion that you are -- which actually gives me a more objective
viewpoint from which to comment. Pair programming is a fantasy -
there is simply no evidence that it works. Those who think it does
are kidding themselves.
\item[{XPer}] How can you say that? There was this university study that
demonstrated experimentally that it works!
\item[{Ed}] Are you talking about the study by Laurie Williams at the
University of Utah?
\item[{XPer}] Yeah -- that's the one.
\item[{Ed}] Tell me -- have you \emph{read} William's thesis?
\item[{XPer}] Well -- no, but I've read \emph{about} it.
\item[{Ed}] So I can't comment on pair programming because I haven't really
tried it, but you \emph{can} comment on experiments that you haven't even
read.
\item[{XPer}] Look -- I may not have read the details, but I know what it
proved.
\item[{Ed}] What it proved is that it's easy to do bad experiments, and
that many software developers like yourself are gullible enough to
believe anything they hear, so long as it fits in with their
preconceptions. If you really knew about pair programming, you'd
already know that the Williams experiment proves absolutely nothing.
\item[{XPer}] I've paired with plenty of developers in the past, but nobody
got upset about it like you. Have you got some kind of problem?
\item[{Ed}] If you think that others should necessarily have the same
preferences as you, then I'd suggest it's \emph{you} that's got the
problem. I'm happy for you to pair program if you want, but I must
decline the offer to participate in your hallucination.
\item[{XPer}] [shaking head] Ed, you've got to learn to "embrace change".
The whole XP thing is taking off -- "agile" is the way software
development is gonna be from now on. Get on board or step aside.
\item[{Ed}] "Change imposed is changed opposed."
\item[{XPer}] How do you mean?
\item[{Ed}] For one so agile, you're a bit slow on the pick-up. In this
context, it means that if you try and force people to work a way they
don't want to, then they'll fight back.
\item[{XPer}] I don't hear anyone fighting against XP.
\item[{Ed}] Then where have you been for the last five minutes? You just
demonstrated my point -- people hear what they want to hear.
\item[{Ed}] Ok, maybe some folks don't get it, but there are plenty of
people who do, and who are achieving success. At least as many people
have tried XP and failed. Some of them go on to claim success anyway,
because admitting to failure would be too embarrassing. Most of them
just say nothing and hope nobody notices their stuff-up. If you think
the success-stories you read about in the media are representative,
you're kidding yourself. The real story is very, very different.
"Success has many fathers, but failure is an orphan."
\item[{XPer}] OK, maybe there's some truth to that. But you can't be saying
that all these XP proponents are lying?
\item[{Ed}] No -- not all of them, but some of them are, and some of them
are exaggerating. The rest are probably what we call "pious frauds" -
that is, they genuinely believe what they're saying, but are really
misconstruing the influence of XP on their projects. It's easy to do
if you play down the negatives and emphasize the positives.
\item[{XPer}] Say -- didn't you tell me once that you're a skeptic?
Shouldn't a skeptic keep an open mind?
\item[{Ed}] Yes, but not so open that their brains fall out.
\end{description}
\end{enumerate}

\subsubsection{The Deflowering of a Pair Programming Virgin  \footnote{First published 16 Sep 2003 at
\url{http://www.hacknot.info/hacknot/action/showEntry?eid=22}}}
\label{sec:orgheadline215}

In your readings of the voluminous XP canon, you will no doubt have
encountered mention of the practice of Pair Programming \footnote{\url{http://www.pairprogramming.com/}}. If, like
me, you are of a solitary disposition, you will have found yourself
thinking -- nice idea, but not for me.

Many of us are attracted to software development as a career because we
enjoy the experience of solitary problem solving. We relish those times
when we are "in the zone" -- where our locus of concern narrows to
exclude everything but ourselves, the keyboard and the problem at hand.
This state can produce a feeling of mild euphoria, and gives us a place
of retreat from the worries and concerns of our immediate environment.

The practice of Pair Programming puts an end to all of this. The problem
solving medium moves from an interior dialogue to an exterior one. The
silence we traditionally associate with deep thought and focused effort
is replaced with the interaction and debate we more usually expect from
a meeting or brainstorming session.

It was with some trepidation then that I recently accepted an offer from
a colleague to engage in some Pair Programming as a way of extending my
knowledge of certain subsystems of our application in which he had a
greater degree of involvement than myself. The activity lasted about
four days -- long enough to complete the implementation and testing of a
minor system feature in its entirety. The experience was an interesting
one, but on the whole, not one that I'd care to repeat with any
regularity.

Pair Programming studies so far conducted have tended to originate from
academic environments, and so focus on novice-novice pairings amongst
students. It is not clear that their findings translate into a
commercial programming context staffed by more mature professionals. By
contrast, myself and the colleague I paired with have been doing
whatever it is that we do for 10+ years each. In the period described
herein, we sat together for approximately six hours on each day, using
the same person's computer each time.

Following is a point-form summary of my experiences over this period,
both positive and negative.

\begin{enumerate}
\item Positives
\label{sec:orgheadline211}

\begin{itemize}
\item When pairing, one programmer keeps the other from goofing off and
wasting time web surfing etc.\\
\item You tend to be more diligent in the construction of unit tests and
more careful in general when you know that someone is watching you
and looking for error. Also, as a matter of professional pride, you
don't want to be seen to be hacking by a colleague.\\
\item The quality of code produced is marginally better than I would
achieve at a first cut when coding individually.\\
\item When two people have participated in the construction process,
familiarity with the code is spread further amongst the team members
which mitigates the dependence upon any individual. If there is no
external documentation, it may be more efficient to acquire
familiarity with a piece of code on this basis, than by the
alternative -- reverse engineering.\\
\item There is the opportunity to pick up tricks and shortcuts from
watching someone else go about the arcana of their job (e.g. learning
to use IDE features that you were previously unaware of).\\
\item Mistakes are picked up more quickly due to the overseeing of one's
partner.
\end{itemize}

\item Negatives
\label{sec:orgheadline212}

\begin{itemize}
\item The constant interaction is very tiring. Most days I went home
absolutely exhausted from the enervating effect of continuous dialog,
and frequently with a headache.\\
\item There is a lot of noise produced, which tends to disturb those in the
surrounding area. A room full of pair programmers, as advocated by
XP, would be very noisy indeed.\\
\item There are numerous ergonomic problems when two people share a
computer. My colleague prefers a conventional keyboard with
international settings activated (he is bilingual), a trackball and a
medium screen resolution. I prefer a split keyboard, no extended
character set capability, a wheelie mouse and a slightly higher
screen resolution. We had to swap hardware whenever we "changed
drivers," which was annoying. Had our preferences in screen
resolution not been similar, working from the one VDU could have been
impossible (for example, if one of us had low vision).\\
\item There is a lot of "pair pressure" created from having someone
watching every character you type. It tends to produce a
self-consciousness that is inhibiting and constitutes a low-level and
constant stressor.\\
\item There is a tendency to feel constantly under time pressure when
typing, because someone is waiting for your every keystroke. This
produces a certain degree of "hurry up" sickness, which discourages
any delay in doing more typing, such as that produced by thoughtful
consideration of design issues.\\
\item Groupthink can occur, even when there are only two people in the
group. When you are working so closely with another, you are very
wary of argument or disagreement, lest it sour the working
relationship. Therefore people tend to agree too readily with one
another, and seek compromise too quickly. Whoever chimes in first
with a suggestion is likely to go unopposed.\\
\item Time spent away from one's pair partner tends to be non-productive as
your thoughts are dominated by the task the pair is currently
tackling. This makes it difficult to effectively interleave other
tasks with an extended Pair Programming session.\\
\item Both myself and my colleague concede that we work in a different way
when pairing than when working individually. Alone, our work patterns
tends to consist of short bursts of productivity, separated by
periods of mental slouching, by way of recuperation and cogitation.
When pairing, those intermittent rest breaks are removed for fear of
hindering someone else's progress, and because the low level details
of different people's work habits will be unlikely to exactly
coincide.
\end{itemize}

\item Conclusions
\label{sec:orgheadline213}

From this brief experience in Pair Programming it seems clear to me that
the appeal (and therefore success) of the practice is likely to vary
significantly between individuals. More gregarious programmers may enjoy
the conversation and teaming effects, whereas more introverted
programmers will find the constant interaction draining.

I am particularly interested to note that reports of Pair Programming
experiences commonly available through the media tend to have a positive
reporting bias. Experience reports of the form "we tried pair
programming and we loved it" are not difficult to come by \footnote{\url{http://www.cs.utah.edu/~lwilliam/Papers/ieeeSoftware.PDF}} (which is
not to say they are significant in number, but simply that a few studies
are very frequently cited), but anecdotes that end "\ldots{} and then he
resigned because he couldn't bear the constant pair programming" are not
as readily available.

I don't believe my take on Pair Programming is likely to be singular. My
personality type and communication preferences are not at all uncommon
amongst developers. In Myers-Briggs terms I am an ISTJ \footnote{\url{http://www.typelogic.com/istj.html}}, which is
the most common personality type in the IT industry. I believe that many
developers will find Pair Programming to be a difficult and ultimately
unsustainable work practice -- one that removes from their work day some
of the basic elements that first attracted them to their occupation.

For a pairing of mature developers, I believe the effect on code quality
is vastly overstated amongst the XP community. That there is some
marginal improvement in the quality of the code when first cut seems
clear. That this improvement justifies the investment of effort required
to produce it, or that it could not be obtained more efficiently through
regular code review techniques, is not at all clear.

Finally, I believe that Pair Programming is a very inefficient way to
share knowledge amongst team members. The total man hours invested in
doubling up can result in at best two people being familiar with the
code being worked on. A good design document could guide an arbitrary
number of future developers to an equivalently detailed understanding of
the code, saving the expense of continual, unassisted reverse
engineering on their parts.

\item Addendum
\label{sec:orgheadline214}

Shortly after posting this, a reader asked for the basis of my statement
that ISTJ is the most common personality type in the IT industry. The
findings of two large studies are relevant here, both of which I found
referenced in "Professional Software Development", Steve McConnell,
Addison Wesley, 2004, p63:

\begin{itemize}
\item "\emph{Effective Project Teams: A Dilemma, a Model, a Solution}", Rob
Thomsett,
\end{itemize}

American Programmer, July-August 1990, pp.25-35

\begin{itemize}
\item "\emph{The DP Psyche}", Michael L. Lyons, Datamation, August 15, 1985, pp.
103-109
\end{itemize}

McConnell cites these two studies as finding the most common personality
type for software developers to be ISTJ. My statement generalizes this
conclusion to the entire IT industry, which is obviously unwarranted.

McConnell cites further studies from Thomsett, Lyons, Bostrom and Kaiser
as finding that ISTJs comprise 25-40 percent of all software developers.
\end{enumerate}

\subsubsection{XP and ESP: The Truth is Out There!  \footnote{First published 5 May 2004 at
\url{http://www.hacknot.info/hacknot/action/showEntry?eid=53}}}
\label{sec:orgheadline219}

\begin{quote}
“/Eclipses occur, and savages are frightened. The medicine men wave
wands -- the sun is cured -- they did it./” -- Charles Fort \footnote{Cited in \emph{Voodoo Science}, Robert Park, Oxford, 2000}
\end{quote}

People have a vast capacity for self-deception. Even members of the
scientific community, from whom we expect objectivity, can unwittingly
allow their personal beliefs and preconceptions to color their
interpretation of data. Professional ambition and wishful thinking can
turn their stance from one of neutral observance into passionate
adherence to a position, sustained by willful ignorance of contrary
evidence. Such attitudes are common amongst the ranks of
pseudo-scientists and paranormal researchers. Enthusiasts in this domain
reward these ersatz scientists by buying their books and journals in
numbers proportionate to the impressiveness of the alleged experimental
findings. In doing so, they become complicit in their own deception.

Many of these enthusiasts labor under the misimpression that the
existence of ESP, PK and other paranormal phenomena has been "proved" by
creditable scientists. Many of the researchers are similarly deceived.

Curiously, we may be seeing exactly the same effects currently at work
in the software development community with regard to XP. If there is
sufficient desire to find "evidence" favorable to XP, it will be found.
If there is sufficient reward for publication of XP success stories,
they will be published. The belief that XP has been "proved" in the
field can develop, if there is sufficient desire to believe it. And if
sustaining that belief makes it necessary to ignore conflicting evidence
and censor stories of failure, then that will also occur.

Be it XP trials or ESP experiments, there are two sorts of bias that
make it possible to find significance where there is none, and sustain
false belief. This post examines how these biases manifest in both
domains.

\begin{enumerate}
\item Positive Outcome Bias: Embrace Change Or Exaggerate Chance?
\label{sec:orgheadline216}

Positive outcome bias is defined as:

\begin{quote}
/The tendency of researchers and journals to publish research with
positive outcomes much more frequently than research with negative
outcomes./ \footnote{\emph{The Skeptic's Dictionary}, Robert Carroll, Wiley, 2003}
\end{quote}

Suppose 100 researchers conduct an experiment in ESP. Each professor
chooses a single subject who believes they have ESP and asks them to
"sense" a series of randomly chosen Zener cards being "sent" to them by
the person who selects the cards. Suppose that in 50\% of these
experiments, the subject achieves an accuracy greater than that which
could be attributed to chance alone. The 50 researchers conducting those
experiments are intrigued, and decide to conduct a further round of
tests with the same subject. The other 50 researchers, knowing that
failed attempts to detect ESP are unlikely to get them published,
abandon their experiments.

In the next round of experiments, the same pattern occurs, and 25 more
researchers give up. Eventually, all the researchers give up, but not
before one has witnessed his subject beat chance in 6 or 7 consecutive
experiments - which is quite a spectacular result! Deciding to neglect
the final experiment that caused him to stop (figuring the subject was
probably tired, anyway) the researcher writes up his results and sends
them to the editor of the \emph{Journal of Parapsychology}, in which they are
published.

Consider the deception which results:

\begin{itemize}
\item The PSI research community's pro-ESP bias has been further confirmed
by their receipt of this latest research evidence
\item The readers of the \emph{Journal of Parapsychology} are impressed with the
evidence, and any preexisting belief in ESP is further cemented.\\
\item Other researchers, perhaps even some outside the PSI community,
conclude "Maybe there's really something to this ESP stuff after all"
and decide to conduct their own experiments in ESP, thereby
propagating the effect into another round of investigations.
\end{itemize}

Note that neither the researcher who was published, the research
community, nor any of the readers of the \emph{Journal of Parapsychology}
ever become aware of the 99 experiments that were abandoned because they
were deemed unpublishable. Taken in isolation, the published result may
be impressive. But taken in the context of the other 99 experiments that
have silently failed, the published result may simply be an outlier
whose occurrence was actually quite likely.

The following factors contribute to positive outcome bias:

\begin{enumerate}
\item Researchers who conduct uncontrolled experiments
\item Researchers who self-censor negative results
\item Researchers who can justify to themselves the imposition of optional
starting and stopping conditions.\\
\item A publication environment that favors success stories
\end{enumerate}

All three of these are features of the environment in which the software
development community examines and reports on your favorite methodology
and mine, XP:

\begin{enumerate}
\item XP is often trialed on a single project, on a non-comparative basis
(controlled experimentation would be prohibitively expensive).\\
\item When an XP project fails, it will probably fail quietly. Companies
and individuals have reputations to protect.\\
\item In a series of XP-related experiences, initial negative experiences
are dismissed as "teething trouble". For an example, see Laurie
William's pair programming experiment. Her dismissal of the last of
four data sets, and devaluing of the first of those four data sets,
is a good example of "optional starting and stopping conditions."
\item There can be no doubt that the IT media just loves those "XP saves
the day" stories. Success stories sell magazines.
\end{enumerate}

In such an environment, XP enthusiasts will declare "Wow, everywhere you
look, XP is succeeding" -- which is true. But it's in the places that
you \emph{haven't} looked that the real story lies.

\item Confirmation Bias
\label{sec:orgheadline217}

\emph{Confirmation bias} is defined as:

\begin{quote}
/The tendency to notice and to look for what confirms one's beliefs,
and to ignore, not look for, or undervalue the relevance of what
contradicts one's beliefs./
\end{quote}

When it is pointed out to PSI researchers who claim to have successfully
demonstrated ESP, that hundreds of non-PSI researchers have tried to
replicate their results and failed, they sometimes attribute this to the
ostensible influence that the attitude of both experimenter and subject
can have over the results. An experimenter who is hostile towards the
concept of ESP, they claim, can exert a negative influence over the
results, thereby counteracting any positive ESP effects that may be
present. This is one of the many "outs" PSI researchers have developed
that enable them to attribute negative results to extraneous causes, and
preserve only the data that is favorable to their preferred hypotheses.

We see exactly the same thing happening in the XP community's evaluation
of experience reports from the field.

When presented with a claim of success using XP, the community accepts
it without challenge, for it is a welcome confirmation of pre-existing
beliefs. However, a claim that XP has failed is an unwelcome affront to
their personal convictions. So these claims are scrutinized until an
"out" is found - some extraneous factor to which the blame for failure
can be assigned. If all else fails, one can claim, as PSI researchers
are wont to do, that the attitude of the participants is to blame for
the failure.

To illustrate, consider the tabulation below of the four types of
experience reports that the XP community can be presented with. The
columns represent the two basic modes of XP usage -- full and partial.
Either you're doing all the XP practices or you're only doing some of
them. The rows represent the claimants assessment of the project outcome
-- success or failure. The table shows the interpretation an XP
proponent can confer upon each type of experience report so as to
\emph{confirm} their pre-existing belief in XP.

\begin{center}
\begin{tabular}{lll}
 & Full XP & Subset of XP\\
\textbf{Success} & "XP has succeeded" & "See how powerful XP is? Even a subset of the practices can yield success"\\
\textbf{Failure} & "You weren't doing xxx as well as you could have",  "You weren't committed enough", "There's something wrong with you" etc. & "You weren't doing all the practices, so you weren't really doing XP"\\
\end{tabular}
\end{center}

The XPers have all their bases covered. No matter what the experience
report, there is no need to ever cast doubt upon XP itself -- there are
always rival causes to be blamed. \footnote{\url{http://c2.com/cgi/wiki?IfXpIsntWorkingYoureNotDoingXp}} In this way, XP becomes
nonfalsifiable.

\item Conclusion
\label{sec:orgheadline218}

There is an "essential tension" \footnote{\emph{Why People Believe Weird Things}, M. Shermer, Owl Books, 2002} between being so skeptical of new
technologies and methods that we miss the opportunity to exploit genuine
innovations, and being so credulous that we are ourselves exploited by
those willing to subjugate integrity to self-interest. Given the
software industries' history of fads, trends and passing enthusiasms, we
would be wise to approach claims of innovation with caution -- where
those claims are accompanied by fanaticism and zeal, doubly so. As
Thomas Henry Huxley warned:

\begin{quote}
/Trust a witness in all matters in which neither his self-interest,
his passions, his prejudices, nor the love of the marvelous is
strongly concerned. When they are involved, require corroborative
evidence in exact proportion to the contravention of probability by
the thing testified./
\end{quote}

There is no logical basis for dismissing out of hand every "next big
thing" that comes along. But an awareness of confirmation bias, positive
outcome bias and their contribution to the development of false beliefs
should encourage us to seek evidence beyond that provided by popular
media and effusive testimonial.
\end{enumerate}

\subsubsection{Thought Leaders and Thought Followers \footnote{First published 24 May 2004 at
\url{http://www.hacknot.info/hacknot/action/showEntry?eid=55}}}
\label{sec:orgheadline222}

\begin{enumerate}
\item Fowler On "Appeals To Authority"
\label{sec:orgheadline220}

For a brief, shining moment there was hope. Through the exaggeration and
braggadocio that so permeates the conversation of the Agile community,
there came a fleeting glimpse of self-awareness -- a flash of social
perspective that could have precipitated a greater moderation and
rationality in the methodological discourse. And then it was gone --
swept aside by the force of yet another ill-considered generalization.

I'm referring to a recent blog entry by Martin Fowler entitled \emph{Appeal
To Authority}. \footnote{\url{http://martinfowler.com/bliki/AppealToAuthority.html}} In this entry, Fowler relates how he occasionally
receives the comment "When a guru like you says something, lots of
people will blindly do exactly what you say." Fowler denies the
existence of such an effect, and counters that what appear to be appeals
to authority may really be just an artifact of lazy argument or sloppy
self-expression.

The argument from authority is everywhere in the Agile and XP
communities, and is a far more potent force than Fowler seems to
appreciate. Here are just a few ways that the various so-called "thought
leaders" and "spokesmen" employ direct and indirect appeals to
authority.

\begin{itemize}
\item Statements prefixed with "In my experience", combined with the
suggestion that this experience is extensive, are attempts to cast
the speaker as a seasoned veteran whose word should be taken
seriously. Having many years of experience only establishes that one
is old, not that one is correct.\\
\item Sweeping statements and broad generalizations can make for
powerful-sounding oratory, and suggest that the speaker possesses
some kind of absolute knowledge i.e. that they are simply declaring
information that they know to be factual. By abandoning the
uncertainty and qualification, the speaker sacrifices accuracy for
the sake of impact and elevates opinion to fact.\\
\item By inventing and promulgating cute slogans, folksy homilies and other
media-friendly sound bites, speakers encourage others to quote them
verbatim and dogmatically. Such quotation invests the statement, and
thereby the speaker, with a faux authority.\\
\item With rare exception, the aforementioned comment from Fowler's being
one such case, the "thought-leaders" and "spokesmen" rarely
acknowledge, let alone reject, their decoration with such grand
titles. There is no attempt to discourage the use of such titles,
beyond the occasional token self-deprecation.\\
\item Speakers claiming to represent the opinions and experiences of a
group are naturally encouraging a view of themselves as leaders. Such
speakers will not hesitate to claim "The Agile community believes X"
or "The XP community does X", even though the communities in question
have not been consulted or surveyed, and in fact may have wildly
varying and inconsistent views on the matter.
\end{itemize}

Fowler's claim that appeals to authority are not a significant influence
strikes me as disingenuous. Not only are such appeals frequent, they are
at the very heart of the rhetoric. It should be kept firmly in mind that
those most outspoken in this space are almost always consultants
specializing in AM/XP. \footnote{Agile Methods / Extreme Programming} Consultants make their money by promoting
themselves as authorities on some subject, so that others will hire them
for their perceived expertise.

\item Ruin Your Career With Agility
\label{sec:orgheadline221}

An interesting blog entry, author unknown, came to my attention
recently. Entitled \emph{How Agile Development Ruined My Career (Sort
Of)} \footnote{\url{http://www.undefined.com/ia/archive/000158.html}} it is the story of a Senior Director's attempts to introduce
Agile work practices into a company, and the consequences for himself. I
have commented on the blog itself, and the XP fraternity has just begun
to dissect it on \texttt{comp.software.extreme-programming} \footnote{<\url{http://groups.google.com/groups?group=comp.software.extreme-programming}} (posted 23 May
\begin{enumerate}
\item which should make for entertaining reading.
\end{enumerate}
\end{enumerate}

\subsection{Requirements}
\label{sec:orgheadline231}

\subsubsection{Dude, Where's my Spacecraft? \footnote{First published 4 Nov 2003 at
\url{http://www.hacknot.info/hacknot/action/showEntry?eid=33}}}
\label{sec:orgheadline227}

The Mars Polar Lander (MPL) that NASA launched in 1999 is now a rather
attractive and very expensive field of tinsel-like shrapnel scattered
over several square kilometers of the Martian surface. It is not
functional in any capacity. It is no more. It has ceased to be.

Its demise was the result of the flight control software incorrectly
answering the question that car-bound children have been plaguing their
parents with for years -- "are we \emph{there} yet?" About 40 meters above
the ground, the software succumbed to the constant nagging of its
digital offspring and answered too hastily "Yes! We're there!" --
triggering the shutdown of the MPL's descent engines. The craft's final
moments were spent free falling towards the Martian soil at 50 mph
(80km/h) -- ten times the impact speed it was designed to withstand.

Monitoring the MPL's progress from Earth, NASA had expected a 12 minute
period of broadcast silence during the descent to the landing area, due
to the cant of the craft during re-entry. Shortly after touchdown, the
MPL was scheduled to begin a 45 minute data transmission to Earth, but
this transmission never occurred. NASA kept attempting contact with the
MPL for the next six weeks, until finally giving up hope of ever hearing
from it again.

Of course, it was not long before the faecal matter hit the rotary air
distribution device.

In-depth mission reviews were conducted at NASA Headquarters, JPL and
Lockheed Martin Astronautics. An independent assessment team was also
established. Initially there were considered to be a number of possible
causes for the mission's failure, but extensive investigations singled
out one of them as being the most likely failure mode, with a high
degree of confidence.

The assessment team concluded that a spurious signal from one or more of
the touchdown sensors at the ends of the MPL's legs caused the software
to conclude incorrectly that the craft had already made contact with the
Martian soil and to therefore shutdown the descent engines prematurely.

However, this wasn't an unexpected hardware fault. The tendency of the
Hall Effect touchdown sensors to generate a false momentary signal upon
leg deployment was well known to NASA engineers, having been discovered
in early testing. The software should have screened out these spurious
signals, but this functionality was never actually implemented.

More precisely, the series of events leading to failure was likely the
following:

\begin{enumerate}
\item 1500m above the surface of Mars, the legs of the MPL deployed. The
touchdown sensor at the end of one or more of the legs generated a
characteristic false touchdown signal while being deployed. The false
touchdown event was registered by the flight control software and
buffered.\\
\item 40m above the surface, the software began continuous sampling of the
values from the touchdown sensors.\\
\item The first value read was the buffered false touchdown event that
occurred upon leg deployment.
\item The software immediately triggered the shutdown of the Lander's
descent engines, believing that the Lander was now on the surface of
Mars.
\end{enumerate}

\begin{enumerate}
\item Reasons For Failure
\label{sec:orgheadline224}

One of the main reasons the flight software did not behave correctly is
because the definition of "correct" was changed in response to field
testing. With respect to detecting touchdown, the system requirements
initially stated:

\begin{quote}
"/The touchdown sensors shall be sampled at 100 Hz rate. The sampling
process shall be initiated prior to Lander entry to keep processor
demand constant/"
\end{quote}

When the false signal characteristic of the touchdown sensors was later
discovered, the following clause was added:

\begin{quote}
"/However, the use of the touchdown sensor data shall not begin until
40 meters above the surface./”
\end{quote}

The intended effect of this addendum was to disregard the false
touchdown signal previously generated during leg deployment at 1500m.
This change was never propagated to the lower level software
requirements.

Also note there is no explicit mention of the spurious signal
generation. Even if this addendum had been propagated into the lower
level requirements correctly, the software engineers would not have been
aware that a false touchdown event might \emph{already} have been registered
at the time the use of the sensor data began.

\item Moral \#1
\label{sec:orgheadline225}

The story contains two obvious lessons about requirements:

\begin{itemize}
\item Requirements tracking is useful in maintaining integrity between
multiple requirements sources.\\
\item Requirements should include a rationale i.e. specify \emph{why}, not just
\emph{what}.
\end{itemize}

And now a few words from some XP spokesmen on requirements tracking:

\begin{quote}
/I think I get, from the term, the idea of what Requirements Tracking
is. It sounds like you keep track of changes to the requirements, who
made the change, why they made it, when, stuff like that. If that's
wrong, correct me now. If that's what Requirements Tracking is, I
don't see the benefit. Please tell me a story where the moral is, “And
that's why I am ever so happy that I tracked requirements
changes./" \footnote{\url{http://c2.com/cgi/wiki?RequirementsTracking}} -- Ron Jeffries, with assistance from Kent Beck
\end{quote}

\item Moral \#2
\label{sec:orgheadline226}

You would think that a thorough testing program would uncover the flight
software's shortcomings. However, later testing did not detect the
software's inability to cope with these signals because the touchdown
sensors were incorrectly wired when the tests were performed. When the
wiring error was discovered and corrected, the tests were not reexecuted
in their entirety. Specifically, the deployment of the Lander leg was
not included in the test re-runs. The moral is: Thou shall \emph{fully}
regression test.
\end{enumerate}

\subsubsection{User is a Four Letter Word \footnote{First published 29 Jan 2006 at
\url{http://www.hacknot.info/hacknot/action/showEntry?eid=82}}}
\label{sec:orgheadline230}

The term "user" is not just a pronoun, it is a powerful buzzword that
pervades the software development literature, to both good and bad
effect. On the up side, the development community has been made aware of
the dominating role that end user experience plays in determining the
success or failure of many projects. On the down side, the message of
the importance of user feedback to the development process has been
adopted by some with uncritical fervor.

In their efforts to be "user focused," guided by simplistic notions of
"usability," many managers and programmers uncritically accept whatever
users tell them as a mandate. "The customer is always right" makes a
nice slogan but a poor substitute for critical thought. If you want to
deliver a product that is genuinely useful, it is important to moderate
the user feedback you receive with your own knowledge of usability
principles, and to seek independent confirmation of the information they
relate. For it is a fact seldom acknowledged in the text books that
users are frequently uninformed, mistaken or deliberately deceptive.

\begin{enumerate}
\item User Fraud
\label{sec:orgheadline228}

There are two types of fraud - the \emph{deliberate} fraud and the \emph{pious}
fraud. Both make false statements; the former knowing that they are
false, the latter believing them to be true. The user community contains
both types.

Suppose you are writing a system that will facilitate the workflow of
some subset of a company's employees. As future users of your software,
you go to them to find out exactly how they do their work each day, so
that you can understand their work processes. Some users find it
difficult to articulate their basic work methods, even though they may
have been in the same role for many years. Their routine becomes so
internalized that it is no longer readily available by introspection.
They may appear unsure and vague when describing how particular tasks
are accomplished, and when you ask why things are done in a given way,
you may get dismissive responses such as “Because that's the way we've
always done it.”

Are you being told the truth? The naive developer will take what the
user offers as gospel, and run away to implement it in software. The
more experienced developer will simply take it on board for
consideration, knowing that the user may be a fraud. Many users are
pious frauds, in that they will give you their opinion on what workflow
they and others are following, but state it as if it were an
incontestable fact. Long-serving employees are very likely to consider
themselves unassailable authorities on their company's processes.

But you must not lose sight of the fact that even the most genuine of
users can be mistaken or have incomplete knowledge. When surveying
employees who all participate in a common workflow, it is not at all
uncommon to find that each participant has a different conception of the
overall process. Sometimes there are only minor discrepancies between
their individual accounts; sometimes there are direct conflicts and
outright contradictions. This is particularly common in small
organizations that function in a "cottage industry" manner, where
nothing is written down and the work processes survive only through
verbal instruction, not unlike the folkloric traditions that exist in
tribes. The "Chinese whispers" effect can give rise to individuals
having significantly different understandings of what is ostensibly a
common work practice. Such users are not much to blame for their status
as pious frauds, having become so through common psychosocial
mechanisms.

Pious fraud also results from the common tendency to over-estimate one's
own level of expertise in relation to others. For example, drivers
involved in accidents or flunking a driving exam predict their
performance on a reaction test less accurately than more accomplished
drivers \footnote{\emph{Incompetent And Unaware Of It}, J. Kruger and D. Dunning, Journal
of Personality and Social Psychology, 1999, Vol. 77, No. 6,
1121-1134}. This \emph{self-serving bias} will be also be present amongst
your users, who may consider themselves experts in their domain and
therefore convey their responses with greater authority and certainty
than their true level of expertise actually justifies.

The user may describe a particular interface mechanism as having greater
usability than another, when they are in fact only acknowledging the
greater similarity of that design to the paper forms they are already
familiar with. Users are not interface designers any more than drivers
are automotive engineers.

On the border of pious and deliberate fraud are those users that are not
lying outright, but neither are they making much effort to help you
gather the information you need. They may simply be apathetic or cynical
-- perhaps having witnessed many failed IT initiatives within their
organization in the past. When interviewed, their participation is
begrudging, and they will make it obvious that they would rather be back
at their post getting on with some "real work". They are only involved
because their management has forced them to be so; they would really
just like you to go away.

The answers you get from them may be the truth, but not necessarily the
whole truth. Rather than describe to you all the variations and
exceptional circumstances they encounter in the course of doing their
job, they will simply give you a basic description of the usual way of
doing things. Then it will be up to you to tease out of them all the
boundary conditions and how they are handled. For the purposes of
process automation, these special cases are particularly important.

Hardest for the software developer to deal with are the deliberate
frauds. The developer is at a distinct disadvantage, for he is reliant
upon the user for information, but is generally not familiar enough with
the domain to be able to adduce that information's authenticity.

Asked to review documents that capture their workflow, the deliberate
fraud may declare the document correct, when in fact they have not even
read it. Or perhaps they actually \emph{have} attempted to read it but are
unwilling to admit that they have failed to understand it. A user may
announce that their job requires judgments too complex or heuristic to
be captured in software, when in fact they are simply unwilling to
release their accumulated wisdom and expertise because they fear
becoming expendable. The user may declare a particular procedure to be
the correct one, but actually describe how they would \emph{like} the
procedure to be, in the hope that your software will result in things
being done in accord with their personal preference.

Perhaps the most common ploy of the passive aggressive user is
procrastination. When asked to participate in interviews or submit to
any demand on their time, the user offers only perfunctory compliance,
complaining that they just can't find the time to put in greater effort,
given the demands of their existing duties. They know that if they demur
frequently enough, you will probably stop assigning them tasks
altogether.

\item Conclusion
\label{sec:orgheadline229}

There is a common tendency in the development community to conflate a
"user focused" approach with one that unquestioningly accepts arbitrary
dictation from users. The result is a gullible and over-confident
development team that has unwittingly compromised their ability to
effect the success of their own project.

While it is essential for developers to maintain a focus on their user's
needs and expectations, they must be careful to think critically about
the feedback they receive. To this end, it is important to independently
verify the user's statements, obtain feedback from as broad a
demographic as possible, and maintain an awareness of the potential for
both deliberate and unintentional user error.
\end{enumerate}

\subsection{Design}
\label{sec:orgheadline248}

\subsubsection{The Folly of Emergent Design \footnote{First published 14 Oct 2003 at
\url{http://www.hacknot.info/hacknot/action/showEntry?eid=29}}}
\label{sec:orgheadline236}

One of the most pernicious ideas to proceed from the current focus on
lightweight methodologies is that of Emergent Design. It's difficult to
find a precise description of emergent design -- most discussion on the
subject carefully avoids committing to any particular definition. One of
the most succinct descriptions I've encountered is this, from the
adaptionsoft web site:

\begin{quote}
/"Many systems eventually require drastic changes. You cannot
anticipate them all, so stop trying to anticipate any of them. Code
for today, and keep your code and your team agile."/ \footnote{\url{http://www.adaptionsoft.com/xp_practices_simple_design.html}}
\end{quote}

Proponents of Emergent Design tout the following advantages of such an
approach:

\begin{itemize}
\item Visible signs of progress appear more quickly .
\item The system reaches a state in which it can be evaluated by customers
sooner, which is useful for verifying existing requirements and
teasing out as yet undiscovered requirements.\\
\item The risk of "analysis paralysis" is eliminated.
\item No effort is wasted in the preparation of infrastructure to support
anticipated requirements that never actually manifest.\\
\item An increased ability / willingness to adapt to changing requirements,
as the development effort is not burdened by prior commitment to a
particular solution approach.
\end{itemize}

Opponents of Emergent Design claim the following disadvantages:

\begin{itemize}
\item Exploration of alternative solutions takes much longer when using
code as the vehicle for exploration, rather than a more abstract
medium such as UML.\\
\item The "code for today" approach discourages the reaping of long term
savings in implementation effort by investing in supporting
functionality in the short term.
\end{itemize}

Proponents will counter these by referencing the incremental nature of
constant refactoring. Opponents will counter this with appeals to the
benefits of a middle ground where "just enough" design is partnered with
early prototyping \footnote{\emph{Extreme Programming Refactored}, M Stephens and D Rosenberg,
Apress, 2003}. Eventually, somebody makes comment on somebody
else's mother and her preference for military footwear, and all hope of
rational discussion is lost.

\begin{enumerate}
\item An Example Of The Hazards Of Emergent Design
\label{sec:orgheadline232}

As near as I can ascertain, the project upon which I am currently
working employs Emergent Design, although there has been no explicit
statement to that effect. At the beginning of the year there were one or
two group design sessions, which identified the major subsystems of the
product and how they would collaborate to achieve one of the principal
use cases. Since then, any design efforts which have occurred have been
of an incremental nature, and generally done "on the back of an
envelope" as individuals have struggled to implement various aspects of
a subsystem's functionality against pressing deadlines. Thus, developers
have only done what was necessary to achieve the functionality need for
the task at hand -- which seems consistent with the philosophy of
Emergent Design.

The resulting code base bears some interesting characteristics which I
believe illustrate some of the difficulties inherent with the practical
application of an Emergent Design approach. To illustrate, consider the
following three classes from the application's current code base,
presented here in abbreviated form:

\begin{verbatim}
public class YearLevel {
  public YearLevel(NormYearLevel,
                   Country, String, String);
  public getNormYearLevel() : NormYearLevel;
  public getCountry()       : Country;
  public getScanText()      : String;
  public getLabel()         : String;
}
public class NormYearLevel {
  public static final NormYearLevel
    NORM_YEAR_1 = new NormYearLevel(1);
  public static final NormYearLevel
    NORM_YEAR_12 = new NormYearLevcel(12);
  private NormYearLevel(int aYearLevel);
}
public class RawYearLevel {
  public RawYearLevel(String aScanText);
}
\end{verbatim}

The main purpose of this application is to process the responses of
junior and secondary school students to multiple choice exams. A given
exam may be taken by students from different countries and therefore
different educational systems. The results are captured in individual
and aggregate reports, which are printed and dispatched to the
participating schools.

It takes as input the data files resulting from the optical scanning of
the exam papers. Students indicate their "year level" as defined by the
educational system in force in their country (a "year" is variously
referred to as a "grade", "form" etc). For example, a student in year 3
in Australia would indicate a "3"; a student in Grade 4 in France would
indicate a "4" and so on.

What is notable about these three classes is that they represent three
different aspects of the same concept, and might well have been
collapsed into a single abstraction. More significant than the choice
and number of abstractions used to represent the concept, is the way
these disparate representations came into being. Each was created by a
different developer, working in a different subsystem from the others,
and employing a philosophy consistent with Emergent Design. A review of
the version control history for each class traces their genesis.

First came \texttt{RawYearLevel}, conceived of and implemented by a developer
concerned with the early stages of the data processing pipeline, as a
way of representing the student's literal indication of what year they
were in.

In parallel with \texttt{RawYearLevel}, the \texttt{YearLevel} class was created by a
second developer working in another subsystem, who was focusing on the
opposite end of the pipeline, where the results are embodied in hard
copy reports. The \texttt{YearLevel} class (without the \texttt{NormYearLevel}
association) captured enough information to print on a report "This
student was in Year 6" or "This student was in Grade 8", depending on
the country and the educational system it employed.

Lastly came the \texttt{NormYearLevel} class, created by a third developer
working in a subsystem between the two mentioned above, that was
responsible for calculating individual and population statistics. In the
course of these calculations it becomes necessary to relate a year level
in one country with its educational equivalent in another country. So
the concept of a Normative Year Level was introduced, and the
\texttt{YearLevel} abstraction was country-specific augmented to be associated
with it's normative equivalent.

Each of these classes has "emerged" from an individual developer's
immediate need to implement some portion of a subsystem's functionality.
To meet that need, they have done the simplest thing that could possibly
work \footnote{\url{http://xp.c2.com/DoTheSimplestThingThatCouldPossiblyWork.html}}. That often means writing a class from scratch. If another
developer creates the same or a similar abstraction in parallel, each
will be unaware of the duplication until their work is integrated.
Sometimes it is considered simpler to get partial leverage from an
existing abstraction. In either case, the imperative is to achieve the
target functionality as quickly as possible, such is the time pressure
the developers are under (a situation common to many development shops).
It is by no means certain that the design issues surrounding these
abstractions will \emph{ever} be revisited.

\item Just Refactor It
\label{sec:orgheadline233}

The inefficiency of maintaining the above three abstractions is
compounded by the amount of surrounding code that does little more than
map from one type to another. Proponents of Emergent Design would
suggest that the problem can be very simply overcome -- just refactor
the code. Of course, this is entirely possible. However there are some
very real reasons why the abstractions have persisted in the application
for 6 months or more, and have not been eliminated through refactoring.

\begin{itemize}
\item Nobody considers the refactoring to be of high enough priority to
warrant spending our limited developer resources on. The task is not
immediately related to any particular operational requirement, and so
it is viewed as being less important than making functional
progress.\\
\item There is considerable psychological inertia associated with a body of
code that is basically functional. Refactoring will mean losing that
functionality for the duration of the refactoring task, and so
superficially appears a retrograde step.\\
\item The classes have become part of the vocabulary of the developers, and
they have come to think of them as being an intrinsic part of the
system i.e. their presence is not openly questioned.
\end{itemize}

\item Constraining Evolution Leads To Mutants
\label{sec:orgheadline234}

Emergent Design is frequently likened to the process of evolution.
Proponents speak of "evolving a design" , the implication being that
some software equivalent of natural selection is weeding out the
inferior mutants, leaving only the fittest to survive. If this is the
case, why have the three classes above not evolved into a better design?
Or is that evolution yet to occur? Or are these three classes actually
the fittest to survive already, for some suitable definition of
"fittest"?

I conject that the practical application of Emergent Design so
constrains the evolution of the design elements that we cannot expect
such an approach to have a reasonable chance of giving rise to a good
design.

Comparing the evolution of a software design with the evolution of a
species, we see the following significant differences:

\begin{itemize}
\item Evolution can take its time exploring as many dead ends and genetic
cul-de-sacs as it likes. There is no supervising authority standing
by looking for visible signs of monotonic progress. There are no time
constraints or fiscal limitations that require evolution to produce a
workable result within a certain number of generations.\\
\item Evolution can explore many alternatives in parallel, but a
development group will rarely have sufficient resources to try a
large number of different design alternatives in parallel. A very
limited number of resources assigned to a design task must try
alternatives in series, if at all. Obviously there is a strong
tendency to stick with the first one tried that appears to hold
promise.\\
\item Evolution is objective in its evaluation of the success of each
alternative. There is no attachment to a genetic alternative that is
nearly good enough. However software developers often favor "pet"
design approaches, or try and force non-optimal designs further than
they should go because there is the promise of success just around
the corner, and the attendant resolution of an uncompleted task. That
is to say, it is very human to normalize deviation.\\
\item Evolution is not required to be predictable. No one has bet their
financial future on the lesser fairy penguin evolving heat
dissipation mechanisms to cope with increasing Arctic temperatures,
and doing it in no more than 3 generations. But stakeholders in
software development efforts will commonly invest large sums to see
successful designs produced (and thereby business problems solved)
within a limited contract period.
\end{itemize}

You will find any number of elegant analogies in the Emergent Design
literature -- but finding one that addresses the above constraints is
quite another matter.

For example, there is the delightful story (probably apocryphal) of the
landscaping engineer who was asked to cement pathways at a University,
after the buildings had been erected. Rather than predict the correct
place to put the pathways, the engineer stood back for one semester and
let the students make their own way between buildings. The furrows they
wore in the ground were adopted as the courses for the cement pathways.

How very Zen\ldots{} really, it's a terrific tale. I love it. But before we
spin our prayer wheels and marvel at the engineers' wisdom, let's think
of the liberties that the landscape engineer was allowed in pursuing
such a solution method. Liberties which would be denied a great many
University contractors in the real world:

\begin{itemize}
\item The landscape engineer was allowed to take the time necessary to wait
for the paths to emerge. What if the University had required
completion sooner than that -- say, before the semester started?\\
\item The landscape engineer was allowed an entire semester in which he was
not required to demonstrate visible progress. What if a competitor
had taken advantage of this lull and offered to complete the job
using best guesses of the correct routes for the pathways.\\
\item The landscape engineer was free to distribute the labor and materials
cost over the course of the project as he saw fit. What if the
budgeting system of the University had made allowances for
expenditure on landscaping in this semester, but not in the following
one?\\
\item An entire University cohort spent a semester walking through the mud
after every rainfall. They were willing to put up with this
discomfort so that the engineer could let his design emerge. I wonder
how the senior lecturers felt about this. More importantly, I wonder
how those students in wheelchairs coped.
\end{itemize}

Emergent Design has the capacity to lead to some very elegant solutions
-- eventually. That design may be wonderfully efficient -- if you have
the financial stamina to await its arrival and the confidence that you
will recognize it when it appears.

\item Conclusion
\label{sec:orgheadline235}

Does Emergent Design work? Of course - just look in the mirror. You and
every other product of evolution is testament to the potential success
of the approach.

Does that imply that it is a suitable model for designing software? No.

While the idea has aesthetic appeal, the practical context in which the
emergence occurs makes all the difference. The requirements for
timeliness and predictability in a software development project,
together with the subjective nature of those who gauge the cost/benefit
of a particular approach, mean that true, uninhibited evolution cannot
occur. If the compromises embodied in an emergent design are consistent
with our corporate priorities, then it will be by coincidence only --
and that's too important a matter to leave to chance.
\end{enumerate}

\subsubsection{The Top Ten Elements of Good Software Design \footnote{First published 18 May 2004 at
\url{http://www.hacknot.info/hacknot/action/showEntry?eid=54}}}
\label{sec:orgheadline247}

\begin{quote}
“/You know you've achieved perfection in design, not when you have
nothing more to add, but when you have nothing more to take away./” --
Antoine de Saint-Exupery
\end{quote}

Much is spoken of "good design" in the software world. It is what we all
aim for when we start a project, and what we hope we still have when we
walk away from the project. But how do we assess the "goodness" of a
given design? Can we agree on what constitutes a good design, and if we
can neither assess nor agree on the desirable qualities of a design,
what hope have we of producing such a design?

It seems that many software developers feel that they can recognize a
good design when they see or produce one, but have difficulty
articulating the characteristics that design will have when completed. I
asked three former colleagues -- Tedious Soporific, Sparky and Willa
Wonga -- for their "Top 10 Elements of Good Software Design". I combined
these with my own ideas, then filtered and sorted them based upon
personal preference and the prevailing wind direction, to produce the
list you see below. A big thanks to the guys for taking the time to
write up their ideas.

Below, for your edification and discussion, is our collective notion of
the \emph{Top 10 Elements of Good Software Design}, from least to most
significant. That is, we believe that a good software design \ldots{}

\begin{enumerate}
\item 10. Considers The Sophistication Of The Team That Will Implement It
\label{sec:orgheadline237}

Does it seem odd to consider the builder when deciding how to build? We
would not challenge the notion that a developer's skill and experience
has a profound effect on their work products, so why would we fail to
consider their experience with the particular technologies and concepts
our design exploits? Given fixed implementation resources, a good design
doesn't place unfamiliar or unproven technologies in critical roles,
where they become a likely point of failure.

Further, team size and their collocation (or otherwise) are considered.
It would not be unusual for such a design's structure to reflect the
high level structure of the team or organization that will implement it.

\item 9. Uniformly Distributes Responsibility And Intelligence
\label{sec:orgheadline238}

Classes containing too much intelligence become both a point of
contention for version control purposes, and a bottleneck for
maintenance and development efforts. They also suggest that a class is
capturing more than a single data abstraction.

\item 8. Is Expressed In A Precise Design Language
\label{sec:orgheadline239}

The language of a design consists of the names of the entities within
it, together with the names of the operations those entities perform. It
is easier to understand a design expressed in precise and specific
terms, as they provide a more accurate indication of the purpose of the
entities and the way they cooperate to achieve the desired
functionality. Look for the following features:

\begin{itemize}
\item The objective of the designed thing can be described in one or two
sentences completely.\\
\item The interface requirements of the entities are stated precisely.
\item The contracts between an entity and its callers are stated precisely
and contract adherence is enforced programmatically (Design by
Contract).\\
\item Entities are named with accurate and concrete terms, and specified
fully enough to form a suitable basis for implementation.
\end{itemize}

\item 7. Selects Appropriate Implementation Mechanisms
\label{sec:orgheadline240}

Certain mechanisms are problematic and more likely to produce
difficulties at implementation time. A good design minimizes the use of
such mechanisms. Examples are:

\begin{itemize}
\item Reflection and introspection
\item Dynamic code generation
\item Self-modifying code
\item Extensive multi-threading
\end{itemize}

Sometimes the use of such mechanisms is unavoidable, but at other times
a design choice can be made to sacrifice more complex, generic
mechanisms for those easier to manage cognitively.

\item 6. Is Robustly Documented
\label{sec:orgheadline241}

As long as a design lies hidden in the complexities of the code, so too
does our ability arrive at an understanding of the code's structure as a
whole. As the abstract structure becomes apparent to us, either through
rigorous examination of the code or study of an accompanying design
document, we gradually develop a course understanding of the code's
topography. A good design document is used before or during
implementation as a justification and guide, and after construction as a
way for those new to the code base to get an overview of it more quickly
than they can through reverse engineering. Captured in abstract form, we
can discuss the pros and cons of different approaches and explore design
alternatives more quickly than we can if we were instead manipulating a
code-level representation of the design.

But as soon as the abstract and detailed records of a design part
company, discrepancy between the two becomes all but inevitable.
Therefore it is essential to document designs at a level of detail that
is sufficiently abstract to make the document robust to changes in the
code and not unnecessarily burdensome to keep up to date. A good design
document should place an emphasis upon temporal and state relationships
(dynamic behavior) rather than static structure, which can be more
readily obtained from automated analysis of the source code. Such a
document will also explain the rationale behind the principal design
decisions.

\item 5. Eliminates Duplication
\label{sec:orgheadline242}

Duplication is anathema to good design. We expect different instances of
the same problem to have the same solution. To do otherwise introduces
the unnecessary burden of understanding two different solutions where we
need only understand one. There are also attendant integrity problems
with maintaining consistency between the two differing solutions. Each
design problem should be solved just once, and that same solution
applied in a customized way to different instances of the target
problem.

\item 4. Is Internally Consistent And Unsurprising
\label{sec:orgheadline243}

We often use the term "intuitive" when describing a good user interface.
The same quality applies to a good design. Something is "intuitive" if
the way you expect (intuit) it to be is in accord with how it actually
is. In a design context, this means using well-known and idiomatic
solutions to common problems, resisting the urge to employ novelty for
its own sake. The philosophy is one of "same but different" -- someone
looking at your design will find familiar patterns and techniques, with
a small amount of custom adaptation to the specific problem at hand.
Additionally, we expect similar problems to be solved in similar ways in
different parts of the system. A consistency of approach is achieved by
employing common patterns, concepts, standards, libraries and tools.

\item 3. Exhibits High Cohesion And Low Coupling
\label{sec:orgheadline244}

Our key mechanism for coping with complexity is abstraction -- the
reduction of detail in order to reduce the number of entities, and the
number of associations between those entities, which must be
simultaneously considered. In OO terms this means producing a design
that decomposes a solution space into a half dozen or so discrete
entities. Each entity should be readily comprehensible in isolation from
the other design elements, to which end it should have a well defined
and concisely stateable purpose. Each entity, be it a sub-system or
class, can then be treated separately for purposes of development,
testing and replacement. \emph{Localization of data and separation of
concerns} are principles which lead to a well decomposed design.

\item 2. Is As Simple As Current And Foreseeable Constraints Will Allow
\label{sec:orgheadline245}

It is difficult to overstate the value of simplicity as a guiding design
philosophy. Every undertaking regarding a design -- be it
implementation, modification or rationalization -- begins with someone
developing an understanding of that design. Both a detailed
understanding of a particular focus area, and a broader understanding of
the focus area's role in the overall system design, are necessary before
these tasks can commence.

It is necessary to distinguish between accidental and essential
complexity \footnote{\emph{The Mythical Man Month, Anniversary Edition}, F. Brooks,
Addison-Wesley,, 1995}. The essential complexity of a solution is that which is
an unavoidable ramification of the complexity of the problem being
solved. The accidental complexity of a solution is the additional
complexity (beyond the essential complexity) that a solution exhibits by
virtue of a particular design's approach to solving the problem. A good
design minimizes accidental complexity, while handling essential
complexity gracefully. Accidental complexity is often the result of the
intellectual conceit of the designer, looking to show off their design
"chops." Sometimes a "simple" approach is misinterpreted as being
"simple-minded." On the other hand, we might make a design too simple to
perform efficiently. This seems to be a rather rare occurrence in the
field. As the scope of software development broadens at the enterprise
level and attracts greater essential complexity, the reduction of
accidental complexity becomes ever more important.

\item 1. Provides The Necessary Functionality
\label{sec:orgheadline246}

The ultimate measure of a design's worth is whether its realization will
be a product that satisfies the customer's requirements. Software
development occurring in a business context must provide business value
that justifies the cost of its construction. Also of significant
importance is the design's ability to accommodate the inevitable
modifications and extensions that follow on from changes in the business
environment in which it operates.

But it is necessary to exercise great caution when predicting future
requirements. An excessive focus upon anticipatory design can easily
result in wasted effort resulting from faulty predictions, and encumber
a design with unnecessary complexity resulting from generic provisions
which are never exploited. Terms like "product line" and "framework" may
be warning signs that the design is making high-risk assumptions about
the future requirements it will be subject to.

It is easy to overlook the non-functional requirements (e.g. performance
and deployment) incumbent upon the design. Taking different "views" of
the design, in the manner of the "4+1" architectural views in RUP \footnote{\emph{Rational Unified Process}, P Kruchten, Addison-Wesley, 1999},
can help provide confidence that there are no gaping holes (functional
or otherwise) and that the design is complete.
\end{enumerate}

\subsection{Documentation}
\label{sec:orgheadline256}

\subsubsection{Oral Documentation: Not Worth the Paper it's Written On \footnote{First published 10 Jun 2004 at
\url{http://www.hacknot.info/hacknot/action/showEntry?eid=57}}}
\label{sec:orgheadline249}

The Agile Manifesto \footnote{\url{http://www.agilemanifesto.org}} states:

\begin{quote}
"\emph{The most efficient and effective method of conveying information to
and within a development team is face-to-face conversation.}"
\end{quote}

Forgive me for questioning a holy proclamation, but isn't it rather well
established that verbal communication is often incomplete and ambiguous,
and that human memory is inaccurate and prone to confabulation? The
plethora of psychological research in such areas as false memories, the
veracity of eyewitness testimony, and the effect of predisposition on
the interpretation of sensory data has surely given us a big hint that
our perceptual and communicative capabilities are erratic and dubitable?

So where comes the apparently wide spread acceptance of (or at least,
lack of challenge to) such outrageous Agile sophistry? For my part, it
is difficult to ignore the manifest problems associated with a
development team's reliance upon face-to-face communication. Over the
last 3 or 4 months, as the inheritor of a code base whose authors
preferred the "verbal tradition" style of documentation, I suffer daily
from the flow-on effects of this laziness. Let me illustrate by
providing you with a summary of a typical day for me in recent months,
so you too can marvel at the feel-good richness and super-duper
efficiency of face to face communication amongst software developers.

\emph{Fade in}.

/Scene 1 - a cubicle. Ed is slouched in an office chair staring
forlornly at the screen in front of him. Except for the occasional
insouciant jab at his keyboard, he gives the appearance of being
comatose/.

The day begins with my desire to extend the functionality of a legacy
application, approximately 600K lines of code. I need to locate that
portion of the code responsible for performing function X, so that I can
insert function Y just after it. I go looking for function X amongst the
code. I can't find it. In fact, I started looking for it sometime
yesterday, and haven't found it yet. I check the folder marked "docs",
to find it contains only a single \texttt{README.txt} file, the sole contents
of which is the teaser "This directory will contain the docs" --
apparently the dying message of a long extinct group of developers whose
brains exploded before being able to make good on their promise. I find
a piece of code that looks like it's in the same ballpark as the code
I'm looking for, and examine the revision history of the file it is in,
to find that it has principally been developed by "Bob". I must find
Bob. I need to find Bob. Bob will know where function X is.

Here is my first problem. I cannot contact Bob directly, because I am
but a lowly contractor. Bob is a valuable and in-demand member of my
client's staff, and I can't just go up to him and steal his valuable
time. There's a chain of command to be observed here! I must lodge a
request with my manager to see Bob, who will forward that request to a
liaison officer, who will forward that request to Bob's manager, who
will then cue it up with Bob. If he's not too busy.

\emph{Scene 2 - a meeting room. Ed sits opposite a brown-skinned man wearing
a turban}.

The next day, I get to meet Bob. He can only spare 15 minutes to talk to
me, because he's busy preparing for the next release of some whiz-bang
new pile of crud. It's at this point that I discover that Bob's real
name is "Sharmati Sanyuktananda", but everyone just calls him "Bob" for
short. Bob is Indian. Bob's formal exposure to English was limited to
the 15 minutes he spent reading "Miffy Learns English" while waiting in
line at Immigration for his visa to be processed.

I try and talk with Bob, but it is like talking with Dr Seuss. At the
end of 15 minutes, I have learnt almost nothing from him, and he keeps
repeating something about public transport, which seems to have no
relevance. His final word is "Sue", who I know is another member of the
client's staff. So I contact my manager to organize some time with Sue.

\emph{Scene 3 - a meeting room. Ed sits opposite a nerdish looking woman
wearing glasses with a very strong prescription.}

Next day, I discover, to my significant relief, that Sue speaks English
quite well. Unfortunately, her memory is a little hazy on the bit of
code I'm asking her about. She remembers dealing with it about a year
ago, but there's been a lot of water under the bridge since then. At
this point, I am beginning to consider tying weights around my feet and
jumping off that bridge. She can't tell me where functionality X is, but
she's pretty sure it isn't where I'm looking. "Have you tried asking
John?", she queries. So I contact my manager and request a meeting with
another client staff member, John.

\emph{Scene 4 - a meeting room. Ed sits opposite a cool dude with sideburns
and shoulder length hair.}

Next day, John is disarmingly candid about the code I'm dealing with.
"Oh yeah, I remember this crap", he begins. "We wrote that it in about a
week, sometime last year, when we were up against the wall. It is
absolute rubbish." "No kidding", I think. John is my guardian angel --
he knows that function X got ripped out at the last moment, so they
could meet their deadline. But then they put it back in a bit later,
when things slowed down, and it's kept in a different module in the
version control system. Which one? "You'll have to ask Declan", says
John in a matter of fact way. I ask my manager to queue up some time
with Declan.

\emph{Scene 5 - a cubicle. Ed is slouched in an office chair, browsing the
advertisements on an employment web site.}

My manager replies a few hours later, saying that Declan left the
company a few months ago -- maybe someone else knows. Have I tried
asking Bob?

\emph{Fade to black.}

And that, ladies and gentlemen, is the delight of face-to-face
communication amongst software developers. See how efficient and
effective it is? No one wasted any time writing nasty old documents,
which saved them a bit of time -- once. Everyone since then has wasted
the time they saved, multiplied tenfold, trying to recover the
information the original author could have documented in an hour or two,
but was too busy, choosing to rely instead on good old "face to face"
communication.

When it comes to the maintenance and extension of legacy code, and
clearing the organizational hurdles associated with the handover of code
from one party to another, a reliance on "face to face" communication is
very convenient for the first generation of developers, and a chain
around the leg of every other developer and organization involved
thereafter.

It all sounds very folksy and appealing when you just say the words. If
you're just talking in general terms about how much easier it is to have
a bit of a chin wag with the bloke sitting next to you, then it sounds
so reasonable to point out how much is being saved by just talking about
stuff rather than writing it down. Of course! We'll just have a little
chat about it and everything will be alright. That same simplicity is a
large part of its appeal to many developers. Unfortunately, reality is
not quite so simple.

For a maintenance programmer, the reality of dealing with your
predecessor's reliance upon "oral documentation" is:

\begin{itemize}
\item The people you need to talk to are often not available -- their time
may be spoken for, or they may have left the company.\\
\item The people that are available to talk with are often inarticulate
techies with the verbal communication skills of a mime.\\
\item The people you talk to have fuzzy memories, particularly where low
level details are concerned. Frequently, they simply can't recall the
information you need.\\
\item The people you talk to all give you a different account of how things
work. You're not getting the facts anymore, you're getting opinions
and best guesses.\\
\item The people you talk to have moved on to new duties and are not
particularly interested in answering your queries about a system they
would prefer to forget.
\end{itemize}

The "out" offered by XP/AM \footnote{Extreme Programming / Agile Methods} and other idealistic retreats is that
you just "do the documentation as needed". Brilliant! If only I'd
thought of that, maybe I could've been a thought leader too! The problem
is, "as needed" and "when time is available" are rarely coincident for
reasons entirely beyond the developer's control. Try and convince a
manager that you need to take a week out to catch up on some
documentation. During that week you won't be writing code, you won't be
making any functional progress towards a measurable or billable outcome,
but the schedule will be taking a hit. Good luck with that one.

Fowler has a few delightful stories of "handover" scenarios in which
face-to-face communication has been achieved by paradropping an
"ambassador" into an enemy territory full of maintenance programmers, so
that knowledge can be still be transferred verbally, and documentation
produced as required by those maintenance programmers. I would like to
enunciate a question that has long been in my mind, but heretofore
unexpressed: "Martin, what part of the Twilight Zone do you live in, and
where can I get a ticket?" Really \ldots{} is it just me or do the folksy
anecdotes and one-off case studies that some Agile enthusiasts put
forward sound just a little too contrived to be realistically
transferred to your average corporate setting? Where are these companies
they speak of, that have the latitude to abandon their normal procedures
and protocols and set about bending over backwards in an effort to
provide just the right climate to support these processes, no matter how
involved the accommodation may be?

Whenever I read these fabulous accounts of the stunning success of AM/XP
in some corporate environment, and how it didn't really matter that the
team prepared no documentation whatsoever, I feel like I'm reading some
sort of fairy tale, where everybody finishes their projects without
difficulty, and then goes off to have a picnic in some bucolic setting,
where they eat cucumber sandwiches and drink lashings of ginger beer.
Hurrah!

By contrast, here's how handover happens in my world. One day --
sometime before you've actually finished what you're working on -- some
pointy-haired manager comes up to you and says "You're changing to
Project W tomorrow". No thought, no discussion, no campfire chat and
singing of old spirituals. Just the immediate transferal of resources
from one emergency to the next emergency. Whatever difficulties you
might leave behind -- too bad. What happens to the programmers that come
after you is of no immediate concern. This dooms the poor sods to
spending inordinate amounts of time, as I have recently, wandering the
halls like a restless spirit, shuffling from one vague and apathetic
source of information to the next.

The reliance upon face-to-face communication that the XP/AM contingent
favor is not the straighttalking, light-weight, near-telepathic
communicative fantasy of the Agile dream, but a prescription for pain
and suffering for every maintenance programmer that has to come along
and clean up after the original programming team has done a hitand-run
on the code base.

Are my experiences unique here, or do others find this whole "fireside
chat" model of developer communication a little hard to swallow?

\subsubsection{FUDD: Fear, Uncertainty, Doubt and Design Documentation \footnote{First published 27 Jan 2004 at
\url{http://www.hacknot.info/hacknot/action/showEntry?eid=46}}}
\label{sec:orgheadline255}

\begin{quote}
"\emph{Think twice, cut once}" -- Carpenter's adage
\end{quote}

In the years that I've been doing software development, the one source
of recurring dispute between myself and colleagues is the issue of
design documentation. I am of the opinion that the production and review
of design documentation significantly increases the chances of producing
quality software, and that such documentation should be an integral part
of the development of any piece of commercial software.

In the course of advancing this argument, I believe I have heard every
counter-argument known to man (or "excuses," as I prefer to call them).
It would require a small book to document them thoroughly, in all their
variation and inventiveness, but the following list covers the main
ones:

\begin{itemize}
\item We have a tight schedule and the sooner I begin coding, the better.
\item The document will quickly drift out of synch with the code.
\item I can always produce a design document later, if I have to.
\item No one looks at design documents anyway.
\item The information you capture can be obtained directly from the code.
\item I'm paid to write software, not technical documents.
\item The customer wants working software, not documents.
\item Nobody does Big Design Up Front anymore.
\item Never had to do it on any of my previous projects.
\item Everyone on the team knows how the system is designed.
\item A good design will emerge once we begin coding.
\item It's better just to write the code, then recover the design later
with a
\end{itemize}

CASE tool.

\begin{itemize}
\item I comment the source code thoroughly.
\item You can't really understand how the software will work until you
write the code.
\end{itemize}

I'm not going to try and disprove any of these statements. The state of
empirical research in the area and the vagueness of many of the
statements themselves forbids disproof. Additionally, it is quite
possible to develop and deliver software without a shred of design
documentation. Indeed, it is common practice.

But I believe that we can do \emph{better} with design documentation than
without it. In other terms, though a tradesman might achieve his end
with blunt tools, the going is harder and the result messier than if he
had used sharp tools. My experience suggests that design documentation
is a sharp tool that we blunt with our own misconceptions and false
beliefs about the role it plays in the development process. Given that I
can't prove that to you, I will try and persuade you of it by
challenging some of the beliefs underlying the above statements.

It should first be acknowledged that for many developers, the notion of
writing documentation of any type is a task they anticipate with the
same distaste as root canal work. In other words, any of the above
stated reasons for eschewing design documentation may really just be an
attempt to rationalize the real reason:

\begin{quote}
\emph{I hate writing documentation}.
\end{quote}

I believe the enmity toward documentation that we see so much of in the
development community derives largely from the cognitive shortcomings
(real or perceived) of the average software developer. Many developers
come from mathematics, science and engineering backgrounds, and talent
in those areas is often accompanied by a proportional lack of ability in
the humanities. Documentation requires expression in natural language,
and a disturbing number of developers have approximately the same
facility with the written word as a high school junior. Nobody enjoys
doing things that they're no good at. It's frustrating and tiring.

From the reasons given above, I have tried to distill the core
underlying beliefs.

\begin{itemize}
\item Well written/commented code substitutes for design documentation
\item The team already knows the design, so there's no need to document it
\item Code is the only meaningful work product and sign of progress
\item The maintenance cost of design documentation is prohibitively high
\item 
\end{itemize}

Let me challenge each of these beliefs in turn.

\begin{enumerate}
\item Well Written/Commented Code Substitutes For Design Documentation
\label{sec:orgheadline250}

Design documentation can provide value before the code is even written.

Senior technical staff frequently maintain an architecture-level view of
the system being developed, leaving front-line developers to focus on
whatever functional area they are currently preoccupied with. These are
two distinctly different mindsets, and switching back and forth between
them is tiring. When you've got your head buried in a complex
multi-threading problem, you're not inclined to be thinking about how
your code fits into the overall scheme of things. Similarly, when you're
sorting out architectural issues, you're not concerned with lower level
implementation details. By having the design of a low level subsystem
reviewed by someone with a high level view of system structure, we can
ensure that individual units of work go together in an architecturally
consistent manner.

Additionally, the very act of externalizing a design to a level of
detail that convinces a reviewer that it is sufficient, can lead the
developer to discover aspects of the problem they might otherwise gloss
over in their haste to begin coding. The problem with "back of the
envelope" designs and hastily scribbled whiteboard designs is that they
make it easy to overlook small but problematic details.

\item The Team Already Knows The Design, So There's No Need To Document
\label{sec:orgheadline251}
It

Those who have taken part in the construction of a system have had the
opportunity to witness the evolution of its design and absorb it in a
piecemeal fashion over a period of time. But new team members and
maintainers are thrown in at the deep end and confronted with the
daunting task of gaining sufficient familiarity with an unknown body of
code to enable them to fix and enhance it. For these developers, design
documentation is a blessing. It enables them to quickly acquire an
abstract understanding of a body of code, without having to tediously
recover that information from the code itself. They can come up to speed
with greater ease and more quickly than they might without the guidance
of the design documentation.

\item Code Is The Only Meaningful Work Product And Sign Of Progress
\label{sec:orgheadline252}

This statement is true if the only lifecycle activity you recognize is
coding, and the only goal towards which you proceed is "code complete."
As a design matures and different aspects of the solution space are
explored, the designers' understanding of the problem deepens. This
progress in understanding is real progress towards a solution, even
though it is not captured in code. The exploration and evaluation of
design alternatives is also real progress, the end result of which is
captured in a design document.

\item The Maintenance Cost Of Design Documentation Is Prohibitively High
\label{sec:orgheadline253}

Many developers view design documentation as a programmatic
after-thought; something that you do after the real work of writing code
is done, perhaps to satisfy a bureaucrat and create a paper trail. Any
type of documentation produced in such a desultory fashion and out of a
sense of obligation is likely to be low in quality, and of little use.
So the preconception becomes a self-fulfilling prophecy.

It's not difficult at all to create useful design documentation, as long
as you know what use you're going to put it to. I've found that useful
design documentation can be achieved by following these two simple
guidelines:

\begin{enumerate}
\item Include only those details that have explanatory power. There's no
need to put every class on a class diagram, or to include every
method and attribute. Only include the most significant classes, and
only those features that are critical to the class's primary
responsibilities; generally, these are the public features. Omit
method arguments if you can get away with it. In other words, seek
minimal sufficiency. This also makes the resulting document more
robust to change.\\
\item Focus on dynamic behavior, not static structure. If possible,
restrict yourself to a single class diagram per subsystem.
Associations and inheritance hierarchies are relatively easy to
recover from source code, but the interactions that occur in order to
fulfill a subsystem's main responsibilities are much harder to
identify from the code alone. This is why reverse engineering of
interactivity diagrams by CASE tools is ubiquitously done poorly. The
primary function of the design document is to explain how the classes
interact in order to achieve the most important pieces of
functionality
\end{enumerate}

That code can be written in such a way as to obviate the need for
documentation is a retort of the documentation-averse that I've been
hearing for many years. Those not keen on commenting their code will
appeal to the notion of "self-commenting code". Those not keen on design
documentation will claim "the code is the design". This phrase, as it is
commonly used, is intended to convey the idea that the code is the only
manifestation/representation of the software's design that can be
guaranteed to be accurate. While a design document will drift out of
synch with the code, the code will always serve as the canonical
representation of the design it embodies.

I believe such reasoning constitutes a scarecrow argument in that it
presents an image of design documentation as necessarily so detailed and
rigorous that it is fragile and brittle. Certainly it is possible to
write design documentation in that manner, but it is also possible to
make it quite robust by exercising some common sense regarding content
and level of detail.

To the XPers who promote such fallacies, I would ask this:

\begin{quote}
“/If you believe you can write code in such a way that the cost of
change becomes negligible, why can't you employ those same techniques
to write design documentation with the same properties? A design
document does not demand the same accuracy or contain the same
complexity as source code; so why can't you just refactor a design
document with the same ease with which you refactor your code?/”
\end{quote}

This inconsistency points to "the code is the design" argument as a
failed attempt to rationalize personal preference. Twiddling with the
code is fun, twiddling with diagrams is not (apparently).

\item Conclusion
\label{sec:orgheadline254}

Explicit consideration of design as a precursor to implementation has
numerous benefits, most of which have their origin in the limited
abilities of the brain to cope with complexity. Embarrassingly, there
are those in our occupation who would deny the applicability of the
mechanisms commonly employed in other fields to cope with these
limitations. Abstraction, planning and forethought are as useful to
software engineers as civil engineers. Design recovery from complex
artifacts is just as difficult for us as for those in other
construction-based occupations.

To get value from design documentation:

\begin{itemize}
\item Make it a part of your development cycle - don't treat it as an
optional afterthought. Document as part of the design of each
subsystem (NB: design documentation does not imply BDUF).
\item Keep it as concise as possible, in the interests of maintainability.
\item Eschew CASE tools offering round trip engineering and use a simple
drawing tool (personally, I like the UML stencils in Visio).\\
\item Concentrate on capturing dynamic behavior rather than static
structure.
\end{itemize}
\end{enumerate}

\subsection{Programming}
\label{sec:orgheadline278}

\subsubsection{Get Your Filthy Tags Out of My Javadoc, Eugene \footnote{First published 6 Aug 2003 at
\url{http://www.hacknot.info/hacknot/action/showEntry?eid=14}}}
\label{sec:orgheadline263}

Recently I've been instituting a code review process on a number of
projects in my workplace. To kick start use of the process, I took a
sample of the Java code written by each of my colleagues and reviewed
it.

While doing so I was struck by the degree of individual variation in the
use of Javadoc comments, and reminded of how easy it is to fulfill one's
obligation to provide Javadoc without really thinking about how
effectively one is actually communicating.

I think the quality of Javadoc commenting is important because - let's
be honest - it's the only form of documentation that many systems will
ever have.

Here are some of the problems in Javadoc usage that I frequently
observe:

\begin{itemize}
\item Developers never actually run the Javadoc utility to generate HTML
documentation, or do so with such irregularity they can have no
confidence that their copy of the HTML documentation is up to date.\\
\item Developers use their IDE's facility to autogenerate a comment
skeleton from a method signature, but then fail to flesh out that
skeleton.\\
\item HTML tags are overused, severely impairing the readability of
comments when viewed as plain text.\\
\item Comment text is diluted with superfluous wording and duplication of
information already conveyed by data types.\\
\item Valuable details are omitted e.g. method preconditions and
post-conditions, the types of elements in Collections and the range
of valid values for arguments (in particular, whether an object
reference can be null).\\
\item The conventional single sentence summary at the beginning of a method
header comment is omitted.\\
\item Non-public class features are not commented.
\end{itemize}

My conclusion is that many developers are just "going through the
motions" when writing Javadoc comments. With a little more thought, more
effective use of both the author's and the reader's time can be made.

I propose the following guidelines for effective Javadoc commenting \ldots{}

\begin{enumerate}
\item Do Not Use HTML Tags
\label{sec:orgheadline257}

This maximizes the readability of the comment when viewed \emph{in situ}, and
saves the author some time (which is better spent adding meaningful text
to the comment). Use simple typographic conventions \footnote{\url{http://docutils.sourceforge.net/rst.html}} to create
tables and lists.

\item Javadoc All Class Features, Regardless Of Scope
\label{sec:orgheadline258}

While third parties using your code as an API don't need it, the
developers and maintainers of your code base do - and they are your
principal audience.

\item Don't Prettify Comments
\label{sec:orgheadline259}

Cute formatting such as lining up the descriptions of \texttt{@param} tags
wastes space you could devote to meaningful description and makes the
comments harder to maintain.

\item Drop The Description For Dimple Accessors
\label{sec:orgheadline260}

For methods that simply set or get the value of a class attribute, this
sentence duplicates the information contained in an \texttt{@param} or
\texttt{@return} clause respectively.

\item Assume \texttt{Null} Is Not OK
\label{sec:orgheadline261}

Adopt the convention that object references can not be \texttt{null} unless
otherwise stated. In the few circumstances where this is not true,
specifically mention that \texttt{null} is OK, and explain what significance
the \texttt{null} value has in that context.

Use Terse Language

Feel free to use phrases instead of full sentences, in the interest of
brevity. Avoid superfluous references to the subject like "This class
does \ldots{}", "Method to \ldots{}", "An integer that \ldots{}", "An abstract class
which \ldots{}".

\item Be Precise
\label{sec:orgheadline262}

\begin{itemize}
\item For classes: precisely describe the object being modeled.
\item For methods: describe the range of valid values for each \texttt{@param} and
\texttt{@return}.
\item For fields: describe the types of objects in Collections and the
range of valid values.
\end{itemize}
\end{enumerate}

\subsubsection{Naming Classes: Do it Once and Do it Right \footnote{First published 9 Mar 2004 at
\url{http://www.hacknot.info/hacknot/action/showEntry?eid=48}}}
\label{sec:orgheadline273}

The selection of good class names is critical to the maintainability of
your application. They form the basic vocabulary in which developers
speak and the language in which they describe the code's every activity.
No wonder then that vague or misleading class names will quickly derail
your best efforts to understand the code base.

Because we are called on to invent class names so frequently, there is a
tendency to become somewhat lackadaisical in our approach. I hope the
following guidelines will assist you in devising meaningful class names,
and encourage you to invest the effort necessary to do so. As always,
these are just guidelines and ultimately you should use your own
discretion.

\begin{enumerate}
\item 1. A Class Name Is Usually A Noun, Possibly Qualified
\label{sec:orgheadline264}

The overwhelming majority of class names are nouns. Sometimes you use
the noun by itself:

\begin{itemize}
\item \texttt{Image}
\item \texttt{List}
\item \texttt{Position}
\item \texttt{File}
\item \texttt{Exception}
\end{itemize}

Other times you qualify the noun with one or more words which help to
specialize the noun:

\begin{center}
\begin{tabular}{ll}
Class Name & Grammatical Breakdown\\
\hline
\texttt{JPEGImage} & The noun \texttt{Image} is qualified by the noun \texttt{JPEG}\\
\texttt{LinkedList} & The noun \texttt{List} is qualified by the adjective \texttt{Linked}\\
\texttt{ParsePosition} & The noun \texttt{Position} is qualified by the verb \texttt{Parse}\\
\texttt{RandomAccessFile} & The noun \texttt{File} is qualified by the adjective \texttt{Random} and the verb \texttt{Access}\\
\texttt{FormException} & The noun \texttt{Exception} is qualified by the noun \texttt{Form}\\
\end{tabular}
\end{center}

When searching for a noun to serve as a class name, consider the
following suffixes which are often used to form nouns from other
words: \footnote{\emph{Bloomsbury Grammar Guide}, Gordon Jarvie}

\begin{center}
\begin{tabular}{ll}
Suffix & Example Class Names\\
\hline
-age & \texttt{Mileage}, \texttt{Usage}\\
-ation & \texttt{Annotation}, \texttt{Publication}, \texttt{Observation}\\
-er & \texttt{User}, \texttt{Broker}, \texttt{Listener}, \texttt{Observer}, \texttt{Adapter}\\
-or & \texttt{Decorator}, \texttt{Creditor}, \texttt{Author}, \texttt{Editor}\\
-ness & \texttt{Thickness}, \texttt{Brightness}, \texttt{Responsiveness}\\
-ant & \texttt{Participant}, \texttt{Entrant}\\
-ency & \texttt{Dependency}, \texttt{Frequency}, \texttt{Latency}\\
-ion & \texttt{Creation}, \texttt{Deletion}, \texttt{Expression}, \texttt{Enumeration}\\
-ity & \texttt{Plasticity}, \texttt{Mutability}, \texttt{Opacity}\\
-ing & \texttt{Tiling}, \texttt{Spacing}, \texttt{Formatting}\\
-al & \texttt{Dismissal}, \texttt{Removal}, \texttt{Committal}\\
\end{tabular}
\end{center}

\item 2. Avoid Class Names That Have Non-Noun Interpretations
\label{sec:orgheadline265}

Suppose that while maintaining an application you come across a class
called \texttt{Empty}. As a noun, instances of \texttt{Empty} might represent a state
in which some vessel is devoid of contents. However the word "empty" can
also function as a verb, being the act of removing all the contents of a
vessel. So there is potential confusion as to whether the class models a
state or an activity. This ambiguity would not arise if the class had
been called \texttt{EmptyState} or \texttt{EmptyActivity}.

\item 3. A Class Name Is Sometimes An Adjective
\label{sec:orgheadline266}

There is a special type of class called a \emph{structural property
class} \footnote{\emph{Object Oriented Software Construction, 2nd Edition}, Bertran Meyer}, which is often named with an adjective. Such classes exist
to confer specific structural properties upon their subclasses (or
implementers, in the case of interfaces). They are often suffixed with
\emph{-able}. Examples include:

\begin{itemize}
\item \texttt{Comparable}
\item \texttt{Undoable}
\item \texttt{Serializable}
\item \texttt{Printable}
\item \texttt{Drawable}
\end{itemize}

\item 4. Use Commonly Accepted Domain Terminology
\label{sec:orgheadline267}

Specialist domains come ready-made with their own vernacular. This can
be both a curse and a blessing. The down side is that newcomers to the
domain have a lot of new terminology to master. The up side is that,
once mastered, that terminology makes for efficient and precise
communication with others fluent in the domain's jargon. Incorporating
domain terminology in your class names is a good idea, as it succinctly
communicates a lot of information to the reader. But you must be careful
to use only terminology that is commonly known and has a precise
definition, and ensure that your usage of the term is consistent with
that definition. Avoid region-specific slang and colloquialisms.
Examples:

\begin{itemize}
\item \texttt{DichotomousItem}
\item \texttt{CorrigendaSection}
\item \texttt{DeweyDecimalNumber}
\item \texttt{AspectRatio}
\item \texttt{OrganicCompound}
\end{itemize}

\item 5. Use Design Pattern Names
\label{sec:orgheadline268}

Incorporating design pattern names like \emph{Factory}, \emph{Proxy} and
\emph{Singleton} into your class names is a good idea, for the same reasons
that it is useful to use terminology from the application domain --
because a lot of information is communicated succinctly. Just be careful
not to get pattern-happy, and start thinking “everything is an instance
of some pattern.” Only refer to design pattern names if they have direct
relevance to the intrinsic nature of the class. Examples:

\begin{itemize}
\item \texttt{ConnectionFactory}
\item \texttt{ClientProxy}
\item \texttt{AccountObserver}
\item \texttt{DocumentBuilder}
\item \texttt{TableDecorator}
\end{itemize}

\item 6. Aim For Clarity Over Brevity
\label{sec:orgheadline269}

Many developers demonstrate a form of scarcity thinking when it comes to
naming classes -- as if there were a shortage of characters in the world
and they should be conserved. The days when we needed to constrain
identifiers to particular length restrictions are long gone. Today we
should be focused upon selecting class names that communicate
effectively, even if at the expense of a little extra length. With many
developers using IDEs that support auto-completion, the traditional
arguments in favor of abbreviation (typographical error and typing
effort) are no longer applicable. The one case where abbreviation is
warranted is specialist acronyms that are commonly used in the
application \texttt{CMOSChip} is clearer than
\texttt{ComplimentaryMetalOxideSemiconductorChip}. Examples:

\begin{itemize}
\item \texttt{ProductionSchedule} is clearer than \texttt{ProdSched}
\item \texttt{LaunchCommand} is clearer than \texttt{LaunchCmd} domain
\item \texttt{ThirdParty} is clearer than \texttt{ThrdPrty}
\item \texttt{ApplicationNumber} is clearer than \texttt{AppNum}
\item \texttt{SystemCorrespondence} is clearer than \texttt{SysCorro}
\end{itemize}

\item 7. Qualify Singular Nouns Rather Than Pluralize
\label{sec:orgheadline270}

When a class represents a collection of some type, it can be tempting to
name it as the plural of the collected type e.g. a collection of \texttt{Part}
classes might be called \texttt{Parts}. Although correct, you can communicate
more about the nature of the collection by using qualifying nouns such
as \texttt{Set}, \texttt{List}, \texttt{Iterator} and \texttt{Map}. Examples:

\begin{center}
\begin{tabular}{ll}
Class Name & Group Semantics\\
\hline
\texttt{PartList} & Parts are ordered\\
\texttt{PartSet} & Parts are unordered and each Part can not appear more than once\\
\texttt{PartPool} & Parts are interchangeable\\
\end{tabular}
\end{center}

\item 8. Find Meaningful Alternatives To Generic Terms
\label{sec:orgheadline271}

Terms like \texttt{Item}, \texttt{Entry}, \texttt{Element}, \texttt{Component} \emph{and} \texttt{Field} are
very common and rather vague. If these terms really are the standard
terminology in your application domain then you should use them. But if
you are free to use class names of your own invention then search for
something more specific and meaningful.

\item 9. Imply Relationships With Other Classes
\label{sec:orgheadline272}

Naming a class provides you with the opportunity to communicate
something about that class's relationship with other classes in the
application. This will help other developers understand that class's
place in a broader application context.

Some techniques that may be helpful in this regard:

\begin{itemize}
\item Use the name of a super-class or interface as a suffix e.g. call
implementations of the \texttt{Task} interface \texttt{PrintTask}, \texttt{ExecuteTask}
and \texttt{LayoutTask}.\\
\item Prefix the name of abstract classes with the word \texttt{Abstract}.
\item Name association classes by pre-pending and appending the class names
on either side of the association e.g. the association between
\texttt{Student} and \texttt{Test} could be called \texttt{StudentTakesTest}.
\end{itemize}
\end{enumerate}

\subsubsection{In Praise of Code Reviews \footnote{First published 27 Feb 2006 at
\url{http://www.hacknot.info/hacknot/action/showEntry?eid=83}}}
\label{sec:orgheadline277}

I have a woeful sense of direction --- the navigational abilities of a
lemming combined with the homing instinct of a drunk. But like much of
my gender, I continue to entertain the fantasy that I possess an
instinctive ability to find my way, an evolutionary artifact of the
male's traditional role as the hunter; an unerring inner compass that
will guide me safely through the hunt of everyday life, despite
voluminous evidence to the contrary. It is a fantasy that gets me in
trouble on a regular basis.

Whenever I am driving to somewhere new, the scenario generally plays out
like this: I begin the journey looking through my street directory,
tracing out the path I need to follow. After memorizing the first few
turns I set the directory down and depart, resolving to stop and consult
the directory again once I've completed those turns. Within a few
minutes I have traveled over the first part of the journey that I've
already memorized, and have reached a decision point. Will I pull over
to the side of the road and reacquire my bearings as planned, or will I
just follow my nose? Invariably, I choose the latter.

"I'm bound to see a relevant sign before too much longer," I think. And
so I drive on, keeping an eye out for the anticipated sign. If it
doesn't shortly appear, I begin to make speculative turns based on my
own "gut feeling" about which way to head. If I'm heading to a popular
destination, I might simply follow the path I perceive most of the
traffic is taking, figuring that they're all probably headed to the same
place as I am. Through a combination of guess-work, dubious reasoning
and random turns I eventually reach the point where I have to admit to
myself that I'm lost. Only then will I pull over to the side of the
road, get the street directory out of the glove compartment to find out
where I am and how to get to my original destination from here.

This insane behavior has been a characteristic of my driving for many
years. It usually manifests when I am driving home alone from some event
which has left me feeling tired and distracted. I slip into a worn out
fugue, adopt a "she'll be right" attitude and head off to
goodness-knows-where. About a year ago, driving home from a job
interview in a distant city, I strayed off course by over 100 kilometers
-- all the while resolutely refusing to pull over and consult my
directory, which I could have done at any time.

Thanks to these unexpected excursions, I have seen parts of the country
side that I might otherwise have missed, but I have no idea where they
were or how to get back there.

So why do I do it? Why not spend five minutes by the side of the road
working out where I've been and where I'm going, rather than just keep
driving aimlessly in hope of finding some visible prompt to get me on
course? As strange as the habit is, I think it's exactly the same
behavior that many people exhibit when they make self-defeating
decisions. It stems in part from short-term thinking.

Driving along in my pleasant reverie, I am faced with a choice. Stopping
to consult my street directory will require some mental energy. I'll
have to break the flow of my journey, find a significant landmark or
intersection, locate it in the directory, and re-plot a path to my
destination. The alternative is just to keep drifting along and hope for
the best. If your scope of consideration is only the next few minutes,
then it's very easy to decide to avoid the short-term inconvenience of
pulling over in favor of continuing to do what you're already doing --
even though it isn't working out and has already got you into
difficulty.

A smoker indulges in similar thinking every time they light up. They
know full well that they're killing themselves by having that next
cigarette, but considering only the next five minutes, what is easier:
Resisting the craving for a cigarette, or giving in?

This desire to minimize small, short-term pain even at the expense of
significantly more pain in the long term is at the core of much
self-defeating behavior.

We'll return to this theme in a moment. But first, a short divergence on
code reviews.

\begin{enumerate}
\item Code Reviews
\label{sec:orgheadline274}

For many types of work it is standard practice to have one's work
checked by another before the work product is put into service. Authors
have editors; engineers have inspectors and so on. But in software
development it is common for code to flow directly from the programmer's
fingertips into the hands of the end users without ever having been seen
by another pair of eyes.

This is despite there being a large body of empirical evidence
establishing the effectiveness of code review techniques as a device for
defect prevention. Since the early history of programming, a number of
different techniques for reviewing code have been identified and
assessed. A \emph{code walkthrough} is any meeting in which two or more
developers review a body of code for errors. A code walkthrough can find
anywhere between 30 and 70 percent of the errors in a program \footnote{\emph{Rapid Development}, Steve McConnell, pg 70, citing Myers 1979,
Boehm 1987b, Yourdon 1989b}.
\emph{Code reading} is a more formal process in which printed copies of a
body of code are distributed to two or more reviewers for independent
review. Code reading has been found to detect about twice as many
defects as testing \footnote{\emph{Ibid}, pg 71, citing Card 1987}. Most formal of all is the \emph{code inspection},
which is like a code walkthrough where participants play pre-defined
roles such as \emph{moderator}, \emph{scribe} or \emph{reviewer}. Participants receive
training prior to the inspection. Code inspections are extremely
effective, having been found to detect between 60 and 90 percent of
defects \footnote{\emph{Ibid}, pg 71}. Defect prevention leads to measurably shorter project
schedules. For instance, code inspections have been found to give
schedule savings of between 10 and 30 percent.

I estimate that about 25 percent of the projects I have worked on
conducted code reviews, even though 100 percent of them were working
against tight schedules. If we can save time and improve quality with
code reviews, why weren't the other 75 percent of projects doing them?

I believe the answer is mostly psychological, and the basic mechanism is
the same one that I engage in every time I go on one of my unplanned
excursions in my car. The essential problems are short-term thinking,
force of habit and hubris.

Suppose you have just finished coding a unit of work and are about to
check it into your project's version control system. You're faced with a
decision -- should you have your code subjected to some review
procedure, or should you just carry on to the next task? Thinking about
just the next five minutes, which option is easier? On the one hand
you'll have to organize the review, put up with criticism from the
reviewers, and probably make modifications to your code based upon their
responses. On the other hand, you can declare the task "finished', get
the feeling of accomplishment that comes along with that, and be an
apparent step closer to achieving your deadlines. So you make the
decision which minimizes discomfort in the short term, the same way I
decide to just keep on driving in search of a road sign rather than pull
over and consult my street directory.

But then, you've got to rationalize this laziness to yourself in some
way. So you reflect on previous experience and think "I've gotten away
with not having my code reviewed in the past, so I'll almost certainly
get away with it again". Similarly, I'm driving along thinking "I've
never failed to eventually get where I'm going in the past, so I'll
almost certainly get there this time as well." Complacency breeds
complacency.

Finally, although it is difficult to admit, there is some comfort in not
having your code reviewed by others. We would like to think that we can
write good code all by ourselves, without the help of others, so
avoiding code reviews enables us to avoid confronting our own
weaknesses. In the same way, by following my nose rather than following
my street directory, I can avoid having to confront the geographically
exact evidence of my hopeless sense of direction that it will provide.
Ignorance is bliss.

Even when you quote the empirical evidence to programmers, many will
still find a way to excuse themselves from performing code reviews, by
assuming that the touted reductions in schedule and improvements in
quality were derived through experimentation upon lesser developers than
themselves. The thinking goes something like "Sure, code reviews might
catch a large percentage of the defects in the average programmer's
work, but I'm way above average, don't write as many defects, and so
won't get the same return on investment that others might."
Unfortunately it is very difficult to tell simply by introspection
whether you really are an above average programmer, or whether you just
think you are. Most people consider that they are "above average" in
ability with respect to a given skill, even though they have little or
no evidence to support that view. For example, most of us consider
ourselves "better than average drivers". The effect is sometimes
referred to as \emph{self-serving bias} or simply the \emph{above average effect}.

Those that have bought into the Agile propaganda (can we call it
"agile-prop"?) may have been deceived into thinking that pair
programming is a substitute for code reviews. To the best of my
knowledge, there is no credible empirical evidence that this is the
case. In fact, there are good reasons to be highly skeptical of any such
assertions -- in particular, that a pair programmer does not have the
independent view of the code that a reviewer uninvolved with its
production can have. Much of the benefit of reviews comes from the
reviewers different psychological perspective on the product under
review, the fact that they have no ego investment in it, and that they
have not gone through the same (potentially erroneous) thought processes
that the original author/s have done in writing it. A pair programmer is
not so divorced from the work product or the process by which it was
generated, and so one would expect a corresponding decrease in ability
to detect faults.

So we sustain self-defeating work practices the same way we sustain many
other sorts of self-defeating behavior -- by lying to ourselves and
putting long term considerations aside.

\item Do Code Reviews Have A Bad Reputation?
\label{sec:orgheadline275}

There is perhaps another factor contributing to a hesitance to perform
code reviews, which is the reputation they have as being confrontational
and ego-bruising experiences. This reputation probably springs from
consideration of the more formal review processes such as code
inspections, in which the reviewing parties can be perceived as "ganging
up" on the solitary author of the code, subjecting them to a famously
unexpected Spanish Inquisition.

This is a legitimate concern, and it is certainly easy for a review of
code to turn into a review of the coder, if a distinct separation is not
encouraged and enforced. I therefore recommend that code reviews be
conducted by individual reviewers in the absence of the code's author.
This tends to depersonalize the process somewhat, and remove some of the
intimidatory effect that a group process can have. There is in fact some
evidence to suggest that an individual reviewer is no less effective
than a group of reviewers in detecting faults in code.

The code can be printed out and written comments attached to it, or
comments can be made in the source file itself, perhaps as "TODO" items
that can be automatically flagged by an IDE. Personally, I prefer
paper-based reviews because a paper-based review system is quick and
easy to institute, and equally applicable to reviews of written
artifacts such as design and requirements documents.

\item Conclusion
\label{sec:orgheadline276}

There is much to recommend the practice of conducting code reviews on a
regular basis, and few negatives associated with them, provided they are
conducted sensitively and with regard for the feelings of the code's
author. All it takes is for one other programmer on your team to be
willing to undertake the task, and you can establish a simple code
review process that will likely produce noticeable benefits in improved
code quality and reduced defect counts. Not everyone is good at
reviewing code, so if you have the option, have your code reviewed by
someone who demonstrates an eye for detail and is known for their
thoroughness. If you have the authority to do so, it is well worth
incorporating code reviews into your team's development practice,
perhaps as a mandatory activity to be undertaken before new code is
committed to the code base, or perhaps on a random basis. It may also be
worthwhile to have junior staff review the code written by their more
experienced counterparts, as a way of spreading knowledge of good coding
techniques and habits.

When introducing code reviews, you will likely encounter some initial
resistance, simply because the short-term thinking which has so far
justified their absence is a habit that is superficially attractive and
requiring of a certain determination to break. However, once they have
had the opportunity to participate in code reviews, many programmers
will concede that it is a habit worth forming.
\end{enumerate}

\subsection{User Interfaces}
\label{sec:orgheadline301}

\subsubsection{Web Accessibility for the Apathetic[ \footnote{First published 2 Nov 2004 at
\url{http://www.hacknot.info/hacknot/action/showEntry?eid=69}}}
\label{sec:orgheadline287}

If you're like me, you approach the subject of accessibility with a
certain self-conscious guilt. On the one hand, you recognize that there
are excellent ethical and legal reasons for making your applications --
be they web-based or rich client -- accessible to those with sensory or
cognitive impairments; but on the other hand you can't ignore the fact
that the extra work required to add that accessibility is only going to
make a difference to a very small percentage of your users.

In recent years, the legal impetus has begun to gain strength, forcing
those of us to action who might otherwise have been willing to put our
internal ethics department on hold in the name of conserving time and
energy. Having spent some time recently working inside a department of
the Australian government, I have learnt that the issue of
accessibility, in particular web accessibility, has a reasonably high
profile. Because government web sites are required to adhere to
accessibility guidelines \footnote{\url{http://www.w3.org/TR/WAI-WEBCONTENT/}}, there has developed a group, comprised of
either moralists or opportunists, who spend their time scouring the web
pages of government web sites looking for non-conformances to use as the
basis for legal prosecution. American courts have recently ruled that
the accessibility requirements pertinent to US governmental web sites
are also applicable to privately held web sites. Even your blog counts
as material that is made "publicly available," and must therefore be
equally available to all.

With these ideas in mind, and also to assuage my growing feelings of
guilt regarding the accessibility (or lack thereof) of this site, I
decided to undertake a bit of a site revamp, the cosmetic results of
which you will already have noticed. This article provides a brief
overview of the process I followed, and thereby gives a general
introduction to the tools and techniques necessary to retro-fit
accessibility to a site that was designed without specific consideration
of that issue.

\begin{enumerate}
\item General Approach
\label{sec:orgheadline279}

In general, web accessibility can be achieved by adhering to the
following two principles:

\begin{itemize}
\item Separate presentation from content by restricting your use of HTML to
the standard structural elements, and using CSS (Cascading Style
Sheets) to control the way that structure is presented.
\item Emphasize textual content. Where non-textual content is used, always
provide a textual equivalent.
\end{itemize}

A good portion of the details appearing below are in support of these
two principles. The steps below show you how to transform a
non-accessible web page into an accessible one.

\item Step 1: Ensure All HTML Elements Are Structural
\label{sec:orgheadline280}

Structural elements those which describe the semantic units of an HTML
document. Examples of structural HTML elements are:

\begin{itemize}
\item \texttt{<h1>} \ldots{} \texttt{<h6>}
\item \texttt{<p>}
\item \texttt{<ol>}
\item \texttt{<ul>}
\item \texttt{<img>}
\item \texttt{<li>}
\item \texttt{<div>}
\item \texttt{<span>}
\end{itemize}

Over the years, browser vendors have added proprietary non-structural
elements and attributes to the HTML grammar their browser understands,
in an effort to differentiate their product from their competitor's. The
result is a tag set which invites misuse, is interpreted differently (or
not at all) in different browsers, and awkwardly combines content and
presentation. By removing elements that specify some aspect of the
document's presentation, accessibility can be improved.

Examples of non-structural HTML elements you should remove are:

\begin{itemize}
\item \texttt{<hr>}
\item \texttt{<i>}
\item \texttt{<b>}
\item \texttt{<u>}
\item \texttt{<big>}
\item \texttt{<small>}
\item \texttt{<font>}
\item \texttt{<basefont>}
\item \texttt{<br>}
\item \texttt{<font>}
\item \texttt{<tt>}
\end{itemize}

The layout effects produced by these nonstructural tags can, and should
be, achieved with style sheets. Using these tags only pollutes your HTML
document with presentation information that may well be useless or
misleading to those with low vision. For instance, \texttt{<b>} elements should
be removed because bold text has no meaning to a blind user. This
doesn't mean that text can't be made bold, but rather that CSS rather
than HTML should be the means by which the bolding is achieved.

Note that in some cases there is a structural tag that you should put in
place of the deleted nonstructural tag. For example:

\begin{itemize}
\item If you have removed \texttt{<b>} tags that were used to emphasize words,
insert \texttt{<strong>} tags where the \texttt{<b>} tags used to be.
\item If you have removed \texttt{<b>} tags that were used to create a heading,
insert a heading tag like \texttt{<h3>} where the \texttt{<b>} tags used to be.
\item If you have removed \texttt{<b>} tags that were used to add emphasis, insert
\texttt{<em>} tags where the \texttt{<b>} tags used to be.
\end{itemize}

In other cases, there is no structural element already defined in HTML
that adequately captures a structural aspect of your web page, so you
must invent your own using the \texttt{<span>} or \texttt{<div>} elements. For
instance, you might create a of class "footnote" to denote footnote
references: \texttt{<span class=”footnote”>This is a footnote.</span>}.

The way that span elements of class footnote are displayed is later
specified in a CSS.

\item Step 2: Ensure All HTML Attributes Are Structural
\label{sec:orgheadline281}

Non-structural attributes should be removed for the same reasons that
structural elements should be removed. Examples of non-structural
attributes you can delete are:

\begin{itemize}
\item \texttt{align}
\item \texttt{link}
\item \texttt{alink}
\item \texttt{halign}
\item \texttt{valign}
\item \texttt{background}
\item \texttt{color}
\item \texttt{text}
\item \texttt{bgcolor}
\item \texttt{vspace}
\item \texttt{height}
\item \texttt{width}
\item \texttt{hspace}
\item \texttt{border}
\end{itemize}

Again, all the layout effects that were produced by these attributes can
be achieved with CSS, leaving the basic HTML document more accessible.

\item Step 3: Remove Misused Structural HTML Elements
\label{sec:orgheadline282}

Structural elements should not be used as ersatz layout mechanisms as
this will confuse those accessing your web page with a text browser.

Examples of the misuse of structural elements for layout purposes
include:

\begin{itemize}
\item Using empty paragraphs (\texttt{<p>}) to put a vertical space between
consecutive blocks of text.
\item Using the \texttt{<table>} element to achieve columnar alignment of material
that is not inherently tabular.\\
\item Drawing lines by stretching a 1-pixel \texttt{<img>}.
\item Using \texttt{<blockquote>} purely to achieve indentation.
\end{itemize}

\item Step 4: Ensure All Non-Textual Content Has A Textual Equivalent
\label{sec:orgheadline283}

Users with visual impairment may use a text browser, Braille bar or
screen reader to access your web page. These mechanisms can only deal
with text as input. So you need to supply a textual equivalent to any
non-textual content on your web page. A common examples is using the
\texttt{alt} attribute of \texttt{<img>} tags to describe the significance of the
image.

There are certain mechanisms which should be used sparingly, if at all,
because they are inherently inaccessible. These include:

\begin{itemize}
\item Image maps
\item Javascript
\item Side-by-side frames
\item Secondary windows
\item Shockwave animations
\end{itemize}

Not only are these mechanisms difficult for some users to access, but
they may be deliberately disabled by any user in their browser.

\item Step 5: Add In Attributes Or Elements That Aid Accessibility
\label{sec:orgheadline284}

There are a few HTML structural elements and attributes that are
particularly helpful from an accessibility perspective:

\begin{itemize}
\item The \texttt{<abbrev>} and \texttt{<acronym>} elements can be used to specify the
expansion of abbreviations and acronyms when they first occur in a
document.
\item The \texttt{<th>} element should be used to identify column headers. Tables
are linearized in text browsers, and knowing which table cells are
headers helps the user interpret them.
\item In HTML forms, use the \texttt{<label>} element around the form labels.
Additionally, field labels should be immediately to the left of, or
immediately above, the field.
\item Provide a logical tab order for elements by specifying the \texttt{tabindex}
   attribute for \texttt{<input>} elements.
\item Use the \texttt{title} attribute of \texttt{<a>} elements to provide more
information about the target of the hyperlink.
\end{itemize}

\item Checkpoint
\label{sec:orgheadline285}

At this point, you should have an HTML document that is marked up solely
with structural elements and attributes. This is a good time to preview
your page in a text browser like Lynx \footnote{\url{http://lynx.browser.org/}}, or with a screen reader like
IBM Home Page Reader \footnote{\url{http://www-3.ibm.com/able/solution_offerings/hpr.html}}.

It is also a good time to run your HTML through one of the automated
accessibility-checker sites on the web. Such sites enable you to provide
your HTML -- either directly with cut/paste, or by nominating a URL --
and then scan the document looking for accessibility problems. I found
\emph{www.bobby.com} to be quite useful.

\item Step 6: Recreate The Layout Using Cantankerous Style Sheets
\label{sec:orgheadline286}

And now for the tricky bit. Converting your web page to use only
structural HTML elements and attributes is easy compared to using CSS to
achieve your desired layout. Mostly the difficulty stems from the
variations in the way different browsers render CSS directives. Behavior
of "floating" elements seems to be particularly problematic. Therefore
it is essential to test the layout in as many different browsers as you
can. This lack of standardization in behavior is the most frustrating
aspect of using CSS. I found the following books useful in getting up to
speed on CSS:

\begin{itemize}
\item \emph{CSS - The Definitive Guide 2nd Edition}, Eric A. Meyer, O'Reilly
Media Inc, 2004
\item \emph{CSS - Designing for the Web 2nd Edition}, H. Lie and Bert Bos,
Addison Wesley, 1999
\item \emph{More Eric Meyer on CSS}, Eric Meyer, New Riders, 2004
\end{itemize}

Once you've got a style sheet that presents the HTML document the way
you want, you're done. Just be sure that your choice of layout effects
doesn't aggravate those suffering from particular medical conditions:

\begin{itemize}
\item Those with light-triggered epilepsy can seizure when subject to
blinking text or images. Sensitivity varies between the 4Hz and 59Hz
frequencies, with peak sensitivity around 20Hz.
\item Color perception problems are quite common -- more so in males than
females. Make sure your layout doesn't rely on color as the sole
discriminator between different objects. The filter available at
\emph{www.visicheck.com} can show you what your page looks like to users
with different color perception difficulties.
\item Do not use text sizes that are too small. The minimum size should
appear to be equivalent to a 10pt font, but 12pt is preferable. Note
that you should not actually use \texttt{pt} or \texttt{px} units to specify font
sizes, as these don't scale up when the user changes the text size in
their browser. The \texttt{em} unit should be used instead.
\end{itemize}
\end{enumerate}

\subsubsection{SWT: So What?[ \footnote{First published 24 Apr 2005 at
\url{http://www.hacknot.info/hacknot/action/showEntry?eid=74}}}
\label{sec:orgheadline300}

If you are about to undertake a major project using SWT, I suggest you
think very carefully before doing so. Compared to its obvious
competitor, Swing, SWT is very lacking in functionality, support and
community development experience. Little wonder that there is not a lot
of detailed information to be found from people who are using SWT in
anger to create serious applications. There is a certain amount of
fan-boy stuff \footnote{\url{http://blogs.bytecode.com.au/glen/2005/02/12/1108169609271.html}}, written by people in the first blush of initial
enthusiasm, convinced that everything is "cool" and "awesome", but very
little from people who have been through a significant implementation
effort extending over months or years. The closest one can get to
finding "veteran" users is on the \texttt{eclipse.org.swt} newsgroup. In
surveying opinions on SWT from the development community, I have found
that people's enthusiasm for SWT is inversely proportional to the amount
of experience they have had with it.

Let me briefly outline the principle differences between SWT and Swing,
at a high level:

\begin{itemize}
\item Sun first released Swing in 1997. It is bundled with Java and is
considered the "standard" for GUI development in Java. Swing creates
a GUI using only emulation - that is, Java draws the buttons, menus
and other widgets on a blank window using primitive graphic
operations. It entirely ignores whatever widgets are made available
by the native platform, but through its pluggable "Look and Feel"
facility it imitates the appearance and behavior of those widgets.
\item IBM released SWT as open source in 2001, having written it to support
development of the Eclipse IDE. IBM began developing Eclipse in
Swing, found it unacceptably slow, and so decided to write their own
widget toolkit instead. In general, SWT wraps the native widgets from
the underlying platform, which is intended to give better performance
than Swing, and make interfaces written with SWT indistinguishable
from native applications.
\end{itemize}

Discussions of the relative merits of Swing and SWT fall tend towards
religious war. SWT advocates champion SWT's fidelity to native
applications, performance and efficiency. They deride Swing's
responsiveness, memory consumption and complexity. Swing advocates
champion Swing's maturity, power and support. They deride SWT's
capabilities, quality and small developer base. Advocates from both
sides consider their opponents to be of questionable parentage.

\begin{enumerate}
\item Problems In Using SWT
\label{sec:orgheadline291}

There are numerous obstacles for the would-be SWT programmer to
overcome. Collectively, you will find them a source of great
frustration.

\begin{enumerate}
\item Bugs
\label{sec:orgheadline288}

Unless you are developing a trivial interface, you will be forced to
become very well acquainted with the Bugzilla at \emph{eclipse.org}. As
further examples, try doing a query on the Bugzilla to find the number
of bugs raised by the principal developers of Azureus \footnote{\url{http://azureus.sourceforge.net/}} and
BitTorrent \footnote{\url{http://www.bittorrent.com/}} - probably the two most well-known SWT applications at
this time. You will see that each has raised fifty or more issues in the
course of developing their products. That may be fine if you're working
on an open source application without strict deadlines or resource
limitations, but in a commercial context, losing so much time and effort
to bugs is a major problem.

You don't want your project to have critical issues to be fixed on a
time line that is beyond your control. The old open source standby of
"just fix it yourself" is a non-sequitur here. In a commercial context,
one is paid to advance the business interests of the client, not to
overcome shortcomings in a widget toolkit. Besides, making additions to
SWT requires a low-level knowledge of the behavior of five different
operating systems and windowing environments, and how many people have
that kind of expertise?

The fact that each bug fix must be made to work for different native
implementations is a significant multiplication of effort, and the
source of often lengthy delays when it comes to bug fixes and functional
enhancements. This was stated by Steve Northover, the original architect
of SWT, in a recent message to the \texttt{eclipse.org.swt} newsgroup. He
responded to one programmer's frustrated complaints about bugs in the
\texttt{Table} widget which had been outstanding for several years, in this
way:

\begin{quote}
If you stop to think about it, we support 5 different operating
systems using totally different code bases and somehow knit together
and implement a portable API to all of them and we do this for free.
It's a full time job, 24-7.
\end{quote}

This problem is an unavoidable byproduct of the architectural decision
that underlies SWT -- the use of native widgets necessitates the
development and maintenance of numerous distinct code bases. The burden
is significant and, to quote James Gosling, "a bad place to be". \footnote{\url{http://www.builderau.com.au/program/work/0,39024650,39176462,00.htm}}

\item Limited Functionality
\label{sec:orgheadline289}

Those coming to SWT from a Swing background will probably be shocked by
the absence of many bits of functionality that they are accustomed to
having at their fingertips. For instance, Swing programmers will think
nothing of having a Button widget that displays both a text label and
image, and be surprised they can't do that in SWT unless the \texttt{Button}
appears within a \texttt{ToolBar} or \texttt{CoolBar} [Ed. 2006 -- This issue has
since been resolved]. They will be used to attaching Borders to widgets
as they see fit, using the Swing \texttt{BorderFactory}, but wonder why borders
are only supported on some SWT widgets such as \texttt{Text} and \texttt{Label}. They
will be accustomed to setting up input masks on text fields using the
facilities on \texttt{JTextField}, but find in SWT they will have to write that
themselves by listening to individual keystrokes on a \texttt{Text} widget.

\item Eclipse Driven Development
\label{sec:orgheadline290}

We do well to remember that SWT was originally developed in service of
Eclipse. Now that Eclipse is open source and SWT is being touted by some
as an alternative to Swing for general interface development, this
heritage is turning out to be quite a burden. There is a bipartite
division in issue response times that seems to be related to relevance
to Eclipse. If a bug is found that effects Eclipse, then there is some
chance of it being attended to in a reasonable time frame. If the bug
doesn't effect Eclipse -- then the situation is quite different. Such
bugs appears to attract a much lower priority. And given the resource
restrictions the Eclipse GUI team struggles with, getting enhancement
requests done is quite an achievement. This Eclipse-centric approach to
maintenance and extension is a problem when the application you're
constructing is not from the same domain as Eclipse. The facilities
required to construct the interface for, say, a warehousing or
inventory-tracking system are different from those required to construct
a programmer's IDE. The former makes demands of SWT not made by the
latter -- but maintenance and enhancement appears to be prioritized
according to relevance to Eclipse. Therefore you'll find SWT less and
less relevant the further away you stray from the programming domain.
\end{enumerate}

\item Myths
\label{sec:orgheadline297}

There is a lot of urban myth and misinformation surrounding both SWT and
Swing. When evaluating the relative merits of these two technologies,
your first task will be to distinguish fact from opinion. There is much
of the latter masquerading as the former. Below, I address a few of the
common misconceptions in this area.

\begin{enumerate}
\item SWT Is Fast, Swing Is Slow
\label{sec:orgheadline292}

Apparently it was performance concerns with Swing that prompted IBM to
begin development of SWT. It would be interesting to know if they would
make this same decision now, especially given the Swing performance
improvements in JDK1.5. In practice, both Swing and SWT applications can
be made to appear unresponsive if you perform long-running operations in
the GUI event thread (a concept shared by both) or if a big garbage
collection cycle arrests the entire application. The best way to compare
Swing and SWT performance would be via benchmarks, however it is
difficult to construct a fair comparison that truly compares like with
like when the underlying technologies differ in such fundamental ways.

\item SWT Exposes The Native Widgets Of The Underlying Platform
\label{sec:orgheadline293}

In general, SWT exposes the behavior of the native platform's own GUI
widget set. However this is only part of the story. There are some
inferences people tend to make based on this, that are incorrect.

Some believe that the entirety of the underlying widget's behavior is
exposed through SWT. This is not necessarily so. SWT must produce the
same behavior across all the platforms it caters to. If widget W has
behaviors A, B and C on its native platform, but C is missing from one
platform's implementation of the widget W, then only A and B are
provided by W on all platforms. In other words, behavior C will be
masked out on its native platform, because it was not available on all
platforms. This "lowest common denominator" approach can be very
limiting. For example, you would not think it a great challenge to put
both an image and a text label on a button. However, unless the button
is in a \texttt{Toolbar} or \texttt{CoolBar}, you can't do it in SWT [Ed. 2006 --
/This issue has since been resolved/]. This is because it's not
permitted on one of the platforms that SWT supports, therefore it can't
be available on any of them. Every few weeks, somebody posts a message
to the SWT newsgroup wanting to know how to do this, and is surprised to
find that they can't \ldots{} they have to write their own button widget if
they want that functionality.

However, the situation is not that simple. Sometimes the "lowest common
denominator" is augmented using emulation in SWT. In other words,
somebody has determined that the lowest common denominator is simply not
acceptable, and those platforms where the behavior is not available
natively have that behavior added on by SWT itself. In some cases this
extends to emulation of an entire widget. For example, Motif has no tree
widget. Rather than hide the tree widget on all platforms, SWT emulates
the entire tree widget for Motif.

There are both advantages and disadvantages to SWT's partial exposure of
native widgets. On the up side, you get fidelity to platform appearance
and behavior. On the down side, that fidelity may not extend to the
inclusion of features outside of the LCD. Further on the down side, not
only do you get the native widget's behavior, you also get its bugs. On
the up side, sometimes SWT can compensate for those bugs so that they
appear fixed to the SWT user.

\item Platform Fidelity Increases Usability
\label{sec:orgheadline294}

The rationalization that SWT proponents constantly offer for attaching
such importance to absolute platform fidelity is that it increases
usability. SWT is meant to offer greater platform fidelity than Swing,
which makes the usability of SWT applications better. I believe this
argument is specious, for several reasons.

First, this argument gets voiced by programmers, not users. This is
significant because what is important to programmers is not necessarily
important to the general user population. There is also the possibility
of programmers letting their technical convictions influence their
perception of usability. Consider, it was feedback from programmers that
drove the development of SWT to begin with. In the forward to "SWT: The
Standard Widget Toolkit", Erich Gamma states:

\begin{quote}
I was part of the team with the mission to build a Java based
integrated development environment for embedded applications that was
shipped as the IBM VisualAge/MicroEdition. \ldots{} We felt pretty good
about what we had achieved! However, our early adopters didn't feel as
good as we did\ldots{} they complained about the performance and most
importantly about the fact that the IDE didn't look, feel and respond
like a native Windows application. Some of the performance problems
were our fault and some of them could be attributed to Swing. The
performance problems didn't bother us that much; they could be
engineered away over time. What worried us more was the nonnative
criticism. While we could implement a cool application in Swing that
runs on Windows, we couldn't build a true Windows application. Fixing
this problem required more drastic measures.
\end{quote}

So SWT sprung from an IDE development effort, and the feedback of the
IDE's early adopters - who are themselves programmers. I suspect that
the issue of platform fidelity is of very little significance to
non-programmers. Personally, I have seen no evidence that whatever
discrepancies exist between Swing's emulation of Windows and the native
Windows appearance make any appreciable difference in usability at all.
Many users don't even notice, and those that do only have a vague
awareness that something is a bit different about the application, but
they're not quite sure what.

Second, due to the LCD effect already described, SWT often doesn't
expose the exact behavior or appearance of the native widget set. Where
is the evidence that the difference in fidelity between the SWT version
of widgets and the Swing emulation of those widgets actually results in
a difference in usability? In fact, there is much to suggest that it is
not the case. Consider the success of applications such as iTunes for
Windows, QuickTime, Winamp and the Firefox browser. All of these have
interfaces very different from that of native Windows applications --
yet they are successfully used by even novice Windows users. When users
upgrade from one version of Windows to another, say from 2000 to XP,
there are numerous cosmetic differences in the interface presented, but
do they suddenly find themselves lost and unable to use the
applications? No, of course not. The reason is that minor aesthetics are
not key determinants of usability. Overall interface structure, task
orientation and affordance are the key factors. Whether a button has a
3-pixel wide or 2-pixel wide shadow is not important. As long as a user
can recognize the controls presented to them, and those controls behave
in a predictable way, then usability is unaffected.

Finally, if usability and platform fidelity are so inextricably linked,
what are we to make of the Flat Look part of SWT -- that subset which
creates interfaces which are similar to web pages in appearance but
exhibit greater functionality? They are entirely unlike anything in any
of the native platforms that SWT supports. If you've seen the PDE in
Eclipse, you've seen Flat Look. If the claim that platform fidelity is
linked to usability is true, shouldn't Flat Look interfaces be usability
nightmares? The inconsistency between philosophy and implementation is
puzzling.

\item SWT Is Quicker To Learn Than Swing
\label{sec:orgheadline295}

SWT enthusiasts claim that it is easier to learn than Swing. Having been
through the learning curve for both, I have not found this to be the
case. There are two main aspects to the ease of learning for any
technology -- the difficulty of the technical concepts themselves, and
the way those concepts are taught. Conceptually, there is a significant
overlap between SWT and Swing. Component hierarchies, layout managers,
threading and separation of data from presentation are concepts present
in both. The basic selection of built-in widgets and layouts is much the
same also. The real differentiator is the quality and quantity of
instructional material available. The Javadoc for SWT is sparse, the
remaining knowledge has to be pieced together from articles, code
snippets and asking questions on the SWT newsgroup. There are perhaps a
half dozen books on SWT available. Beyond that, you need to look at the
SWT code itself and reverse engineer an understanding of what's going
on. The situation with Swing is very different. The Javadoc is
extensive, there is a vast amount of tutorial information available
online, and a large number of books are dedicated to the topic.
Therefore learning Swing is generally easier than learning SWT, because
of the greater amount of plain English information available.

\item Limited Third Party Widget Selection Is A Good Thing
\label{sec:orgheadline296}

Any comparison of SWT and Swing must unearth the fact that there is next
to nothing in the way of third part widgets available for SWT, but there
are a number of such offerings available for Swing. This can have a
profound effect on programmer productivity, forcing one to write by hand
what might otherwise be available off the shelf for considerably less
cost.

Probably the most desperate pro-SWT argument I've heard to date is the
claim that this reduced selection of COTS widgets is a good thing
because it reduced the opportunities for programmers to do the wrong
thing. If there is a wide selection of widgets available, the argument
goes, then programmers will fill their interfaces with every cute widget
they can get their hands on. This is not a problem when using SWT, as
few such widgets are available in the first place.

The argument is so ridiculous as to beggar belief, but it is one I have
heard SWT zealots voice, in a desperate attempt to rationalize their
ideological convictions. Its main failing is to confuse widget
availability and widget usage. The usability of an interface is not a
function of how many different types of widgets it contains, but of the
way those widgets are organized and used in the interface. A good
interface designer knows that novel widgets may confuse users unfamiliar
with them, and so does not employ them unless they offer a radical
functional improvement in return for lesser intuitiveness. A bad
interface designer will construct an interface with poor usability
regardless of how few widgets they have at their disposal. To understand
why, consider the following analogy.

Suppose you take a good artist and a bad artist, give them each a
palette of one thousand colors then ask them to paint a picture. The
good artist produces a work of art, the bad artist an eyesore. Now, in
an attempt to make it harder for the bad artist to do the wrong thing,
you restrict them both to a palette of ten colors. What results? The
good artist produces another work of art, perhaps less subtle than the
first, and the bad artist produces another eyesore, just with less
variation in hue. By restricting the color selection, you haven't made
it harder for the bad artist to create a mess, you've just made it more
difficult for the good artist to use their talent to the fullest. The
worth of the final painting is a function of the artist's talent much
more than it is the availability of colors. So it is too with user
interfaces. The usability of the interface is mostly a function of the
designer's talent and experience, not the number of widgets available to
them.
\end{enumerate}

\item Conclusion
\label{sec:orgheadline299}

There has been a revival of interest lately in rich client interfaces.
It seems that the obsession with web applications that the industry has
experienced in recent years may be starting to thaw. It is finally being
appreciated that it is not OK to squeeze all interaction through the
restrictions currently imposed by web browsers. Even though programmers
may be temporarily enamored of web-based development, their enthusiasm
is not necessarily shared by the user population who must struggle with
the results of their IT department's technical and ideological
enthusiasms.

So now it is time for programmers to impose their technical preferences
regarding rich client interfaces upon an unsuspecting user group, for
which they will need some ostensible justification - hence the cattle
call to SWT, and the unsubstantiated claims in its favor.

For those interested in what actually benefits their organization,
rather than what looks best on their CV and is "cool", there is really
no competition between Swing and SWT. SWT is simply not ready for
generalized interface development, and given that its development lags
behind Swing some seven years, one has to wonder how its use and
continued development can be rationalized.

If you are developing a rich interface in Java, and considering both SWT
and Swing, I urge you to consider the following issues:

\begin{itemize}
\item If you believe that the greater platform fidelity of SWT will make
for a more usable application, what actual evidence do you have to
support that conclusion? Have you put both in front of your user
population?
\item It's hard to find good GUI developers. Finding good GUI developers
with SWT skills is even harder. Where are you going to find the staff
to develop your GUI in SWT? If you anticipate getting Swing
developers to cross-train in SWT, get ready for staff turnover.
Taking a Swing developer and giving them SWT is like taking someone
used to riding a Harley Davidson and giving them a Vespa motor
scooter. They're not likely to be delighted.
\item How close is your target GUI to the Eclipse GUI? Be aware that every
time you step even a little way beyond the functional demands of
Eclipse, you are on your own. You will likely have to start writing
custom widgets in order to get the behavior you want. Can your
organization justify spending time and money writing widgets that in
Swing, would be available off the shelf?
\item Due to the bugs and shortcomings in SWT, your developers will be
working with a lowered productivity, and so you should expect project
delays and/or increased resource requirements. Can your organization
justify this extra investment?
\item Before deciding that Swing applications are slow and ugly, take the
time to look at products like Netbeans and GUI libraries such as
JIDE. I have heard people voice these opinions, having not looked at
Swing since the days of AWT.
\item Is your source of information about SWT the blogs of novice GUI
developers, or those who have had only a fleeting encounter with SWT.
Let me suggest you subscribe to the SWT newsgroup and mailing list
where you will get the perspective of those who have been struggling
with it for a longer period of time, and are past that initial phase
of enthusiasm.
\end{itemize}

Of course, just because SWT is the technically inferior solution doesn't
mean that it will go away. Hype, marketing, vendor over-enthusiasm and
managerial stupidity can propel a second-rate solution to prominence.
This may yet prove to be the case for SWT.

\begin{enumerate}
\item SWT Resources
\label{sec:orgheadline298}

\begin{itemize}
\item \emph{Professional Java Native Applications with SWT/JFace}, J.L. Guojie
\item \emph{Definitive Guide to SWT and JFace}, R. Harris, R. Warner
\item \emph{SWT/JFace in Action}, M. Scarpino et.al.
\item \emph{SWT Developers Notebook}, T. Hatton
\item \emph{Developing Quality Plugins for Eclipse}, E. Clayberg
\item \emph{Contributing to Eclipse}, E. Gamma, K. Beck, Addison Wesley, 2004
\item \emph{SWT: The Standard Widget Toolkit, Volume 1}, S. Northover, M.
Wilson, Addison Wesley, 2004
\item \emph{SWT Designer}, \url{http://www.swt-designer.com/}
\item \emph{SWT Sightings},
\url{http://www.oneclipse.com/Members/admin/news/swt-sightings}
\end{itemize}
\end{enumerate}
\end{enumerate}

\subsection{Debugging and Maintenance}
\label{sec:orgheadline345}

\subsubsection{Debugging 101 \footnote{First published 17 Apr 2006 at
\url{http://www.hacknot.info/hacknot/action/showEntry?eid=85}}}
\label{sec:orgheadline336}

\begin{quote}
“/An interactive debugger is an outstanding example of what is not
needed -- it encourages trial-and-error hacking rather than systematic
design, and also hides marginal people barely qualified for precision
programming./”-- Harlan Mills
\end{quote}

Recently, a colleague and I were working together to resolve a bug in a
piece of code she had just written. The bug resulted in an exception
being thrown and looking at the stack trace, we were both puzzled about
what the root cause might be. Worse yet, the exception originated from
within an open source library we were using. As is typical of open
source products, the documentation was sparse, and wasn't providing us
with very much help in diagnosing the problem before us. It was
beginning to look like we might have to download the source code for
this library and start going through it -- a prospect that appealed to
neither of us.

As a last resort before downloading this source code, I suggested that
we try doing a web search on the text of the exception itself, by
copying the last few lines of the stack trace into the search field for
a web search engine. I hoped the search results might include pages from
online forums where someone else had posted a message like "I'm seeing
the following exception, can anyone tell me what it means?", followed by
all or part of the stack trace itself. If the original poster had
received a helpful response to their query, then perhaps that response
would be helpful to us too.

My colleague, who is reasonably new to software development, was
surprised by the idea and commented that it was something she would
never have thought to try. Her response got me to thinking about
debugging techniques in general, and the way we acquire our knowledge of
them.

Reflecting on my formal education in computer science, I cannot recall a
single tutorial or lecture that discussed how I should go about
debugging the code that I wrote. Mind you, I cannot remember much of
anything about those lectures, so perhaps it really was addressed and
I've simply forgotten. Even so, it seems that the topic of debugging is
much neglected in both academic and trade discussions. Why is this?

It seems particularly strange when you consider what portion of their
time the average programmer spends debugging their own code. I've not
measured it for myself, but I wouldn't be surprised if one third or more
of my day was spent trying to figure out why my code doesn't behave the
way I expected. It seems strange that I never learnt in any structured
way how to debug a program. Everything I know about debugging has been
acquired through experience, trial and error, and from watching others.
Unless my experience is unique, it seems that debugging techniques
should be a topic of vital interest to every developer. Yet some
developers seem almost embarrassed to discuss it.

I suspect the main reason for this is hubris. The ostensible ability to
write bug-free code is a point of pride for many programmers. Displaying
a knowledge of debugging techniques is tantamount to admitting
imperfection, acknowledging weakness, and that really sticks in the craw
of those developers who like to think of themselves as "l337 h4x0r5".
But by avoiding the topic, we lose a major opportunity to learn methods
for combating our inevitable human weaknesses, and thereby improving the
quality of the work we do.

So I've taken it upon myself to list the main debugging techniques that
I am aware of. For many programmers, these techniques will be old hat
and quite basic. But even for veteran debuggers there may be value in
bringing back to mind some of these tried and true techniques. For
others, there might be one or two methods that you hadn't thought of
before. I hope they save you a few hours of frustrating fault-finding.

\begin{enumerate}
\item General Principles
\label{sec:orgheadline306}

Regardless of the specific debugging techniques you use, there are a few
general principles and guidelines to keep in mind as your debugging
effort proceeds.

\begin{enumerate}
\item Reproduce
\label{sec:orgheadline302}

The first task in any debugging effort is to learn how to consistently
reproduce the bug. If it takes more than a few steps to manually trigger
the buggy behavior, consider writing a small driver program to trigger
it programmatically. Your debugging effort will proceed much more
quickly as a result.

\item Progressively Narrow Scope
\label{sec:orgheadline303}

There are two basic ways to find the origin of a bug -- \emph{brute force}
and \emph{analysis}. \emph{Analysis} is the thoughtful consideration of a bug's
likely point of origin, based on detailed knowledge of the code base. A
\emph{brute force} approach is a largely mechanical search along the
execution path until the fault is eventually found.

In practice, you will probably use a combination of both methods. A
preliminary analysis will tell you the area of the code most likely to
contain the bug, then a brute force search within that area will locate
it precisely.

Purists may consider any application of the brute force approach to be
tantamount to hacking. It may be so, but it is also the most expedient
method in many circumstances. The quickest way to search the path of
execution by brute force is to use a binary search, which progressively
divides the search space in half at each iteration.

\item Avoid Debuggers
\label{sec:orgheadline304}

In general, I recommend you avoid symbolic debuggers of the type that
have become standard in many IDEs. Debuggers tend to produce a very
fragile debugging process. How often does it happen that you spend an
extended period of time carefully stepping through a piece of code,
statement by statement, only to find at the critical moment that you
accidentally "step over" rather than "step into" some method call, and
miss the point where a significant change in program state occurs? In
contrast, when you progressively add trace statements to the code, you
are building up a picture of the code in execution that cannot be
suddenly lost or corrupted. This repeatability is highly valuable --
you're monotonically progressing towards your goal.

I've noticed that habitual use of symbolic debuggers also tends to
discourage serious reflection on the problem. It becomes a knee-jerk
response to fire up the debugger the instant a bug is encountered and
start stepping through code, waiting for the debugger to reveal where
the fault is.

That said, there are a small number of situations where a debugger may
be the best, or perhaps only, method available to you. If the fault is
occurring inside compiled code that you don't have the source code for,
then stepping through the just-in-time decompiled version of the
executable may be the only way of subjecting the faulty code to
scrutiny. Another instance where a debugger can be useful is in the case
of memory overwrites and corruption, as can occur when using languages
that permit direct memory manipulation, such as C and C++. The ability
most debuggers provide to "watch" particular memory segments for changes
can be helpful in highlighting unintentional memory modifications.

\item Change Only One Thing At A Time
\label{sec:orgheadline305}

Debugging is an iterative process whereby you make a change to the code,
test to see if you've fixed the bug, make another change, test again,
and so on until the bug is fixed. Each time you change the code, it's
important to change only one aspect of it at a time That way, when the
bug is eventually fixed, you will know exactly what caused it -- namely,
the very last thing you touched. If you try changing several things at
once, you risk including unnecessary changes in your bug fix (which may
themselves cause bugs in future), and diluting your understanding of the
bug's origin.
\end{enumerate}

\item Technical Methods
\label{sec:orgheadline322}

Debugging is a manually intensive activity more like solving logic
problems or brain teasers than programming. You will find little use for
elaborate tools, instead relying on a handful of simple techniques
intelligently applied.

\begin{enumerate}
\item Insert Trace Statements
\label{sec:orgheadline308}

This is the principle debugging method I use. A trace statement is a
human readable console or log message that is inserted into a piece of
code suspected of containing a bug, then generally removed once the bug
has been found. Trace statements not only trace the path of execution
through code, but the changing state of program variables as execution
progresses. If you have used Design By Contract (see “\hyperref[sec:orgheadline307]{Introduce
Design By Contract}” below) diligently, you will already know what
portion of the code to instrument with trace statements. Often it takes
only half a dozen or so well chosen trace statements to pinpoint the
cause of your bug. Once you have found the bug, you may find it helpful
to leave a few of the trace statements in the code, perhaps converting
console messages into file-based logging messages, to assist in future
debugging efforts in that part of the code.

\item Consult The Log Files Of Third Party Products
\label{sec:orgheadline309}

If you're using a third party application server, servlet engine,
database engine or other active component then you'll find a whole heap
of useful information about recently experienced errors in that
component's own log files. You may have to configure the component to
log the sort of information you're interested in. In general, if your
bug seems to involve the internals of some third party product that you
don't have the source code for (and so can't instrument with trace
statements), see if the vendor has supplied some way to provide you with
a window into the product's internal operation. For example, an ORM
library might produce no console output at all by default, but provide a
command line switch or configuration file property that makes it output
all SQL statements that it issues to the database.

\item Search The Web For The Stack Trace
\label{sec:orgheadline310}

Cut the text from the end of a stack trace and use it as a search string
in the web search engine of your choice. Hopefully this will pick up
questions posted to discussion forums, where the poster has included the
stack trace that they are seeing. If someone posted a useful response,
then it might relate to your bug. You might also search on the text of
an error message, or on an error number. Given that search engines might
not discover dynamically generated web pages in discussion forums, you
might also find it profitable to identify web sites likely to host
discussions pertaining to your bug, and use the site's own search
facilities in the manner just described.

\item Introduce Design By Contract
\label{sec:orgheadline307}
In my opinion, DBC is one of the best tools available to assist you in
writing quality code. I have found rigorous use of it to be invaluable
in tracking down bugs. If you're not familiar with DBC, think of it as
littering your code with assertions about what the state of the program
should be at that point, if everything is going as you expect it to.
These assertions are checked programmatically, and an exception thrown
when they fail. DBC tends to make the point of program failure very
close to the point of logical error in your code. This avoids those
frustrating searches where a program fails in function C, but the actual
error was further up the call chain in function A, which passed on
faulty values to function B, which in turn passed the values to function
C, which ultimately failed. It's best to use DBC as a means of bug
prevention, but you can also use it as a means of preventing bug
recurrence. Whenever you find a bug, litter the surrounding code with
assertions, so that if that code should ever go wrong again, a nearby
assertion will fail.

\item Wipe The Slate Clean
\label{sec:orgheadline311}

Sometimes, after you've been hunting a bug for long enough, you begin to
despair of ever finding it. There may be an overwhelming number of
possible sources yet to explore, or the behavior you're observing is
just plain bizarre. On such occasions it can be useful to wipe the slate
clean and start again. Create an entirely new mini-application whose
sole function is to demonstrate the presence of your bug. If you can
write such a demo program, then you're well on your way to tracking down
the cause of the bug. Now that you have the bug isolated in your demo
program, start removing potentially faulty components one by one. For
example, if your demo program uses some database connection pooling
library, cut it out and run the program again. If the bug persists, then
you've just identified one component that doesn't contribute to the
buggy behavior. Proceed in that manner, stripping out as many possible
fault sources as you can, one at a time. When you remove a component
that makes the bug disappear, then you know that the problem is related
to the last component you removed.

\item Intermittent Bugs
\label{sec:orgheadline312}

A bug that occurs intermittently and can't be consistently reproduced is
the programmer's bane. They are often the result of asynchronous
competition for shared resources, as might occur when multiple threads
vie for shared memory or race for access to a local variable. They can
also result from other applications competing for memory and I/O
resources on the one machine.

First, try modifying your code so as to serialize any operations
occurring in parallel. For example, don't spawn N threads to handle N
calculations, but perform all N calculations in sequence. If your bug
disappears, then you've got a synchronization problem between the blocks
of code performing the calculations. For help in correctly synchronizing
your threads, look first to any support for threading that is included
in your programming language. Failing that, look for a third party
library that supports development of multi-threaded code.

If your programming language doesn't provide guaranteed initialization
of variables, then uninitialized variables can also be a source of
intermittent bugs. 99\% of the time, the variable gets initialized to
zero or \texttt{null} and behaves as you expected, but the other 1\% of the time
it is initialized to some random value and fails. A class of tools
called "System Perturbers" \footnote{\emph{Rapid Development}, Steve McConnell, Microsoft Press, 1996} can assist you in tracking down such
problems. Such tools typically include facility for zero-filling memory
locations, or filling memory with random data as a way of teasing out
initialization bugs.

\item Exploit Locality
\label{sec:orgheadline313}

Research shows that bugs tend to cluster together. So when you encounter
a new bug, think of those parts of the code in which you have previously
found bugs, and whether nearby code could be involved with the present
bug.

\item Read The Documentation
\label{sec:orgheadline314}

If all else fails, read the instructions. It's remarkable how often this
simple step is foregone. In their rush to start programming with some
class library or utility some developers will adopt a trial-and-error
approach to using a new API. If there is little or no API documentation
then this may be an appropriate approach. But if the API has some decent
programmer-level documentation with it, then take the time to read it.
It's possible that your bug results from misuse of the API and the
underlying code is failing to check that you have obeyed all the
necessary preconditions for its use.

\item Introduce Dummy Implementations And Subclasses
\label{sec:orgheadline315}

Software designers are sometimes advised to "write to interfaces". In
other words, rather than calling a method on a class directly, call a
method on an interface that the class implements. This means that you
are free to substitute in a different class that implements the same
interface, without needing to change the calling code. While dogmatic
application of this guideline can result in a proliferation of
interfaces that are only implemented once, it does point to a useful
debugging technique. If the outcome of the collaboration between several
objects is buggy, look to the interfaces that the participating objects
implement. Where an object is invoked only via interfaces, consider
replacing the object with a simple, custom object of your own that is
hard-wired to perform correctly under very specific circumstances. As
long as you limit your testing to the circumstances that you know your
custom object handles correctly, you know that any buggy behavior you
subsequently observe must be the fault of one of the other objects
involved. That is, you've eliminated one potential source of the bug.
You can achieve a similar effect by substituting a custom subclass of a
participant class, rather than a custom implementation of an interface.

\item Recompile And Relink
\label{sec:orgheadline316}

A particularly nasty type of bug arises from having an executable image
that is a composite of several different compile and/or relink
operations. The failure behavior can be quite bizarre and it can appear
that internal program state is being corrupted "between statements".
It's like gremlins have crept into your code and started screwing around
with memory.

Most recently, I have encountered this bug in Java code when I change
the value of string constants. It seems the compiler optimizes
references to string constants by inserting them literally at the point
of reference. So the constant value is copied to multiple class files.
If you don't regenerate all those class files after changing the string
constant, those class files not regenerated will still contain the old
value of that constant. Performing a complete recompilation prevents
this from occurring. Finally, set the compiler to include debugging
information in the generated code, and set the compiler warning level to
the maximum.

\item Probe Boundary Conditions And Special Cases
\label{sec:orgheadline317}

Experienced programmers know that it's the limits of an algorithmic
space that tend to get forgotten or mishandled, thereby leading to bugs.
For example, the procedure for deleting records 1 to N might be slightly
different from the procedure for deleting record 0. The algorithm for
determining if a given year is a leap year is slightly different if the
year is divisible by 400. Breaking a string into a list of
space-separated words requires consideration of the cases where the
string contains only one word, or is empty. The tendency to code only
the general case and forget the special cases is a very common source of
error.

\item Check Version Dependencies
\label{sec:orgheadline318}

One of the most obscure sources of a bugs is the use of incompatible
versions of third party libraries. It is also one of the last things to
check when you've exhausted other debugging strategies. If version 1.0.2
of some library has a dependency on version 2.4 of another library, but
you supply version 2.5 instead, the results may be subtle failures that
are difficult or impossible to diagnose. Look particularly to any
libraries that you have upgraded just prior to the appearance of the
bug.

\item Check Code That Has Changed Recently
\label{sec:orgheadline319}

When a bug suddenly appears in functionality that has been working for
some time, you should immediately wonder what has recently changed in
the code base that might have caused this regression.

This is where your version control system comes into its own, providing
you with a way of looking at the change history of the code, or
recreating successively older versions of the code base until you get
one in which the regression disappears.

\item Don't Trust The Error Message
\label{sec:orgheadline320}

Normally you scrutinize the error messages you get very carefully,
hoping for a clue as to where to start your debugging efforts. But if
you're not having any luck with that approach, remember that error
messages can sometimes be misleading. Sometimes programmers don't put as
much thought into the handling and reporting of error conditions as one
would like, so it may be wise to avoid interpreting the error message
too literally, and to consider possibilities other than the ones it
specifically identifies.

\item Graphics Bugs
\label{sec:orgheadline321}

There are a few techniques that are particularly relevant when working
on GUIs or other graphics-related bugs. Check if the graphics pipeline
you are using includes a debugging mode -- a mode which slows down
graphics operations to a speed where you can observe individual drawing
operations occurring. This mode can be very useful for determining why a
sequence of graphic operations don't combine to give the effect you
expected.

When debugging problems with layout managers, I like to set the
background colors of panels and components to solid, contrasting colors.
This enables you to see exactly where the edges of the components are,
which highlights the layout decisions made by the layout managers
involved.
\end{enumerate}

\item Psychological Methods
\label{sec:orgheadline329}

I think it's fair to say that the vast majority of bugs we encounter are
a result of our own cognitive limitations. We might fail to fully
comprehend the effects of a particular API call, forget to free memory
we've reserved, or simply fail to translate our intent correctly into
code. Indeed, one might consider debugging to be the process of finding
the difference between what you instructed the machine to do, and what
you thought you instructed the machine to do. So given their basis in
faulty thinking, it makes sense to consider what mental techniques we
can employ to think more effectively when hunting bugs.

\begin{enumerate}
\item Wooden Indian
\label{sec:orgheadline323}

When you're really stuck on a bug, it can be helpful to grab a colleague
and explain the bug to them, together with the efforts you've made so
far to hunt down its source \footnote{\emph{The Pragmatic Programmer}, A. Hunt and D. Thomas, AddisonWesley,
2000}. It may be that your colleague can
offer some helpful advice, but this is not what the technique is really
about. The role of your colleague is mainly just to listen to your
description in a passive way. It sometimes happens that in the course of
explaining the problem to another, you gain an insight into the bug that
you didn't have before. This may be because explaining the bug's origin
from scratch forces you to go back over mental territory that you
haven't critically examined, and challenge fundamental assumptions that
you have made. Also, by verbalizing you are engaging different sensory
modalities which seems to make the problem "fresh" and revitalizes your
examination of it.

\item Don't Speculate
\label{sec:orgheadline324}

Arthur C. Clarke once wrote "Any sufficiently advanced technology is
indistinguishable from magic." And so it is for any sufficiently
mysterious bug. One of the greatest traps you can fall into when
debugging is to resort to superstitious speculation about its cause,
rather than engaging in reasoned enquiry \footnote{\emph{Code Complete}, Steve McConnell, Microsoft Press, 1993}. Such speculation yields a
trial-and-error debugging effort that might eventually be successful,
but is likely to be highly inefficient and time consuming. If you find
yourself making random tweaks without having some overall strategy or
approach in mind, stop straight away and search for a more rational
method.

\item Don't Be Too Quick To Blame The Tools
\label{sec:orgheadline325}

Perhaps you've had the embarrassing experience of announcing "it must be
a compiler bug" before finding the bug in your own code. Once you've
done it, you don't rush to judgment so quickly in the future. Part of
rational debugging is realistically assessing the probability that there
is a bug in one of the development tools you are using. If you are using
a flaky development tool that is only up to its beta release, you would
be quite justified in suspecting it of error. But if you're using a
compiler that has been out for several years and has proven itself
reliable in the field over all that time, then you should be very
careful you've excluded every other possibility before concluding that
the compiler is producing a faulty executable.

\item Understand Both The Problem And The Solution
\label{sec:orgheadline326}

It's not uncommon to hear programmers declare "That bug disappeared" or
"It must've been fixed as a side-effect of some other work". Such
statements indicate that the programmer isn't seeking a thorough
understanding of the cause of a bug, or its solution, before dismissing
it as no longer in need of consideration. Bugs don't just magically
disappear. If a bug seems to be suddenly fixed without someone having
deliberately attended to it, then there's a good chance that the fault
is still somewhere in the code, but subsequent changes have changed the
way it manifests. Never accept that a bug has disappeared or fixed
itself. Similarly, if you find that some changes you've made appear to
have fixed a bug, but you're not quite sure how, don't kid yourself that
the fix is a genuine one. Again, you may simply have changed the
character of the bug, rather than truly fixing its cause.

\item Take A Break
\label{sec:orgheadline327}

In both bug hunting and general problem solving I've experienced the
following series of events more times than I can remember. After
struggling with a problem for several hours and growing increasingly
frustrated with it, I reach a point where I'm too tired to continue
wrestling with it, so I go home. Several choice expletives are muttered.
Doors are slammed. The next morning, I sit down to continue tackling the
problem and the solution just falls out in the first half hour.

Many have noted that solutions come much easier after a period of
intense concentration on the problem, followed by a period of rest.
Whatever the underlying mechanism might be, if you have similar
experiences its worth remembering them when you're faced with a decision
between bashing your head against a problem for another hour, or having
a rest from it.

Another way to get a fresh look at a piece of code you've been staring
at for too long is to print it out and review it off the paper. We read
faster off paper than off the screen, so this may be why it's slightly
easier to spot an error in printed code than displayed code.

\item Consider Multiple Causes
\label{sec:orgheadline328}

There is a strong human tendency to oversimplify the diagnoses of
problems, attributing what may be multi-causal problems to a single
cause. The simplicity of such a diagnosis is appealing, and certainly
easier to address. The habit is encouraged by the fact that many bugs
really are the result of a single error, but that is by no means
universally the case.
\end{enumerate}

\item Bug Prevention Methods
\label{sec:orgheadline334}

"Prevention is better than cure," goes the maxim; as true of sicknesses
in code as of sicknesses in the body. Given the inevitability and cost
of debugging during your development effort, it's wise to prepare for it
in advance and minimize it's eventual impact.

\begin{enumerate}
\item Monitor Your Own Fault Injection Habits
\label{sec:orgheadline330}

After time you may notice that you are prone to writing particular kinds
of bugs. If you can identify a consistent weakness like this, then you
can take preventative steps. If you have a code review checklist,
augment the checklist to include a check specifically for the type of
bug you favor. Simply maintaining an awareness of your "favorite"
defects can help reduce your tendency to inject them.

\item Introduce Debugging Aids Early
\label{sec:orgheadline331}

Unless you've somehow attained perfection prior to starting work on your
current project, you can be confident that you have numerous debugging
efforts in store before you finish. You may as well make some
preparation for them now. This means inserting logging statements as you
proceed, so that you can selectively enable them later, before
augmenting them with bug-specific trace statements. Also think about the
prime places in your design to put interfaces. Often these will be at
the perimeters of significant subsystems. For example, when implementing
client-server applications, I like to hide all client contact with the
server behind interfaces, so that a dummy implementation of the server
can be used in place of the server, throughout client development. It's
not only a convenient point of interception for debugging efforts, but a
development expedient, as the test-debug cycle can be significantly
faster without the time cost of real server deployment and
communication.

\item Loose Coupling And Information Hiding
\label{sec:orgheadline332}

Application of these two principles is well known to increase the
extensibility and maintainability of code, as well as easing its
comprehension. Bear in mind that they also help to prevent bugs. An
error in well modularized code is less likely to produce unintended
side-effects in other modules, which obfuscates the origin of the bug
and impedes the debugging effort.

\item Write A Regression Test To Prevent Reoccurrence
\label{sec:orgheadline333}

Once you've fixed the bug it's a good idea to write a regression test
that exercises the previously buggy code and checks for correct
operation \footnote{\emph{Writing Solid Code}, Steve Maguire, Microsoft Press, 1993}. If you wish, you can write that regression test before
having fixed the bug, so that a successful bug fix is indicated by the
successful run of the test.
\end{enumerate}

\item Conclusion
\label{sec:orgheadline335}

If you spend enough time debugging, you start to become a bit blasé
about it. It's easy to slip into a rut and just keep following the same
patterns of behavior you always have, which means that you never get any
better or smarter at debugging. That's a definite disadvantage, given
how much of the average programmer's working life is consumed by it.
There's also a tendency not to examine debugging techniques closely or
seriously, as debugging is something of a taboo topic in the programming
community.

It's wise to acknowledge your own limitations up front, the resultant
inevitability of debugging, and to make allowances for it right from the
beginning of application development. It's also worth beginning each
debugging effort with a few moments of deliberate reflection, to try and
deduce the smartest and quickest way to find the bug.
\end{enumerate}

\subsubsection{Spare a Thought for the Next Guy \footnote{First published 26 Apr 2004 at
\url{http://www.hacknot.info/hacknot/action/showEntry?eid=52}}}
\label{sec:orgheadline337}

I just had a new ISDN phone line installed at my house. It was an
unexpectedly entertaining event, and provided the opportunity for some
reflection on the similarity between the problems faced by software
developers and those in other occupations.

The installation was performed by a technician who introduced himself in
a Yugoslavian brogue as "Ranko." Ranko looked at the existing phone
outlets, declared that it would be a straight forward job and should
take 30 - 45 minutes.

I sat down to read a book and let Ranko go about his work.

Things seemed to be going well for him until 15 minutes into the
procedure when I heard some veiled mutterings coming from the kitchen.
Putting my book down to listen more carefully, I heard Ranko talking to
himself in angry tones - "What have they done? What have they done to
poor Ranko?" (he had an unusual habit of referring to himself in the
third person).

Curious, I sauntered into the kitchen on the pretence of making myself a
cup of coffee. I found Ranko still muttering away and staring in angry
disbelief at the display of some instrument. Before him were a half
dozen cables spewing out of the wall like so many distended plastic
intestines set loose from the house's abdominal cavity. Ranko asked if
he could get up into the ceiling cavity of the house, and I assented --
pointing him in the direction of the access cover. He strode outside to
his van and reappeared in my front door a few moments later with a step
ladder under one arm.

I returned to my reading while he pounded around above me. Shortly I
heard a few exclamations of "Bloody hell!" followed by more thumping.
After a brief pause, there came a series of "You bloody idiots!" /
"Bastards!" two-shots in rapid succession, punctuated by some
unnecessarily loud pounding of feet upon the ceiling joists. Underneath,
I listened with growing amusement, choking back laughter with one hand
over my mouth.

For the next 10 minutes or so I was lost in my reading, and looked up in
surprise to find Ranko standing in front of me looking slightly
disheveled but rather proud of himself.

"I have found the problem" he declared proudly, and proceeded to
explain. It appeared the previous residents of the house had
self-installed one of the telephone extensions in my house. Rather than
daisy-chain the additional outlet on from another outlet, they had
simply spliced into the phone line up in the ceiling cavity and run
cabling from the splice point to the new outlet. This was easier for
them than daisychaining, as it halved the number of times they had to
run a phone cable through a wall cavity.

But for future technicians, it meant that any wiring changes of the type
Ranko was attempting would necessitate access to the ceiling cavity
where the splice-point was located. If done in daisy-chain style, as is
regular practice amongst phone technicians, the wiring changes could've
been done without having to ascend into the crawl space above. The job
took nearly twice as long as what Ranko initially estimated, because he
had the unexpected tasks of diagnosing the problem with the existing
installation, determining the location of the splice and then working
around it.

Sound familiar?

Ranko has experienced the same problem that maintenance programmers face
every day. We estimate the duration of a maintenance task based on some
assumptions about the nature of the artifact we will be altering. We
begin the maintenance task, only to find that those assumptions don't
hold, due to some unexpected shortcuts taken by those that came before
us. Then we have to develop an understanding of those shortcuts, before
we can perform our maintenance task.

And if we chose to work around the shortcut rather than fix it, future
maintenance programmers will have the same problems. And so a short-term
expediency made by a programmer long ago becomes the burden of every
programmer that follows.

And the very need to make assumptions at all stems from the absence of
any information about the morphology of the existing system. Those that
hack into the phone line are not of the nature to document their
efforts, nor to keep existing documentation up to date.

So next time you're under dead line pressure and have your fingers
poised above the keyboard ready to take a shortcut -- spare a thought
for the next guy who will have to deal with that shortcut. He might be
you.

\subsubsection{Six Legacy Code AntiPatterns \footnote{First published 2 Feb 2004 at
\url{http://www.hacknot.info/hacknot/action/showEntry?eid=47}}}
\label{sec:orgheadline344}

I recently began work on a J2EE project -- a workflow assistance tool
that has been under development for a few years. The application is
totally new to me and yet is immediately familiar, for it bears the
scars and wounds so common to a legacy system. Browsing through the code
base and playing with the GUI, the half dozen legacy code anti-patterns
that leave me with déjà vu are listed below. How many do you recognize?

\begin{enumerate}
\item Nadadoc
\label{sec:orgheadline338}

The Javadoc has been written in a perfunctory, content-free manner,
giving rise to what I call \emph{Nadadoc}. Here's an example of Nadadoc:

\begin{verbatim}
/**
 * Process an order
 *
 *
 * @param orderID
 * @param purchaseID
 * @param purchaseDate
 * @return
*/
public int processOrder(
  int  orderID, 
  int  purchaseID, 
  Date purchaseDate);
\end{verbatim}

Just enough text is used to assuage any niggling professionalism the
author might be experiencing, without the undertaking the burden of
having to communicate useful information to the reader. Commenting of
code is an afterthought, achieved by invoking the IDE facility for
generating a Javadoc template and performing some token customization of
the result.

\item Abandoned Framework
\label{sec:orgheadline339}

With school boy enthusiasm, the original authors have decided they know
enough about their application domain to build a framework for the
construction of similar applications, the first use of which will be the
product they are trying to write. Such naivety is driven by grand
notions of reuse not yet tarnished by contact with the real world.
Classes constructed early in this project are so insanely generic that
even fundamental types such as \texttt{java.util.Enumeration} are rewritten
with bespoke versions that are ostensibly more general purpose. Classes
constructed later in the project, after the team realizes that
constructing a framework within the time allowed is totally infeasible,
are application specific hack-fests.

\item GUI - Designed By Programmers And Written By Borland
\label{sec:orgheadline340}

Software developers seem to ubiquitously suffer the self-deception that
it is easy to design a good user interface. Perhaps they confuse the
ability to program a GUI with the ability to design one. Perhaps the
commonality of GUIs leads them to think "everyone's doing it, so it must
be easy." In any case, you can often spot a GUI designed by programmers
at a glance. This is certainly the case with my current project. Common
usability guidelines are violated everywhere - no keyboard access to
fields, no keyboard accelerators, group boxes around single controls, no
progress indicators for long operations, illogical and misaligned
layouts.

At the code level, the story is even worse. Many elements of the UI have
been generated by the GUI builder in an IDE -- in this case JBuilder.
Although it is possible to generate semi-acceptable code from these
things, they are rarely used to good effect. When the default control
names and layout mechanisms are used, the generated code becomes a real
maintenance burden, consisting of a complex combination of components
with names like \texttt{panel7}, \texttt{label23} and the like.

\item Oral Documentation Is Mostly Laughter
\label{sec:orgheadline341}

If you can't be bothered writing documentation, the lads at \emph{Fantasy
Central} (otherwise known as XP-land) have provided you with a
ready-made out in the form of the oxymoron "oral documentation". When
maintenance programmers ask "Where's the documentation?" you need only
say (preferably with smug self assurance) "We use oral documentation."

The developers of this system relied very heavily on oral documentation,
and there are just a few problems with it that the XP dreamers generally
neglect to mention:

\begin{itemize}
\item The documentation set becomes self-referential. If you ask John about
component X, he'll refer you to Darren, who refers you to James, who
refers you back to John. Not because they don't have the answers, but
because explaining the inner workings of systems they've left behind
is boring.
\item Parts of the documentation set keep walking out the door due to
attrition. Some chapters are unavailable due to illness.
\item The documentation fades rather quickly. As developers move on and
become ensconced in new projects, the details of the projects they've
left behind quickly fade.
\item Certain pages in an oral documentation set are bookmarked with
laughter. In this system, a great many of them are so marked. The
laughter disguises the embarrassment of the original developers when
you uncover the hacks and shortcuts in their work. Not surprisingly,
developers are loathe to discuss the details of work they know is
sub-standard, and enquiries in these areas result in information that
is a guilty mix of admission and excuse.
\end{itemize}

\item Cargo-Cult Development Idioms
\label{sec:orgheadline342}

When developers can't understand how the code works, they tend to add
functionality by just cutting and pasting segments of existing code that
appear to be relevant to their development goal. There develops a series
of application-specific idioms that are justified with the phrase
"that's just how we do it." No one really knows why - sufficiently
detailed knowledge of the code base to choose amongst implementation
alternatives on a rational basis is lost or not readily available, so
the best chance of success seems to be to follow those implementation
idioms already present in the code.

\item Architecture Where Art Thou?
\label{sec:orgheadline343}

Many developers are not very enthusiastic about forethought. It just
delays the start of coding, and that's where the real fun is. Alas, when
there is no pre-planned structure for that code it tends to grow in a
haphazard, organic and often chaotic way. Rather like growing a vine -
if you train the vine up a trellis, then the resulting plant exhibits at
least a modicum of structure. Without the trellis, the vine wanders
randomly without purpose or regularity. My current project was grown
without a trellis and is riddled with weeds and straggling limbs. The
original developers have, presumably against their will, attempted to
document the project as if there were some intentional underlying
structure. But there is too little accord and too many inconsistencies
between the structure described and the reality of the code base for the
one to have guided the construction of the other.
\end{enumerate}

\subsection{Skepticism}
\label{sec:orgheadline404}

\subsubsection{The Skeptical Software Development Manifesto \footnote{First published 19 Oct 2003 at
\url{http://www.hacknot.info/hacknot/action/showEntry?eid=30}}}
\label{sec:orgheadline360}

\begin{quote}
“Argumentation cannot suffice for the discovery of new work, since the
subtlety of Nature is greater many times than the subtlety of
argument.” -- Francis Bacon
\end{quote}

The over-enthusiastic and often uncritical adoption of XP and Agile
tenets by many in the software development community is worrying.

It is worrying because it attests to the willingness of many developers
to accept claims made on the basis of argument and rhetoric alone. It is
worrying because an over-eagerness to accept technical and
methodological claims opens the door to hype, advertising and wishful
thinking becoming the guiding forces in our occupation. It is worrying
because it highlights the professional gulf existing between software
engineering and other branches of engineering and science, where claims
to discovery or invention must be accompanied by empirical and
independently verifiable experiment in order to gain acceptance.

Without skepticism and genuine challenge, we may forfeit the ability to
increase our domain's body of knowledge in a rational and verifiable
way; instead becoming a group of fashion followers, darting from one
popular trend to another.

What is needed is a renewed sense of skepticism towards the claims our
colleagues make to improved practice or technology. To that end, and to
lend a little balance to the war of assertion initiated by the Agile
Manifesto \footnote{\url{http://agilemanifesto.org/}}, I would like to posit the following alternative.

\begin{enumerate}
\item The Skeptical Software Development Manifesto
\label{sec:orgheadline358}

We are always interested in claims to the invention of better ways of
developing software. However we consider that claimants carry the burden
of proving the validity of their claims. We value:

\begin{itemize}
\item Predictability over novelty
\item Empirical evidence over anecdotal evidence
\item Facts and data over rhetoric and philosophy
\item 
\end{itemize}

That is, while there is value in the items on the right, we value the
items on the left more.

Our skepticism is piqued by claims and rhetoric exhibiting any of the
following characteristics:

\begin{itemize}
\item An imprecision that does not permit further scrutiny or enquiry
\item The mischaracterization of doubt as fear or cynicism
\item Logical and rhetorical fallacies such as those listed below: \footnote{\emph{The Demon-Haunted World}, C. Sagan and A. Druyan, Ballantine
Books, 1996}
\end{itemize}

\begin{enumerate}
\item Argumentum Ad Hominem
\label{sec:orgheadline346}

Reference to the parties to an argument rather than the arguments
themselves.

\item Appeal To Ignorance
\label{sec:orgheadline347}

The claim that whatever has not been proved false must be true, and vice
versa.

\item Special Pleading
\label{sec:orgheadline348}

A claim to privileged knowledge such as "you don't understand", "I just
know it to be true" and "if you tried it, you'd know it was true." \footnote{\emph{How To Win An Argument, 2nd Edition}, M. Gilbert, Wiley, 1996}

\item Observational Selection
\label{sec:orgheadline349}

Drawing attention to those observations which support an argument and
ignoring those that counter it.

\item Begging The Question
\label{sec:orgheadline350}

Supporting an argument with reasons whose validity requires the argument
to be true.

\item Doubtful Evidence
\label{sec:orgheadline351}

The use of false, unreasonable or unverifiable evidence.

\item False Generalization
\label{sec:orgheadline352}

The unwarranted generalization from an individual case to a general
case; often resulting from their being no attempt to isolate causative
factors in the individual case.

\item Straw-Man Argument
\label{sec:orgheadline353}

The deliberate distortion of an argument to facilitate its rebuttal.

\item Argument From Popularity
\label{sec:orgheadline354}

Reasoning that the popularity of a view is indicative of its truth. e.g.
"everybody's doing it, so there must be something to it."

\item Post Hoc Argument
\label{sec:orgheadline355}

Reasoning of the form "B happened after A, so A caused B". i.e.
confusing correlation and causation.

\item False Dilemma
\label{sec:orgheadline356}

Imposing an unnecessary restriction on the number of choices available.
e.g. "either you're with us or you're against us.”

\item Arguments From Authority
\label{sec:orgheadline357}

Arguments of the form "Socrates said it is true, and Socrates is a great
man, therefore it must be true".

We are especially cautious when evaluating claims made by parties who
sell goods or services associated with the technology or method that is
the subject of the claim.
\end{enumerate}

\item Principles Behind The Skeptical Software Development Manifesto
\label{sec:orgheadline359}

We follow these principles:

\begin{itemize}
\item Propositions that are not testable and not falsifiable are not worth
much.
\item Our highest priority is to satisfy the customer by adopting those
working practices which give us the highest chance of successful
software delivery.
\item We recognize that changing requirements incur a cost in their
accommodation, and that claims to the contrary are unproven. We are
obliged to apprise both ourselves and the customer of the realistic
size of that cost.
\item It is our responsibility to identify the degree/frequency of customer
involvement required to achieve success, and to inform our customer
of this. Our customer has things to do other than help us write their
software, so we will make as efficient use of their time as we are
able.
\item We recognize that controlled experimentation in the software
development domain is difficult, as is achieving isolation of
variables, but that is no excuse for not pursuing the most rigorous
examination of claims that we can, or for excusing claimants from the
burden of supporting their claims.
\item Quantification is good. What is vague and qualitative is open to many
interpretations.
\end{itemize}
\end{enumerate}

\subsubsection{Basic Critical Thinking for Software Developers \footnote{First published 18 Jan 2004 at
\url{http://www.hacknot.info/hacknot/action/showEntry?eid=45}}}
\label{sec:orgheadline365}

\begin{enumerate}
\item Vague Propositions
\label{sec:orgheadline361}

A term is called “vague” if it has a clear meaning but not a clearly
demarcated scope. Many arguments on Usenet groups and forums stem from
the combatants having different interpretations of a vaguely stated
proposition. To avoid this sort of misunderstanding, before exploring
the truth of a given proposition either rhetorically or empirically, you
should first state that proposition as precisely as possible.

Consider this proposition: \emph{P(1): Pair Programming works}.

If I were to voice that proposition on the Yahoo XP group \footnote{\url{http://groups.yahoo.com/group/extremeprogramming}}, I would
expect it to receive enthusiastic endorsement. I would also expect no
one to point out that this proposition is non-falsifiable.

It is non-falsifiable because the terms "pair programming" and "works"
are so vague. There are an infinite number of scenarios that I could
legitimately label "pair programming", and an infinite number of
definitions of what it means for that practice to "work." Any specific
argument or evidence you might advance to disprove P(1) will imply a
particular set of definitions for these terms, which I can counter by
referencing a different set of definitions -- thereby preserving P(1).

A vast number of arguments about software development techniques are no
more than heated and pointless exchanges fueled by imprecisely stated
propositions. There is little to be gained by discussing or
investigating a non-falsifiable proposition such as P(1). We need to
formulate the proposition more precisely before it becomes worthy of
serious consideration.

Let's begin by rewording P(1) to clarify what we mean by "works": \emph{P(2):
Pair Programming results in better code}.

Now at least we know we're talking about code as being the primary
determinant of whether pair programming works. However P(2) is now
implicitly relative, which is another common source of vagueness. An
implicitly relative statement makes a comparison with something without
specifying what that something is. Specifically, it proposes that pair
programming produces better code, but better code than what?

Let's try again: \emph{P(3): Pair Programming produces better code than that
produced by individuals programming alone}.

P(3) is now explicitly relative, but still so vague as to be
non-falsifiable. We have not specified what attribute/s we consider
distinguish one piece of code as being "better" than another.

Suppose we think of defect density as being the measure of programmatic
worth: \emph{P(4): Pair programming produces code with a lower defect density
than that produced by individuals programming alone}.

Now we've cleared up what we mean by the word "works" in P(1), let's
address another common source of vagueness -- quantifiers. A quantifier
is a term like "all", "some", "most" or "always". We tend to use
quantifiers very casually in conversation and frequently omit them
altogether. There is no explicit quantifier in P(4), so we do not know
whether the claimant is proposing that the benefits of pair programming
are always manifest, occasionally manifest, or just more often than not.

The quantifier chosen governs the strength of the resulting proposition.
If the proposition is intended as a hard generalization (one that
applies without exceptions), then a quantifier like "always" or "never"
is applicable. If the proposition is intended as a soft generalization,
then a quantifier like "usually" or "mostly" may be appropriate.

Suppose P(4) was actually intended as a soft generalization: /P(5): Pair
programming usually produces code with a lower defect density than that
produced by individuals programming alone/.

P(5) nearly sounds like it could be used as a hypothesis in an empirical
investigation. However the term "pair programming" is still rather
vague. If we don't clarify it, we might conduct an experiment that finds
the defect density of pair programmed code to be higher than that
produced by individuals programming alone, only to find that advocates
of pair programming dismiss our experimental method as not being real
pair programming. In other words, the definition of the term "pair
programming" can be changed on an ad hoc basis to effectively render
P(5) non-falsifiable.

"Pair programming" is a vague term because it carries so many secondary
connotations. The primary connotations of the term are clear enough: two
programmers, a shared computer, one typing while the other advises. But
when we talk of pair programming we tend to assume other things that are
not amongst the primary connotations. These secondary connotations need
to be made explicit for the proposition to become falsifiable. To the
claimant, the term "pair programming" may have the following secondary
connotations:

\begin{itemize}
\item The pair partners contribute more or less equally, with neither one
dominating the activity
\item The pair partners get along with each other i.e. there is a minimum
of unproductive conflict.
\item The benefits of pair programming are always manifest, but to a degree
that may vary with the experience and ability of the particular
individuals.
\end{itemize}

To augment P(5) with all of these secondary connotations will make for a
very wordy statement. At some point we have to consider what level of
detail is appropriate for the context in which we are voicing the
proposition.

\item Non-Falsifiable Propositions
\label{sec:orgheadline362}

Why should we seek to refine a proposition to the point that it becomes
falsifiable? Because a proposition that can not be tested empirically
and thereby determined true or false is beyond the scrutiny of rational
thought and examination. This is precisely why such propositions are
often at the heart of irrational, pseudo-scientific and metaphysical
beliefs.

I contend that such beliefs have no place in the software engineering
domain because they inhibit the establishment of a shared body of
knowledge -- one of the core features of a true profession. Instead,
they promote a miscellany of personal beliefs and superstitions. In such
circumstances, we cannot reliably interpret the experiences of other
practitioners because their belief systems color their perception of
their own experiences to an unknown extent. Our body of knowledge
degrades into a collective cry of "says who?".

Here are a few examples of non-falsifiable propositions that many would
consider incredible:

\begin{itemize}
\item There is a long-necked marine animal living in Loch Ness.
\item The aliens have landed and walk amongst us perfectly disguised as
humans.
\item Some people can detect the presence of water under the ground through
use of a forked stick.
\end{itemize}

Try as you might, you will never prove any of these propositions false.
No matter how many times you fail to find any evidence in support of
these propositions, it remains true that "absence of evidence is not
evidence of absence." If we are willing to entertain non-falsifiable
propositions such as these, then we admit the possibility of some very
fanciful notions indeed.

Here a few examples of non-falsifiable propositions that many would
consider credible:

\begin{itemize}
\item Open source software is more reliable than commercial software
\item Agile techniques are the future of software development
\item OO programming is better than structured programming.
\end{itemize}

These three propositions are, as they stand, just as worthless as the
three propositions preceding them. The subject areas they deal with may
well be fruitful areas of investigation, but you will only be able to
make progress in your investigations if you refine these propositions
into more specific and thereby falsifiable statements.

\item Engage Brain Before Engaging Flame Thrower
\label{sec:orgheadline363}

Vagueness and non-falsifiable propositions are the call to arms of
technical holy wars. When faced with a proposition that seems set to
ignite the passions of the zealots, a useful diffusing technique is to
identify the non-falsifiable proposition and then seek to refine it to
the point of being falsifiable. Often the resulting falsifiable
proposition is not nearly as exciting or controversial as the original
one, and zealots will call off the war due to lack of interest. Also,
the very act of argument reconstruction can be informative for all
parties to the dispute. For example:

\begin{description}
\item[{Zealot}] Real programmers use Emacs
\item[{Skeptic}] How do you define a "real programmer?"
\item[{Zealot}] A real programmer is someone who is highly skilled in
writing code.
\item[{Skeptic}] So what you're claiming is "people who are highly skilled
in writing code use Emacs"?
\item[{Zealot}] Correct.
\item[{Skeptic}] Are you claiming that such people always use Emacs?
\item[{Zealot}] Well, maybe not all the time, but if they have the choice
they'll use Emacs.
\item[{Skeptic}] In other words, they \emph{prefer} to use Emacs over other text
editors?
\item[{Zealot}] Yep.
\item[{Skeptic}] So you're claim is really "people who are highly skilled
in writing code prefer Emacs over other text editors?"
\item[{Zealot}] Fair enough.
\item[{Skeptic}] Are you claiming that \emph{all} highly skilled coders prefer
Emacs, or could there be \emph{some} highly skilled coders that prefer
other text editors?
\item[{Zealot}] I guess there might be a few weird ones who use something
else, but they'd be a minority.
\item[{Skeptic}] So you're claim is really "Most people who are highly
skilled in writing code prefer Emacs over other text editors?”
\item[{Zealot}] Yep.
\item[{Skeptic}] Leaving aside the issue of how you define "highly
skilled", what evidence do you have to support your proposition?
\item[{Zealot}] Oh come on -- \emph{everyone} knows it's true.
\item[{Skeptic}] I don't know it's true, so clearly not \emph{everyone} knows
it's true.
\item[{Zealot}] Alright -- I'm talking here about the programmers that I've
worked with.
\item[{Skeptic}] So are you saying that most of the highly-skilled
programmers you've worked with preferred Emacs, or that they shared
your belief that most highly-skilled programmers prefer Emacs?
\item[{Zealot}] I'm talking about the editor they used, not their beliefs.
\item[{Skeptic}] So your claim is really "Of the people I've worked with,
those who were highly skilled in writing code preferred to use Emacs
over other text editors".
\item[{Zealot}] Yes! That's what I'm saying, for goodness sake!
\item[{Skeptic}] Not quite as dramatic as "real programmers use Emacs", is
it?
\end{description}

You may find that it is not possible to get your opponent to formulate a
specific proposition. They may simply refuse to commit to any specific
claim at all. This reaction is common amongst charlatans and con men.
They only speak in abstract and inscrutable terms (sometimes of their
own invention), always keeping their claims vague enough to deny
disproof. They discourage scrutiny of their claims, preferring to cast
their vagueness as being mysterious and evidence of some deep, unspoken
wisdom. If they cannot provide you with a direct answer to the question
"What would it take to prove you wrong?" then you know you are dealing
with a non-falsifiable proposition, and your best option may simply be
to walk away.

\item Summary
\label{sec:orgheadline364}

Before engaging in any debate or investigation, ensure that the
proposition being considered is at least conceivably falsifiable. A
common feature of non-falsifiable propositions is vagueness.

Such propositions can be refined by:

\begin{itemize}
\item Defining any broad or novel terminology in the proposition
\item Making implicit quantifiers explicit
\item Making implicitly relative statements explicitly relative
\item Making both primary and secondary connotations of the terminology
explicit
\end{itemize}
\end{enumerate}

\subsubsection{Anecdotal Evidence and Other Fairy Tales \footnote{First published 22 Mar 2004 at
\url{http://www.hacknot.info/hacknot/action/showEntry?eid=49}}}
\label{sec:orgheadline379}

As software developers we place a lot of emphasis upon our own
experiences. This is natural enough, given that we have no agreed upon
body of knowledge to which we might turn to resolve disputes or inform
our opinions. Nor do we have the benefit of empirical investigation and
experiment to serve as the ultimate arbiter of truth, as is the case for
the sciences and other branches of engineering - in part because of the
infancy of Empirical Software Engineering as a field of study; in part
because of the difficulty of conducting controlled experiments in our
domain.

Therefore much of the time we are forced to base our conclusions about
the competing technologies and practices of software development upon
our own (often limited) experiences and whatever extrapolations from
those experiences we feel are justified. An unfortunate consequence is
that personal opinion and ill-founded conjecture are allowed to
masquerade as unbiased observation and reasoned inference.

So absolute is our belief in our ability to infer the truth from
experience that we are frequently told that personal experience is the
primary type of evidence that we should be seeking. For example, it is a
frequent retort of the XP/AM\ref{sec:orgheadline223} crowd that one is not entitled to
comment on the utility of XP/AM\ref{sec:orgheadline223} practices unless one has had
first hand experience of them. Only then are you considered suitably
qualified to make comment on the costs and benefits of the practice -
otherwise "you haven't even tried it."

Such reasoning always makes me smile, for two reasons:

\begin{enumerate}
\item It contains the logical fallacy called an "appeal to privileged
knowledge". This is the claim that through experience one will
realize some truth that forbids \emph{a priori} description.\\
\item If a trial is not conducted under carefully controlled conditions, it
is very likely you will achieve nothing more than a confirmation of
your own preconceptions and biases.
\end{enumerate}

This post is concerned with the second point. It goes to the capacity
humans have to let their personal needs, prior expectations, attitudes,
prejudices and biases unwittingly influence the outcomes of technology
and methodology evaluations -- both researchers and subjects. There are
a number of statistical and psychological effects whose influence must
be eliminated, or at least ameliorated, before one can draw valid
deductions from human experiences. Some of these effects are briefly
described in the table below. Conclusions drawn from anecdotal evidence
are frequently invalid precisely because the evidence has been gathered
under circumstances in which no such efforts have been made.

\begin{enumerate}
\item Observational Bias
\label{sec:orgheadline366}

When a researcher allows their own biases to color their interpretation
of experimental results. Selective observation is a common type of
observational bias in which the researcher only acknowledges those
results which are consistent with their pre-formulated hypothesis.

\item Population Bias
\label{sec:orgheadline367}

When experimental subjects are chosen nonrandomly and the resulting
population exhibits some unanticipated characteristic that is an
artifact of the selection process, which influences the outcome of an
experiment in which they participate.

\item The Hawthorne Effect
\label{sec:orgheadline368}

Describes the tendency for subjects to behave uncharacteristically under
experimental conditions where they know they are being watched.
Typically this means the subjects improve their performance in some
task, in an attempt (deliberate or otherwise) to favorably influence the
outcome of the experiment.

\item The Placebo Effect
\label{sec:orgheadline369}

Describes the tendency of strong expectations, particularly among highly
suggestible subjects, to bring about the outcome expected through purely
psychological means.

\item Logical Fallacies
\label{sec:orgheadline373}

Conclusions drawn from anecdotal evidence often exhibit one or more of
the following deductive errors:

\begin{enumerate}
\item Post Hoc Ergo Propter Hoc
\label{sec:orgheadline370}

Meaning "after this, therefore because of this". When events A and B are
observed in rapid succession, the post hoc fallacy is the incorrect
conclusion that A has caused B. It may be that A and B are correlated,
but not necessarily in a causal manner.

\item Ignoring Rival Causes
\label{sec:orgheadline371}

To disregard alternative explanations for a particular effect, instead
focusing only upon a favorite hypothesis of the researcher. It is common
to look for a simple cause of an event when it is really the result of a
combination on many contributory causes.

\item Hasty Generalization
\label{sec:orgheadline372}

The unwarranted extrapolation from limited experimentation into a
broader context.
\end{enumerate}

\item Examples
\label{sec:orgheadline377}

The following scenarios demonstrate how easily one or more of the above
factors can invalidate the conclusions that we reach based on our own
experience - thereby reducing the credibility of those experiences when
later offered as anecdotal evidence in support of a conclusion.

\begin{enumerate}
\item The Linux Enthusiast
\label{sec:orgheadline374}

Chris is a Linux enthusiast. On his home PC he uses Linux exclusively,
and often spends hours happily toying with device drivers and kernel
patches in an effort to get new pieces of hardware working with his
machine. In his work as a software developer he is frequently forced to
use Microsoft Windows, which he has a very low opinion of. He is prone
to waxing lyrical on the unreliability and insecurity of Windows, and
the evil corporate tactics of Microsoft. Whenever he experiences a Blue
Screen of Death on his work machine, his cubicle neighbors know that
once the cursing subsides they are in for another of his speeches about
the massive productivity hit that Windows imposes on the corporate
developer. When surfing the web during his lunch hours, if he should
come across a reference to Linux being used successfully as an
alternative to Windows, then he will print out the article and file it
away for future reference. He is confident that it is only a matter of
time before Linux replaces Windows on the desktop, both in business and
at home.

\textbf{Analysis}: Chris exhibits \emph{observational bias} in a few ways. The hours
he spends getting his Linux machine to recognize a new piece of hardware
is enjoyable to him, and so he chooses not to observe that the same
outcome might be achieved on a Windows system in a minute, thanks to
plug-and-play. When he gets a BSOD, he chooses to observe its negative
effect on his productivity while he waits for a reboot, but chooses to
disregard the productivity cost of his subsequent anti-Microsoft
pontifications. When surfing the web, he \emph{selectively observes} those
stories which are pro-Linux and/or anti-Microsoft in nature. Indeed, the
media is complicit in this practice, because such stories make good
press. There may be many more occasions in which Linux was unsuccessful
in usurping Windows, but they are unremarkable and unlikely to attract
media coverage. His confidence in Linux's ultimate victory based upon
his selective observations is a very \emph{hasty generalization}.

\item The XP Proponent
\label{sec:orgheadline375}

Ryan and his team have been reading a lot about XP recently and are keen
to try it out on one of their own projects. They have had difficulty
getting permission to do so from their management, who are troubled by
some aspects of XP such as pair programming and the informal approach to
documentation. Through constant badgering, Ryan finally gets permission
to use XP on a new project. But he is warned by his management that they
will be watching the project's progress very carefully and reserve the
right to switch the project over to the company's standard methodology
if they think XP is not working out. Overjoyed, Ryan's team begins the
new project under XP. They work like demons for the next six months,
doing everything in their power to make the project a success. At the
end of that time, the project delivers a high quality first release into
the hands of a few carefully chosen customers. Feedback from these
customers is unanimously positive. Management is suitably impressed.
Ryan and his team breathe a sigh of relief.

Analysis: The participants are a self-selected group of enthusiasts,
which is an obvious source of \emph{population bias}. It could be that they
have an above-average level of ability in their work, and a
commensurately higher level of enthusiasm and dedication - which drives
them to try new approaches like XP. Their project's success may be
partly or entirely attributable to these greater capabilities they
already had. Knowing they are being closely evaluated by management and
have put their necks on the line by trying XP despite management's
concerns, they are also victims of the \emph{Hawthorne Effect}. They are very
motivated to succeed, because they perceive potential adverse
consequences for themselves individually if they should fail. If Ryan's
team or their management attributes the project's success to XP itself,
then they are guilty of \emph{ignoring the rival causes} just described. It
may be that they succeeded \emph{despite} XP, rather than because of it.

\item The Revolutionary
\label{sec:orgheadline376}

Seymour thinks there is something wrong with the way university
computing students are taught to program. He feels there is insufficient
exposure to the sorts of problems and working conditions they will
encounter when they finish their degrees. He strongly believes that
students would become better programmers and better employees if there
were a greater emphasis upon group programming assignments in the
academic environment. This would enable them to develop the skills
necessary to function effectively in a team, which is the context in
which they will spend most of their working lives. To demonstrate the
effectiveness of the group approach, he asks for some volunteers from
his third year software engineering class to participate in an
experiment. Rather than do the normal lab work for their course, which
is focused on assignments to be completed by the individual, they will
do different labs designed to be undertaken in groups of four or five.
These labs will be conducted by Seymour himself. About 30 students
volunteer to take part. At the end of the semester, these students sit
the same exams as the other students. Their average mark is 82\% while
the average mark of the other students is 71\%. Seymour feels vindicated
and the volunteer students are pleased to have taken part in a landmark
experiment in the history of computing education.

\textbf{Analysis}: Here is a case of \emph{population bias} that any competent
researcher would be ashamed of. The volunteer group is self-selected,
and so may be biased toward those students that are both more interested
and more capable. Poor performing, disinterested students would be
unlikely to volunteer. The \emph{Hawthorne Effect} comes into play due to the
extra focus that Seymour places upon his volunteer group. They may
receive extra attention and instruction as part of their labs, which may
be enough in itself to improve their final grades. Additionally, knowing
they are part of a select group, at some level they will be motivated to
please the researcher and demonstrate that they have performed well in
their role as "lab rats." Their superior performance in the final exam
may be a result of these confounding factors, and have nothing to do
with the difference between individual and group instruction. It would
certainly be a hasty generalization to conclude that their better \emph{exam
results} will translate into better performance in the workforce.
\end{enumerate}

\item Conclusion
\label{sec:orgheadline378}

I hope this post will give you pause for thought when you next conduct a
technology trial, and when you are next evaluating anecdotal evidence
supplied to you by friends and colleagues. Because personal experiences
are particularly vivid, we often tend to over-value them. From there, we
can easily make unwarranted generalizations and overlook the confounding
effect of our own preconceptions and biases.

In particular, next time one of the XP/AM crowd voice the familiar
retort of "How could you know? You haven't even tried it" - bear in mind
that in the absence of quantification and controlled experimental
technique, they don't know either.
\end{enumerate}

\subsubsection{Function Points: Numerology for Software Developers \footnote{First published 28 Jun 2004 at
\url{http://www.hacknot.info/hacknot/action/showEntry?eid=59}}}
\label{sec:orgheadline167}
\begin{quote}
"\emph{Where else can one get such a marvelous return in conjecture from
such a modest investment of fact?}" -- Mark Twain
\end{quote}

\emph{Numerology} is the study of the occult meanings of numbers and their
influence on human life1. Numerologists specialize in finding numeric
relationships between otherwise disparate figures, and attributing to
them some greater significance.

For instance, some claim that by adding up the component numbers in your
birth date, together with the numeric equivalent of your name (where
A=1, B=2 etc) then a figure is derived that, if properly interpreted,
can yield insight into your personality. \footnote{\emph{The Skeptic's Dictionary}, R. Carroll, Wiley and Sons, 2003.
\url{http://www.skepdic.com/}}

Others consider that the reoccurrence of the number 19 in Islamic texts
is evidence of their authorship by a higher being \footnote{\emph{Did Adam and Eve Have Navels?}, M. Gardner, W.W. Norton and
Company, 2000}. The Koran has
114 (6 x 19) chapters and 6346 verses (19 x 334) and 329,156 (19 x
17,324) letters. The word "Allah" appears 2,698 (19 x 142) times. The
sum of the verse numbers that mention Allah is 118,123 (19 x 6,217).

Pyramids are a favorite topic for numerologists, and there are dozens of
"meaningful" numeric relationships to be found in their dimensions. For
instance, the base perimeter of the Great Pyramid of Cheops is 36,515
inches -- 100 times the number of days in the solar year. And so on.

We can laugh at such desperate searches for meaning, but before we laugh
too hard we should consider that software development has its own brand
of numerology, which we have given the grand name of Function Point
Analysis (FPA).

\begin{enumerate}
\item Overview Of Function Points
\label{sec:orgheadline380}

FPs were proposed in 1979 as a way of finding the size of a piece of
software given only its functional specification. It was intended that
the FP count of an application would be independent of the technology,
people and methods eventually used to implement the application,
focusing as it did upon the functionality the application provided to
the user. Broadly speaking, basic FPs are calculated by following these
steps:

\begin{enumerate}
\item Divide a functional view of the system into components.
\item Classify each component as being one of five types -- external input,
external output, external inquiry, internal logical file or external
interface file.\\
\item Classify the complexity of each component as low, average or high.
The rules for performing this classification vary by component
type.\\
\item For each type of component, multiply the number of components of that
type by a numeric equivalent of the complexity e.g. low = 3, average
= 4, high = 6. The numeric equivalents that apply vary by component
type.\\
\item Sum the results of step 4 across all five component types. The total
is a figure called Unadjusted Function Point count (UFP).You can then
multiply the UFP by a Value Adjustment Factor (VAF) which is based on
consideration of 14 general system characteristics, to yield the
final Function Point count.
\end{enumerate}

I won't bore you with the excruciating specifics of the component
calculations. The above gives you some idea of the nature of FP counting
and it's reliance upon subjective judgments. Specifically, the placement
of component boundaries and the values chosen for the many weighting
factors and characteristics are all determined on a subjective basis.
Some of that subjectivity has been embodied in the standardized FP
counting rules that are issued by the International Function Point Users
Group (IFPUG). \footnote{\url{http://www.ifpug.org/}}

So lacking have FPs been found, that there has been a steady stream of
proposed improvements and alternatives to them since 1979. But none of
these have challenged the basic FP ethos of modeling functional size as
a weighted sum of arbitrarily selected attributes. They simply change
the number and definition of those attributes, and the means by which
they are mangled together into a final figure. The basic chronology of
the FP family tree has been:

\begin{center}
\begin{tabular}{rl}
1979 & Function Points (Albrecht)\\
1986 & Feature Points (Jones)\\
1988 & Mark II Function Points (Symons)\\
1989 & Data Points (Sneed)\\
1991 & 3 D Function Points (Boeing)\\
1994 & Object Points (Sneed)\\
1997 & Full Function Points (St. Pierre et. al)\\
1999 & COSMIC Full Function Points (IFPUG)\\
\end{tabular}
\end{center}

To understand why the FP and its many variants are fundamentally flawed,
it is first necessary to understand the difference between \emph{measuring}
and \emph{rating}.

\item Measurement Vs. Rating
\label{sec:orgheadline381}

To \emph{measure} an attribute of something is to assign numbers to it on an
objective and empirical basis, so that the relationships between the
numbers preserve any intuitive notions and empirical observations about
that attribute. \footnote{\emph{Software Measurement: A Necessary Scientific Basis}, N. Fenton,
IEEE Trans. Software Eng., Vol. 20, No. 3, 1994}

For example, the metric meter is a measure, which implies:

\begin{itemize}
\item 4 meters is twice as long as 2 meters, because 4 is twice 2
\item The difference between 9 and 10 meters is the same as the difference
between 1 and 2 meters, because 10-9 = 2-1
\item If you moved 4 meters in 2 seconds (at constant velocity) then you
moved 2 meters in the first second and 2 meters in the last second.\\
\item If two different people measure the same length to the nearest meter,
they will get the same number.
\end{itemize}

To \emph{rate} an attribute of something is to assign numbers to it on a
subjective and intuitive basis. The relationships between the numbers do
\emph{not} preserve the intuitive and empirical observations about the
attribute. In contrast to the above example, consider the rating out of
10 that a reviewer gives a movie:

\begin{itemize}
\item A movie that gets a 4 is not twice as good as a movie that gets a 2.
\item The difference between movies that get 9 and 10 is not the same as
the difference between movies that get 1 and 2.\\
\item A 2 hour movie that gets a 6 did not rate 3 for the first hour and 3
for the second hour.\\
\item Two different people rating the same movie may award different
ratings.
\end{itemize}

To clarify, suppose a reviewer expresses their assessment of a movie in
words rather than numbers. Instead of rating a movie from 1 to 10, they
rate it from "abysmal" to "magnificent". We might be tempted to think a
movie that gets an 8 is twice as good as a movie that gets a 4, but we
would surely not conclude that "very good" is twice as good as
"disappointing". We can express a rating using any symbols we want, but
just because we choose numbers for our symbols does not mean that we
confer the properties of those numbers upon the attribute we are rating.

In summary:

\begin{itemize}
\item A \emph{measurement} is objective and can be manipulated mathematically.
\item A \emph{rating} is subjective and cannot be manipulated mathematically.
\end{itemize}

\item Function Points Are A Rating, Not A Measurement
\label{sec:orgheadline382}

From the above, it is clear that FPs are a rating and not a measurement,
due to the subjective manner in which they are derived. Hence, they
cannot be manipulated mathematically. And yet the software literature is
rife with examples of researchers attempting to do just that. Many
researchers and reviewers continue to ignore the fundamental
implications of the non-mathematical nature of the FP \footnote{\emph{The Problem with Function Points}, B. Kitchenhas, IEEE Software,
March/April 1997}, such as:

\begin{itemize}
\item \emph{You cannot measure productivity using FPs} -- If a team completes an
application of 250 FP in 10 weeks, their productivity is not 25
FP/week. The figure "25" has no meaning. Similarly, a given team need
not take 50\% longer to write a 1800 FP application as they will a
1200 FP application.\\
\item \emph{You cannot compare FP counts numerically} -- An application of 1000
FP is not twice as big, complex or functional as an application of
500 FP. The first application is not "twice" the second in any
meaningful sense.\\
\item \emph{You cannot compare FPs from disparate sources} -- The subjectivity
of FP analysis makes it sensitive to contextual variations in
application domain, technology, organization and counting method.
\end{itemize}

Given such limitations, there are very few valid uses of an
application's FP count. If the FP counts of two applications differ
markedly, and their contexts are sufficiently similar, then you \emph{may} be
justified in saying that one is functionally bigger than the other, but
not by how much. The notion that FPs can participate in mathematical
calculations, and thereby be used for scheduling, effort and
productivity measures, is without theoretical or empirical basis.

\item Why Are Function Points So Popular?
\label{sec:orgheadline383}

\begin{itemize}
\item Although their use may have declined in recent years, Function Points
are still quite popular. There are several factors which might
account for their continued usage, despite their essential
invalidity:
\item The fact that other organizations use FPs is enough to encourage some
to follow suit. However, we should be aware that an \emph{argument from
popularity} has no logical basis. There are many beliefs that are
both widely held and false. The popularity of FPs may only be
indicative of how desperately the industry would like there to be a
single measure of functional size that can be calculated at the
specification stage. It certainly would be desirable for such a
measure to exist, but we cannot wish such a metric into existence, no
matter how many others have the same wish.
\item Some researchers claim to have validated function points (in their
original form, or some later variant thereof). However, if you
examine the details of these experiments, what you will find is
pseudo-science, ignorance of basic measurement theory and statistics,
and much evidence of "fishing for results." There is a lot of fitting
of models to historical data, but not a lot of using those models to
predict future data. This is not so surprising, for the general
standard of experimentation in software is very poor, as Fenton
observes. Altman makes an observation \footnote{\emph{Statistical Guidelines for Contributors to Medical Journals},
Altman, Gore, Gardner, Pocock, British Medical Journal, Vol. 286,
1983} about the legion of
errors that occur in medical experimentation that could apply equally
well to software development:
\item "The main reason for the plethora of statistical errors is that the
majority of statistical analyses are performed by people with an
inadequate understanding of statistical methods. They are then peer
reviewed by people who are generally no more knowledgeable."
\item Hope springs eternal. Rather than concede that efforts to embody
functional size in a single number are misguided, it is consoling to
think that FPs are "nearly there", just a few more tweaks away from
being useful. Hence the many FP variants that have sprung up.
\item FP enthusiasts selectively quote the "research" that is in their
favor, and ignore the rest. For example, the variance between FP
counts determined by different analysts is often quoted as "plus or
minus 11 percent." \footnote{\emph{Why We Should Use Function Points}, S Furey, IEEE Software,
March/April 1997} However other sources \footnote{\emph{Comparison of Function Point Counting Techniques}, J.Jeffery, G.
Low, M. Barnes, IEEE Trans. Software Eng., Vol. 19, No. 5, 1993} have reported
worse figures, such as a 30\% variation \emph{within} an organization,
rising to more than 30\% \emph{across} organizations.
\item Some choose to dismiss the theoretical invalidities of FPs as
irrelevant to their practical worth. Their excuses may have some
appeal to the average developer, but don't withstand scrutiny.
Examples of such excuses are:
\item \emph{As long as FPs work, who cares what basis they have or don't have?}
\begin{itemize}
\item The problem is that in general, FPs \emph{don't} work. Even FP adherents
\end{itemize}
will admit to the numerous shortcomings of FPs, and the need to
constrain large numbers of contextual factors when applying them.
Witness the various mutations of FP that have arisen, each attempting
to address some subset of the numerous failings of FPs.
\item \emph{It doesn't matter if you're wrong, as long as you're wrong
consistently} \footnote{\emph{Measurement and Estimation}, Burris} -- Unfortunately, unless you know why you're
wrong, you have no way of knowing if you are indeed being
\emph{consistently} wrong. FPs are sensitive to a great many contextual
factors. Unless you know what they are and the precise way they
effect the resulting FP count, you have no way of knowing the extent
to which your results have been influenced by those factors, let
alone whether that influence has been consistent.
\end{itemize}

\item Function Point's True Believers
\label{sec:orgheadline384}

FPs have attracted their own league of True Believers -- like many
technical schools whose tenets, lacking an empirical basis, can only be
defended by the emotional invective of their adherents. I encountered
one such adherent recently in David Anderson, author of "Agile Project
Management." Anderson made some rather pompous observations \footnote{\url{http://www.agilemanagement.net/Articles/Weblog/WorldClassVelocity.html}} on his
blog as to how surprising it was that people should express disbelief
regarding his claims to 5 and 10-fold increases in productivity using
TDD, AM and (insert favorite acronym here)FDD. I replied that their
incredulity might stem from the boldness of his claims or the means by
which he collected his data, rather than an inherently obstreperous
attitude. He indicated his productivity data was expressed in FPs per
unit time! I tried explaining to him that FPs cannot be used to measure
productivity, because not all FPs are created equal, as explained above.
He wasn't interested. That discussion has now been deleted from his
blog. He also denied me permission to reproduce that portion of it which
occurred in email.

Such is the attitude I typically encounter when dealing with self-styled
gurus and experts. There is much talk of science and data, but as soon
as you express doubt regarding their claims, there is a quick resort to
insult and posture. Ironic, given that doubt and criticism are the basic
mechanisms that give science the credibility that such charlatans seek
to cloak themselves in.

\item Why Must Functional Size Be A Single Number?
\label{sec:orgheadline385}

The appeal, and hence the popularity, of FPs is their reduction of the
complex notion of software functional size to a single number. The
simplicity is attractive. But what basis is there for believing that
such a single-figure expression of functional size is even possible?

Consider this analogy. When you walk into a clothing store, you
characterize your size using several different measures. One figure for
shirt size, another for trouser size, another for shoe size and another
for hat size. What if, by way of misguided reductionism, we were to try
and concoct a single measure of clothing size and call it \emph{Clothing
Points}. We could develop all sorts of rules and regulations for
counting Clothing Points, including weighting factors accounting for
age, diet, race, gender, disease and so on. We might even find that if
we sufficiently controlled the influence of external factors, given the
limited variations of the human form, we might eventually be able to
find some limited context in which Clothing Points were a
semi-reasonable assessment of the size of all items of clothing. We
could then walk into a clothing store and say "My size is 187 Clothing
Points" and get a size 187 shirt, size 187 trousers, size 187 shoes and
size 187 hat. The items might even fit, although we would likely
sacrifice some comfort for the expediency and convenience of having
reduced four dimensions down to a single dimensionless number.

The search for a grand unified "measure" of functional size may be just
as foolhardy as the quest for uni-metric clothing.

\item Conclusion
\label{sec:orgheadline386}

The continued use and acceptance of Function Point Analysis in software
development should be a source of acute embarrassment to us all. It is a
prime example of muddle-headed, pseudo-scientific thinking, that has
persisted only because of the general ignorance of measurement theory
and valid experimental methodology that exists in the development
community. We need to stop fabricating and embellishing arbitrary sets
of counting rules. In doing so, we are treating these formulae as if
they were incantations whose magic can only manifest when precisely the
correct wording has been discovered, but whose inner workings must
forever remain a mystery. Rather, we need to go back to basics and work
towards understanding the fundamental technical dimensions that
contribute to the many and varied notions of an application's functional
size. How can we hope to measure something when we can't even precisely
define what that something is? Empiricism holds some promise as a means
to improve software development practices, but the pseudo-empiricism of
Function Point Analysis is little more than numerological voodoo.
\end{enumerate}

\subsubsection{Programming and the Scientific Method \footnote{First published 21 Aug 2004 at
\url{http://www.hacknot.info/hacknot/action/showEntry?eid=64}}}
\label{sec:orgheadline400}

In 1985 Peter Naur wrote a rather cryptic piece entitled \emph{Programming as
Theory Building} \footnote{\emph{Programming as Theory Building}, Peter Naur} in which he drew an analogy between software
development and the scientific method. Since then, other authors have
attempted to co-opt this analogy as a means of enhancing the perceived
credibility of particular programming practices. This post aims to
explain the analogy between the scientific method and programming, and
to explore the limitations of that analogy.

\begin{enumerate}
\item The Scientific Method
\label{sec:orgheadline387}

There is no canonical representation of the scientific method. Different
sources will explain it in different ways, but they are all referring to
the same logical process. For the purposes of this discussion, I will
adopt a simplified definition of the scientific method, considering it
to be comprised of the following activities repeated in a cyclic manner:

\begin{enumerate}
\item \emph{Model} -- Form a simplified model of a system by drawing general
conclusions from existing data.\\
\item \emph{Predict} -- Use the simplified model to make a specific prediction
about how the system will behave when subject to particular
conditions.\\
\item \emph{Test} -- Test the prediction by conducting an experiment.
\end{enumerate}

If the test confirms our prediction, we return to step 2 and make a new
prediction based upon the same model. Otherwise, we return to step 1 and
revise our model so that it accounts for the results of our most recent
test (and all preceding tests).

More formal descriptions of the scientific method often include the
following terms:

\begin{description}
\item[{\emph{Hypothesis}}] A testable statement accounting for a set of
observations. It is equivalent to the model in the above description.
\item[{\emph{Theory}}] A well supported and well tested hypothesis or set of
hypotheses.
\item[{\emph{Fact}}] A conclusion confirmed to such an extent that it would be
reasonable to offer provisional agreement. \footnote{\emph{Why People Believe Weird Things}, Michael Shermer}
\end{description}

\item An Example Of The Scientific Method
\label{sec:orgheadline392}

Suppose you are given a sealed black box that has only three external
features -- two toggle switches marked A and B, and a small lamp. By
playing around with the switches you notice that certain combinations of
switch positions result in the lamp lighting up. Your task is to use the
scientific method to develop a theory of how the box operates. In other
words, to create a model which can account for the observed behavior of
the box.

\begin{enumerate}
\item Round 1
\label{sec:orgheadline388}

\begin{description}
\item[{\emph{Model}}] Casual observation suggests that the switches and lamp are
connected in circuit with an internal power source. Let's suppose
that this is the case, and that the two toggle switches are wired in
series.
\item[{\emph{Predict}}] If our model is accurate, then we should find that
turning both switches on causes the lamp to light up.
\item[{\emph{Test}}] We get the box, turn both switches on and find that the
lamp does indeed light up. Our model has been partially verified. But
there are other predictions we can make based upon it.
\end{description}

\item Round 2
\label{sec:orgheadline389}

\begin{description}
\item[{\emph{Model}}] As in experiment 1.
\item[{\emph{Predict}}] If our model is accurate, then we should find that
turning switch A off and switch B on causes the lamp to go out.
\item[{\emph{Test}}] We get the box, turn switch A off and switch B on and find
that the lamp actually lights up. Our prediction was incorrect,
therefore our model is wrong.
\end{description}

\item Round 3
\label{sec:orgheadline390}

\begin{description}
\item[{\emph{Model}}] Now we need to rework our model so that it correctly
accounts for all our observations thus far. Then we can use it as a
basis for further prediction. Suppose the box were wired with the two
toggle switches in parallel. That would account for our observations
from rounds 1 and 2. Let's make that our new model.
\item[{\emph{Predict}}] If this new model is accurate, then we should find that
turning switch A on and switch B off causes the lamp to light up.
\item[{\emph{Test}}] We get the box, turn switch A on and switch B off and find
that the lamp actually goes off. Our prediction was incorrect;
therefore our new model is wrong.
\end{description}

\item Round 4
\label{sec:orgheadline391}

\begin{itemize}
\item \emph{Model}: Once again, we need to reformulate our model so that
correctly accounts for all of our existing observations. After some
thought, we realize that if the box were wired so that only switch B
effected the lamp, with switch A out of the circuit entirely, then
this would account for all of our existing observations, as well as
giving us a new prediction to test.

\item[{\emph{Predict}}] If this latest hypothesis is true, then we should find
that turning switch A off and switch B off causes the lamp to go out.
\item[{\emph{Test}}] We get the box, turn switch A off and switch B off and
observe that the lamp does indeed go out. Our prediction was correct,
and our model is consistent with our observations from all four
experiments.
\end{itemize}

You can see why the scientific method is sometimes described as being
very inefficient -- there is a lot of trial and error involved. But it's
important to note that it's not random trial and error. If we just made
random predictions and then tested them through experiment, all we would
end up with is a disjoint set of cause/effect observations. We would
have no way of using them to predict how the system would behave under
situations that we hadn't already deserved. Instead, we choose our
predictions deliberately, guided by the intent of testing a particular
aspect of the model currently being considered. In this way, each
experiment either goes some way toward confirming the model, or
confuting it.

Note that all we can ever have is a model of the system. We make no
pretense to know the truth about the system in any absolute sense. Our
model is simply \emph{useful}, at least until new observations are made that
our model can't account for. Then we must change it to accommodate the
new observations. This is why all knowledge in science (even that
referred to as fact) is actually provisional and continually open to
challenge.
\end{enumerate}

\item A Programming Example
\label{sec:orgheadline397}

The following example demonstrates how software development is similar
to the scientific method.

The task is to develop an application which models the behavior of the
black box in the above example. The software will present a simple GUI
with two toggle buttons marked A and B, and an icon which can adopt the
appearance of a lamp turned on or off. The lamp icon should appear to be
turned on as if the lamp were a real lamp connected to an internal power
source, and the toggle buttons were toggle switches, with switch B in
circuit with the lamp, and switch A out of circuit.

The table below compares the activities in the scientific method with
their programming counterparts. Keep these analogs in mind as you read
through the following example. Scientific Method

\begin{center}
\begin{tabular}{lll}
 & Scientific Method & Programming\\
\textbf{Model} & Form a simplified model of a system by drawing general conclusions from existing data. & Developing a mental model of how the software works.\\
\textbf{Predict} & Use the simplified model to make a specific prediction about how the system will behave when subject to particular conditions. & Taking a particular case of interaction with that model, and predicting how the software will respond.\\
\textbf{Test} & Test the prediction by conducting an experiment. & Subjecting software to a test and getting a result.\\
\end{tabular}
\end{center}

\begin{enumerate}
\item Round 1
\label{sec:orgheadline393}

\begin{description}
\item[{\emph{Model}}] Unlike experimentation, we begin by assuming our model is
correct. It is created from our requirements definition and states
"The lamp icon should appear to be turned on as if the lamp were a
real lamp connected to an internal power source, and the toggle
buttons were toggle switches, with switch B in circuit with the lamp,
and switch A out of circuit."
\item[{\emph{Predict}}] If the software is behaving correctly, toggling both
buttons on should result in the lamp icon going on.
\item[{\emph{Test}}] We run the software, toggling the buttons A and B on, and
observe that the lamp icon does indeed come on. So far our hypothesis
has been confirmed; which is to say, the software behaves as the
requirements say it should. But there are other behaviors specified
by the requirements.
\end{description}

\item Round 2
\label{sec:orgheadline394}

\begin{description}
\item[{\emph{Model}}] As per round 1
\item[{\emph{Predict}}] If the software is behaving correctly, then toggling
button A off and button B on will cause the lamp icon to go on.
\item[{\emph{Test}}] We run the software, toggle button A off and button B on,
and find that the lamp icon actually turns off. Our prediction was
incorrect; therefore our software is not behaving as per its
requirements. Instead of adjusting our model to suit the software, we
adjust the software to suit the model i.e. we debug the software. In
the software world, we can change the "reality" we are observing to
behave however we want unlike the real world where we have to adjust
our model to fit an invariant reality. Once the software behaves in a
manner consistent with the above prediction, we have to repeat our
test from round 1 (i.e. regression test), to confirm that the
prediction made there still holds i.e. that we haven't "broken" the
software reality.
\end{description}

\item Round 3
\label{sec:orgheadline395}

\begin{description}
\item[{\emph{Model}}] As per round 1.
\item[{\emph{Predict}}] If the software is behaving correctly, then toggling
button A on and button B off should cause the lamp icon to turn off.
\item[{\emph{Test}}] We run the software, toggle button A on and button B off
and find that the lamp icon actually turns on. Our prediction was
incorrect; therefore our software is in error. Once again we debug
the software until it behaves in a manner consistent with the above
prediction. Then we regression test by repeating the tests in rounds
2 and 3.
\end{description}

\item Round 4
\label{sec:orgheadline396}

\begin{description}
\item[{\emph{Model}}] As per round 1.
\item[{\emph{Predict}}] If the software is behaving correctly, then toggling
buttons A and B off should cause the lamp icon to turn off.
\item[{\emph{Test}}] We run the software, toggle buttons A and B off and find
that the lamp icon does indeed turn off. Our prediction was correct;
therefore the software is behaving as per its requirements.
\end{description}

Notice the critical difference between programming and experimentation.
In experimentation, reality is held invariant and we adjust our model
until the two are consistent. In programming, the model is held
invariant and we adjust our reality (the software) until the two are
consistent.
\end{enumerate}

\item Limits Of The Analogy
\label{sec:orgheadline398}

Rote performance of the model/predict/test cycle does not mean that one
is doing science, or even that one's activities are science-like. There
are critical attributes of the way these activities are carried out that
must be met before the results have scientific validity. Two of these
are objectivity and reproducibility. Some authors have taken the analogy
between scientific method and programming too far by neglecting these
attributes.

McCay3 contends that pair programming is analogous to the peer review
process that scientific results undergo before being published. The
reviewers of a scientific paper are chosen so that they are entirely
independent of the material being reviewed, and can perform an objective
review. They must have no vested interest in the material itself, and no
relationship to the researcher or anyone else involved in the conduct of
the experiment. To this end, scientific peer reviews are often conducted
anonymously. Clearly this independence is missing in pair programming.
Both parties have been intimately involved in the production of the
material being reviewed, and as a coauthor each has a clear personal
investment in it. They have participated in the thought processes that
lead to the code being developed, and so can no longer analyze the
material in an intellectually independent manner.

Mugridge \footnote{3/if (extremeProgramming.equals(scientificMethod))/, Larry McCay} contends that the continuous running of a suite of
regression tests is equivalent to the concept of scientific
reproducibility. But here again, the independence is missing. A single
researcher arriving at a particular result is not enough for those
results to be considered credible by the scientific community.
Independent researchers must successfully replicate these results, as a
way of confirming that they weren't just a chance occurrence, or an
unintentional byproduct of situational factors. But running regression
tests does not provide such confirmation, because each run of the
regression tests is conducted under exactly the same circumstances as
the preceding ones. The same tests are executed in the same environment
over and over again, so there is no independence between one execution
and the next. Thus the confirming effect of scientific reproducibility
is lost.

Both Mugridge and McCay try and equate the XP maxim "do the simplest
thing that could possibly work" (DTSTTCPW) with Occam's Razor. Occam's
razor is a principle applied to hypothesis selection that says "Other
things being equal, the best hypothesis is the simplest one, that is,
the one that makes the fewest assumptions." Because the scientific
hypothesis is analogous to the \emph{system metaphor} in XP, the XP
equivalent of Occam's Razor would be "Other things being equal, the best
system metaphor is the simplest one, that is, the one that makes the
fewest assumptions." However XPers often invoke DTSTTCPW with regard to
implementation decisions, not choice of metaphor. Indeed, the metaphor
is one of the least used of XP practices. \footnote{\emph{Agile and Iterative Development}, C. Larman}

Additionally, the "all other things being equal" part of Occam's razor
is vital, and neglected in XP's DTSTTCPW slogan. We evaluate competing
hypotheses with respect to the criteria of adequacy \footnote{\emph{How to Think About Weird Things, 3rd edition}, T. Schik and L.
Vaughn, McGraw Hill, 2002} -- which
provide a basis for assessing how well each hypothesis increases our
understanding. The criteria include testability, fruitfulness,
simplicity and scope. Note that simplicity is only one of the factors to
consider. The scope of a hypothesis refers to its explanatory power; how
much of reality it can explain and predict. We have a preference for a
hypothesis of broader scope, because it accounts for more natural
phenomena. In a programming context, suppose we have two competing
models of a piece of software's operation. One is more complex than the
other, but the more complex one also has greater scope. Which one is
better? It's a subjective decision; but it should be clear that
considering simplicity alone is a naive basis for hypothesis selection.

\item Conclusion
\label{sec:orgheadline399}

OK, so there are parallels between the scientific method and
programming. Aside from the intellectual interest, what value is there
in recognizing these parallels?

Naur claims that the theory of a piece of software corresponds to the
model that the programmer builds up in their head of how it works. Such
a theory might say "The software is like a box with two toggle buttons
and a lamp", or "The software is like an assembly line with pieces being
added on as the item proceeds". Perhaps multiple metaphors are used.
Once a programmer has a theory (model) of the software in their head,
they can talk about and explain its behavior to others. When they make
changes to the code, they do so in a way that is consistent with the
theory and therefore "fits in" with the existing code base well. A
programmer not guided by such a theory is liable to make modifications
and extensions to the code that appear to be "tacked on" as an
afterthought, and not consistent with the design philosophy of the
existing code base. I believe there is some validity in this notion.

Cockburn then extends this by claiming that this theory is what is
essential to communicate (in documentation or otherwise) from one
generation of programmers to the next: "What should you put into the
documentation? That which helps the next programmer build an adequate
theory of the program". He also sees this as validation of the "System
Metaphor" practice from XP. Perhaps so, but I think there is only
limited utility in identifying what has to be communicated. The real
problem is identifying how to communicate; how to persist that knowledge
in a robust form, and transfer it from one programmer to another as new
programmers arrive on a project and old ones leave.
\end{enumerate}

\subsubsection{From Tulip Mania to Dot Com Mania \footnote{First published 5 Jun 2004 at
\url{http://www.hacknot.info/hacknot/action/showEntry?eid=56}}}
\label{sec:orgheadline403}

\begin{quote}
“Those who cannot remember the past are condemned to repeat it.” --
George Santayana
\end{quote}

Those of us working in IT tend to think of ourselves as being modern,
savvy and much more advanced than our forebears. This conviction is
often accompanied by a certain degree of hubris, and a somewhat derisive
attitude towards older technologies and practitioners. You've probably
encountered this ageist bias in your own work place, or even displayed
it yourself. Older members of our profession are viewed as out-dated and
irrelevant. Older programming languages such as C and FORTRAN are viewed
as inherently inferior to those more recently introduced such as Java
and C\#. Contempt for that which has come before us is as common place as
the fascination with novelty and invention that breeds it.

In our struggle to stay abreast of the rapid rate of change in our
industry, our focus is so intensely upon the present and immediate
future, that we neglect the lessons of the past. We excuse our
parochialism by kidding ourselves that the pace of technological makes
any comparison with the past all but irrelevant anyway. But here lies a
serious error in thinking -- for although technology changes rapidly,
people do not. For example, throughout history there are numerous
examples of large groups of people succumbing to mass panics, group
delusions and popular myths. Notable events are:

\begin{itemize}
\item The \emph{Martian Panic} of 1938, in which many Americans became convinced
that a radio broadcast of H.G. Well's War of the Worlds was a news
broadcast of an actual Martian invasion, leading some to flee their
homes to escape the alien terror. \footnote{\emph{Hoaxes, Myths and Manias}, R. Bartholomew and B. Radford,
Prometheus Books, 2003}
\item The \emph{Roswell Flying Saucer} crash of 1947, a myth sustained by many
even today.\\
\item The widespread belief in \emph{Satanic Ritual Abuse} of children in
America in the 1970's and 1980's.\\
\item The \emph{Witch Mania} of the 15th-17th centuries on multiple continents.
\end{itemize}

Exemplified by the Salem witch trials of 1692.

\begin{itemize}
\item The \emph{Face on Mars} myth of 1976
\end{itemize}

It is easy to dismiss such phenomena as unique to their times, the like
of which could never be experienced by modern, technology-aware,
scientifically informed people such as ourselves. But we view our modern
world with old brains. Psychologically, we have the same predilections
and foibles as the witch-hunters and alchemists of centuries past. We
still experience greed, we still feel a need to belong to a group, and
we can still sustain false and irrational beliefs if we see others doing
the same.

To illustrate our continuing susceptibility to irrational group
behaviors, consider the Tulip Mania of the 1630s, which exhibits
striking parallels with the dot-com mania that would follow it some 400
years later.

\begin{enumerate}
\item Tulip Mania
\label{sec:orgheadline401}

The collecting of tulips began as a fashion amongst the wealthy in
Holland and Amsterdam in the late 16th century \footnote{\emph{Extraordinary Populat Delusions and The Madness of Crowds},
Charles Mackay, Wordsworth Editions, 1995}. The popularity of
the flower spread to England in 1600, and filtered down from the upper
class to the middle class. By 1635 the mania had reached its peak
amongst the Dutch, and preposterous sums were being paid for bulbs of
the rarer varieties. A single bulb of the species Admiral Liefken sold
for 4400 florins, and a Semper Augustus for 5500 florins, at a time when
a sheep cost 10 florins.

In 1636 the demand for rare tulips became so great that regular marts
for their sale were established on the Stock Exchange of Amsterdam. At
this time, speculation in tulip bulbs appeared, and those fortunate
enough to buy low and sell high quickly grew rich. Seeing their friends
and colleagues profiting from the tulip mania, ordinary citizens began
converting their property into cash and investing in bulbs. All were
convinced that Europe's current infatuation with tulips would continue
unabated for the foreseeable future and that vast wealth awaited those
who could satiate the frenzied demands that were sure to come from the
rest of Europe.

But the more prudent began to see that this artificial price inflation
could not be sustained for much longer. As confidence dropped, so too
did the market price of tulips -- never to rise again. Those caught with
significant investments in bulbs were ruined, and Holland's economy
suffered a blow from which it took many years to recover.

There are obvious similarities with the dot com boom -- the artificial
escalation of value, the widening scope of investors, the confusion of
popularity with substance, the progression from investor over-confidence
to widely held belief, and finally, the sudden deflation of value
promoted by the growing awareness of the role that non-financial factors
were playing in the trend.

\item Conclusion
\label{sec:orgheadline402}

It has always been the province of recent generations to view the
mistakes of earlier generations with a contempt derived from the
assumption that they are somehow immune to such follies. Those of us who
are more technology-aware than some others are particularly prone to
this. And yet, even the geekiest techno-junkie can fall prey to the same
psychological and sociological traps that have plagued our species for
centuries. Indeed, far from inuring us to metaphysical thinking, it
seems that the sheer success of science has lead many to deliberately
pursue "alternative" beliefs as a way of restoring some feeling of
mystery and wonder into their lives. A 1990 Gallup poll of 1236 adult
Americans found that 52\% believed in astrology, 46\% in ESP and 19\% in
witches. \footnote{\emph{Why People Believe Weird Things}, Michael Shermer, Henry Holt and
Company, 2002} The result is that superstition and technology are both
coexistent and symbiotic. As software developers, we need to heed the
lessons of the mass manias of the past, acknowledge that we are still
psychologically vulnerable to them today, and guard against their
re-emergence by making a deliberate effort to think critically about the
trends, fashions and hype which so predominate our industry.
\end{enumerate}

\subsection{The Industry}
\label{sec:orgheadline422}

\subsubsection{The Crooked Timber of Software Development \footnote{First published 7 Aug 2005 at
\url{http://www.hacknot.info/hacknot/action/showEntry?eid=77}}}
\label{sec:orgheadline408}

\begin{quote}
“/Out of the crooked timber of humanity no straight thing was ever
made./” -- Immanuel Kant
\end{quote}

Imagine you are a surgeon. You are stitching a wound closed at the end
of a major procedure, when you are approached by the chief surgeon, clad
in theatre garb. He explains that, inkeeping with recently introduced
hospital policy, you are required to use a cheaper, generic brand of
suture material, rather than the more common (and more expensive) brand
you are accustomed to using. He orders you to undo the stitching you've
done, and redo it using the generic brand.

Now you are in an ethical quandary. You know that the cheaper suture
material is not of the same strength and quality as the usual type. You
also know that it is a false economy to skimp on sutures, given that the
amount of money to be saved is trivial, but the increased risk to the
patient is decidedly non-trivial. Further, it seems unconscionable to be
miserly on such critical materials. But on the other hand, the chief
surgeon wields a lot of political might in the hospital, and it would no
doubt be a career-limiting move to ignore his instruction. So what do
you do?

As a health professional, there is simply no question. You are legally
and ethically obliged to act in the best interests of the patient and
there are serious consequences if you fail to do so. The penalties for
malpractice include financial, legal and professional remedies. You can
be fined, sued for malpractice, or struck from the register and rendered
unable to practice. In the light of the system's support and enforcement
of good medical practice, you complete the stitching using the standard
suture material, then express your concerns to the chief surgeon. If you
don't get satisfaction, you can take the matter further.

Now let's examine a similar situation in our own industry. Suppose a
software developer is trying to decide which of a set of competing
technologies should be used on a project. One technology stands out as
clearly superior to the others in terms of its suitability to the
project's circumstances. Upon hearing of the technology chosen, the
company's senior architect informs the developer that they have made the
wrong decision, although they cannot explain why that is the case. The
architect directs you to use a technology you know to be inferior, and
makes it clear that it would be a career-limiting move to ignore his
instruction. Again, what do you do?

My observations over the last twelve years working as a software
developer leave me in no doubt what the probable outcome is. You shake
your head in disbelief, and use the technology you are instructed to
use, knowing that the best interests of both the project and its
sponsors has just been seriously compromised. Why is the situation so
different from the previous medical scenario? The basic answer is this:
medicine is a profession, but software development merely an occupation.

\begin{enumerate}
\item A Profession Is More Than An Occupation
\label{sec:orgheadline405}

As it is used in common parlance, the word "profession" refers to the
principle occupation by which you earn an income. But this is not its
true meaning. A true profession has at least the following
characteristics: \footnote{\emph{After The Gold Rush}, Steve McConnell, Microsoft Press, 1999}

\begin{itemize}
\item \emph{Minimum educational requirements} -- Typically an accredited
university degree must be completed.\\
\item \emph{Certification \& licensing} -- Exams are taken to ensure that a
minimum level of knowledge has been obtained. These exams target an
agreed upon body of knowledge that is considered central to the
profession.\\
\item \emph{Legally binding code of ethics} -- Identifies the behaviors and
conduct considered appropriate and acceptable. Failure to observe the
code of ethics can result in ejection from professional societies,
loss of license, or a malpractice suit.\\
\item \emph{Professional experience} -- A residency or apprenticeship with an
approved organization to gain practical skills.\\
\item \emph{Ongoing education} -- Practitioners are required to undertake a
minimum amount of selfeducation on a regular basis, so that they
maintain awareness of new developments in their field.
\end{itemize}

Notice that software development has none of these elements. Anyone,
regardless of ability, education or experience can hang out a shingle
calling themselves a "software developer," without challenge. Worse,
practitioners may behave in any manner they choose, without restraint.
The strict ethical requirements of a medical practitioner aim to ensure
that the patients needs are best served. In the absence of such
requirements, a software developer is free to scheme, manipulate, lie
and deceive as suits their purpose -- consequently we see a great deal
of exactly this type of behavior in the field.

\item Integrity
\label{sec:orgheadline406}

The key concept in any profession is that of integrity. It means, quite
literally, "unity or wholeness." A profession maintains its integrity by
enforcing standards upon its practitioners, ensuring that those
representing the profession offer a minimum standard of competence.
Viewed from the perspective of a non-practitioner, the profession
therefore offers a consistent promise of a certain standard of work, and
creates the public expectation of a certain standard of service.

Individuals, also, are required to act with integrity. It is not
acceptable for them to say one thing and do another e.g. to promise to
always act in the best interests of a patient or client, but then let
personal interests govern their action. What is said and what is done
must be consistent.

This cultural focus upon integrity is entirely missing from the field of
software development, and demonstrates the vast gap in maturity that
exists between our occupation and the true professions. If we are ever
to make a profession of software development, to move beyond the
currently fractured and uncoordinated group of individuals motivated by
self-interest, with little or no concern for the reputation or
collective future of their occupation, then some fundamental changes in
attitude must occur. We must begin to value both personal and
professional integrity and demonstrate a strong and unwavering
commitment to it in our daily professional lives.

Think about it -- what are your ethical and professional obligations in
your current position. Are you fulfilling them? Look to ethical codes
such as those offered by the ACM \footnote{\url{http://www.acm.org/constitution/code.html}} and the IEEE-CS \footnote{\url{http://www.ieee.org/portal/pages/about/whatis/code.html}}, even if
you are not a member of these societies. Although not legally binding,
they at least demonstrate the sorts of concerns you should be
championing in your everyday work. You will find that their central
focus is upon always acting with integrity; always representing the best
interests of the client. Specifically, you will note that the following
behaviors, as commonplace as they are amongst developers, are
antithetical to ethical conduct:

\begin{itemize}
\item Choosing technologies and solutions because they are "cool", have
novelty value or look good on your CV.\\
\item "Going with the flow" or "keeping a low profile: i.e. remaining
deliberately distant from or ignorant of issues which affect the
quality of service delivered to the customer. You must be willing to
voice unpopular facts or express controversial opinions if you have
reason to believe that not doing so will compromise the service
delivered to a client.\\
\item Distancing yourself from others who are attempting to maintain a
minimum standard of work or conduct, so as to avoid any political
risk yourself. If you are aware of a challenge to the ethical
standards of your profession, you are obliged to defend those
standards, even if you have not been directly involved.\\
\item Letting unethical conduct go unchallenged. To observe unethical
conduct and say nothing is to offer a tacit endorsement of that
behavior. Saying "It's not my problem," "It's none of my business" or
"I'm glad that didn't happen to me" is not acceptable. Next time, it
may be happening to you.
\end{itemize}

There's no denying that acting ethically can have a personal cost,
perhaps quite a profound one. It would be naive to think that attempts
to contradict or combat unethical behavior are not likely to result in
some attempt at retribution. Even in professions with legally binding
codes of ethics, this is the case. In software development, where it is
a moral free-for-all, it is particularly so. Raising ethical objections,
voicing unpopular facts, standing up for the client's rights where they
conflict with some manager's self-interest -- all of these actions bring
a very real risk of retribution from offended parties, that may include
losing your job. Because ours is not a true profession, there is no
protection -- legal or otherwise --- for a developer who speaks the
truth and in so doing defies authority. Whoever is most adept at
bullying, intimidation and political manipulation is likely to hold
sway.

I suspect that more than a few of the incidents we have recently seen
involving the termination of bloggers for alleged indiscretions on their
blogs have been excuses for employers to remove inconvenient employees
who threaten the status quo. Although superficially plausible reasons
may be offered for such action, they may well be nothing more than an
excuse for retribution against the employee for challenges they have
made to the employer's unethical behavior.

\item There Was A Crooked Man
\label{sec:orgheadline407}

In assessing the personal cost of ethical action, it helps to maintain a
broader perspective. In our industry, jobs come and go like the seasons.
Due to the prevalence of contract work, many software developers will
likely have dozens of employers in their careers. Rather than viewing
our work as a series of unrelated engagements, I believe we need to view
our efforts as part of a larger process -- the maturation of an
occupation into a true profession. Seen from this angle, the
significance of any particular job (or the loss of it) is lessened and
the importance of the over-arching principles becomes more obvious.

As they say, the chain is only as strong as its weakest link. The
strength of our reputation and worth as a burgeoning profession is
therefore dependant upon the strength of the individual's commitment to
maintaining a high personal standard of ethics. The integrity of the
whole is contingent upon the integrity of the parts.

Some years ago I read the following statement, which for its truth and
boldness has stuck with me ever since:

\begin{quote}
\emph{The best managers are the ones that come into work each day prepared
to lose their job.}
\end{quote}

In other words, unless you remain willing to walk away from a job, the
threat of termination can always be used against you, and used as
leverage to encourage or excuse unethical behavior. The same reasoning
applies to developers as it does to managers. The same ethical
obligations and the same obstacles to fulfilling them are present.

In 1985, David Parnas resigned his position as member of a U.S. Defense
Department Committee advising on the Strategic Defense Initiative (SDI).
He felt, with good reason, that the goals set for the SDI were entirely
unachievable, and that the public was being misled about the program's
potential. Others urged him to continue, and continued with it
themselves, even though they shared his beliefs about the feasibility of
the programs fundamental objectives. They reasoned that, even though the
desired outcomes wouldn't be achieved, there was good funding to be had
that might be put into ostensibly "contributing efforts", and the
opportunity was too good to miss. When Parnas resigned, he wrote a
series of eight papers \footnote{\emph{Software Fundamentals: Collected Papers} by David L. Parnas,
Addison-Wesley, 2001} outlining both his reasons for doing so,
and the fundamental issues about software professionalism that the SDI
issue had bought to light. Unfortunately, he have very few men of his
quality in our occupation.

Parnas summarized a professional's responsibility in three statements,
which I conclude with here:

\begin{itemize}
\item I am responsible for my own actions and cannot rely on any external
authority to make my decisions for me.\\
\item I cannot ignore ethical and moral issues. I must devote some of my
energy to deciding whether the task that have been given is of
benefit to society.\\
\item I must make sure that I am solving the real problem, not simply
providing short-term satisfaction to my supervisor.
\end{itemize}
\end{enumerate}

\subsubsection{From James Dean to J2EE: The Genesis of Cool \footnote{First published 11 Jan 2004 at
\url{http://www.hacknot.info/hacknot/action/showEntry?eid=43}}}
\label{sec:orgheadline410}

It has always been the purview of the young to define what "cool" means
to their generation. In the fifties, cool was epitomized by James Dean.
Teenagers rushed to emulate him in looks and manner. Cigarettes, leather
jackets, sports cars and a crushing sense of parent-induced angst were
the hallmarks by which these youth declared both their distance from the
previous generation and unity within their own.

In the sixties, the hippy generation stepped off the path to maturity
their parents had planned out for them, put flowers in their hair and
went on a drug assisted exploration of their own psyche to the
soundtrack of Jimi Hendrix and The Jefferson Airplane. The meaning of
cool became a little more diffuse. As an adjective of laid back
approval, it still carried the antiauthoritarian flavor of the previous
decade; but was broad enough to include almost anything of an
unconventional nature.

In the seventies, bigger was better. Wide collars and ties, flared
trousers and ostentatious jewelry were the adornments of the young and
cool. Disco was king and the Bee Gees were the kings of disco. The
definition of cool could only be broadened to accommodate the crass
symbols of consumerism that the cultural elite filled their home and
their wardrobes with. For the first time, cool was as much about earning
capacity as it was about rebellion.

In the eighties, consumerism and technology joined forces to highjack
cool from the hands of the kids. It became an adjunct to the management
buzzwords and marketing neologisms that littered the corporate lingo.
The electronics companies created synthesizers that dominated the music
of the decade, and sold them back to the youth who were wondering what
had become of cool. "Behold", they said, "this is technology and verily,
it is cool."

In the nineties, cool went through its final stage of deconstruction to
become the meaningless mouth-noise that we have today. With the
unexpected rise in popularity of the Web and its accompanying soap
bubble of financial optimism, cool became the adjective of choice for
the technically literate. In keeping with their unfettered enthusiasm
and cavalier attitude, dot-com entrepreneurs everywhere looked up only
briefly from their Palm Pilots to heap uncritical praise upon every new
technology and gadget that passed across their expansive desks.

\begin{enumerate}
\item The Future Of Cool
\label{sec:orgheadline409}

This decade, “cool” means nothing. It is a label applied so ubiquitously
and indiscriminately that it could compete with "nice" for the title of
“Most Ineffectual Adjective in Common Usage.” The retro punk rockers
with their double basses and Gibson Epiphones think they have it. The
Feng Shui consultants and the new age drop-outs think it has something
to do with Atlantis. The advertising executives and middle managers know
that they had it once, but then it slipped between the cushions of their
leather lounges along with their ridiculously miniature mobile phones.

But most laughably of all, we the techies think that we have it.
Surprised to find that technology is now cool, we feel justified in
labeling the geekiest of our enthusiasms with this meaningless
endorsement. Pop quiz: Which of the following are cool?

\begin{itemize}
\item Open source
\item Linux
\item Visual Basic
\item Windows XP
\item Extreme Programming
\item MP3
\item Quake
\item J2EE
\item .NET
\end{itemize}

There are no correct answers to this quiz, and your response means
nothing -- unless you voice it with breathless enthusiasm while gazing
in a shop window.

In the coming year, cool will lead us everywhere and nowhere, with the
following predictable detours:

\begin{itemize}
\item Many software projects will be initiated by software developers with
a cool hammer looking for some business-case nails to justify their
expenditure. Projects thus founded will fail, but not before the
developers have had a nice time playing with their new hammers and
increasing their market appeal to future employers in search of the
latest coolness.\\
\item Many vendors will grunt out another selection of half-baked products
that promise a world of coolness but deliver instead a slew of bugs,
patches and service packs. The products these same vendors previously
marketed as cool will be mysteriously absent from their catalog,
although many of the newer products will bare an uncanny resemblance
to their predecessors.\\
\item The shelves of technical book stores will overflow with 500 page
tomes promising a quick path to mastery of these latest technologies.
The speed with which these books are issued and revised will equal or
exceed the release rate of the technologies they describe.\\
\item Many legacy systems that have been providing satisfactory service for
years will be decommissioned and replaced with systems based on newer
and cooler technologies. These replacements will be less reliable
than their predecessors.\\
\item Technology selection based on hard-headed empiricism will be viewed
as impossibly expensive and time consuming, and abandoned in favor of
emotive decision making based on marketing promises and perceived
tech appeal. We will be too busy climbing the learning curves of the
latest software development gear to have any time remaining in which
to quantifying the costs and benefits of doing so. Hamsters \ldots{}
exercise wheels \ldots{} same old story.
\end{itemize}

The overall success and failure rates of software projects will remain
much as it was last decade, and everyone will bemoan the sad state of
software development.
\end{enumerate}

\subsubsection{IEEE Software Endorses Plagiarism \footnote{First published 2 Oct 2004 at
\url{http://www.hacknot.info/hacknot/action/showEntry?eid=67}}}
\label{sec:orgheadline415}

\begin{quote}
\emph{plagiarize} -- take (the work or an idea of someone else) and pass it
off as one's own. -- The New Oxford Dictionary of English
\end{quote}

Ours is an occupation obsessed with invention and novelty. Every week it
seems that some new technology or development technique arrives,
heralded by a fanfare of hype and a litany of neologisms. So keen are we
to exploit the community's enthusiasm for newness that we will even take
old ideas and rebadge them, offering them up to our colleagues as if
they were original.

Every time I see such reinvention, I feel a certain discomfort. There
seems to me something fundamentally wrong with positing work as being
entirely your own, when it in fact borrows, duplicates or derives from
the work of others.

In science, precedence counts for a great deal and authors are usually
generous and fastidious in providing correct attribution and
acknowledgement of former discoveries which their own work has benefited
from. Indeed, a broad indication of the significance of a paper is the
number of subsequent citations that the work receives. In software
development, there appears to be rather less respect for the
contributions that others make; perhaps even a certain contempt for
prior art.

\begin{enumerate}
\item Fail Fast
\label{sec:orgheadline411}

A particularly egregious example of this disrespect for precedence
appeared in the Sept/Oct 2004 issue of IEEE Software, in an article in
the Design section by Jim Shore called \emph{Fail Fast} \footnote{\emph{Fail Fast}, Jim Shore, IEEE Software, Sept*Oct 2004, pg 21}. The section
editor is Martin Fowler.

Shore describes "\emph{a simple technique that will dramatically reduce the
number of bugs in your software}". His technique, which he considers
"\emph{nonintuitive}" is to write your code so that it fails "\emph{immediately
and visibly}." This is achieved by putting assertions at the beginning
of each method, that check the validity of the values passed to the
method's arguments, throwing a run-time exception if invalid values are
encountered.

For example, if you write a method for finding the positive square root
of a non-negative argument, you make the expectation of "non-negativity"
explicit at the beginning of the method, like this:

\begin{verbatim}
public void squareRoot(float value) {
  if (value < 0.0) {
    throw new SomeException(value);
  }
  // More code goes here
}
\end{verbatim}

This technique is the antithesis of \emph{defensive programming}, which would
encourage us to make the method as tolerant of unexpected input as
possible.

Shore then goes to some lengths to enumerate the strengths of this
technique, such as:

\begin{itemize}
\item When failure occurs, the result is a stack trace that leads directly
to the source of error. Code that doesn't fail-fast can sometimes
propagate errors to other portions of the call hierarchy, finally to
fail in a location quite distant from the original point of error.\\
\item Reduced use or elimination of a debugger; the messages from the
assertion failures are sufficient to localize the error.\\
\item Logging of assertion failures provide excellent debugging information
for maintenance programmers who later diagnose a production failure
from log files.\\
\item Reduced time and cost of debugging.
\end{itemize}

There are no citations anywhere within the article; nor does it specify
any references. The author (and by extension, the editor) are apparently
content to have you believe that this concept is new and original.

\item Design By Contract
\label{sec:orgheadline412}

You may well be familiar with the term \emph{Design by Contract} (DBC). The
term was coined by Bertrand Meyer, and a full exposition of it may be
found in Chapter 11 of his excellent text \emph{Object Oriented Software
Construction 2}. Shore's Fail Fast technique is nothing more than a
re-naming of a subset of the concepts within DBC. In short, “Fail Fast”
is entirely derivative in nature.

For those who have not previously encountered it, DBC is a technique for
specifying the relationship between a class and its clients as a formal
agreement \footnote{\emph{Object Oriented Software Construction, 2nd Edition}, Bertrand
Meyer, Prentice Hall, 1997} -- a \emph{contract}. A contract is expressed as an assertion
of some boolean conditional statement. When the condition is false, the
contract is said to fail; which results in the throwing of a runtime
exception.

Broadly speaking there are three types of contracts -- preconditions,
postconditions and invariants. The \emph{Fail Fast} technique relies only
upon preconditions -- assertions placed at the beginning of a method
that specify the conditions the method assumes to be true. The topic of
DBC is fairly involved, particularly with regard to the way that
contracts accumulate across inheritance relationships. Meyer's exegesis
of DBC is vastly superior to the limited discussion of preconditions
(under the new name “Fail Fast”) given by Shore.

Not only does Shore co-opt the work of others, he combines it with bad
advice regarding the general use of assertions. Shore claims:

\begin{quote}
When writing a method, avoid writing assertions for problems in the
method itself. Tests, particularly test-driven development, are a
better way of ensuring the correctness of individual methods.
\end{quote}

This is the purest nonsense. Assertions are an \emph{excellent} way of
documenting the assumed state of a method mid-way through its operation,
and are helpful to anyone reading or debugging the method body. This was
first pointed out by Alan Turing back in 1950:

\begin{quote}
How can one check a large routine in the sense that it's right? In
order that the man who checks may not have too difficult a task, the
programmer should make a number of definite assertions which can be
checked individually, and from which the correctness of the whole
program easily follows. \footnote{\emph{Checking A Large Routine}, Talk delivered by Alan Turing,
Cambridge, 24 June 1950.}
\end{quote}

In contrast to Shore, Meyer is generous in his acknowledgement of
predecessors and contributors to DBC itself. Section 11.1 of his text
has an entire page of "Bibliographical Notes" in which he acknowledges
the work of Turing, Floyd, Hoare, Dijkstra, Mills and many others.
Indeed, he has delivered an entire presentation on the conceptual
history of DBC prior to his own involvement. \footnote{\emph{Eiffel's Design by Contract: Predecessors and Original
Contributions}, Bertrand Meyer}

\item Giving Credit Where Credit Is Due
\label{sec:orgheadline413}

Such misattribution and inattention to precedence as Shore's harms our
profession in several ways:

\begin{itemize}
\item It is professionally discourteous in that it denies those who develop
and originate work their proper credit.\\
\item It discourages modern readers from exploring the history of the
concepts they are presented with, thereby denying them an opportunity
to deepen their knowledge through exploration of the prior art. Meyer
has already expounded the benefits of "fail fast" versus "defensive
programming" at length. If Shore's article had appropriate citations,
readers would be directed towards this better and more detailed
explanation, and would realize that the concept can be taken much,
much further through postconditions, invariants, and inheritance of
contracts.\\
\item It garners false credit for those who ignore the precedence of
other's work, encouraging others to do the same -- diverting energy
into the re-labeling of already known concepts that could otherwise
be directed into new areas.\\
\item It creates confusion amongst the readership and obfuscates links with
the existing body of knowledge. Central to any epistemological effort
is a consistent naming scheme, so that links between new discoveries
and existing concepts can be identified. Renaming makes it difficult,
particularly for those new to the field, to distinguish new from old
concepts.
\end{itemize}

\item Conclusion
\label{sec:orgheadline414}

To have work published in a peer reviewed journal is a significant
achievement. It means that one's work has been found to make a
worthwhile contribution to the literature, and to be of a high
professional standard. By these criteria, the Fail Fast article by Jim
Shore in the Sept/Oct 2004 issue of IEEE Software should not have been
published. The material it presents as being new and original is a
superficial (and flawed) restatement of earlier work by Meyer, Hoare and
others. It should be cause for concern for us all that a high profile,
professional journal should publish work that is derivative and
misrepresentative. Those who reviewed Shore's article prior to
publication, and the editor/s who approved its publication deserve the
harshest admonishment for effectively endorsing plagiarism.
\end{enumerate}

\subsubsection{Early Adopters or Trend Surfers? \footnote{First published 25 Sep 2003 at
\url{http://www.hacknot.info/hacknot/action/showEntry?eid=24}}}
\label{sec:orgheadline416}

\begin{quote}
\textbf{Q}: What are the most exciting/promising software engineering ideas
or techniques on the horizon?

\textbf{A}: I don't think that the most promising ideas are on the horizon.
They are already here and have been here for years but are not being
used properly.

-- Interview with David L Parnas
\end{quote}

Many software developers pride themselves on being up to date with the
latest software technologies. They live by the credo "beta is better"
and willingly identify themselves as early adopters. The term "early
adopter" comes from the seminal work on technology transfer Diffusion of
Innovations by Everett M. Rogers (1962). He categorizes the users of a
new innovation as being innovators, early adopters, early majority, late
majority and laggards. Innovators and early adopters constitute about
16\% of the user population.

Amongst the software development population, that percentage must be
significantly higher, given the technological orientation of most
practitioners. Consider the following selection of recent technologies
and their respective dates of introduction. Observe how quickly these
technologies have become main stream. In about five years a technology
can go from unknown to common place. In ten years it is passé?

\begin{center}
\begin{tabular}{lr}
\textbf{Technology} & \textbf{Introduced}\\
JSP & 1998\\
EJB & 1998\\
.NET & 2002\\
Java & 1995\\
J2EE & 1999\\
SOAP & 2000\\
Microsoft Windows & 1993\\
GUI & 1974\\
\end{tabular}
\end{center}

Now consider the following software development practices:

\begin{center}
\begin{tabular}{lr}
\textbf{Practice} & \textbf{First Noted}\\
Source code control & 1980\\
Inspections & 1976\\
Branch coverage testing & 1979\\
Software Metrics & 1977\\
Throwaway UI prototyping & 1975\\
Information Hiding & 1972\\
Risk Management & 1981\\
\end{tabular}
\end{center}

Why is it that after, in some cases, 20 years worth of successful
application in the field, often accompanied by repeated empirical
verification of their worth, many of these practices are yet to be
considered even by the early majority?

Adopting new technologies is easy, but changing work practices is hard.
Technologies are "out there" but work practices are distinctly personal.
And new technologies promise immediate gratification by way of
satisfying the hunger for novelty.

\subsubsection{Reuse is Dead. Long Live Reuse. \footnote{First published 4 Aug 2003 at
\url{http://www.hacknot.info/hacknot/action/showEntry?eid=13}}}
\label{sec:orgheadline417}

Reuse is one of the great broken promises of OO. The literature is full
of empirical and anecdotal evidence to this effect. The failure to
realize any significant benefit from reuse is variously ascribed to
technical, organizational and people factors. Observation of the habits
and beliefs of my fellow software engineers over many years leads me to
believe that it is the latter which poses the principle obstacle to
meaningful reuse, and which ultimately renders it unachievable in all
but the most trivial of cases.

Hubris is a common trait amongst software developers and brings with it
a distrust and disrespect for the work of others. This "not invented
here" attitude, as it is commonly known, leads developers to reinvent
solutions to problems already solved by others, driven by the conviction
that the work of anonymous developers must be of dubious quality and
value. Some simply prefer "the devil you know" - figuring that whatever
the shortcomings of a solution they may write themselves, their
familiarity with it will sufficiently reduce the cost of subsequent
maintenance to justify the cost of duplicating the development effort.
Evidence of this drive to reinvention is everywhere. Indeed, the
collective output of the open source movement is proof of the "I can do
better" philosophy in action.

Consider what it is about software development that attracts people to
it. In part, it is the satisfaction that comes from solving technical
problems. In part, it is attraction to the novelty of new technologies.
In part, it is the thrill of creating something that has a life
independent of its original author. Reuse denies the developer all of
these attributes of job satisfaction. The technical problem is already
solved, the new technology has already been mastered (by somebody else),
and the act of creation has already occurred. On the whole, the act of
reuse is equivalent to surrendering the most satisfying aspects of one's
job.

So what degree of reuse can coexist with such a mindset? Certainly we
may abandon hope for any broad reuse such as that promised by
frameworks. Instead, we may expect frameworks themselves to proliferate
like flowers in spring. The greater the scope of the potential reuse,
the greater the opportunity to disguise technology lust and hubris as
genuine concerns over scalability or applicability.

I believe the only reuse likely to be actually realized is in the form
of limited utility libraries and perhaps small GUI components. If the
problem the potentially reusable item solves is seen as technically
novel or intriguing, then reinvention will result. If there is no
entertainment, novelty or career value in reinvention then begrudging
reuse may result simply as a way of avoiding "the boring stuff." But as
long as developers are willing to use their employer's time and money to
satisfy their personal ambitions; and as long as they continue to
believe they hold a personal monopoly on reliable implementation, then
the cost advantage of reuse will remain a gift that we are too proud to
accept.

\subsubsection{All Aboard the Gravy Train \footnote{First published 27 Aug 2006 at
\url{http://www.hacknot.info/hacknot/action/showEntry?eid=89}}}
\label{sec:orgheadline421}

\begin{quote}
“/Hype is the plague upon the house of software./” -- Richard Glass
\end{quote}

It is interesting to watch the software development landscape change
underfoot. As with many geographies, the tremors and shifts which at
first appear random, when more closely examined reveal an underlying
order and structure that is more familiar and less mysterious.

Recently, some of the loudest rumblings have been coming from that
quarter whose current fascination is the scripting language Ruby, and
its database framework Rails. Think back to the last cycle of hype you
saw in our industry -- perhaps the Extreme Programming craze -- and
you'll recognize many of the phenomena from that little reality
excursion now reoccurring in the context of Rubyism. There are wild and
unverifiable claims of improved productivity amidst the breathless
ravings of fan boys declaring how cool it all is. There are comparisons
against precursor technologies, highlight faults that are apparently
obvious in hindsight, but were unimportant while those technologies were
in fashion. And above all there is the frenetic scrambling of the "me
too" crowd, rushing to see what the fuss is all about, desperately
afraid that the bandwagon will pass them by, leaving them stranded in
Dullsville, where nothing is cool and unemployment is at a record high.

But this crowd faces a real dilemma, for there are multiple bandwagons
just ripe for the jumping upon. Which to choose?

The Web 2.0 juggernaut has been on tour for some time, despite the lack
of a cogent definition. The AJAX gang have also been making a lot of
noise, mainly because the Javascript weenies can't contain their
excitement at being in the popular group again.

But how and why does all this techno-fetishism get started?

\begin{enumerate}
\item Now Departing On Platform One
\label{sec:orgheadline418}

\begin{quote}
“/Welcome aboard the gravy train, ladies and gentleman. Our next stop
is Over-enthusiasm Central. Please be advised that critical thought
and a sense of perspective are not permitted in the passenger
compartment. Please ensure that your safety belt is unfastened while
the red dollar sign is illuminated. We know that you have a choice of
bandwagons, and thank you for your choice to bet the farm upon this
one. We promise -- this time it'll be different./"
\end{quote}

The endless cycle of technological and methodological fashions that so
characterizes our industry is the result of a symbiotic relationship
between two groups -- the sellers and the buyers.

The sellers are the parties who are out to create a "buzz," generating a
desire for some technology-related product. They include the corporate
vendors of new technologies such as Sun and IBM. Alongside them are the
pundits and self-promoters who are looking to make a name for
themselves. They attach themselves to particular trends in order to
cross-sell themselves as consultants, authors and speakers. Hot on their
heels are the book publishers and course vendors, who appear with
remarkable speed at the first hint of something new, with a selection of
500 page books and offsite training courses to ease your transition to
the next big thing.

The buyers are the developers who hear the buzz and are drawn to it. And
for many, that draw is very strong indeed, for a variety of reasons.
First, many developers are fascinated with anything new simply because
it is a novelty. The desire to play with new tech toys is what got many
into IT to begin with, and is still their main source of enjoyment in
their working lives. For others, the lure of a new technology lies in
the belief that it might solve all their development woes (rarely is it
stated directly, but that's the tacit promise). It's classic "silver
bullet" thinking of the sort Fred Brooks warned against 25 years ago,
but which is just as deceptively attractive now as then.

Incoming technologies have the same advantage over their predecessors
that opposition political parties have over the governing party; the
shortcomings of the existing option have been revealed through
experience, but the shortcomings of the incoming option are unknown
because nobody has any experience with it. This makes it easy to make
the incoming option look good by comparison. You just focus on the
problems with the old technology, while saying nothing of the problems
that will inevitably accompany the new one. The newer option has an
image that is unblemished by the harsh light of experience. The new
technology is promoted as a key ingredient of forthcoming software
success stories, but those pieces of software are just vaporware, and
vaporware doesn't have any bugs or suffer any performance or
interoperability problems.

It should also be acknowledged that there is a psychological and
emotional appeal to placing such emphasis upon the technological aspect
of software development. It alleviates the burden of self-examination
and introspection upon work practices. It is much easier and more
comfortable to think of all one's problems as being of external origin,
leaving one's self blame free. "As long as the problem is "out there"
somewhere, rather than "in here", we can just jump from one silver
bullet to the next in the hope that maybe this time the vendors have got
it right. Heaven forbid that the way we apply those technologies should
actually have something to do with the sort of outcome we achieve.

But think of this:

\begin{quote}
\emph{Of all the failed and troubled software development efforts you've
been involved in, there is one common element \ldots{} you.}
\end{quote}

\item Your Regularly Scheduled Program
\label{sec:orgheadline419}

Some developers enjoy this perpetual onslaught of marketing efforts, for
it keeps them well supplied with new toys to play with. But some of us
are both tired of the perpetual call to revolution, and concerned for
the long term effect it has upon our profession. I belong to the latter
group.

The main danger that this ever-changing rush to follow technological
fashion has upon us is to distract us from focusing on those aspects of
our work that really matter -- the people who are doing the work and the
working methods they employ. Do you think that the technologies you use
really make much difference to the outcomes your achieve? I suggest they
are generally quite incidental. To understand why, consider this
analogy.

Suppose a group of professional writers gather together for a conference
discussing the nature of the writing activity. You would expect them to
broach such topics as plot, character development, research methods,
editing techniques and so on. But suppose they spent their time
discussing the brand of pen that they preferred to write with. If one
author claimed "My writing has got so much better since I started using
Bic pens" - would you not think that author might be missing something?
If another claimed "That book would have been so much better if it'd
been written with a Parker pen" - you might again think that the speaker
has missed the point. If a third claimed "I write twice as much when I
use a Staedtler pen," you might think that the author is probably making
things up, or at least trying to rationalize a behavior that is really
occurring for emotional or psychological reasons. But isn't this exactly
what we developers do when we claim "This project would have been so
much better if we'd written it in Ruby" or "I'm twice as productive
writing in Java as I am in C++"? In other words, our focus is all wrong.
We're preoccupied with the tools we use, but we should be focused on the
skills and techniques with which we wield those tools.

At the organizational level, this fixation with novelty often works to
create a bad impression of IT's capabilities and proclivities. If those
that make the strategic technology decisions for a company are the type
to get carried away with the latest fads, then that company can find
itself buffeted by the ever-changing fashions of the technical industry,
always switching from one "next big thing" to another, with no concern
for long term maintenance burden and skills investment. It is easy to
create a portfolio of projects implemented in a broad range of diverse
technologies, requiring an equally diverse set of skills from anyone
hoping to later maintain the project. A broad skill base is seldom very
deep, so staff become neophytes in an ever-increasing set of
technologies, none of which have been used for a sufficient time for
them to gain a high level of expertise. From an outsider's perspective,
the IT section seems to be a bunch of boys playing with toys, terminally
indecisive, that for some reason needs to keep re-implementing the same
old applications in progressively newer and cooler technologies, though
successive reimplementations don't seem to be getting any better or more
reliable. It seems that every six to twelve months they suddenly
"realize" that the technologies they're currently using aren't adequate
and a new technology direction is spawned. All that is really happening
is that the novelty of one technology selection has worn off and the
hype surrounding some new novelty is beckoning.

Think of the organizational detritus this leaves behind. You've got
legacy VB applications that can only be maintained by the VB guys,
legacy J2EE systems that can only be maintained by the J2EE guys, a few
.NET applications that only the .NET guys can comprehend, and that
Python script that turned out to be unexpectedly useful, which no one
has been game to touch since the Python enthusiast that wrote it
resigned last year.

How many companies, do you suppose, are now left with monolithic J2EE
systems containing entity beans galore, that were written as the result
of some consultant's fascination with application servers, and their
compulsion to develop a distributed system even if one wasn't required.
And how impressed are the executives in those companies who find
themselves with an enormous, sluggish system that appears to have gone
"legacy" about five minutes after the consultants left the building. Can
we be surprised at their cynicism when they're told their system will
have to be rewritten because it was done poorly by people who didn't
really understand the technologies they were working with (how could
they -- they were learning as they went). How can they leverage their
technology and skill investments when both seem to become irrelevant so
rapidly?

\item What's The Better Way?
\label{sec:orgheadline420}

Thankfully, it doesn't have to be like this. But avoiding the harmful
effects of technology obsession requires some clarity.

At the organizational level, it requires senior technicians to have the
maturity and professional responsibility to put the interests of the
business before their personal preferences. It means developing
technology strategies and standards based solely upon benefit to the
business. It means remembering that there is no ROI on "cool."

At the individual level, it means adopting a skeptical attitude towards
the hype generated by vendors and pundits; and turning one's focus to
the principles and techniques of software development, which transcend
any technology fashion. Your time and energy is better invested in
improving your abilities and skills than in adding another notch to your
technology belt.
\end{enumerate}
\end{document}
